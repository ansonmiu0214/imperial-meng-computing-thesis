\documentclass[12pt,twoside]{report}

\newcommand{\reporttitle}{Type-safe Webservices Generation: Interim Report}
\newcommand{\reportauthor}{Anson Miu}
\newcommand{\reporttype}{MEng Individual Project}
\newcommand{\supervisor}{Prof. Nobuko Yoshida}
\newcommand{\sndmarker}{TBC}
\newcommand{\cid}{your college-id number}

% include files that load packages and define macros
\input{includes} % various packages needed for maths etc.
% quick way of adding a figure
\newcommand{\fig}[3]{
 \begin{center}
 \scalebox{#3}{\includegraphics[#2]{#1}}
 \end{center}
}

%\newcommand*{\point}[1]{\vec{\mkern0mu#1}}
\newcommand{\ci}[0]{\perp\!\!\!\!\!\perp} % conditional independence
\newcommand{\point}[1]{{#1}} % points 
\renewcommand{\vec}[1]{{\boldsymbol{{#1}}}} % vector
\newcommand{\mat}[1]{{\boldsymbol{{#1}}}} % matrix
\newcommand{\R}[0]{\mathds{R}} % real numbers
\newcommand{\Z}[0]{\mathds{Z}} % integers
\newcommand{\N}[0]{\mathds{N}} % natural numbers
\newcommand{\nat}[0]{\mathds{N}} % natural numbers
\newcommand{\Q}[0]{\mathds{Q}} % rational numbers
\ifxetex
\newcommand{\C}[0]{\mathds{C}} % complex numbers
\else
\newcommand{\C}[0]{\mathds{C}} % complex numbers
\fi
\newcommand{\tr}[0]{\text{tr}} % trace
\renewcommand{\d}[0]{\mathrm{d}} % total derivative
\newcommand{\inv}{^{-1}} % inverse
\newcommand{\id}{\mathrm{id}} % identity mapping
\renewcommand{\dim}{\mathrm{dim}} % dimension
\newcommand{\rank}[0]{\mathrm{rk}} % rank
\newcommand{\determ}[1]{\mathrm{det}(#1)} % determinant
\newcommand{\scp}[2]{\langle #1 , #2 \rangle}
\newcommand{\kernel}[0]{\mathrm{ker}} % kernel/nullspace
\newcommand{\img}[0]{\mathrm{Im}} % image
\newcommand{\idx}[1]{{(#1)}}
\DeclareMathOperator*{\diag}{diag}
\newcommand{\E}{\mathds{E}} % expectation
\newcommand{\var}{\mathds{V}} % variance
\newcommand{\gauss}[2]{\mathcal{N}\big(#1,\,#2\big)} % gaussian distribution N(.,.)
\newcommand{\gaussx}[3]{\mathcal{N}\big(#1\,|\,#2,\,#3\big)} % gaussian distribution N(.|.,.)
\newcommand{\gaussBig}[2]{\mathcal{N}\left(#1,\,#2\right)} % see above, but with brackets that adjust to the height of the arguments
\newcommand{\gaussxBig}[3]{\mathcal{N}\left(#1\,|\,#2,\,#3\right)} % see above, but with brackets that adjust to the height of the arguments
\DeclareMathOperator{\cov}{Cov} % covariance (matrix) 
\ifxetex
\renewcommand{\T}[0]{^\top} % transpose
\else
\newcommand{\T}[0]{^\top}
\fi
% matrix determinant
\newcommand{\matdet}[1]{
\left|
\begin{matrix}
#1
\end{matrix}
\right|
}



%%% various color definitions
\definecolor{darkgreen}{rgb}{0,0.6,0}

\newcommand{\blue}[1]{{\color{blue}#1}}
\newcommand{\red}[1]{{\color{red}#1}}
\newcommand{\green}[1]{{\color{darkgreen}#1}}
\newcommand{\orange}[1]{{\color{orange}#1}}
\newcommand{\magenta}[1]{{\color{magenta}#1}}
\newcommand{\cyan}[1]{{\color{cyan}#1}}


% redefine emph
\renewcommand{\emph}[1]{\blue{\bf{#1}}}

% place a colored box around a character
\gdef\colchar#1#2{%
  \tikz[baseline]{%
  \node[anchor=base,inner sep=2pt,outer sep=0pt,fill = #2!20] {#1};
    }%
}%

%%%%%%%%%%%%%%%%%%%%%%%%%%%%%%%

\usepackage{ bm }			% \bm
\usepackage{ cmll }		% \with

% GENERAL
\newcommand{\role}[1]{
{\color{blue}\bm{\mathsf{#1}}}
}
\newcommand{\pt}[1]{
\texttt{pt}\left({#1}\right)
}

% NEW GENERAL
\newcommand{\centroid}[2]{
{#1} \circledast \role{#2}
}

% GLOBAL TYPES
\newcommand{\gcomm}[4]{
\role{#1} \to \role{#2} : \left\{ {#3} \right\}_{#4}
}

% NEW GLOBAL TYPES
\newcommand{\groute}[5]{
\role{#1} \xrightarrow[\role{#3}]{} \role{#2} : \left\{ {#4} \right\}_{#5}
}

% LOCAL TYPES
\newcommand{\tbra}[3]{
\role{#1} \with \left\{ #2 \right\}_{#3}
}
\newcommand{\tsel}[3]{
\role{#1} \oplus \left\{ {#2} \right\}_{#3}
}

% NEW LOCAL TYPES
\newcommand{\router}[4]{
\texttt{route}\left(\role{#1} \to \role{#2}: \left\{{#3}\right\}_{#4}\right)
}

\newcommand{\tselproxy}[4]{
\tsel{\role{#1}_{\role{#2}}}{#3}{#4}
}

\newcommand{\tbraproxy}[4]{
\tbra{\role{#1}_{\role{#2}}}{#3}{#4}
} % short-hand notation and macros


\usepackage[nottoc]{tocbibind}
\setcitestyle{numbers}

% Includes

% Infer
\usepackage{proof}

% Figure and table counter
\usepackage{chngcntr}
\counterwithout{figure}{chapter}

% Commands
\newcommand{\term}[1]{\textit{#1}}
\newcommand{\mathref}[1]{\textsection #1}

\newcommand{\rulename}[1]{[\textsc{#1}]}

\newcommand{\piin}[2]{#1(#2)}
\newcommand{\piout}[2]{\bar{#1}\langle #2 \rangle }
\newcommand{\parti}[1]{{\color{blue}\mathbf{#1}}}

\newcommand{\send}[2]{#1!\langle #2 \rangle}
\newcommand{\recv}[2]{#1?(#2)}
\newcommand{\bra}[2]{#1 \rhd \left\{#2\right\}}
\newcommand{\sel}[2]{#1 \lhd #2}

% Citation style
\setcitestyle{square}


%%%%%%%%%%%%%%%%%%%%%%%%%%%%

\begin{document}
% front page
% Last modification: 2016-09-29 (Marc Deisenroth)
\begin{titlepage}

\newcommand{\HRule}{\rule{\linewidth}{0.5mm}} % Defines a new command for the horizontal lines, change thickness here


%----------------------------------------------------------------------------------------
%	LOGO SECTION
%----------------------------------------------------------------------------------------

\includegraphics[width = 4cm]{./figures/imperial}\\[0.5cm] 

\begin{center} % Center remainder of the page

%----------------------------------------------------------------------------------------
%	HEADING SECTIONS
%----------------------------------------------------------------------------------------
\textsc{\LARGE \reporttype}\\[1.5cm] 
\textsc{\Large Imperial College London}\\[0.5cm] 
\textsc{\large Department of Computing}\\[0.5cm] 
%----------------------------------------------------------------------------------------
%	TITLE SECTION
%----------------------------------------------------------------------------------------

\HRule \\[0.4cm]
{
\huge \bfseries \reporttitle
}\\ % Title of your document
\HRule \\[1.5cm]
\end{center}
%----------------------------------------------------------------------------------------
%	AUTHOR SECTION
%----------------------------------------------------------------------------------------

\noindent
\begin{minipage}{.5\linewidth}
\begin{flushleft} \large
\textit{Author:}\\
\reportauthor
\end{flushleft}
\end{minipage}%
\begin{minipage}{.5\linewidth}
\begin{flushright} \large
\textit{Supervisor:}\\
\supervisor
\end{flushright}
\begin{flushright} \large
\textit{Second Marker:}\\
\sndmarker
\end{flushright}
\end{minipage}

\vspace{2cm}
\makeatletter
\centering
\@date 


\vfill % Fill the rest of the page with whitespace



\makeatother


\end{titlepage}



%%%%%%%%%%%%%%%%%%%%%%%%%%%% 

%\doublespacing
\tableofcontents
%\singlespacing

%%%%%%%%%%%%%%%%%%%%%%%%%%%% Main document

\chapter{Introduction}

% Rise of distributed programs -> main challenges

% How to ensure correctness in general: type system, data types

% Type discipline for concurrent programs - behavioural types (--> session types)

% Mainstream example of distirbuted programs: web services (microservice architecture), interactive web application; 

% Objective: 

\chapter{Background}

\section{Session Types}

\subsection{Overview}
% Motivate different ways of modelling concurrency, narrow the focus to message passing
Web applications are one of many examples of distributed systems in practice. Distributed systems are built upon the interaction between concurrent processes, which can be implemented using the two main communication abstractions in \term{shared memory} and \term{message passing}. 

Shared memory provides processes with the impression of a logical single large monolithic memory but requires programmers to understand consistency models in order to correctly reason about the consistency of shared state.

Message passing interprets the interaction between processes as the exchange of messages, and best describes the communication transports found in web applications, ranging from the stateless request-response client-server interactions via HTTP to full-duplex communication channels via the WebSocket protocol \cite{WebSocketRFC}.

% Outline relevance of process algebra and session types as the relevant typing discipline
The process algebra $\pi$-calculus introduced by Milner \cite{Milner1989} provides a formalism of the message passing abstraction in terms of the basic building blocks of sending and receiving processes, along with inductively defined continuation processes. The composition of these primitives allow us to describe more complex communication sessions.
Session types define the typing discipline for the $\pi$-calculus and provide reliability guarantees for communication sessions; the latter addresses a key challenge when reasoning about the correctness of distributed systems. 

% Outline existing implementations of session types in practice
Many studies are done on the practical applications of session types, from developing languages providing native session type support \cite{ATS2016} to implementing session types in existing programming languages across different paradigms.
Implementations of the latter approach differ by how they leverage the design philosophy and features provided by the programming language. For example, King et al. \cite{PureScript2019} leveraged the expressive type system of PureScript to perform static session type checking, whilst Neykova and Yoshida \cite{Python2017} opted for dynamic approaches to check the conformance of Python programs with respect to session types.

\subsection{Asynchronous $\pi$-calculus}
\begin{itemize}
\item Introduce asynchronous pi calculus
\item Motivate how this can model interaction between concurrent processes through reduction rule in operational semantics
\end{itemize}

\subsection{Binary Session Types}
\begin{itemize}
\item Introduce synchronous pi calculus with branching and selection
\item Motivate with travel agency example
\item Introduce binary session types, notion of duality and subtyping for well-typed binary proesses
\item Guarantees for well-typed sessions?
\end{itemize}

\subsection{Multiparty Session Types}
\begin{itemize}
\item Motivate the extension from BST to MPST (example of battleship game, existing distributed protocols that involve multiple participants?)
\item Introduce global type, projection
\item How does the notion of compatibility (which used to be duality) adapt to MPST?
\end{itemize}

Gentle introduction to MPST - \cite{MPST} 

\section{The Scribble Protocol Language}

\subsection{Overview}
% Introduce Scribble as an implementation of MPST
Whilst session type theory represents the type language for concurrent processes, it also forms the theoretical basis of proposals introduced to implement session types for real-world application development: the Scribble language is one such proposal.

Scribble \cite{Scribble} is a platform-independent description language for the specification of message-passing protocols. The language describes the behaviour of communicating processes at a high level of abstraction: more importantly, the description is independent from implementation details in the same way that the type signature of a function declaration is decoupled from the corresponding function definition.

% Protocol specification: show parallels between Scribble protocol and MPST global type
A Scribble protocol specification describes an agreement of how participating systems, referred to as \term{roles}, interact. The protocol stipulates the sequence of structured messages exchanged between roles; each message is labelled with a name and the type of payload carried by the message.

We present an example of a Scribble protocol in Figure \ref{fig:adder_scr} adapted from \cite{Hybrid2016}. The protocol specifies an arithmetic web service offered by a server to a client, represented by roles $S$ and $C$ respectively. The client is permitted to either:

\begin{itemize}
\item Send two \texttt{int}s attached to an \texttt{Add} message, where the server will respond with an \texttt{int} in a message labelled \texttt{Res}, and the protocol recurses; or,
\item Send a \texttt{Quit} message, where the server will respond with a \texttt{Terminate} message and the protocol ends.
\end{itemize}

The platform-independent nature of Scribble can be observed from the \texttt{type} declaration on Line 1: the developer has the freedom to specify message payload formats and data types from the target language of the implementation - in this case, aliasing the built-in Java integer as \texttt{int} throughout the protocol.

\begin{figure}[!h]
\begin{lstlisting}
type <java> "java.lang.Integer" from "rt.jar" as int;

global protocol Adder(role C, role S) {
	choice at C {
		Add(int, int)	from C to S;
		Res(int)		from S to C;
		do Adder(C, S);
	} or {
		Quit()		from C to S;
		Terminate()	from S to C;	
	}
}
\end{lstlisting}
\caption{Adder Protocol in Scribble}
\label{fig:adder_scr}
\end{figure}

The simplicity of the protocol specification language reflects the design goals for Scribble, as outlined in \cite{Scribble}, to be easy to read, write and maintain, even for developers who are not accustomed to the formalities in protocol specification. Moreover, we clearly observe the parallels between the Scribble language and multiparty session type theory (MPST), from the homomorphic mapping between Scribble roles and MPST participants to the syntactic similarities between the specification in Figure \ref{fig:adder_scr} and the global type below written in the calculus.

\[
\begin{array}{rl}
G = \mu\mathbf{t}.\parti{C}\to\parti{S}:\{
& \text{Add}(\texttt{int}, \texttt{int}): \parti{S}\to\parti{C}: \{\text{Res}(\texttt{int}): \mathbf{t}\}, \\
& \text{Quit}(): \parti{S}\to\parti{C}: \{\text{Terminate}(): \mathbf{end}\}\\
\} &
\end{array}
\]


\subsection{Endpoint Finite State Machines}
\begin{itemize}
\item Show parallels between Scribble protocol-to-EFSM and MPST global-to-local projections
\item Formalise syntax and properties of EFSM derived from well-formed protocols
\end{itemize}

\begin{figure}[!hb]
\doublespacing
\[
\begin{array}{rlr}

\text{EFSM} ::= & \mathbb{R} \times \mathbb{L} \times \mathbb{T} \times \Sigma \times \mathbb{S} \times \delta & \text{Endpoint FSM} \\

\mathbb{R} ::= & r,~r',~\dots & \text{Role Identifiers} \\

\mathbb{L} ::= & l,~l',~\dots & \text{Message Label Identifiers} \\

\mathbb{T} ::= & \texttt{int},~\texttt{bool},~T,~T',~\dots & \text{Payload Format Types} \\

\Sigma ::= & & \text{Actions} \\
     & r!l(\tilde{T}) \quad \text{where } \tilde{T} \subseteq \mathbb{T} & \text{Output} \\
\mid & r?l(\tilde{T}) \quad \text{where } \tilde{T} \subseteq \mathbb{T} & \text{Input} \\

\mathbb{S} ::= & S,~S',\dots & \text{State Identifiers} \\

\delta ::= & \mathbb{S} \times \Sigma \rightharpoonup \mathbb{S} & \text{State Transition Function} \\

\end{array}
\]
\singlespacing
\caption{Syntax for Endpoint FSM}
\end{figure}

\begin{figure}[!hb]
\doublespacing
\[
\begin{array}{rl}

\texttt{initial}_{\mathbb{R}, \mathbb{L}, \mathbb{T}, \Sigma, \mathbb{S}, \delta}(S) \iff & \nexists S' \in \mathbb{S},\alpha \in \Sigma.~\delta(S',\alpha) = S \\
\texttt{terminal}_{\mathbb{R}, \mathbb{L}, \mathbb{T}, \Sigma, \mathbb{S}, \delta}(S) \iff & \delta(S) = \emptyset \\
\texttt{output}_{\mathbb{R}, \mathbb{L}, \mathbb{T}, \Sigma, \mathbb{S}, \delta}(S) \iff & \delta(S) = \{\,\alpha \in \Sigma \mid \exists r \in \mathbb{R}, l \in \mathbb{L}, \tilde{T} \subseteq{\mathbb{T}}.~\alpha = r!l(\tilde{T})\,\} \\
\texttt{input}_{\mathbb{R}, \mathbb{L}, \mathbb{T}, \Sigma, \mathbb{S}, \delta}(S) \iff & \delta(S) = \{\,\alpha \in \Sigma \mid \exists r \in \mathbb{R}, l \in \mathbb{L}, \tilde{T} \subseteq{\mathbb{T}}.~\alpha = r?l(\tilde{T})\,\} \\
\end{array}
\]
\singlespacing
\caption{Types of EFSM States}
\end{figure}

\clearpage

\section{Code Generation}
The EFSM describes the local session type and provides guidance to developers for verifying that their endpoint implementation conforms to the communication protocol. However, a direct encoding of the local session type into the target language of the implementation is usually not feasible as the EFSM assumes IO objects as first-class citizens and communication channels are linear resources, features that are left to be desired in mainstream programming languages (such as Java and Python) used by the developers' implementations. We note that languages with native session type support \cite{ATS2016} do exist, but their usage largely remains for research purposes as supposed to real-world application development.

Code generation is a common approach for verifying implementations written in the aforementioned mainstream languages against the EFSM. Approaches in the literature differ by how they leverage features in the target language (such as the type system), but generally define some interpretation of the EFSM in the target language and generate APIs which the developer can use to implement a target application that guarantees the following two properties:

\begin{itemize}
\item \textbf{Behavioural typing}: The execution trace of messages sent and received by the application is accepted by the EFSM.
\item \textbf{Channel linearity}: Each transition in the EFSM represents a channel resource. When the application transitions from some state $S$ to some successor state $S'$, it must no longer be able to access a reference (e.g. have an alias) to $S$.
\end{itemize}

We outline the different existing approaches and summarise how they verify the aforementioned properties in \mathref{\ref{section:codegencompare}}.

\subsection{Runtime Monitors}
% What
Neykova and Yoshida targeted the MPST methodology for Python programs in \cite{Python2017} and proposed to generate \term{runtime monitors} from the EFSM. These monitors expose APIs for sending and receiving messages, which is used by the developer in their implementation. The runtime monitor is an abstraction between the developer's implementation and the actual communication channel, and ``executes'' the EFSM internally to ensure protocol conformance. When the developer sends a message (with some label and payload) using the API, the runtime monitor checks whether this send action conforms to the current EFSM state, and if so, performs the send and advances to the successor state. Likewise, when the developer invokes a receive, the runtime monitor verifies that this is permitted at the current EFSM state before returning the received payload.

We observe that this approach complements the dynamic typing nature of the Python language, which makes it sensible to perform behavioural typing at runtime. As the send and receive IO primitives are made available to the developer, there are no ``instances'' of channel resources created, so the developer cannot explicitly hold a reference to some state in the EFSM (let alone keep aliases), so channel linearity is trivially guaranteed here.

\subsection{Type-Level Encoding}
King et al. presented an approach in \cite{PureScript2019} for integrating session types into web development using the PureScript language, which takes advantage of its expressive type system to provide static guarantees on implementation conformance with respect to the protocol. We outline the main components of their EFSM encoding:

\paragraph{Actions as type classes} The semantics of the state transition function in the EFSM formalism express that a tuple of state-action \textit{uniquely defines} a successor state. These semantics can be expressed by \textit{multi-parameter type classes (MPTC) with functional dependencies}: \texttt{class Send r s t a | s -> t r a} defines \texttt{Send} as a MPTC parameterised by recipient \texttt{r}, current state \texttt{s}, successor state \texttt{t} and payload type \texttt{a}, and \texttt{s -> t r a} expresses the functional dependency that, for an instance of this type class, the current state uniquely determines the successor state, the recipient and payload type. These type classes are independent of the EFSM.

\paragraph{Transitions as instances of type classes} By encoding states as data types, valid EFSM transitions are encoded as \textit{instances} of the type classes. If \texttt{S1} is an output state, sending an \texttt{Int} to \texttt{Svr} with successor \texttt{S2}, we would encode \texttt{instance SendS1 :: Send Svr S1 S2 Int}. Because of the functional dependency, the developer cannot instantiate an invalid transition (e.g. \texttt{Send Svr S1 S3 Bool}, since \texttt{S1} uniquely determines the other type parameters. 

This proposal is relevant to the problem we are tackling, and we appreciate that the intricacy of the library design in the communication combinators to conceal the channel resources is something we can build upon in our solution. We also observe the challenges of applying session types to the front-end environment, as shown by the careful choice made in \cite{PureScript2019} to use the \textit{Concur UI} framework which builds UI elements sequentially to model sequential sessions; not doing so would require binding channel resources into event listeners on UI elements, which makes linearity violation possible (e.g. by binding a send transition to a button click event, the user might click the button twice, thus reusing the channel).

\subsection{Hybrid Session Verification}
Whilst runtime monitors and type-level encoding achieve communication safety guarantees via dynamic and static approaches respectively, Hu and Yoshida implemented a workflow in \cite{Hybrid2016} that performs hybrid verification of communication protocol conformance for Java applications. We highlight the two main components below:

\paragraph{Static session typing} States are represented as classes; supported transitions on each state are represented as instance methods, parameterised by the role, label and payload involved in the message exchange. A send method takes the payload to send as a parameter, whilst a receive method is a blocking call that requires the caller to allocate a \texttt{Buf<T>} wrapper on the stack (where \texttt{T} is the expected payload type), then the receive method populates the payload into the wrapper and returns upon receiving from the channel. These instance methods return a new instance of the successor state class.

\paragraph{Runtime linear channel usage checks} Each state keeps track of its usage in a private boolean flag and throws an exception when the instance method is called twice, which signifies channel reuse and a linearity violation. Similarly, the \texttt{SessionEndpoint} class keeps track of whether the connection is open or close, and throws an exception when program execution exits the scope of the session endpoint and a terminal state has yet to be reached, signifying a protocol violation as the session hasn't completed yet but is now out of scope.

We find that this proposal strikes a good balance between maximising static communication safety guarantees whilst providing an intuitive set of APIs for developers to efficiently write their applications. For example, the encoding of transitions as instance methods that return new instances of the successor state exposes the channel resource, and since Java does not build support for linear resources and does not monitor aliasing of variables, linear channel usage is monitored dynamically. However, this complements the imperative style of application code written in Java, and takes advantage of the type system to statically enforce valid transitions in the code.

\subsection{Comparison}
\label{section:codegencompare}

We compare how these existing code generation approaches provide communication safety guarantees in Table \ref{table:comparison}.


\begin{figure}[!h]
\centering
\begin{tabular}{l || p{0.35\textwidth} | p{0.35\textwidth}}
%\begin{tabular}{l||l|l}
Language & Behavioural typing & Channel linearity \\
\hline\hline
Python \cite{Python2017} & Dynamically enforced - runtime monitor represent the EFSM and only execute supported transitions. & Trivially guaranteed - channel resources not exposed, developer uses \texttt{send()} and \texttt{receive()} primitives. \\
\hline
PureScript \cite{PureScript2019} & Type-level encoding - EFSM states are encoded as types, IO actions are encoded as multi-parameter type classes (to express the semantics of the state transition function) and EFSM transitions are encoded as instances of said type classes. & Statically guaranteed - channels are not exposed in the provided combinators to prevent reuse, and session constructor requires a \texttt{Session} continuation parameterised by initial and terminal state to prevent incomplete sessions.  \\
\hline
Java \cite{Hybrid2016} & Statically guaranteed - states are encoded as classes; transitions are encoded as instance methods on the state class and return a new instance of the successor state class. & Checked at runtime - states keep \texttt{used} boolean flag to detect and prevent reuse, Session API implements \texttt{AutoCloseable} interface to be used in resource try-catch block to prevent unused.
\end{tabular}
\captionof{table}{Comparison between existing MPST code generation approaches.}
\label{table:comparison}
\end{figure}

\clearpage

\section{TypeScript}
\label{section:typescript}

We introduce the TypeScript language 
\cite{UnderstandingTypeScript}
as our choice of target language
for session type API generation.
Developed by \textit{Microsoft Research},
TypeScript is an extension to
JavaScript to address
the deficiencies of the latter in
\textit{developing} and \textit{maintaining}
large-scale complex applications.
Syntactically,
TypeScript is a \textit{superset} of JavaScript,
so every JavaScript program is a
TypeScript program.
The TypeScript Compiler is used
to compile a TypeScript program into JavaScript
source code, with full type erasure.

We introduce specific language
features used to implement our API generation
solution throughout the report as needed;
here, we highlight the key properties
of the type system implemented in the language.

\paragraph{Structural Typing}
In a structural type system, type equivalence
is determined by \textit{shape} rather than by name
(which is the case in a \textit{nominal} type system).

Consider the following TypeScript code:

\begin{lstlisting}[language=javascript]
class ThisSquare {
	constructor(public side: number) { }
};

class ThatSquare {
	constructor(public side: number) { }
};

const area = (sq: ThisSquare) => sq.side * sq.side;

area(new ThisSquare(2));	// ok
area(new ThatSquare(2));	// ok (*@\label{line:structural1}@*)
area({ side: 2 });			// ok (*@\label{line:structural2}@*)
\end{lstlisting}

The \texttt{area} function takes a \texttt{ThisSquare}
as parameter.
Under a structural type system,
\cref{line:structural1,line:structural2}
will type-check, because \texttt{ThatSquare} and the object
literal created from scratch matches the \textit{shape}
of \texttt{ThisSquare} -- all of them have a \texttt{side}
property typed \lstonelinejs{number}.

In languages (e.g. Java) that use a nominal type system, 
\cref{line:structural1}
will not type-check because \texttt{ThatSquare} is not named
\texttt{ThisSquare}.

\paragraph{Gradual Typing}
In a gradual type system, a program can have parts
that are statically typed and other parts are dynamically
typed \cite{GradualTyping}.
TypeScript distinguishes dynamically typed code using
the \lstonelinejs{any} type.

\begin{lstlisting}[language=javascript]
// Invoke remote API
fetch('https://jsonplaceholder.typicode.com/todos/1')
	// Convert to JavaScript Object Notation
	.then((response: Response) => response.json())
	.then((json: any) => {
		// Up to the developer to correctly deserialise;
		// incorrect implementations will cause
		// runtime type error.
	});
\end{lstlisting}

The rationale for this decision in \cite{UnderstandingTypeScript}
is that, JavaScript programs tend to interact with
data of unspecified types (such as fetching data
from API calls over the network); these parts
need to be dynamically typed in order to give developers
a smooth transition into TypeScript, and for TypeScript
to be usable in a distributed system setting. \\

\noindent
Based on the compatibility of TypeScript
with JavaScript, 
we believe that TypeScript API generation for session-typed
web development best achieves our objective
of providing
developers with a workflow
that provides communication safety guarantees in \textit{modern
web programming}
through multiparty session types.
We argue that the type system of TypeScript,
along with other language features we introduce
throughout the course of the report, is sufficient
for implementing session type theory in a manner that
complements idiomatic web development practices.

\chapter{Project Plan}
We aim to develop a multiparty session type-safe development workflow for building interactive full-stack TypeScript applications that conform to a communication protocol. This involves encoding session types using the TypeScript language and implementing a code generation workflow to generate the encodings.

%As we target full-stack applications, we tackle server-side and browser-side targets separately:
%
%\paragraph{Server-side targets} We will analyse the approaches presented in \cite{Hybrid2016, PureScript2019, Python2017} and assess their applicability with respect to the language features and type system offered by TypeScript and general back-end web development patterns. At the point of writing, we have prototyped several approaches for server-side session type TypeScript encodings and will compare their effectiveness through implementing example protocols, such as the \textit{Adder} example in \cite{Hybrid2016} and the \textit{Battleships} example in \cite{PureScript2019}.
%
%\paragraph{Browser-side targets} We will analyse how \cite{PureScript2019} handles API generation for front-end implementations, and in particular, study their approach to behavioural typing and guaranteeing channel linearity in an event-driven environment. \cite{MVU2019} also presents an architecture for verifying binary session type conformance of fornt-end implementations through formalisms of the \textit{Model-View-Update} (MVU) architecture and \textit{model types}. We will investigate how to extend their approach for multiparty session types using TypeScript. At the point of writing, we have explored methods to express the methodology in \cite{MVU2019} using React.js for a multiplayer game.

\section{Delivery}
At the point of writing, we have analysed the existing code generation approaches presented in \cite{Hybrid2016, Scribble, Python2017} and assessed their applicability with respect to verifying the communication aspects of interactive web applications. We have also explored approaches for encoding session types into TypeScript, motivated by existing works along with similar proposals specific to web development in \cite{PureScript2019, MVU2019}.

We plan to deliver our code generation implementation incrementally, aiming to get a working version of the end-to-end workflow first, then iteratively add support for more complex session type primitives. This minimises the risks involved in the project by ensuring we have a functional deliverable at the early stages of the project. We plan to deliver this basic end-to-end workflow by the end of February.

The subsequent months will involve adding support for other primitives (i.e. selection, choice and recursion) for multiparty sessions. We plan to measure our progress by writing example protocols for these milestones and checking that our implementation supports those protocols by the end of each month.

We will also try to connect with web developers in the community to experiment with our implementation to get feedback on ways to make it more applicable and compatible with industry practices, such that we can apply their feedback on the next iterations of our deliverables.

If time permits, we may explore possible extensions of applying supporting \textit{Explicit Connection Actions} presented in \cite{FASE2017} in our workflow to support more protocols.

\section{Timetable}
We attach a preliminary timetable (Table \ref{table:timetable}) for guidance.

\begin{figure}
\centering
\begin{tabular}{l|p{0.4\textwidth}|p{0.4\textwidth}}
Month & Milestones & Deadlines \\
\hline\hline
Jan & \begin{itemize}
\item Explore approaches for TypeScript encoding
\item Investigate methods for code generation from EFSM
\end{itemize} & \begin{itemize}
\item 24th: Interim Report
\item 25th: PLACES 2020
\end{itemize} \\
\hline
Feb & \begin{itemize}
\item Develop API generation end-to-end toolchain
\item Support binary sessions with simple send/receive
\end{itemize} & \begin{itemize}
\item 14th: Project Review
\end{itemize} \\
\hline
Mar & \begin{itemize}
\item Support multiparty sessions with simple send/receive
\end{itemize} & \begin{itemize}
\item 16th-20th: Examinations
\end{itemize}\\
\hline
Apr & \begin{itemize}
\item Support binary sessions with selection, choice and recursion
\item Example: calculator service
\end{itemize} & \begin{itemize}
\item 6th-17th: Easter break
\end{itemize}\\
\hline
May & \begin{itemize}
\item Support multiparty sessions with selection, choice and recursion
\item Example: Tic Tac Toe game
\end{itemize} & \begin{itemize}
\item 15th: Health check-up
\end{itemize}\\
\hline
Jun & \begin{itemize}
\item Complete report write-up
\end{itemize} & \begin{itemize}
\item 15th: Final Report
\item 26th: Final Archive
\end{itemize}
\end{tabular}
\captionof{table}{Project Timetable.}
\label{table:timetable}
\end{figure}


\chapter{Evaluation Plan}
\section{Theory}
\section{Implementation}


\clearpage

\bibliographystyle{acm}
\bibliography{interim}

\end{document}
%%% Local Variables: 
%%% mode: latex
%%% TeX-master: t
%%% End: 
