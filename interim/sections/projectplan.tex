
\chapter{Project Plan}
We aim to develop a multiparty session type-safe development workflow for building interactive full-stack TypeScript applications that conform to a communication protocol. This involves encoding session types using the TypeScript language and implementing a code generation workflow to generate the encodings.

\section{Delivery}
At the point of writing, we have analysed the existing code generation approaches presented in \cite{Hybrid2016, Scribble, Python2017} and assessed their applicability with respect to verifying the communication aspects of interactive web applications. We have also explored approaches for encoding session types into TypeScript, motivated by existing works along with similar proposals specific to web development in \cite{PureScript2019, MVU2019}.

We plan to deliver our code generation implementation incrementally, aiming to get a working version of the end-to-end workflow first, then iteratively add support for more complex session type primitives. This minimises the risks involved in the project by ensuring we have a functional deliverable at the early stages of the project. We plan to deliver this basic end-to-end workflow by the end of February.

The subsequent months will involve adding support for other primitives (i.e. selection, choice and recursion) for multiparty sessions. We plan to measure our progress by writing example protocols for these milestones and checking that our implementation supports those protocols by the end of each month.

We will also try to connect with web developers in the community to experiment with our implementation to get feedback on ways to make it more applicable and compatible with industry practices, such that we can apply their feedback on the next iterations of our deliverables.

If time permits, we may explore possible extensions of applying supporting \textit{Explicit Connection Actions} presented in \cite{FASE2017} in our workflow to support more protocols.

\section{Timetable}
We attach a preliminary timetable (Table \ref{table:timetable}) for guidance.

\begin{figure}
\centering
\begin{tabular}{l|p{0.4\textwidth}|p{0.4\textwidth}}
Month & Milestones & Deadlines \\
\hline\hline
Jan & \begin{itemize}
\item Explore approaches for TypeScript encoding
\item Investigate methods for code generation from EFSM
\end{itemize} & \begin{itemize}
\item 24th: Interim Report
\item 25th: PLACES 2020
\end{itemize} \\
\hline
Feb & \begin{itemize}
\item Develop API generation end-to-end toolchain
\item Support binary sessions with simple send/receive
\end{itemize} & \begin{itemize}
\item 14th: Project Review
\end{itemize} \\
\hline
Mar & \begin{itemize}
\item Support multiparty sessions with simple send/receive
\end{itemize} & \begin{itemize}
\item 16th-20th: Examinations
\end{itemize}\\
\hline
Apr & \begin{itemize}
\item Support binary sessions with selection, choice and recursion
\item Example: calculator service
\end{itemize} & \begin{itemize}
\item 6th-17th: Easter break
\end{itemize}\\
\hline
May & \begin{itemize}
\item Support multiparty sessions with selection, choice and recursion
\item Example: Tic Tac Toe game
\end{itemize} & \begin{itemize}
\item 15th: Health check-up
\end{itemize}\\
\hline
Jun & \begin{itemize}
\item Complete report write-up
\end{itemize} & \begin{itemize}
\item 15th: Final Report
\item 26th: Final Archive
\end{itemize}
\end{tabular}
\captionof{table}{Project Timetable.}
\label{table:timetable}
\end{figure}
