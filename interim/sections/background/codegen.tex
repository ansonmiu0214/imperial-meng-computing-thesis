\section{Code Generation}

\subsection{Overview}
\begin{itemize}
\item Motivate the practical implementation of session types in programming
\item What needs to be verified - I/O behaviour (introduce the term behaviour typing) and linear channel usage
\end{itemize}

\subsection{Static Session Typing}
\begin{itemize}
\item PureScript implementation that addresses the 2 criteria statically
\begin{itemize}
\item Relevance for web development
\item Static behavioural typing
\item Static channel linearity by construction
\item Pros - type dependencies (expressing dependent types)
\item Cons
\end{itemize}
\item Other initiatives -- creating language with first-class channel primitives (e.g. SILL)
\end{itemize}

PureScript - \cite{PureScript2019}

\subsection{Runtime Monitors}
\begin{itemize}
\item Python dynamic runtime monitors that addresses the 2 criteria dynamically
\begin{itemize}
\item Channel abstraction
\item How invitation/conversation messages are handled
\end{itemize}
\end{itemize}

Python - \cite{Python2017}

\subsection{Hybrid Session Verification}
\begin{itemize}
\item How the aforementioned 2 approaches motivated hybrid session verification
\item Workflow using Scribble toolchain 
\item Static behavioural typing, dynamic channel usage runtime checks
\item Pros - Abstract I/O interfaces
\item Pros - Input futures for `complex communication patterns' motivated by SMTP example
\item Cons - ``practical compromise'' as outlined in paper
\end{itemize}

Java - \cite{Hybrid2016}

\subsection{Comparison}
\begin{itemize}
\item Comparison table between the 3 approaches for how they deal with behavioural typing 
\end{itemize}

\begin{figure}[!h]
\centering
\begin{tabular}{l p{0.4\textwidth} p{0.4\textwidth}}
Language & Behavioural typing & Channel linearity \\
\hline\hline
Java \cite{Hybrid2016} &  & \\
\hline
Python \cite{Python2017} &  & \\
\hline
TypePureScript \cite{PureScript2019} &  & \\
\end{tabular}
\captionof{table}{Comparison between existing MPST code generation approaches.}
\label{table:comparison}
\end{figure}
