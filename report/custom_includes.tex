% Paragraph === skip-line && no-indent
%\usepackage[parfill]{parskip}

\usepackage{bookmark}

% Change chapter number size
\usepackage{sectsty}
\chapternumberfont{\Large}
\chaptertitlefont{\Huge}

% Change table/figure counter
\usepackage{chngcntr}

% Highlighting
\usepackage{soul}

% Graphics path
\graphicspath{ {figures/} }

\usepackage{underscore}

\usepackage{listings}

% Fancy names
\newcommand{\fancyname}[1]{\textsc{#1}}
\newcommand{\trole}[1]{\texttt{#1}}
\newcommand{\tmsg}[1]{\texttt{#1}}
\newcommand{\tprotocol}[1]{\textsc{#1}}
\newcommand{\filename}[1]{\texttt{#1}}

% Consistent names
\usepackage{xspace}
\newcommand{\newtheory}{\fancyname{RoutedSessions}\xspace}
\newcommand{\nodecodegen}{\fancyname{NodeMPST}\xspace}
\newcommand{\reactcodegen}{\fancyname{ReactMPST}\xspace}
\newcommand{\codegen}{\fancyname{SessionTS}\xspace}

% Nicer references
\usepackage[noabbrev, capitalise]{cleveref}

% Font?
%\usepackage{ascii}
%\usepackage[T1]{fontenc}
%\renewcommand\ttfamily{\asciifamily}


% Includes

% Proof trees
%\usepackage{amsthm}
%\theoremstyle{plain}
%
%\newtheorem{thm}{Theorem}[section]
%\newtheorem{cor}{Corollary}[theorem]
%\newtheorem{lem}[theorem]{Lemma}
%\usepackage{proof}
\usepackage{bussproofs}

% Code listing

\lstdefinelanguage{Scribble}{
  keywords={module, type, from, as, global,protocol, role, to, choice, at, or, do},
  keywordstyle=\color{blue}\bfseries,
  identifierstyle=\color{black},
  sensitive=false,
  comment=[l]{//},
  morecomment=[s]{/*}{*/},
  commentstyle=\color{darkgray}\ttfamily,
  stringstyle=\color{purple}\ttfamily,
  morestring=[b]',
  morestring=[b]"
}
\lstdefinelanguage{JavaScript}{
  keywords={typeof, new, true, false, catch, function, return, null, catch, switch, var, if, in, while, do, else, case, break, class, export, boolean, throw, implements, import, this, const, let, extends, async, await, namespace, type, abstract, any, number, string, instanceof, typeof, declare, never,enum,constructor,public,private,void,super,undefined, as,protected,interface,keyof,infer,is,try,default},
  keywordstyle=\color{blue}\bfseries,
%  ndkeywords={class, export, boolean, throw, implements, import, this},
%  ndkeywordstyle=\color{darkgray}\bfseries,
  identifierstyle=\color{black},
  sensitive=true,
  comment=[l]{//},
  morecomment=[s]{/*}{*/},
  commentstyle=\color{darkgray}\ttfamily,
  stringstyle=\color{purple}\ttfamily,
  morestring=[b]',
  morestring=[b]"
}
\lstdefinelanguage{PureScript}{
  keywords={class,instance,Int,Bool},
  keywordstyle=\color{blue}\bfseries,
%  ndkeywords={class, export, boolean, throw, implements, import, this},
%  ndkeywordstyle=\color{darkgray}\bfseries,
  identifierstyle=\color{black},
  sensitive=true,
  comment=[l]{//},
  morecomment=[s]{/*}{*/},
  commentstyle=\color{darkgray}\ttfamily,
  stringstyle=\color{purple}\ttfamily,
  morestring=[b]',
  morestring=[b]"
}
\lstdefinelanguage{Python}{
  keywords={class, self, def, cls, int, str, bool, pass, return, from, import, and, if, else, None, is, not},
  keywordstyle=\color{blue}\bfseries,
  identifierstyle=\color{black},
  comment=[l]{\#},
  commentstyle=\color{darkgray}\ttfamily,
  stringstyle=\color{purple}\ttfamily,
  morestring=[b]',
  morestring=[b]"
}
\lstset{
	basicstyle=\footnotesize\ttfamily,
	numberstyle=\tiny\ttfamily,
	numbers=left,
	frame=single,
	tabsize=4,
    xleftmargin=1.25em,
    framexrightmargin=-0.25em,
    escapeinside={(*@}{@*)}
}
\def\lstonelinejs{\lstinline[language=javascript,basicstyle=\ttfamily]}
\def\lstonelineps{\lstinline[language=purescript,basicstyle=\ttfamily]}

% Figure and table counter
\usepackage{chngcntr}
\counterwithout{figure}{chapter}
\counterwithout{table}{chapter}

% Commands
%\newcommand{\term}[1]{\textit{#1}}
%\newcommand{\mathref}[1]{\textsection #1}
%
%\newcommand{\rulename}[1]{[\textsc{#1}]}
%
%\newcommand{\piin}[2]{#1(#2)}
%\newcommand{\piout}[2]{\bar{#1}\langle #2 \rangle }
%\newcommand{\parti}[1]{{\color{blue}\mathbf{#1}}}
%
%\newcommand{\send}[2]{#1!\langle #2 \rangle}
%\newcommand{\recv}[2]{#1?(#2)}
%\newcommand{\bra}[2]{#1 \rhd \left\{#2\right\}}
%\newcommand{\sel}[3]{#1 \lhd #2 \langle #3 \rangle}
%
%\newcommand{\tysend}[2]{#1![#2]}
%\newcommand{\tyrecv}[2]{#1?[#2]}
%\newcommand{\tybra}[2]{#1 \& \left\{#2\right\}}
%\newcommand{\tysel}[2]{#1 \oplus \left\{#2\right\}}


% Typewriter font
% \renewcommand{\ttdefault}{pcr} % selects Courier font

% Citation style
\setcitestyle{square}
