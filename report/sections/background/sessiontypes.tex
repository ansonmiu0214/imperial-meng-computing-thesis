\section{Session Types}

Web applications are one of many examples of 
distributed systems in practice. 
Distributed systems are built upon the interaction 
between concurrent processes, which can be implemented using 
the two main communication abstractions in 
\textit{shared memory} and \textit{message passing}. 

Shared memory provides processes with the impression of 
a logical single large monolithic memory 
but requires programmers to understand consistency models 
in order to correctly reason about the consistency of shared state.

Message passing interprets the interaction between processes 
as the exchange of messages.
This best describes the communication transports 
found in web applications, ranging from the 
stateless request-response client-server interactions via HTTP 
to full-duplex communication channels via the WebSocket protocol 
\cite{WebSocketRFC}.

The process algebra $\pi$-calculus 
introduced by Milner in \cite{Milner1999} 
formalises the message passing abstraction in terms of 
the basic building blocks of sending and receiving processes,
along with inductively defined continuation processes. 
The composition of these primitives allow us to 
describe more complex communication sessions.
\textit{Session types} define the typing discipline 
for the $\pi$-calculus and provide reliability guarantees 
for communication sessions; 
the latter addresses a key challenge when 
reasoning about the correctness of distributed systems. 

Practical application of session types in software
engineering range from
developing languages providing native session type support 
\cite{ATS2016} to implementing session types 
in existing programming languages across different paradigms.
Implementations of the latter approach differ by 
how they leverage the design philosophy and features 
provided by the programming language.
For example, King et al. leveraged the 
expressive type system of PureScript to perform 
\textit{static} session type checking in \cite{PureScript2019},
whilst Neykova and Yoshida introduced dynamic approaches to 
check the conformance of Python programs 
with respect to session types in \cite{Python2017}.
We discuss these related work, among others, in \cref{section:bgrelated}.

\subsection{Process Calculus}
\label{subsection:picalculus}

The $\pi$-calculus models concurrent computation, 
where processes can execute in parallel and communicate 
via shared names.
We first consider the \textit{asynchronous} 
$\pi$-calculus introduced by 
Honda and Tokoro in \cite{AsyncHonda}.
Among the many flavours of the calculus 
which vary depending on the application domain,
we outline the variant as presented in \cite{C406Lecture}. 

We define the syntax of processes in \cref{fig:async};
the asynchrony comes from the lack of 
continuation in the output process.

\begin{itemize}
\item $\mathbf{0}$ is the nil process and represents inactivity.
\item $\pout{u}{v}$ is the output process that will send value $v$ on $u$.
\item $\pin{u}{x}.P$ is the input process that,
 upon receiving a message on $u$, 
 will bind the message to $x$ and 
 carry on executing $P$ under this binding.
\item $P\mid Q$ represents the parallel composition of 
processes executing simultaneously.
\item $!P$ represents the parallel composition of 
\textit{infinite} instances of $P$; 
more specifically, $!P \equiv P \mid {!P}$.
\item $(\nu a)~P$ represents a name restriction 
where any occurrence of $a$ in $P$ is local
 and will not interfere with other names outside the scope of $P$.
\end{itemize}

\begin{figure}[!hb]
\doublespacing
\[
\begin{array}{rlr}

P,Q ::= & & \text{Processes} \\
     & \mathbf{0} & \text{Nil Process} \\
\mid & \pout{u}{v} & \text{Output} \\
\mid & \pin{u}{x}.P & \text{Input} \\     
\mid & P \mid Q & \text{Parallel Composition} \\
\mid & !P & \text{Replication} \\
\mid & (\nu a)~P & \text{Restriction} \\

u,v ::= & & \text{Identifiers} \\
     & a, b, c,~\dots & \text{Names} \\
\mid & x, y, z,~\dots & \text{Variables} \\

\end{array}
\]
\singlespacing
\captionof{figure}{Syntax of Asynchronous $\pi$-calculus}
\label{fig:async}
\end{figure}

The operational semantics model the interaction 
between parallel processes. 
Whilst \cite{C406Lecture} presents the full operational semantics, 
we highlight the \rulename{Comm} reduction rule which 
specifically models message passing:
if the parallel composition of an input process and output process 
share the same name, the composition reduces to the 
continuation of the input process, 
substituting the variable $x$ with the message received $v$. 
We omit the definitions of substitution, free variables and free names, 
$\alpha$-equivalence and structural congruence; 
the interested reader may refer to \cite{C406Lecture}.

\begin{prooftree}
\AxiomC{}
\RightLabel{\rulename{Comm}}
\UnaryInfC{$
\pout{a}{v} \mid \pin{a}{x}.P ~\longrightarrow ~ P[v/x]
$}
\end{prooftree}

We additionally define a process $P$ to be {stuck} 
if $P$ is not the nil process and $P$ cannot be reduced any further. 
For example, the process 
$P = \pin{a}{x}.\mathbf{0} \mid \pout{b}{v}$ 
is stuck as the parallel composition of an input process 
and an output process that do not share the same name cannot 
be reduced using \rulename{Comm}. 
In practice, a stuck process contains communications 
that will never be executed.

\subsection{Binary Session Types}
\label{subsection:bgbst}

A \textit{binary session} is a parallel composition 
of two processes, each representing a distinct participant.
In the context of web applications, a binary session may
describe the interactions between client and server.
Without loss of generality, a \text{session} represents 
the sequence of send and receive actions of a single participant.

We introduce a \text{synchronous} session calculus 
in \cref{fig:sync},
inspired by \cite{MPST}. 
We briefly discuss the main components and 
how it differs from the variant introduced in \cite{C406Lecture}:

\begin{itemize}
\item \textbf{Synchronous communication}: 
Output processes have a continuation that will be executed upon
a successful send.

\item \textbf{Polyadic communication}: 
More than one values can be communicated at once.
We refer to these as a \textit{vector} of values.

\item \textbf{Branching and selection}: 
A branching process can offer a set of branches, 
each defined by its own label identifier and continuation process. 
A selection process can select a branch by 
sending the corresponding label identifier 
alongside the payload to the branching process.

\item \textbf{Labelled messages}: 
A label identifier is attached to all messages; 
the input process in \cref{fig:async}
is generalised as a branching process that offers one branch.
\end{itemize}

The \rulename{Comm} rule in the operational semantics 
for this calculus exemplifies these new additions: 
given a binary session between distinct participants 
$\role{p}$ and $\role{q}$ 
where $\role{q}$ offers a set of labelled branches,
if $\role{p}$ selects a label offered by $\role{q}$ and 
sends a vector of expressions $e_1, \dots, e_n$ 
that evaluate\footnote{
We adopt the operational semantics for 
expression evaluation $e \downarrow v$ 
as defined in \cite{C406Lecture}.
} to the corresponding vector of values 
$v_1, \dots, v_n$, 
the session reduces to a session with 
the continuation from the selection process 
composed with the continuation from the selected branch 
of the branching process.
The branching process binds the received values
$v_1, \dots, v_n$ to the variables $x_1, \dots, x_n$.

\begin{prooftree}
\AxiomC{$\exists	j \in I. l_j = l$}
\AxiomC{$e_1 \downarrow v_1 \dots e_n \downarrow v_n$}
\AxiomC{$\mrole{p} \neq \mrole{q}$}
\RightLabel{\rulename{Comm}}
\TrinaryInfC{$
\mrole{p} :: \sel{q}{l}{e_1 \dots e_n}.~P \mid 
\mrole{q} :: \bra{p}{l_i(x_1 \dots x_n): Q_i}{i\in I} 
~\longrightarrow~ 
\ptprocess{p}{P} \mid \ptprocess{q}{Q_j[v_k/x_k]^n_{k=1}}
$}
\end{prooftree}

\begin{figure}[!hb]
\doublespacing
\[
\begin{array}{rlr}
v ::= & \underline{n}~\mid~\texttt{true}~\mid~\texttt{false} 
	& \text{Values} \\
e, e' ::= & & \text{Expressions} \\
	& v & \text{Values} \\
\mid	 & x & \text{Variables} \\
\mid & e + e'~\mid~e - e' & \text{Arithmetic Operators} \\
\mid & e = e'~\mid~e < e' ~\mid~e > e' & \text{Relational Operators} \\
\mid & e \wedge e'~\mid~e \vee e' ~\mid~\neg e & \text{Logical Operators} \\
\mid & e \oplus e' & \text{Non-Determinism} \\

\mrole{p} ::= & \mrole{Client},~\mrole{Server} & \text{Participant} \\

P,Q ::= & & \text{Processes} \\
     & \mathbf{0} & \text{Nil Process} \\
\mid & \sel{p}{l}{e_1 \dots e_n}.~P & \text{Selection} \\
\mid & \bra{p}{l_i(x_1 \dots x_n): P_i}{i\in I} & \text{Branching} \\
\mid & \pcond{e}{P}{Q} & \text{Conditional} \\
\mid & \precursion{P} & \text{Recursive Process} \\
\mid & \precvar & \text{Process Variable} \\

l, l' ::= & \texttt{``str''} & \text{Label Identifiers} \\

\mathcal{M} ::= & \ptprocess{p}{P} ~\mid~ \ptprocess{q}{Q} 
	& \text{Binary Session} \\
\end{array}
\]
\singlespacing
\captionof{figure}{Syntax of Session Calculus with 
Branching, Selection and Recursion}
\label{fig:sync}
\end{figure}

Additionally, the calculus introduces:

\begin{itemize}
\item \textbf{Conditionals}: 
If $e \downarrow \texttt{true}$, 
the process $\pcond{e}{P}{Q}$ reduces to $P$; 
if $e \downarrow \texttt{false}$, 
the process $\pcond{e}{P}{Q}$ reduces to $Q$.

\item \textbf{Recursion}: 
Following the \textit{equirecursive} approach, 
the occurrence of the process variable $\precvar$ 
in the recursive process can be 
expanded into the process transparently;
more specifically, $\precursion{P} \equiv P[(\precursion{P})/X]$.
\end{itemize}

{Session types} represent the type theory for our session calculus.
We define the syntax of session types for binary sessions in
\cref{fig:bst}. 

\begin{figure}[!hb]
\doublespacing
\[
\begin{array}{rlr}

U ::= & \text{\tt{int}}~\mid~\text{\tt{bool}} & \text{Sorts} \\

T ::= & & \text{Session Types} \\
     & \mathbf{end} & \text{Termination} \\
\mid & \tbra{\role{p}}{l_i(U_1 \dots U_n):T_i}{i\in I} & \text{Branching} \\
\mid & \tsel{\role{p}}{l_i(U_1 \dots U_n):T_i}{i\in I} & \text{Selection} \\
\mid & \trec{T} & \text{Recursive Type} \\
\mid & \trecvar & \text{Type Variable} \\
\end{array}
\]

\singlespacing
\caption{Syntax of Session Types}
\label{fig:bst}
\end{figure}

We derive the type of a process with a {typing judgement} 
of the form $\Gamma \vdash P: S$, which reads, 
\textit{under the typing context $\Gamma$, 
process $P$ has session type $S$}. 

The \textit{typing context} records typing assumptions 
used as part of the derivation: in the case of binary session types, 
the context maps expressions to sorts, 
and process variables to session types. 
A typing judgement is constructed in terms of inference rules 
defined inductively on the structure of 
processes and expressions.

We present the rules for \rulename{Ty-Sel} and \rulename{Ty-Bra}; 
the remaining rules follow from \cite{C406Lecture} 
and can be trivially defined as 
they leverage the syntactic similarities between 
session types and our session calculus.

\begin{prooftree}
\AxiomC{$
\Gamma \vdash e_1 : U_1 \dots \Gamma \vdash e_n : U_n
$}
\AxiomC{$\Gamma \vdash P : T$}
\RightLabel{\rulename{Ty-Sel}}
\BinaryInfC{$
\Gamma \vdash \sel{p}{l}{e_1 \dots e_n}.~P : 
\tselone{p}{l(U_1 \dots U_n)}: T
$}
\end{prooftree}

\begin{prooftree}
\AxiomC{$
\forall i \in I. ~ \left(
\Gamma, x_1: U_1, \dots, x_n: U_n \vdash P_i : T_i
\right)
$}
\RightLabel{\rulename{Ty-Bra}}
\UnaryInfC{$
\Gamma \vdash \bra{p}{l_i(x_1 \dots x_n): P_i}{i\in I} : 
\tbra{p}{l_i(U_1 \dots U_n):S_i}{i\in I}
$}
\end{prooftree}

The definition of stuck processes 
from \cref{subsection:picalculus} 
motivate the discussion of communication errors 
that may occur during interactions among participants. 
We outline two of the main classes of errors:

\begin{itemize}
\item \textbf{Deadlock}: 
Progress cannot be made when the two participants 
expect to be receiving a message from each other at the same time.
\item \textbf{Communication mismatch}: 
Progress cannot be made when the selection process 
sends a message with a label identifier not 
offered by the branching process; 
likewise, the payload sent must be compatible 
with the sort expected by the branching process 
for the selected branch.
\end{itemize}

Session types ensure that 
well-typed binary sessions are guaranteed 
to be free from these communication errors 
through the concept of \textit{duality}. 
Duality defines a notion of \textit{compatibility}
 between processes: two session types are dual 
 with respect to each other if the communication 
 between them (i.e. pairs of sending and receiving actions)
 always match (i.e. with respect to the selected label 
 and message payload type). 
We define $\dual{S}$ as the dual type of $S$ in \cref{table:dual};
message payload types are omitted for brevity.

% Row height in tables
\renewcommand{\arraystretch}{1.6}
\begin{center}
\begin{tabular}{rcl}
$\dual{\tend}$ & = & $\tend$ \\
$\dual{\tbra{p}{l_i: T_i}{i\in I}}$ & = & 
	$\tsel{q}{l_i: \dual{T_i}}{i\in I}$ \\
$\dual{\tsel{p}{l_i: T_i}{i\in I}}$ & = & 
	$\tbra{q}{l_i: \dual{T_i}}{i\in I}$ \\
$\dual{\trec{T}}$ & = & $\trec{\dual{T}}$ \\
$\dual{\trecvar}$ & = & $\trecvar$ \\ 
\end{tabular}
\captionof{table}{Duality of Binary Session Types involving participants $\role{p}$ and $\role{q}$}
\label{table:dual}
\end{center}
\renewcommand{\arraystretch}{1}

Consequently, a binary session is well-typed 
if the participating processes are typed to be 
dual with respect to each other: 
we illustrate this in \rulename{MTy}.

\begin{prooftree}
\AxiomC{$
\cdot \vdash P : T
$}
\AxiomC{$
\cdot \vdash Q : \dual{T}
$}
\RightLabel{\rulename{MTy}}
\BinaryInfC{$
\vdash \ptprocess{p}{P} \mid \ptprocess{q}{Q}
$}
\end{prooftree}

The definition of duality alone 
restricts the definition of 
well-typed binary sessions to those
where the two processes are derived to be 
\textit{exactly} dual types of one another. 
Consider the pair of session types below:

\[
\begin{array}{rl}
T_{\text{Client}} &= 
	\tselone{Server}{\tmsg{Succ(nat)}}.~
	\tbraone{Server}{\tmsg{Succ(int)}}.~
	\tend \\
T_{\text{Server}} &= \tbraone{Client}{
\begin{cases}
	\tmsg{Succ(int)}: \tselone{Client}{\tmsg{Succ(int)}} \\
	\tmsg{Quit()}:  \tend \\
\end{cases}
} \\
\end{array}
\]

Whilst $\dual{T_{\text{Client}}} \neq T_{\text{Server}}$, 
this pair of session types is intuitively compatible 
as the client is selecting a branch offered by the server, 
where the session types for the continuations of 
this branch for both participants are indeed dual.
Regarding payload, the server is expecting \texttt{int},
but the client sends \texttt{nat}, which is a subtype of \texttt{int}.

This motivates the concept of \textit{subtypes} in session types,
which allows a process to be typed by its ``supertype'' when required. 
$\subtype$\footnote{
The $\subtype$ operator is also an 
overloaded relation on 
sorts to express subsorting, 
i.e. $\texttt{nat} \subtype \texttt{int}$.
} defines the subtyping relation: 
$T \subtype T'$ reads \textit{$T$ is a subtype of $T'$},
and is defined coinductively on the structure of session types.

We present the inference rules for 
\rulename{Sub-Sel} and \rulename{Sub-Bra}
inspired by \cite{MPST} but adapted for polyadic communication; 
the intuition behind subtyping and subsorting is outlined below:

\begin{itemize}
\item \textbf{Branching}: 
The supertype of a branching process offers 
a subset of the branches and 
expects more specific types of payload; 
intuitively, if a process expects to 
receive an \texttt{int}, 
it can handle a \texttt{nat} payload.

\item \textbf{Selection}: 
The supertype of a selection process offers a 
superset of the internal choices and 
can send more generic types of payload; 
intuitively, if a process sends a \texttt{nat}, 
the payload is compatible with receivers expecting 
a more generic \texttt{int} payload.
\end{itemize}

\begin{prooftree}
\AxiomC{$\forall i \in I.$}
\AxiomC{$U'_1 \subtype U_1 ~ \dots ~ U'_n \subtype U_n$}
\AxiomC{$S_i \subtype T'_i$}
\RightLabel{\rulename{Sub-Bra}}
\doubleLine
\TrinaryInfC{$
	\tbra{p}{l_i(U_1 \dots U_n):T_i}{i\in I \cup J}
	\subtype
	\tbra{p}{l_i(U'_1 \dots U'_n):T'_i}{i\in I}
$}
\end{prooftree}

\begin{prooftree}
\AxiomC{$\forall i \in I.$}
\AxiomC{$U_1 \subtype U'_1 ~ \dots ~ U_n \subtype U'_n$}
\AxiomC{$T_i \subtype T'_i$}
\RightLabel{\rulename{Sub-Sel}}
\doubleLine
\TrinaryInfC{$
	\tsel{p}{l_i(U_1 \dots U_n):T_i}{i\in I}
	\subtype
	\tsel{p}{l_i(U'_1 \dots U'_n):T'_i}{i\in I \cup J}
$}
\end{prooftree}

We also introduce \textit{subsumption} in \rulename{Ty-Sub} 
to incorporate the subtyping relation into the typing judgement.

\begin{prooftree}
\AxiomC{$\Gamma \vdash P: T$}
\AxiomC{$T \subtype T'$}
\RightLabel{\rulename{Ty-Sub}}
\BinaryInfC{$\Gamma \vdash P : T'$}
\end{prooftree}

This allows us to construct a derivation to show that the binary session 

\[
\mathcal{M} = \ptprocess{Client}{P_\text{Client}}
~ \mid ~
\ptprocess{Server}{P_\text{Server}}
\]

is well-typed, assuming $P_\text{Client}$ and $P_\text{Server}$ 
are typed 
$T_\text{Client}$ and $T_\text{Server}$ respectively.

\begin{prooftree}
\AxiomC{\vdots}
\UnaryInfC{$\cdot \vdash P_\text{Client}: T_\text{Client}$}
\AxiomC{\vdots}
\UnaryInfC{$\cdot \vdash P_\text{Server}: T_\text{Server}$}
\AxiomC{\vdots}
\doubleLine
\UnaryInfC{$T_\text{Server} \subtype \dual{T_\text{Client}}$}
\RightLabel{\rulename{Ty-Sub}}
\BinaryInfC{$\cdot \vdash P_\text{Server}: \dual{T_\text{Client}}$}
\RightLabel{\rulename{MTy}}
\BinaryInfC{$
	\vdash \ptprocess{Client}{P_\text{Client}} ~ \mid ~
	\ptprocess{Server}{P_\text{Server}}
$}
\end{prooftree}

\subsection{Multiparty Session Types}
\label{subsection:bgmpst}

% need LTS semantics here too

