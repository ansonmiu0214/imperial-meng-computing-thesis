\section{TypeScript}
\label{section:typescript}

We introduce the TypeScript language 
\cite{UnderstandingTypeScript}
as our choice of target language
for session type API generation.
Developed by \textit{Microsoft Research},
TypeScript is an extension to
JavaScript to address
the deficiencies of the latter in
\textit{developing} and \textit{maintaining}
large-scale complex applications.
Syntactically,
TypeScript is a \textit{superset} of JavaScript,
so every JavaScript program is a
TypeScript program.
The TypeScript Compiler is used
to compile a TypeScript program into JavaScript
source code; since
JavaScript is a dynamically typed langauge,
the compilation process features full type erasure.

We introduce specific language
features used to implement our API generation
solution throughout the report as needed;
here, we highlight the key properties
of the type system implemented in the language.

\paragraph{Structural Typing}
In a structural type system, type equivalence
is determined by \textit{shape} rather than by name
(which is the case in a \textit{nominal} type system).

Consider the following TypeScript code:

\begin{lstlisting}[language=javascript]
class ThisSquare {
	constructor(public side: number) { }
};

class ThatSquare {
	constructor(public side: number) { }
};

const area = (sq: ThisSquare) => sq.side * sq.side;

area(new ThisSquare(2));	// ok
area(new ThatSquare(2));	// ok (*@\label{line:structural1}@*)
area({ side: 2 });			// ok (*@\label{line:structural2}@*)
\end{lstlisting}

The \texttt{area} function takes a \texttt{ThisSquare}
as parameter.
Under a structural type system,
\cref{line:structural1,line:structural2}
will type-check, because \texttt{ThatSquare} and the object
literal created from scratch matches the \textit{shape}
of \texttt{ThisSquare} -- all of them have a \texttt{side}
property typed \lstonelinejs{number}.

In languages (e.g. Java) that use a nominal type system, 
\cref{line:structural1}
will not type-check because \texttt{ThatSquare} is not named
\texttt{ThisSquare}.

\paragraph{Gradual Typing}
In a gradual type system, a program can have parts
that are statically typed and other parts are dynamically
typed \cite{GradualTyping}.
TypeScript distinguishes dynamically typed code using
the \lstonelinejs{any} type.

\begin{lstlisting}[language=javascript]
// Invoke remote API
fetch('https://jsonplaceholder.typicode.com/todos/1')
	// Convert to JavaScript Object Notation
	.then((response: Response) => response.json())
	.then((json: any) => {
		// Up to the developer to correctly deserialise;
		// incorrect implementations will cause
		// runtime type error.
	});
\end{lstlisting}

The rationale for this decision in \cite{UnderstandingTypeScript}
is that, JavaScript programs tend to interact with
data of unspecified types (such as fetching data
from API calls over the network); these parts
need to be dynamically typed in order to give developers
a smooth transition into TypeScript, and for TypeScript
to be usable in a distributed system setting. \\

\noindent
Based on the compatibility of TypeScript
with JavaScript, 
we believe that TypeScript API generation for session-typed
web development best achieves our objective
of providing
developers with a workflow
that provides communication safety guarantees in \textit{modern
web programming}
through multiparty session types.
We argue that the type system of TypeScript,
along with other language features we introduce
throughout the course of the report, is sufficient
for implementing session type theory in a manner that
complements idiomatic web development practices.