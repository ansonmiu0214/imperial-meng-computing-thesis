\section{Related Work}
\label{section:bgrelated}

Whilst session type theory represents 
the type language for concurrent processes,
it also forms the theoretical basis for proposals
introduced to implement session types for 
real-world application development:
the \textit{Scribble} project is one such proposal.
We discuss related work that 
implement session types for software
engineering using the Scribble project
in \cref{subsection:bgscribble},
and focus on existing work for
session-typed web development
in \cref{subsection:sessiontypewebdev}.

\subsection{Scribble and Endpoint API Generation}
\label{subsection:bgscribble}

% Introduce Scribble as an implementation of MPST
Scribble \cite{Scribble} is a
platform-independent description language 
for the specification of message-passing protocols.
The language describes the behaviour of 
communicating processes at a high level of abstraction:
more importantly, the description is independent from 
implementation details in the same way that 
the type signature of a function declaration is decoupled 
from the corresponding function definition.

A Scribble \textit{protocol specification} 
describes an agreement of how participating systems, 
referred to as \text{roles}, interact. 
The protocol stipulates the sequence of structured messages 
exchanged between roles; 
each message is labelled with a name and the
type of payload carried by the message.

We present an example of a Scribble protocol in 
\cref{fig:adder} adapted from \cite{Hybrid2016}.
The protocol specifies an arithmetic web service 
offered by a \trole{S}erver to a \trole{C}lient.
The \trole{C}lient is permitted to either:

\begin{itemize}
\item 
Send two \texttt{int}s attached to an \tmsg{Add} message, 
where the server will respond with an \texttt{int} in a 
message labelled \tmsg{Res}, and the protocol restarts; or,
\item
Send a \tmsg{Quit} message, where the server will 
respond with a \tmsg{Terminate} message and the protocol ends.
\end{itemize}

\begin{figure}[!ht]
\begin{lstlisting}[language=Scribble]
type <java> "java.lang.Integer" from "rt.jar" as int; (*@\label{line:typedecl}@*)

global protocol Adder(role C, role S) {
	choice at C {
		Add(int, int)	from C to S;
		Res(int)		from S to C;
		do Adder(C, S);
	} or {
		Quit()		from C to S;
		Terminate()	from S to C;	
	}
}
\end{lstlisting}
\caption{Adder Protocol in Scribble}
\label{fig:adder}
\end{figure}

The platform-independent nature of Scribble 
can be observed from the \texttt{type} declaration 
on \cref{line:typedecl}: 
the developer has the freedom to specify message payload formats 
and data types from the target language of the implementation 
-- in this case, aliasing the built-in Java integer as 
\texttt{int} throughout the protocol.

To observe the parallels between MPST theory
and the Scribble language,
we present the corresponding global type for
the \tprotocol{Adder} protocol below.

\[
G_\text{Adder} = \trec{
\gcommone{C}{S}{
	\begin{cases}
		\tmsg{Add(int, int)} & :
			\gcommone{S}{C}{\tmsg{Res(int)}}. \trecvar \\
		\tmsg{Quit()} & : 
			\gcommone{S}{C}{\tmsg{Terminate()}}. \tend \\		
	\end{cases}
}} 
\]

The protocol specification language is a component of the broader 
\textit{Scribble project} initiated by Honda et al. in \cite{Scribble}, 
through which the project also facilitates the development of 
\textit{endpoint applications} that conform to user-specified protocols.

\subsection{Session Types in Web Development}
\label{subsection:sessiontypewebdev}

We discuss the current state-of-the-art proposals
for integrating session types in web development.

\subsubsection{API Generation in PureScript}
\dots

\subsubsection{MVU using Links}
\dots