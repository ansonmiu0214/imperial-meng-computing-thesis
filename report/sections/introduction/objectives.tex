\section{Objectives}

The objective of this project is to provide
developers with a development workflow
that \textit{provides communication safety guarantees in modern
web programming} through multiparty session types.

We tackle the limitations of the current
state-of-the-art with two solutions:

\subparagraph{S1: \codegen}
We implement an end-to-end framework
for writing full-stack \textit{TypeScript} applications that
\textbf{statically} conform to a communication protocol.

Developers specify their communication protocol using
the Scribble protocol description language \cite{Scribble}
which is based on multiparty session type (MPST) theory \cite{MPST}.
The Scribble protocol is used as a basis for generating
TypeScript APIs targeted for the Node.js server-side runtime
and React.js browser-side framework.
By implementing the generated APIs in their full-stack applications,
developers can enjoy static guarantees with respect 
to communication safety: 
there will \textit{not} be any \textbf{deadlocks} or
\textbf{communication mismatches},
and \textbf{channel linearity} will be respected.

\subparagraph{S2: \newtheory}
We present an extension of the canonical MPST
theory that supports \textit{routed communication}.
We prove that \newtheory also extends the communication safety
properties of the canonical MPST theory, so
processes that implement \newtheory also enjoy
guarantees of \textbf{deadlock-freedom}, \textbf{channel linearity},
and the absence
of \textbf{communication mismatches}.

Through \newtheory, we can express interactions to be
routed via an intermediate endpoint.
By defining an encoding from canonical MPST theory
to \newtheory, we show that we do not lose expressiveness
using routed communication.
We also extend \codegen to implement \newtheory.
\\

By solution \textbf{S1}, we address the limitation \textbf{L1}.
Our work provides the same communication safety guarantees
as the current state-of-the-art, but 
is compatible with modern web programming tools and idioms.
Among the languages that compile to JavaScript,
we select TypeScript as it is arguably the most intuitive
to use, given it is a \textit{superset} of JavaScript 
\cite{UnderstandingTypeScript}.
It provides developers
with type-safety through its gradual, structural type system,
and is also interoperable with existing JavaScript libraries
that the developer may incorporate into their application.
Whilst some \cite{MVU2020} point out this limits its usability for
encoding multiparty session types, we believe that the language
offers sufficient features that we can use to provide developers with
communication safety guarantees whilst preserving a flexible, natural
and idiomatic workflow.
The TypeScript APIs we generate also support idioms commonly
found in modern web development (such as callbacks and
asynchronous operations), and provides an error handling
mechanism tailored to the dynamic web-based environment
(where roles can disconnect prematurely).

In solution \textbf{S2},
we show that routing communication interactions through an intermediate
role, such as a web server, preserves the communication
safety properties and the semantics from the original
interactions.
By extending \codegen to implement \newtheory,
our workflow can implement protocols of
arbitrary communication structures over a server-centric
network topology using WebSockets; 
this tackles limitation \textbf{L2}.