\section{Contributions}

We present an implementation of the MPST framework for developing
interactive web applications over WebSocket transport. 
Related work \citep{PureScript2019, MVU2020, LINKS, Exceptional} requires the 
developer to implement their application using tools that are 
unconventional for web development.
To the best of our knowledge, this is the first work on multiparty session
type-safe web development that provides the same communication safety 
guarantees using familiar industry tools in TypeScript and React.

In \cref{chap:background}, we introduce the core session type theory
and provide an overview of related work on integrating session types into
application development. We also introduce TypeScript and highlight some 
interesting features and caveats in the language.

The remainder of the report is organised in three parts as such:

\paragraph{\cref{part:server}}
We present an end-to-end workflow for generating TypeScript APIs for 
\emph{server-centric} communication protocols over WebSockets. Server-centric 
protocols feature exactly one server role that is involved in all interactions.

In \cref{chap:codegen}, we introduce the system architecture for our
code generator, \fancyname{SessionTS}. 
The implementation integrates with the Scribble
toolchain to generate handler style APIs that the developer can use to write 
their web applications to guarantee protocol conformance by construction.

We present two code generation strategies for back-end and front-end
TypeScript web development as \fancyname{NodeMPST} and \fancyname{ReactMPST}
in \cref{chap:node} and \cref{chap:react} respectively.

In \cref{chap:ext}, we extend \fancyname{SessionTS} to support advanced
practices commonly found in full-stack web development in a way that
preserves communication safety. We modify the API generation strategies
to support asynchronous implementations and introduce how to 
gracefully handle errors and session cancellation 
in the context of interactive web applications.

\paragraph{\cref{part:general}} 
We present a method to relax the server-centric constraint over WebSocket
transport in order to support peer-to-peer interactions amongst 
client roles, through the concept of \emph{routing}.

In \cref{chap:p2p}, we introduce the \tprotocol{Two Buyer} multiparty
protocol to motivate our proposal of providing the server role with 
routing responsibilities in addition to its existing interactions
specified in the protocol.

In \cref{chap:theory}, we introduce a variant of multiparty 
session type theory with routing. We prove that our extensions preserve 
properties required for communication safety, 
session fidelity and deadlock freedom.
We provide an encoding for global types into routed multiparty 
session type theory, and prove the preservation of well-formedness
and communication. Detailed lemmas and proofs in this section can
be found in \cref{section:proofs}.

In \cref{chap:impl}, we demonstrate how our API generation strategies
implement our routed session type theory to support arbitrary communication
topologies.

\paragraph{\cref{part:evalconcl}}
In \cref{chap:eval}, we evaluate our work through 
case studies of communication protocols 
implemented as webservices. Through examples of binary sessions,
multiparty sessions and routed multiparty sessions, we demonstrate how the
developer may implement the generated APIs and consider the impact of our work
on developer productivity.

Finally, we summarise our work in \cref{chap:concl} and propose potential
areas of improvement and future work.

\begin{remark}
This project was published as \emph{Generating Interactive WebSocket 
Applications in TypeScript} in Volume 314 of the Electronic Proceedings
in Theoretical Computer Science (EPTCS) \cite{PLACES2020}.
\end{remark}