\chapter{\fancyname{ReactMPST}: Front-End Session Type Web Development}
\label{chap:react}

In this chapter, we present \fancyname{ReactMPST},
our session type API generation strategy for browser-side endpoints
implemented using React \cite{React}.
We will refer to the FSM of the \trole{Client}
endpoint (\cref{fig:adderclientfsm})
from the \tprotocol{Adder} protocol 
throughout this chapter.

\begin{figure}[!b]
\centering
\includegraphics[width=0.55\textwidth]{AdderClientFSM}
\captionof{figure}{\trole{Client} Endpoint FSM
in \tprotocol{Adder} protocol}
\label{fig:adderclientfsm}
\end{figure}

\section{Challenges}
\label{section:nodechallenges}
\begin{itemize}
\item session type implementation on the node runtime
\item need to provide consistency under the single-threaded event loop model
\item provide idiomatic event-driven APIs
\item provide static guarantees where possible -- communication mismatch and channel linearity
\item typescript's type system is gradual, structural and non-linear -- session type api generation examples that provide static guarantees on channel usage build upon target language type system's support for linear resources
\end{itemize}

As outlined in \cref{section:typescript}, TypeScript's type system
is neither affine nor linear, so we need to work around the 
limitations of the language to enforce linear usage of channel resources.
Similarly, we need to 
the language does not provide
built-in constructs for enforcing linear channel usage.

\section{Approach}
\begin{itemize}
\item leverage existing front-end frameworks that support reactive programming -- identify the parallel ideas that the UI element rendered on the screen needs to be reactive with respect to -- not only user interaction, but also channel actions.
\item for example, UI should react to a receive state in the EFSM, and once the message is received on the channel, transition to a different UI with the supported actions
\item main goal -- statically ensure that whatever IO actions are `present' on the window at any given time must respect the protocol
\end{itemize}

\subsection{The React Framework}
Our browser-side API generation strategy builds upon the 
\emph{React.js} framework developed by Facebook \cite{React} for the
\textit{Model-View-Controller} (MVC) architecture.
React is widely used in industry to create scalable single-page
TypeScript applications, so this makes our workflow beneficial in an
industrial context.

We introduce the key features of the framework
through illustrating a web-based counter in \cref{lst:counter}.
The browser shows a counter (initialised to zero) 
and an ``Increment'' button:
when the user clicks on the ``Increment'' button,
the count is incremented and the UI shows the updated count.

\begin{figure}[!h]
\begin{lstlisting}[language=javascript,tabsize=2]
type Props = { count: number };
class Count extends React.Component<Props>{
	render() { (*@\label{line:childrender}@*)
		return <strong>{this.props.count}</strong>; (*@\label{line:childprops}@*)
	}
}

type State = { count: number };
class App extends React.Component<{}, State>{
	constructor(props: {}) {
		super(props);
		this.state = { count: 0 }; (*@\label{line:parentstate}@*)
	}
	
	increment() { this.setState({ count: this.state.count + 1 }); (*@\label{line:parentsetstate}@*)
	
	render() { (*@\label{line:parentrender}@*)
		return (<div>
			<button onClick={this.increment.bind(this)}>
				Increment
			</button>
			<Count count={this.state.count} /> (*@\label{line:childcomponent}@*)
		</div>);	
	}
}
\end{lstlisting}
\captionof{lstlisting}{Simple Counter in React}
\label{lst:counter}
\end{figure}

\subparagraph{Components}
A \textit{component} is a reusable UI element which
contains its own mark-up and logic.
Components implement a \texttt{render()} method which returns
a ReactNode, the smallest building blocks of a React application.
This is analogue to the \textit{view} function in the MVU architecture.
React uses \textit{JSX} syntax to interpolate TypeScript logic 
(enclosed in curly braces)
within HTML mark-up: 
in \cref{line:childrender}, the \texttt{Count} component
evaluates the TypeScript expression 
\lstonelinejs{this.props.count} and renders it in bold on the web page.
Components can render other components, which give rise to
a tree of UI elements. \cref{line:parentrender} shows that our
\texttt{Count} component is rendered by 
another component, \texttt{App}.

\subparagraph{Uni-directional Data Flow}
User-defined components are derived from the abstract
\texttt{React.Component} generic base class

\subparagraph{Virtual DOM (VDOM) and Reconciliation}
\dots

\paragraph{Components}
A component is a reusable UI element which
contains its own markup and logic.
Components implement a \texttt{render()} function which returns a React
element, the smallest building blocks of a React application, analogous to the
view function in the MVU architecture.
Components can keep \textit{state}s and the \texttt{render()} function is
invoked upon a change of state.

For example, a simple counter can be implemented as a component,
with its \texttt{count} stored as state.
When rendered, it displays a button which increments \texttt{count}
when clicked and a \texttt{div} that renders the current
\texttt{count}.
If the button is clicked, the \texttt{count} is incremented, which triggers a
re-rendering (since the state has changed), and the updated \texttt{count} is
displayed.

Components can also render other components, which gives rise
to a parent/child relationship between components.
Parents can pass data to children as \textit{props} (short for properties).
Going back to the aforementioned example, the counter component could
render a child component \texttt{<StyledDiv count=\{this.state.count\} />} in
its \texttt{render()} function, propagating the \texttt{count} from its state
to the child.
This enables reusability, and for our use case, gives control to the parent
on what data to pass to its children (e.g. pass the payload of a received
message to a child to render).



\begin{itemize}
\item use a class-based Counter example to explain the terms: props, state, uni-directional data flow, virtual DOM, reconciliation
\end{itemize}

\paragraph{Note: from PLACES}
Our browser-side session type encodings for browser-side targets build upon the
\emph{React.js} framework, developed by Facebook \cite{React} for the
\textit{Model-View-Controller} (MVC) architecture.
React is widely used in industry to create scalable single-page TypeScript
applications, and we intend for our proposed workflow to be beneficial in an
industrial context.
We introduce the key features of the framework.



\subsection{Model Types Revisited}
\paragraph{Note: from PLACES paper}
For browser-side targets, we extend the approach presented in \cite{MVU2019} on
\textit{multiple model types} motivated by the \textit{Model-View-Update} (MVU)
architecture.
% FEEDBACK: introduce MVU and model type
An MVU application features a \textit{model} encapsulating application
state, a \textit{view function} rendering the state on the Document Object Model (DOM), and an
\textit{update function} handling \textit{messages} produced by the
rendered model to produce a new model.
The concept of model types express type dependencies between these
components: a \emph{model type} uniquely defines a \textit{view function},
set of \textit{messages} and \textit{update function} -- rather than
producing a new model, the update function defines valid transitions to
other model types.
We leverage the correspondence between model types and states in the EFSM:
each state in the EFSM is a model type, the set of messages represent
the possible (IO) actions available at that state,
and the update function defines which successor state to transition to,
given the supported IO actions at this state.
\section{EFSM Encoding}
\label{section:nodeefsm}

We show the structure of the generated EFSM.ts file in
\cref{lst:nodeefsmfile}.
Note that the formal definition of the EFSM in 
\cref{section:scribbleefsm}
contains more than just states and the state transition function,
so we encode the additional information as well.
Each type of information is grouped into their own
\textit{namespace}, and are collectively exported in
the EFSM \textit{module} for the developer to use.

In addition to states and transitions, we generate 
TypeScript constructs
for relevant metadata in the state machine: this includes
\textit{roles}, \textit{labels} and \textit{message structures}.


The EFSM contains more than just states and transitions

The generated \textbf{EFSM.ts} file encodes all information
available 

\begin{figure}
\begin{lstlisting}[language=javascript,tabsize=2]
export namespace Roles {...};
export namespace Labels {...};
export namespace Message {...};

export namespace Handler {...};

abstract class ISend {...};
abstract class IReceive {...};
abstract class ITerminal {...};
export namespace Implementation {...};

export type EfsmTransitionHandler =
	(implementation: Implementation.Type) => void;
export type MessageHandler = (message: any) => void;
\end{lstlisting}
\captionof{lstlisting}{Structure of generated EFSM.ts file 
for server endpoints}
\label{lst:nodeefsmfile}
\end{figure}

\begin{itemize}
\item callbacks -- type aliases for each state (handler)
\end{itemize}

\begin{figure}[!ht]
\begin{lstlisting}[language=javascript, tabsize=2]
// Inside the Message namespace...
export interface S54THANKS {
	label: Labels.S54.THANKS,
	payload: [string],
};
export interface S54TERMINATE {
	label: Labels.S54.TERMINATE,
	payload: [],
};

export type S54 = | S54THANKS | S54TERMINATE;
\end{lstlisting}
\label{lst:addersvrmsg}
\captionof{lstlisting}{Generated Message Type Definition for State 54}
\end{figure}

By expressing the payload type as a \textit{tuple}\footnote{
In TypeScript, a tuple is an array with fixed size
and known types for elements at each position.
},
we easily generalise our type definition to polyadic payloads.

\subsection{Handler APIs}

We collect the APIs that the developer needs to implement
under the \texttt{Handler} namespace. 
As a design choice, we \textit{do not} generate handlers for
terminal states, because the semantics of inactivity mean
there is nothing to handle.
%Hence, all generated handler APIs are for non-terminal states,
%and will hold some reference to the successor state encoding.
%The reader will notice that, in the listings below, the successor
%state is stated to be under the 
We outline how the generated API differs between sending
and receiving states.

\subparagraph{Send}
We model selections using \textit{union types} to
encapsulate the possible send actions. 
Each send action is encoded as a tuple of
the label, the payload, and the successor state.
We see some benefits from defining Message Types as interfaces:
with deterministic properties, we can index into the message type
to access the property's type, 
e.g. \texttt{Message.S54THANKS['payload']} would resolve to
\texttt{[string]}, based on \cref{lst:addersvrmsg}.
We provide an example in \cref{lst:addersvrsendhandler}.

We generalise deterministic send actions as a trivial \textit{selection}, 
as motivated from the theory (\cref{fig:globaltypes}),
so the encoding for State 53 in the \trole{Svr} FSM would be
the union of a single tuple.

\begin{figure}[!ht]
\begin{lstlisting}[language=javascript, tabsize=2]
// Inside the Handler namespace...
export type S54 = 
	| [Labels.S54.THANKS, Message.S54THANKS['payload'],
		Implementation.S52] 
	| [Labels.S54.TERMINATE, Message.S54TERMINATE['payload'], 
		Implementation.S52];
\end{lstlisting}
\captionof{lstlisting}{Generated Type for \trole{Svr} Send State
in \tprotocol{Adder} protocol}
\label{lst:addersvrsendhandler}
\end{figure}

\subparagraph{Receive}
\begin{itemize}
\item very short -- go over the structure and give examples
\item show how it extends for polyadic payloads
\end{itemize}

\begin{figure}[!ht]
\begin{lstlisting}[language=javascript,tabsize=2]
// Inside the Handler namespace...
export type S51 = {
	[Labels.S51.ADD]: (...payload: Message.S51ADD['payload']) =>
		Implementation.S53,
	[Labels.S51.QUIT]: (...payload: Message.S51QUIT['payload']) => 
		Implementation.S54,
}
\end{lstlisting}
\captionof{lstlisting}{Generated Type for \trole{Svr} Receive State
in \tprotocol{Adder} protocol}
\end{figure}

As with send states,
we generalise deterministic receive actions as a trivial \textit{branch},
which would be an interface with one property. 

\subsection{Wrapping Handlers in ``Implementations''}
\begin{itemize}
\item need a way for the runtime to distinguish between states (implementation)
\item attempt at using conditional types, but because they handle conditional union types in a distributed way (give example), cannot be statically typed
\item alternative -- use discriminated union by wrapping in an Implementation class
\item all non-terminal states will `return' the successor implementation -- very difficult to resolve in the type-checker in the runtime (give example of what it may look like) -- solve by giving each state implementation the `advance' function to respect the event-driven nature of everything, so they can call it on completion
\item visualise the interaction between runtime and states as a sequence diagram for message passing
\end{itemize}

\section{Runtime}
\label{section:noderuntime}

\begin{itemize}
\item public API -- provide seam for websocket server and implementation
\item private API -- carry out the session
\end{itemize}

\subsection{Managing Connections}
\begin{itemize}
\item use set and partial object to keep track of pending connections
\item ws event listeners
\item forward reference that we need to manage cancellations here too
\end{itemize}

\subsection{Executing the EFSM}
\begin{itemize}
\item implementation details delegated to the actual implementation states
\item constructor binds all methods (that will be passed to other components0 to `this' because of how javascript works with respect to the scoping of `this' (give short example -- or ignore, if this is in the background for typescript)
\item advance -- use discriminated union to figure out what to pass to the state (send -- sendMessage, receive -- register)
\end{itemize}

\begin{figure}[!ht]
\centering
\includegraphics[width=0.5\textwidth]{NodeRuntimeEFSM}
\captionof{figure}{``Message Passing'' Abstraction of EFSM Execution for
Server Endpoints}
\label{fig:noderuntimeefsm}
\end{figure}

\subsection{Handling Message Sends}
\begin{itemize}
\item very short (just for completeness in the report) -- messages are serialised JS objects (in JSON notation) sent and decoded on the other end, type-correct by construction
\end{itemize}

\subsection{Handling Message Receives}
\begin{itemize}
\item motivate the edge case that message can arrive in the websocket (in succession) before the handler is registered
\item motivate the edge case that message can arrive `out of protocol order' 
\item explain the double queue system and how that provides consistency -- emphasising that this works because of the single-threaded typescript runtime
\end{itemize}

\begin{figure}[!ht]
\centering
\begin{subfigure}[b]{0.8\textwidth}
\centering
\includegraphics[width=\textwidth]{NodeRuntimeReceive2}
\caption{Message processed before transitioning to receive state}
\label{subfig:nodereceivemsgfirst}
\end{subfigure}
\hfill
\begin{subfigure}[b]{0.8\textwidth}
\centering
\includegraphics[width=\textwidth]{NodeRuntimeReceive1}
\caption{Message processed after transitioning to receive state}
\label{subfig:nodereceivehandlefirst}
\end{subfigure}
\captionof{figure}{Possible Orderings for Handling Message Event and Preparing
Receive State}
\label{fig:nodereceivecompare}
\end{figure}

\subsection{Handling Termination}
\begin{itemize}
\item design choice that the client is the one that closes the connection because of the centralised role of the server
\item forward reference that this gives opportunity to extend support for general protocols, as in those protocols, the server's terminal state doesn't imply it's terminal for everyone else
\end{itemize}

\section{Alternative Designs}

For send states, a simpler approach would be
to provide the developer with a \texttt{send()} function 
for each permitted selection.
The factory approach would still apply -- the runtime could pass something
of the form

\begin{lstlisting}[language=javascript, numbers=none]
declare function buildSend<T>(label: string): (payload: T) => void
\end{lstlisting}

as a prop to the send state component.
By passing the payload type and label, the send state component
can build a handler (bound to the payload type) that performs a
send over the WebSocket when called.
This provides more flexibility for the developer's implementation,
but this clearly exposes channel resources and 
comes at the cost of not being able to provide guarantees
on affine channel usage.

For receive states, \dots

%
%\begin{itemize}
%\item not sure whether to put this in the subsection above so it flows better?
%\item naive implementation is to pass the send() function as a prop for the developer to incorporate in their event listener -- this is most flexible, but we cannot provide static guarantees
%\item factory approach -- the runtime could pass a factory function which, takes an UI element and performs the binding; this is better, as channel resources are not exposed, but the adversarial-minded developer can later override the event listener in the DOM etc
%\end{itemize}
\section{Limitations}
\begin{itemize}
\item affine channel usage guarantees -- cannot statically guarantee that the factory is attached to an event visible on the DOM
\item decoupling communication state management from business logic state management -- runtime design does not permit passing non-communication-related data as props; need to use other methods for propagating applicaton data (such as React Context API or Redux)
\item the point above might not be a `limitation', but more of an encouragement to change practices because it makes sense, architecturally, to separate concerns
\end{itemize}


%\section{Summary}
%\begin{itemize}
%\item static guarantees on avoiding communication mismatch
%\item static guarantees on affine channel usage
%\item leverage the react framework to express the model types approach from Fowler 2020 appropriately, to provide static guarantees, by construction, that the channel actions made available at each state respect the protocol, and by definition, multiparty session type safety
%\end{itemize}