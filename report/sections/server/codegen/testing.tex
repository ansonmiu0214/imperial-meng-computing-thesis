\section{Testing}

The usual unit testing practices from software engineering
apply to our code generator -- the built-in \texttt{unittest} 
package is more than sufficient for our use case. 
Instead, we detail our methodology for \textit{system testing} -- 
executing \fancyname{SessionTS} end-to-end and testing
the generated code.

We implement a test suite on top of \texttt{unittest} APIs
for verifying that \fancyname{SessionTS}
generates \textit{valid TypeScript code}. This is especially useful as
our templates contain both TypeScript syntax and Jinja markup, 
so we cannot easily make sure that each template generates valid code,
let alone checking that the collection of templates generate a valid
TypeScript project altogether.

\begin{figure}[!ht]
\begin{lstlisting}[language=python,tabsize=4]
def test_code_generation(self):
	flags = [scr, protocol, role, target]
	if svr is not None:
		flags.append('-s')
		flags.append(svr)

	rc = mpst_ts.main(flags) (*@\label{line:testmain}@*)
	self.assertEqual(rc, 0)  (*@\label{line:testmainrc}@*)

	completion = subprocess.run(self.npm_test_cmd, shell=True) (*@\label{line:testtsc}@*)
	self.assertEqual(completion.returncode, 0) (*@\label{line:testtscrc}@*)

	shutil.rmtree(self.output_dir)
\end{lstlisting}
\captionof{lstlisting}{Main logic in \fancyname{SessionTS} system testing
test case}
\label{lst:systemtest}
\end{figure}

We provide a collection of Scribble protocols under \texttt{protocols/}
to generate test cases, one per protocol participant.
For each test case, the test suite (\cref{lst:systemtest}) will:

\begin{enumerate}
\item Invoke \fancyname{SessionTS} to generate the TypeScript project 
(\cref{line:testmain}), 
expecting a zero exit code (\cref{line:testmainrc});
\item Run the TypeScript Compiler on the generated directory 
(\cref{line:testtsc}),
passing the \texttt{noEmit} flag to purely perform type-checking,
and expecting a zero exit code (\cref{line:testtscrc}).
\end{enumerate}

As the generated TypeScript code makes assumptions about the
environment in which it is used (for example, having the \texttt{ws}
WebSocket package installed on server-side endpoints), we require a
sandbox environment to type-check the generated code. 
The sandbox contains the minimal boilerplate required
for testing -- this involves having the WebSocket package installed
for server-side endpoints, the React.js framework (\cref{chap:react})
instantiated for browser-side endpoints,
and corresponding \texttt{tsconfig.json} files for both targets to
be picked up by the TypeScript Compiler.

For convenience, we extend the
Dockerfile and \texttt{build.sh} script 
to set up the sandbox environments. 
We also make use of the optional
\texttt{--output} flag exposed by the \fancyname{SessionTS} CLI
to redirect the generated code to the correct sandbox environment
to simply the testing process.