\section{Implementation}

\fancyname{SessionTS} is written in Python. 
It offers flexible syntax, a rich standard library, and 
a healthy ecosystem of DOT graph parsers and template engines,
making it a suitable choice for implementing our code generator.
Its rich standard library also simplifies many tasks: we use
the \texttt{argparse} package to generate an informative
command line interface (CLI) for developers to 
supply the correct information (\cref{lst:cmdline})
to use \fancyname{SessionTS},
and \texttt{subprocess} package provides an abstraction to
invoke external tools, such as Scribble and the TypeScript Compiler.

\begin{figure}[!ht]
\begin{lstlisting}
root@mpst_ts:/home# python3.7 -m mpst_ts --help
usage: __main__.py [-h] [-s SERVER] [-o OUTPUT]
                   filename protocol role {browser,node}

positional arguments:
  filename       Path to Scribble protocol
  protocol       Name of protocol
  role           Role to project
  {browser,node} Code generation target

optional arguments:
  -h, --help     show this help message and exit
  -s SERVER, --server SERVER
                 Server role (only applicable for browser targets)
  -o OUTPUT, --output OUTPUT
                 Output directory for generation
\end{lstlisting}
\captionof{lstlisting}{Entry Point for \fancyname{SessionTS}}
\label{lst:cmdline}
\end{figure}

We use Docker \cite{docker} to encapsulate our code generator
and its dependencies -- we found the canonical Python virtual environment
solution to be insufficient, as the Scribble toolchain is
a standalone Java executable with non-trivial setup procedures.
The Dockerfile builds a Docker image with Scribble and
the Python dependencies all pre-configured, and the provided
\texttt{build.sh} script instantiates the image as a container 
for development.
The \texttt{start.sh} script enters the Docker container
development environment and mounts the local project directory
onto the container as a volume to synchronise changes between the
two environments.