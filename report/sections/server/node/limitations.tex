\section{Limitations}
\label{section:nodelimitations}

Providing \textbf{static} communication safety guarantees have
come at a cost of providing relatively verbose APIs to the developer,
which introduces a learning curve. In particular,
the distinction between the \texttt{Implementation} and
\texttt{Handler} API should be an internal detail, but
the developer still needs to use the \texttt{Implementation} API
in their logic. 

Our APIs also rely on state identifiers, which also
does not help code readability, as it is neither apparent nor
self-documenting what \texttt{S51} refers to -- Hu and Yoshida
\cite{Hybrid2016} augments the state identifiers with the channel
action and labels involved (e.g. 
\texttt{S51_Receive__Add_Number__Quit_String}).

Using handler-style APIs also mean that we require
nested scoping to propagate values between states,
which does not scale elegantly as the levels of nesting
increase. For example, the following implementation of
the \tprotocol{One Adder} protocol would be invalid if
we define the handlers for the continuations separately,
because the received payload are no longer in scope
(\cref{line:notinscope}).

\begin{lstlisting}[language=javascript]
const first = new S4({ NUM1: (x) => secondOperand, });
const second = new S6({ NUM2: (y) => sum, });
const sum = [Labels.S7.SUM, (*@\label{line:notinscope}\hl{???}@*), new S5()];
\end{lstlisting}

However, the developer may choose to propagate values
between states through persistent storage APIs (such as
storing values in a database and retrieving them for
handlers of continuation states), so our generated APIs
do not \textit{strictly} enforce scoping to be the only way to
propagate values in the application logic.