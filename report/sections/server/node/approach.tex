\section{Approach}
\label{section:nodeapproach}

We motivate our approach from \cite{PureScript2019, HuJava}
to generate \textbf{handler-style APIs} to be implemented
by developers.
This aligns with idiomatic TypeScript practices of defining
\textit{callbacks} in the application logic.
We type the parameters and return values of the handlers to
reflect the message types specified in the protocol.
By strictly specifying handlers for send and receive actions,
we do not expose send and receive APIs to the developer,
contrary to \cite{HuJava},
making it impossible for the developer to reuse channels,
hence saving the need for linearity checks.

The responsibility of guaranteeing linear channel usage now falls
under the runtime that executes the EFSM.
As we generate this for the developer,
we can provide \textbf{static}
linearity guarantees by construction.

When executing \nodecodegen to generate code for the
\trole{Svr} endpoint specifying the \texttt{node} target,
the developer obtains the following files:

\begin{itemize}
\item \textbf{EFSM.ts:} TypeScript types constructed for 
the EFSM encoding (\cref{section:nodeefsm});
\item \textbf{Svr.ts:} Session runtime for executing the EFSM 
(\cref{section:noderuntime}).
\end{itemize}