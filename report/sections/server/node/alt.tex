\section{Alternative Designs}
\label{section:nodealt}

The main alternative EFSM encoding would be similar to those
presented in \cite{Hybrid2016}, which encodes
each EFSM state as a separate class and expose communication APIs
(e.g. send and/or receive) that respect the permitted transitions at
that state. 

\begin{lstlisting}[language=scribble]
global protocol OneAdder(role Client, role Svr) {
	NUM1(number) from Client to Svr;
	NUM2(number) from Client to Svr;
	SUM(number)  from Svr to Client;
}
\end{lstlisting}

\begin{figure}[!h]
\centering
\begin{subfigure}{0.49\textwidth}
\begin{lstlisting}[language=javascript,tabsize=2]
const logic = async (init) => {
	const [x, num2] = 
		await init.receive();
 	const [y, sum] = 
 		await num2.receive();
 		
 	// init.receive(); (*@\label{line:nodebreaklinear}@*)
	return sum.send(x + y);
}
\end{lstlisting}
\caption{Alternative Version}
%\label{...}
\end{subfigure}
\hfill
\begin{subfigure}{0.49\textwidth}
\begin{lstlisting}[language=javascript,tabsize=2]
const logic = new S4({
	NUM1: (x) => new S6({
		NUM2: (y) => ([
			Labels.S7.SUM,
			[x + y],
			new S5(),
		]),
	}),
});

\end{lstlisting}
\caption{Current Version}
%\label{...}
\end{subfigure}
\captionof{figure}{\dots}
%\label{...}
\end{figure}

\begin{itemize}
\item exposing channel resources like the java example -- less idiomatic (not-event-driven), need runtime linearity checks like the java example
\item `hardcoded' switch case runtime like the TSOP papers -- doesn't best leverage the type system
\end{itemize}