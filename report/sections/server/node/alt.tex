\section{Alternative Designs}
\label{section:nodealt}

The main alternative EFSM encoding would be similar to those
presented in \cite{Hybrid2016}, which encodes
each EFSM state in its own class 
and expose communication APIs (e.g. send and/or receive) 
for permitted state transitions.
We would still a session runtime to provide
abstractions over the WebSocket APIs and handle the 
``out-of-order'' arrival of non-causally-related messages.

We conceptualise this alternative design (\cref{fig:nodealt})
using the \tprotocol{One Adder} protocol (\cref{subfig:oneadder}), 
a simplified version of the \tprotocol{Adder} protocol 
but with \textit{curried} addition and without recursion.
An implementation using the alternative API design
(\cref{subfig:oneadderalt}) could have the state transition 
API return a tuple of payload and continuation,
and we could express the event-driven nature of a receive action
using built-in async/await concurrency primitives.
But ultimately, exposing channel resources without a linear type system
means that the developer could violate channel linearity
(i.e. by un-commenting \cref{line:nodebreaklinear}), 
so we would need runtime checks as part of the session runtime.

\begin{figure}[!h]
\centering
\begin{subfigure}{\textwidth}
\begin{lstlisting}[language=scribble]
global protocol OneAdder(role Client, role Svr) {
	NUM1(number) from Client to Svr;
	NUM2(number) from Client to Svr;
	SUM(number)  from Svr to Client;
}
\end{lstlisting}
\caption{The \tprotocol{OneAdder} Protocol}
\label{subfig:oneadder}
\end{subfigure}
\begin{subfigure}{0.49\textwidth}
\begin{lstlisting}[language=javascript,tabsize=2]
const logic = async (init) => {
	const [x, num2] = 
		await init.receive();
 	const [y, sum] = 
 		await num2.receive();
 		
 	// init.receive(); (*@\label{line:nodebreaklinear}@*)
	return sum.send(x + y);
}
\end{lstlisting}
\caption{Alternative API Design}
\label{subfig:oneadderalt}
\end{subfigure}
\hfill
\begin{subfigure}{0.49\textwidth}
\begin{lstlisting}[language=javascript,tabsize=2]
const logic = new S4({
	NUM1: (x) => new S6({
		NUM2: (y) => ([
			Labels.S7.SUM,
			[x + y],
			new S5(),
		]),
	}),
});
\end{lstlisting}
\caption{Current API Design}
\end{subfigure}
\caption{Comparing Alternative \fancyname{NodeMPST} 
API Design using \tprotocol{One Adder} Protocol}
\label{fig:nodealt}
\end{figure}

Looking at alternative designs for the runtime,
we could adopt the approach presented in \cite{TSOP},
which eliminates the need for our \texttt{Implementation} wrapper API
by pushing the workload back to the runtime in the form of a
dense switch statement enumerating all EFSM states.
This may simply the developer API (by removing the need for
the \texttt{Implementation} wrapper), but results in a
session runtime that increases in complexity as the size of the EFSM
increases.