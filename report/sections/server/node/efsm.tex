\section{EFSM Encoding}
\label{section:nodeefsm}

In addition to states and transitions, we generate 
TypeScript constructs
for relevant metadata in the state machine: this includes
\textit{roles}, \textit{labels} and \textit{message structures}.


The EFSM contains more than just states and transitions

The generated \textbf{EFSM.ts} file encodes all information
available 

\begin{figure}
\begin{lstlisting}[language=javascript,tabsize=2]
export namespace Roles {...};
export namespace Labels {...};
export namespace Message {...};

export namespace Handler {...};

abstract class ISend {...};
abstract class IReceive {...};
abstract class ITerminal {...};
export namespace Implementation {...};

export type EfsmTransitionHandler =
	(implementation: Implementation.Type) => void;
export type MessageHandler = (message: any) => void;
\end{lstlisting}
\captionof{lstlisting}{Structure of generated EFSM.ts file 
for server endpoints}
\label{lst:nodeefsmfile}
\end{figure}

\begin{itemize}
\item callbacks -- type aliases for each state (handler)
\item need a way for the runtime to distinguish between states (implementation)
\item attempt at using conditional types, but because they handle conditional union types in a distributed way (give example), cannot be statically typed
\item alternative -- use discriminated union by wrapping in an Implementation class
\item all non-terminal states will `return' the successor implementation -- very difficult to resolve in the type-checker in the runtime (give example of what it may look like) -- solve by giving each state implementation the `advance' function to respect the event-driven nature of everything, so they can call it on completion
\item visualise the interaction between runtime and states as a sequence diagram for message passing
\end{itemize}

\begin{figure}[!ht]
\begin{lstlisting}[language=javascript, tabsize=2]
// Inside the Message namespace...
export type S54THANKS = {
	label: Labels.S54.THANKS,
	payload: [string],
};
export type S54TERMINATE = {
	label: Labels.S54.TERMINATE,
	payload: [],
};

export type S54 = | S54THANKS | S54TERMINATE;
\end{lstlisting}
%\label{}
\captionof{lstlisting}{Generated Message Type Definition for State 54}
\end{figure}

\subsection{Handler APIs}

We collect the APIs that the developer needs to implement
under the \texttt{Handler} namespace. 
As a design choice, we \textit{do not} generate handlers for
terminal states, because the semantics of inactivity mean
there is nothing to handle.
We outline how the generated API differs between sending
and receiving states.

\paragraph{Send}
We generalise deterministic send actions as a trivial \textit{selection}, 
as motivated from the theory (\dots).

\begin{figure}[!ht]
\begin{lstlisting}[language=javascript, tabsize=2]
// Inside the Handler namespace...
export type S54 = 
	| [Labels.S54.THANKS, Message.S54THANKS['payload'],
		Implementation.S52] 
	| [Labels.S54.TERMINATE, Message.S54TERMINATE['payload'], 
		Implementation.S52];
\end{lstlisting}
\captionof{lstlisting}{Generated Type for \trole{Svr} Send State
in \tprotocol{Adder} protocol}
\end{figure}

\paragraph{Receive}
\begin{itemize}
\item very short -- go over the structure and give examples
\item show how it extends for polyadic payloads
\end{itemize}

\begin{figure}[!ht]
\begin{lstlisting}[language=javascript,tabsize=2]
// Inside the Handler namespace...
export type S51 = {
	[Labels.S51.ADD]: (...payload: Message.S51ADD['payload']) =>
		Implementation.S53,
	[Labels.S51.QUIT]: (...payload: Message.S51QUIT['payload']) => 
		Implementation.S54,
}
\end{lstlisting}
\captionof{lstlisting}{Generated Type for \trole{Svr} Receive State
in \tprotocol{Adder} protocol}
\end{figure}

As with send states,
we generalise deterministic receive actions as a trivial \textit{branch}, 
as motivated from the theory (\dots).

\subsection{Terminal}
\begin{itemize}
\item basically nothing here -- can highlight the provided type alias to provide a better developer experience
\end{itemize}