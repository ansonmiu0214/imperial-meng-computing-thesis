\section{Alternative Designs}

For send states, a simpler approach would be
to provide the developer with a \texttt{send()} function 
for each permitted selection.
The factory approach would still apply -- the runtime could pass something
of the form

\begin{lstlisting}[language=javascript, numbers=none]
declare function buildSend<T>(label: string): (payload: T) => void
\end{lstlisting}

as a prop to the send state component.
This allows the send state component to build a handler,
bound to the payload type and label identifier, that performs a 
send action when called with the payload value.
This provides more flexibility for the developer's implementation,
but this clearly exposes channel resources and 
comes at the cost of not being able to provide guarantees
on affine channel usage.

An alternative design for receive states
would be to define message receive handlers on the
\textit{successor} state.
Using \cref{fig:adderclientfsm} as an example,
the receive handler for \lstonelinejs{RES(number)}
would be defined on state S51 instead -- in fact, it wouldn't
be defined as a callback, but the runtime would simply pass
the received payload to S40 via props.
This allows the developer to render UI changes
based on received messages more easily, but comes at the
cost of less robust typing:
the \tmsg{RES} prop would have to be an optional value,
as an EFSM state can be preceded by more than one receive state,
so the developer has to implement logic (i.e. checking
for \lstonelinejs{undefined} props) to determine whether
the predecessor state was a receive state (and if so,
\textit{which} receive state) and render accordingly.