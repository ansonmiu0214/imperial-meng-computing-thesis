\section{EFSM Encoding}
\label{section:reactefsm}

We encode each EFSM state as a React component.
This encoding is consistent with the semantics of model types,
as outlined in \cref{table:efsmmodeltypes}.
\fancyname{ReactMPST} generates an abstract React component class
for each EFSM state. The developer implements the generated API
by extending the abstract class to define their own
view function and implement any abstract methods
required by the EFSM state.

\renewcommand{\arraystretch}{1.5}
\begin{table}[!ht]
\centering
\begin{tabularx}{\linewidth}{||c|X||}
\hline
Model Type Property & \multicolumn{1}{c||}{Equivalent Abstraction for EFSM} \\
\hline\hline
View Function & 
Each EFSM state is a React component 
with its own \texttt{render()} method. \\ 
\hline
Set of Messages & The permitted channel actions are defined either as 
component props (for send states) 
or abstract instance methods (for receive states). \\
\hline
Update Function & EFSM state components are rendered by a session runtime,
so they notify the runtime to trigger a transition. \\
\hline
\end{tabularx}
\captionof{table}{Implementing Model Types as React Components}
\label{table:efsmmodeltypes}
\end{table}
\renewcommand{\arraystretch}{1}

Our EFSM encoding for browser-side endpoints remain
consistent with those for server-side endpoints, in the sense that
channel resources are abstracted away from the APIs generated
for the developer, so it is impossible to trigger
channel actions \textit{more than once} at any given EFSM state,
by construction.

\subsection{Send States}

For React components implementing send states,
we need to guarantee that the channel actions on the DOM
precisely correspond to the permitted transitions.
We require that send actions can only be invoked
by user-triggered events on UI elements, such as clicking a button or
typing in a textbox -- this generally aligns
with idiomatic front-end web development practices.

Without \fancyname{ReactMPST},
an implementation of state S40 from \cref{fig:adderclientfsm}
may resemble the following, where the developer attaches
event listeners on UI elements to trigger channel actions:

\begin{lstlisting}[language=javascript,numbers=none]
<button onClick={ev => (*@\hl{Add}@*)(this.state.num1, this.state.num2)}>
	Add
</button>
<button onMouseOver={ev => (*@\hl{Quit}@*)(this.state.message)}>
	Quit
</button>
\end{lstlisting}

If we directly provide the highlighted channel actions 
(i.e. via props), we cannot guarantee linear usage in the code 
without a linear type system.

What we want to work towards is something similar to the following,
where the highlighted component encapsulates the required information: 
\textbf{(1)} what event to react to, 
and \textbf{(2)} what payload to send when the event is triggered.
\texttt{<AddOnClick>} should guarantee, by construction,
that clicking on any child element will send the \tmsg{ADD}
message with the specified payload.

\begin{lstlisting}[language=javascript,numbers=none]
(*@\hl{<AddOnClick payload=\{[this.state.num1, this.state.num2]\}>}@*)
	<button>Add</button>
(*@\hl{</AddOnClick>}@*);
\end{lstlisting}

This elegantly hides channel actions, but at the same time,
we cannot make excessive assumptions about the 
event (\texttt{onClick}) and UI element (\texttt{<button>})
that the send action will be attached to --
these should be left as implementation details that 
the developer can customise for their application.

Instead, we present a novel approach for binding
channel actions to UI elements in a way that \textit{never
exposes channel resources}:

\begin{enumerate}
\item
The abstract send state React component defines
\textbf{factory properties}
that allow the developer to \textit{generate React components}
for performing send actions, such that the API lets the developer
specify which UI event should trigger the channel action,
and the payload to send. It defines one factory property per
permitted send action at that state.

\item
The runtime provides a \textit{factory method}
which allows the abstract send state component to
generate the aforementioned factory properties
for the channel actions supported at that state.
\end{enumerate}

We walk through the implementation details by working backwards:
showing how the developer would use the API first,
before deriving the type signatures that we need to
generate for the abstract send state component inherited by the developer.

\subsubsection{Developer API}

We show the developer's implementation 
of a send state for the \trole{Client} endpoint
from the \tprotocol{Adder} protocol in \cref{lst:adderclientsenddev}.
Component state is an implementation detail 
that is not related to EFSM execution, so we allow
the developer to customise this through specifying an 
interface (\cref{line:clientsendstatestate}).

\begin{figure}[!h]
\begin{lstlisting}[language=javascript,tabsize=2]
type State = { num1: number, num2: number, message: string }; (*@\label{line:clientsendstatestate}@*)
class InputWindow extends S40<State> {
	render() {
		const { num1, num2, message } = this.state;
		const (*@\hl{Add}@*) = this.ADD('onClick', (*@\label{line:clientsendfactory1}@*)
			(ev: React.MouseEvent) => [num1, num2] (*@\label{line:clientsendpayload1}@*)
		);
		const (*@\hl{Quit}@*) = this.QUIT('onMouseOver', (*@\label{line:clientsendfactory2}@*)
			(ev: React.MouseEvent) => [message] (*@\label{line:clientsendpayload2}@*)
		);		
		return (<div>
			// Omitting <input>s for entering numbers, message
			(*@\hl{<Add>}@*)<button>Sum</button>(*@\hl{</Add>}@*)
			(*@\hl{<Quit>}@*)<button>Quit</button>(*@\hl{</Quit>}@*)
		</div>);
	}
}
\end{lstlisting}
\captionof{lstlisting}{\dots}
\label{lst:adderclientsenddev}
\end{figure}

The factory API for binding send actions to UI elements appear on
\cref{line:clientsendfactory1,line:clientsendfactory2}.
For example, \cref{line:clientsendfactory1} reads: 
\textit{build a React component that sends the \tmsg{ADD}
message with \texttt{[num1, num2]} as payload,
when the user clicks on it}.
\tmsg{ADD} is the factory API we generate: it takes an event identifier
and an event handler which must return the payload 
-- \textit{appropriately typed with respect to the Scribble protocol} -- 
for sending the \tmsg{ADD} message.
Swapping \cref{line:clientsendpayload1,line:clientsendpayload2}
will result in a compile-time type error due to invalid 
payload types, which is the intended behaviour.

We will define the factory API implementation in a way that
calls the event handler to obtain the payload,
sends the payload message, and transitions to the successor state,
such that the send action is never performed more than once.

\subsubsection{Typing the Abstract Send Component}
Referring back to \cref{lst:adderclientsenddev},
a first attempt for typing \tmsg{ADD} could be

\begin{lstlisting}[language=javascript,numbers=none]
(ev: string, handler: (e: any) => [number, number]) 
	=> Constructor<React.Component>
\end{lstlisting}

However, the type of \texttt{ev} depends on the value of
\texttt{event} -- for example, an \texttt{onClick} event
is handled by a \texttt{React.MouseEvent} handler.
React defines the supported event handlers under the
\texttt{React.DOMAttributes} interface (\cref{lst:reactdomattrs}).
We can use an index type query to express the dependency,
and type inference in conditional types\footnote{
We can \lstonelinejs{infer} type variables inside the
\lstonelinejs{extends} clause of a conditional type.
Specifically, \lstonelinejs{FunctionArguments<T> = 
T extends (...args: infer R) ? R : never}.
}
to extract the function argument types:

\begin{lstlisting}[language=javascript,numbers=none]
<K extends keyof React.DOMAttributes>
(ev: K, handler: (e: FunctionArguments<React.DOMAttributes[K]>) 
		=> [number, number])
	=> Constructor<React.Component>)
\end{lstlisting}

Contrary to our discussions about modelling dependent types in
\cref{subsection:nodeefsmimplementation}, we can do this here
because: \textbf{(1)} we are using index type queries instead
of conditional types, and \textbf{(2)} we know the type
of \texttt{ev} at compile-time from the developer's implementation,
so we are not resolving index type queries with union types.

\begin{figure}[!h]
\begin{lstlisting}[language=javascript,numbers=none]
interface DOMAttributes<T> {
	children?: ReactNode; (*@\label{line:reactdomattrsnonfunc}@*)
	onChange?: FormEventHandler<T>;
	onClick?: MouseEventHandler<T>;
	...
};
\end{lstlisting}
\captionof{lstlisting}{Snippet of the \texttt{React.DOMAttributes} Interface}
\label{lst:reactdomattrs}
\end{figure}

However, \texttt{React.DOMAttributes} also contains
non-function properties -- on \cref{lst:reactdomattrs}
\cref{line:reactdomattrsnonfunc},  ``children'' is not a valid event.
We define utility types under Types.ts 
to extract function properties from an interface 
(\cref{lst:extractfuncprops}): we first extract
the \textit{names} of function properties into an union type 
by mapping property names of non-function properties to 
\lstonelinejs{never} (\cref{line:extractfuncprops1})
and performing an index type query (\cref{line:extractfuncprops2})
to obtain an union type of function property names\footnote{
The union type technically contains \lstonelinejs{never}
as the names for non-function properties, but 
TypeScript removes \lstonelinejs{never} from union types
in the same way that $A \vee \bot \equiv A$.
}
which removes all \lstonelinejs{never} constituents;
then extract the properties from the interface that are
indexed by the filtered names (\cref{line:extractfuncprops3})

\begin{figure}[!h]
\begin{lstlisting}[language=javascript,tabsize=2]
type FunctionPropertyNames<T> = {
	[K in keyof T]: T[K] extends Function | undefined ? K : never; (*@\label{line:extractfuncprops1}@*)
}[keyof T]; (*@\label{line:extractfuncprops2}@*)
type FunctionProperties<T> = Pick<T, FunctionPropertyNames<T>>; (*@\label{line:extractfuncprops3}@*)
type DOMEvents = FunctionProperties<React.DOMAttributes<any>>;
\end{lstlisting}
\captionof{lstlisting}{\dots}
\label{lst:extractfuncprops}
\end{figure}

We construct a generic type parameterised on the payload type, 
which lets us define the \tmsg{ADD} and \tmsg{QUIT}
factory property easily.

\begin{lstlisting}[language=javascript,numbers=none]
type EventHandler<Payload, K extends keyof DOMEvents> = 
	(event: FunctionArguments<DOMEvents[K]>) => Payload;
	
type SendComponentFactory<Payload> = <K extends keyof DOMEvents>
	(event: K, handler: EventHandler<Payload, K>) => 
		Constructor<React.Component>;
\end{lstlisting}

The factories are instantiated by the abstract class
as \lstonelinejs{protected} properties, in order to allow 
access by the developer's subclass implementation.
Channel resources are managed by the runtime, so
we need the runtime to pass a ``higher-order factory''
(a factory method that generates a factory)
which lets the send state component generate the required
component factories (i.e. \tmsg{ADD} and \tmsg{QUIT})
with channel actions pre-injected into the component.
We discuss the runtime in \cref{section:reactruntime},
but for now, it is sufficient to understand that
the generated send component abstract class receives
the higher-order factory as a prop, and that this
prop binds the channel action into the factory that it returns.

Now we are in a position to accurately type the abstract
send component \texttt{S40} in \cref{lst:reactsendcomponent}.
We hint at the usage of the higher-order factory on
\cref{line:factory1,line:factory2}:
just using \cref{line:factory1} as an example,
the send component asks the runtime to generate a 
component factory for sending a message 
labelled \lstonelinejs{'ADD'}
with payload typed
\lstonelinejs{[number, number]},
then transition to the \texttt{ReceiveState.S42}
successor state.

\begin{figure}[!h]
\begin{lstlisting}[language=javascript,tabsize=2]
type Props = { factory: SendComponentFactoryFactory };
abstract class S40<State> extends React.Component<Props, State> {
	protected ADD: SendComponentFactory<[number, number]>;
	protected QUIT: SendComponentFactory<[string]>;
	
	constructor(props: Props) {
		super(props);
		this.ADD = props.factory<[number, number]>( (*@\label{line:factory1}@*)
			'ADD', ReceiveState.S42);
		this.QUIT = props.factory<[string]>( (*@\label{line:factory2}@*)
			'QUIT', ReceiveState.S43);
	}
}
\end{lstlisting}
\captionof{lstlisting}{Generated Type for \trole{Client} Send State
in \tprotocol{Adder} protocol}
\label{lst:reactsendcomponent}
\end{figure}

\subsection{Receive States}
\begin{itemize}
\item receive state is `shown' when waiting for a receive event
\item base class has abstract methods -- one for each possible receive -- so the user must implement these when they extend the abstract base class
\item each abstract method is named after the message label expected to be received, and parameterised by the payload
\item no channel actions are passed to the state -- instead, once the receive state is mounted on the DOM (i.e. rendered), the abstract base class will register the event handlers to the runtime (using the register method passed as props)
\item when the message arrives, the runtime will invoke the registered function (+ forward reference the next section with the edge case that the message arrives before handler registration occurs)
\end{itemize}

\subsection{Terminal States}
\begin{itemize}
\item by convention and design choice, the client gracefully closes the connection with the server role
\end{itemize}