\section{EFSM Encoding}
\label{section:reactefsm}

\begin{itemize}
\item motivated by the idea of model types, each state is encoded as a react component, where the user provides a view function (i.e. render())
\item channel actions are abstracted by the runtime and passed to state implementations as props
\item the developer provides an implementation of the state, i.e. ``uses the APIs'', by extending the abstract base classes and providing their own implementations
\end{itemize}

\subsection{Send}
\begin{itemize}
\item make the assumption that channel send actions are triggered by UI elements -- for example, clicking a button or typing enter in a textbox
\item pre-session-types -- the developer will attach an event listener
\item enhanced factory approach -- given an event and a event handler, return a react component that binds the handler to that event, and injects the channel action; the developer can use this API by putting the UI elements that represent the event action as children.
\item the event handler is conditionally parameterised by the event type (achievable by typescript conditional types -- eg, if event is click, then the event handler parameter is a React.MouseEvent), and the return type is the payload type
\item the abstract base class will set up the factory methods to be used by the developer
\end{itemize}

\subsubsection{Alternative Designs}
\begin{itemize}
\item not sure whether to put this in the subsection above so it flows better?
\item naive implementation is to pass the send() function as a prop for the developer to incorporate in their event listener -- this is most flexible, but we cannot provide static guarantees
\item factory approach -- the runtime could pass a factory function which, takes an UI element and performs the binding; this is better, as channel resources are not exposed, but the adversarial-minded developer can later override the event listener in the DOM etc
\end{itemize}

\subsection{Receive}
\begin{itemize}
\item receive state is `shown' when waiting for a receive event
\item base class has abstract methods -- one for each possible receive -- so the user must implement these when they extend the abstract base class
\item each abstract method is named after the message label expected to be received, and parameterised by the payload
\item no channel actions are passed to the state -- instead, once the receive state is mounted on the DOM (i.e. rendered), the abstract base class will register the event handlers to the runtime (using the register method passed as props)
\item when the message arrives, the runtime will invoke the registered function (+ forward reference the next section with the edge case that the message arrives before handler registration occurs)
\end{itemize}

\subsection{Terminal}
\begin{itemize}
\item by convention and design choice, the client gracefully closes the connection with the server role
\end{itemize}