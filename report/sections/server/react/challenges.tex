\section{Challenges}
Our goal with \fancyname{ReactMPST} is to
generate session type implementations for the web browser.
Session typing the user interface (UI) logic is inherently difficult --
assuming channel actions are bound to user interface events 
(i.e. clicking a button sends a message), how does one formalise
channel linearity? How do we guarantee that, if the button
triggers a send at some EFSM state, that it triggers not more than
one channel action, \textit{and} the user cannot trigger the action
at another EFSM state?

We recap how existing work (\cref{subsection:sessiontypewebdev})
tackle browser-side session typing.
Fowler introduced the concept of model types to prevent
channel linearity violation in his proposal for
integrating session types with
graphical user interface (GUI) programming \cite{MVU2020}, but
it requires the linear type system available from
the proprietary Links web programming language \cite{LINKS}.
The session type-safe web development framework presented
by King et al. \cite{PureScript2019} generates APIs for
a functional target language in PureScript, and relies on
the Concur UI framework that constructs UIs sequentially.

We find these proposals to come at the cost of
limiting developer productivity by adopting unconventional practices
that may require a learning curve.
Through our work, we aim to distil the key
concepts from existing work that provide session type safety
for web-based GUI programming, and implement them using
mainstream front-end web development tools (e.g. TypeScript
and React), to provide developers with an intuitive way to implement
browser-side endpoints that guarantee communication safety.