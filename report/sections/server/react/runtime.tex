\section{Runtime}
\label{section:reactruntime}

There are many parallels between the runtime we generate
for \fancyname{ReactMPST} compared to that of \fancyname{NodeMPST}.
We focus on the details unique to React endpoints.

We define the public API for the session runtime
as a React component. The developer provides implementation details via
props.
We show the public API for the \trole{Client} endpoint
of the \tprotocol{Adder} protocol below.

\begin{lstlisting}[language=javascript]
type Props = {
	endpoint: string,
	states: {						// mapping of state identifier
		S40: Constructor<S40>,		// to EFSM state components
		S42: Constructor<S42>,		// implemented by developer
		S43: Constructor<S43>,
		S41: Constructor<S41>,
	},
	waiting: React.ReactNode,		// render when waiting to start
	connectFailed: React.ReactNode,	// render if connection failed
};

export class Session extends React.Component<Props, ...
\end{lstlisting}

The developer provides the WebSocket URL for the runtime
to instantiate the connection.
As for the EFSM state implementations derived from the abstract
classes, the developer constructs an object to map state identifiers
to concrete implementations. 
The developer only needs to pass the \textit{constructor} function,
as the runtime will instantiate the React components separately
and pass the required props based on the type of state.
Additionally, the developer needs to define what to render
whilst \texttt{waiting} for the session to begin, and possibly
an error screen if the connection has failed.
We show an example below.

\begin{lstlisting}[language=javascript,tabsize=2]
// Developer implementations
class InputWindow extends S40 { ... };
class QuitReceived extends S41 { ... };
class WaitScreen extends S42 { ... };
class PendingQuit extends S43 { ... };

// Main application component
class MainApp extends React.Component {
	render() {
		return <div>
			<h1>Adder Client</h1>
			<Client
				endpoint='ws://localhost:8080'
				states={{
					S40: InputWindow,
					S41: QuitReceived,
					S42: WaitScreen,
					S43: PendingQuit,		
				}}>
				waiting={<h1>Pending Connection</h1>}
				connectFailed={<p>Connection Failed</p>}
			/>
		</div>;
	}
}
\end{lstlisting}

When the session starts, the private API takes over and executes
the EFSM by rendering the React component corresponding to the
current EFSM state, as well as performing channel actions
that adhere to the protocol.

\subsection{Connecting to the Session}
We connect to the session by creating a new \texttt{WebSocket}.
Traditionally, this would be done in the constructor,
but for React components, the constructor may be invoked more
than once depending on how reconciliation works.
For this reason, we create the \texttt{WebSocket} in the
\texttt{componentDidMount}\footnote{
The \texttt{componentDidMount} method is part
of the ``commit'' phase in the lifecycle of a React component.
This is when the component is mounted on the browser DOM and
methods in the phase are guaranteed to be called exactly once.
} method as it is guaranteed
to be only called once.
In fact, under React's \textit{Strict Mode},
constructors are \textit{explicitly invoked twice} to 
prevent impure constructors \cite{ReactStrictMode}. 

This means the WebSocket must have optional type as it is 
strictly undefined on construction.
To avoid having to deal with an optional WebSocket value
in the EFSM execution (even though we know for sure that it
has been instantiated when we use it),
we define the public API (the React component named 
\texttt{Session}) to manage the (possibly \lstonelinejs{undefined})
WebSocket in its state, and the private API (the React component
named after the role -- in this case, \trole{Client}) after
as a separate non-exported class
with the WebSocket available via props. 
Once the WebSocket
is instantiated in the \trole{Session} component,
it renders the
\trole{Client} component, 
passing the non-optional WebSocket value via props.

\begin{lstlisting}[language=javascript,tabsize=2]
type Transport = { ws: WebSocket };
class Session extends React.Component<Props, Partial<Transport>> {
	componentDidMount() {
		this.setState({ ws: new WebSocket(this.props.endpoint) });
	}

	render() {
		const { ws } = 	this.state;
		return ws === undefined ? this.props.waiting 
			: <Client ws={ws} {...this.props} />;
	}
}	
export default Session;
\end{lstlisting}

The connection phase is managed in the same way as \nodecodegen:
when the \texttt{Client} component is mounted,
it sends a connection request to the server and overrides the
\texttt{onmessage} event handler to listen for 
the connection confirmation,
before advancing the EFSM to the initial state to 
begin executing the protocol.

\subsection{Executing the EFSM}
Unlike \nodecodegen, our EFSM state encoding does not define
continuation states using the actual implementation.
This is because state encodings are React components
that need to be instantiated by the runtime.
Instead, as previewed in \cref{lst:reactsendcomponent,lst:reactreceivecomponent},
we define an enum-based abstraction for specifying
successor state, and this is also used for executing the EFSM.
Our enum-based abstraction defines a string enum for each type of EFSM state,
and collects all state identifiers in an union type.

\begin{lstlisting}[language=javascript,numbers=none]
enum SendState { S40 = 'S40' };
enum ReceiveState { S42 = 'S42', S43 = 'S43' };
enum TerminalState { S41 = 'S41' };
type State = ReceiveState | SendState | TerminalState;
\end{lstlisting}

The React runtime component also defines a transition function
parameterised by the current state's enum. The runtime renders
the current EFSM state component with different props depending
on the type of the current state.

\begin{itemize}
\item
For \textbf{send} states, the runtime provides the higher-order factory
method to allow the send state to construct its own send component
factory properties for each permitted send transition.

\item
For \textbf{receive} states, the runtime provides the \texttt{register}
function to let the developer pass back the receive handlers
for each permitted receive transition, which is used
to process incoming messages received via the WebSocket.

\item
For the \textbf{terminal} state, the runtime provides the
\texttt{terminate} function to close the WebSocket connection.
\end{itemize}

Enum unions cannot be used in the same manner as discriminated unions
to distinguish between \texttt{SendState} and \texttt{ReceiveState},
so we provide utility functions under EFSM.ts that define
\textit{type guards} to narrow the type of \texttt{State}.

\begin{lstlisting}[language=javascript,numbers=none]
function isReceiveState(state: State): state is ReceiveState {
	return Object.values(ReceiveState).includes(state)
}
\end{lstlisting}

We use these type guards in the transition function,
as shown in \cref{lst:reacttransitionfunc}.
The choice of a string enum for state identifiers
allow us to use it to index into the EFSM state component mapping
provided by the developer, as seen on 
\cref{line:reactadvance1,line:reactadvance2,line:reactadvance3}.
Note that the value of \texttt{state} can be an union type
(depending on the possible values in the enum),
so the type of \texttt{View} may also be union types.
We can still deterministically pass props and render the component,
as all states of each particular type agree on what props to receive.

\begin{figure}[!h]
\begin{lstlisting}[language=javascript]
private advance(state: State) {
	if (isSendState(state)) {
		const View = this.props.states[state]; (*@\label{line:reactadvance1}@*)
		this.setState({	
			elem: <View factory={this.buildFactory} />
		});
	} else if (isReceiveState(state)) {
		const View = this.props.states[state]; (*@\label{line:reactadvance2}@*)
		this.setState({
			elem: <View register={this.registerReceiveHandler} />
		});
	} else if (isTerminalState(state)) {
		const View = this.props.states[state]; (*@\label{line:reactadvance3}@*)
		this.setState({
			elem: <View terminate={this.terminate} />
		});
	}
}
\end{lstlisting}
\captionof{lstlisting}{EFSM Transition Function for Browser-Side Endpoint}
\label{lst:reacttransitionfunc}
\end{figure}

We keep track of the current EFSM state in the state of the
private \texttt{Client} component.
This enables the UI to be re-rendered, and therefore ``react'',
to EFSM state transitions, which is precisely the goal of 
\reactcodegen.

\subsection{Sending Messages}
\label{subsection:reactruntimesend}

Messages are sent through the WebSocket in the same way 
as \nodecodegen (see \cref{subsection:noderuntimesend}) 
-- objects marked with label and payload
are serialised into JSON. The process of sending is always followed
by a transition to the successor state, so we define the
\texttt{sendMessage()} method on the runtime to also take
the successor state as a function argument, so it can make that transition
immediately after.\\

\begin{lstlisting}[language=javascript,numbers=none]
private sendMessage(label: string, payload: any, next: State) {
	this.props.ws.send(JSON.stringify({ label, payload }));
	this.advance(next);
}
\end{lstlisting}

We focus on the higher-order factory that the runtime passes
to the send component as a prop. The implementation shown in
\cref{lst:reactsendfactory} is less straightforward,
so we will walk through the steps.

The prop allows the send component to build factory functions
for sending messages with a particular \texttt{label}
followed by a transition to some \texttt{successor} state
(\cref{line:sendfactoryparams}).
The higher-order factory relies on \textit{closures}
to bind the channel action into the generated component.
Here, \texttt{buildFactory} returns a (factory) function that
builds an anonymous React component which wraps the 
component's children into a \texttt{<div>} with special props 
(\cref{line:sendfactorydiv}).
The props for this anonymous React component comprises of a new
event listener bound to the UI event (\texttt{ev})
specified by the developer (such as \lstonelinejs{'onClick'}).
This new event listener is defined in terms of the
\texttt{handler} from the developer's implementation:
\textbf{(1)} it invokes the handler to obtain the payload to send 
(\cref{line:sendfactoryhandler}), which is constrained by
the generic type parameter to ensure it matches the protocol specification;
then, \textbf{(2)} it sends the message, which triggers the EFSM
to advance to the successor state.
We cannot directly call \lstonelinejs{this.sendMessage} inside
the generated React component because \lstonelinejs{this} refers
to the anonymous class; instead, we define a partially applied
version of the \texttt{sendMessage} method as a local variable 
(\cref{line:sendfactorycurrysend}).

\begin{figure}[!h]
\begin{lstlisting}[language=javascript,tabsize=2]
private buildFactory<T>(label: string, successor: State) { (*@\label{line:sendfactoryparams}@*)
	return <K extends keyof DOMEvents>
				 (ev: K, handler: EventHandler<E, K>) => {
		const send = (payload: T) => (*@\label{line:sendfactorycurrysend}@*)
			this.sendMessage(label, payload, successor);
		return class extends React.Component {
			render() {
				const props = {
					[ev as string]: (e: FunctionArguments<DOMEvents[K]>) => {
						const payload = handler(e); (*@\label{line:sendfactoryhandler}@*)
						send(payload);
					}
				};
				return <div {...props}>{this.props.children}</div> (*@\label{line:sendfactorydiv}@*)
			}		
		}	
	}
}
\end{lstlisting}
\captionof{lstlisting}{Higher-Order Factory Function in 
Browser-Side Runtime}
\label{lst:reactsendfactory}
\end{figure}

\subsection{Receiving Messages}
\label{subsection:reactruntimereceive}

We take the same approach as \fancyname{NodeMPST} --
the runtime keeps track of the ``active'' receive handler,
and the \texttt{onmessage} event listener for the WebSocket
is defined to dynamically process the incoming message
using the active receive handler.
We also need to address the same consistency problem, since
a message arrival event can be handled \textit{before}
the receive handler is registered.
The receive handler is registered when the receive state component
is rendered on the DOM, which occurs when \lstonelinejs{this.setState}
is invoked in the transition function.
\lstonelinejs{this.setState} may be \textit{asynchronous}
as React will internally batch multiple calls into a single update
for performance \cite{ReactState}, so it is possible for the WebSocket message
event to be queued (and hence, procesesd) before the receive handler
is registered.

We address this problem using a pair of queues to keep track
of messages waiting for handlers, and handlers waiting for messages.
Unlike \fancyname{NodeMPST}, we do \textit{not} need a mapping
of queues because browser-side endpoints only receive from the server.

\begin{lstlisting}[language=javascript,numbers=none]
private messageQueue: any[];
private handlerQueue: MessageHandler[];
\end{lstlisting}

This lets us define the WebSocket \texttt{onmessage} event
listener and the \texttt{register} function (which we pass as a prop
to the receive state component) in a similar way compared with
\fancyname{NodeMPST} -- we show this in \cref{lst:reactreceiveruntime}.
Recall that the message handler registered
by the receive state component returns the successor state,
so we invoke the transition function on the result directly 
(\cref{line:reactreceiveadvance1,line:reactreceiveadvance2}).

\begin{figure}[!h]
\begin{lstlisting}[language=javascript]
private register(handler: MessageHandler) {
	const message = this.messageQueue.shift();
	if (message !== undefined) { this.advance(handler(message)); } (*@\label{line:reactreceiveadvance1}@*)
	else { this.handlerQueue.push(handler); }
}

private receive({ data }: MessageEvent) {
	const handler = this.handlerQueue.shift();
	if (handler !== undefined) { this.advance(handler(data); } (*@\label{line:reactreceiveadvance2}@*)
	else { this.messageQueue.push(data); }
}
\end{lstlisting}
\captionof{lstlisting}{Receive Handler Registration and Message Event
Listener in \fancyname{ReactMPST}}
\label{lst:reactreceiveruntime}
\end{figure}

\subsection{Handling Termination}

The terminal state component notifies the runtime (through
the \texttt{terminate} function passed down as a prop)
to close the connection. The implementation is straightforward --
we simply invoke the \texttt{close()} method on the WebSocket.

\begin{lstlisting}[language=javascript,numbers=none]
private terminate() {
	this.props.ws.close();
}
\end{lstlisting}