\section{Runtime}
\begin{itemize}
\item public API - accept websocket endpoint URL, a mapping of state to user implementation, and UI elements to render when waiting for connection to occur, and connection failure
\item private API - carry out the session (just like Node)
\end{itemize}

\subsection{Session Joining}
\begin{itemize}
\item separate into two react components, since the websocket connection cannot be opened in the constructor by React StrictMode
\item `outer' class sets websocket to undefined, but upon componentDidMount, creates the actual connection
\item renders the `inner' class component upon websocket not being undefined, so it can be typed appropriately (otherwise, need to always use bang operator)
\item on connection, send join request message to the server, wait for serber response before executing the EFSM at the initial state
\end{itemize}

\subsection{Executing the EFSM}
\begin{itemize}
\item same concept as node -- implementation details delegated to the actual classes
\item constructor binds methods
\item advance takes the enum instead of the class (since unlike the node example, it is not instantiated) -- uses custom type guards to distinguish between the type of state
\end{itemize}

\subsection{Handling Message Sends}
\begin{itemize}
\item passes in the factory prop, which is a factoryfactory to be precise;
\item sendMessage is bound in the closure and sends the message in the same way as how Node codegen does, by convention
\item (if not already mentioned in the node section) send is asynchronous
\end{itemize}

\subsection{Handling Message Receives}
\begin{itemize}
\item same edge case as the back-end code generation, as message can arrive before handlers are registered
\item use the same two-queue mechanism to keep track of `what comes first' -- i.e. if message comes first, then put in message queue, so handler can pop off when registered; vice-versa, if handler is registered first, put in handler queue so message can be processed when that is received
\item importantly -- only one queue is required, as server-centric topologies only have the client talk to the server
\end{itemize}

\subsection{Handling Termination}
\begin{itemize}
\item close the websocket with a clean exit code
\end{itemize}