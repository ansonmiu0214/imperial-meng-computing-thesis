\section{Approach}
\label{section:reactapproach}

We motivate our approach from \cite{MVU2020} by extending
their work on \textit{multiple model types} motivated by the
\textit{Model-View-Update} architecture (MVU),
introduced in \cref{section:bgrelated}.
The concept of model types express type dependencies between these
components: a \emph{model type} uniquely defines a \textit{view function},
set of \textit{messages} and \textit{update function} -- rather than
producing a new model, the update function defines valid transitions to
other model types.

We leverage the correspondence between model types 
and states in the EFSM:
each state in the EFSM is a model type, the set of messages represent
the possible channel actions available at that state,
and the update function defines which successor state to transition to,
given the supported channel actions at this state.

We implement model types for the EFSM on top of the 
\emph{React.js} (React) framework developed by Facebook \cite{React}.
React is widely used in industry to create scalable single-page
web applications, so this makes our workflow beneficial in an
industrial context. 
The framework defines a way for data to flow
between UI elements, and empowers the UI to subscribe and
``react'' to data changes.
We aim to implement similar behaviour with respect to the EFSM:
the UI should react to EFSM state transitions,
so we can \textbf{statically} ensure that the
channel actions ``present'' on the browser at any given time
are those permitted by the current EFSM state.
We introduce the framework in \cref{subsection:react}.

When executing \fancyname{SessionTS} to generate code
for the \trole{Client} endpoint specifying the
\texttt{browser} target, the developer obtains the following 
groups of files:

\begin{itemize}

\item 
\textbf{S[40-43].tsx\footnote{
The \filename{.tsx} file extension allows
for embedding \textit{JSX} \cite{JSX} elements inside the file.
JSX is a XML-like syntax extension to JavaScript
for elements and components.
}:}
Developer APIs for implementing EFSM states 
(\cref{section:reactefsm});

\begin{itemize}
\item
\textbf{EFSM.ts}, \textbf{Message.ts}, \textbf{Roles.ts:}
Utility types for EFSM encoding;
\end{itemize}

\item 
\textbf{Client.tsx:} 
Session runtime for executing the EFSM 
(\cref{section:reactruntime});

\begin{itemize}
\item
\textbf{Session.ts}, \textbf{Types.ts:}
Utility types for session runtime.
\end{itemize}

\end{itemize}

\subsection{The React Framework}
\label{subsection:react}

We introduce the key features of the framework
through illustrating a web-based counter in \cref{lst:counter}.
The browser shows a counter (initialised to zero) 
and an ``Increment'' button:
when the user clicks on the ``Increment'' button,
the count is incremented and the UI shows the updated count.

\begin{figure}[!h]
\begin{lstlisting}[language=javascript,tabsize=2]
type Props = { count: number };
class Count extends React.Component<Props>{
	render() { (*@\label{line:childrender}@*)
		return <strong>{this.props.count}</strong>; (*@\label{line:childprops}@*)
	}
}

type State = { count: number };
class App extends React.Component<{}, State>{
	constructor(props: {}) {
		super(props);
		this.state = { count: 0 }; (*@\label{line:parentstate}@*)
	}
	
	increment() { this.setState({ count: this.state.count + 1 }); (*@\label{line:parentsetstate}@*)
	
	render() { (*@\label{line:parentrender}@*)
		return (<div>
			<button onClick={this.increment.bind(this)}>
				Increment
			</button>
			<Count count={this.state.count} /> (*@\label{line:childcomponent}@*)
		</div>);	
	}
}
\end{lstlisting}
\captionof{lstlisting}{Simple Counter in React}
\label{lst:counter}
\end{figure}

\paragraph{Components}
A \textit{component} is a reusable UI element which
contains its own mark-up and logic.
Components implement a \texttt{render()} method which returns
a React element, the smallest building blocks of a React application.
This is analogue to the \textit{view} function in the MVU architecture.
React uses the \textit{JSX} syntax extension \cite{JSX}
to interpolate TypeScript logic 
(enclosed in curly braces)
within HTML mark-up: 
in \cref{line:childrender}, the \texttt{Count} component
evaluates the TypeScript expression 
\lstonelinejs{this.props.count} and renders it in bold on the web page.

Components can render other components, which give rise to
a tree of UI elements. \cref{line:parentrender} shows that our
\texttt{Count} component is rendered by 
the \texttt{App} component.

\paragraph{Uni-directional Data Flow}
User-defined components derive from the abstract base class
\texttt{React.Component<P, S>},
which is an abstract base class with generic type parameters
\texttt{<P, S>} for \textit{props} (short for properties) and 
\textit{state} respectively.

The \texttt{App} component maintains \texttt{count} in its
state (\cref{line:parentstate}). Clicking on the increment button
updates the state (\cref{line:parentsetstate}), which invokes
a re-render, so the UI ``reacts'' to state change.

Data flows from parent components down to their children,
in the form of props. 
The \texttt{App} component passes the \texttt{count}
from its state to the \texttt{Count} component 
(\cref{line:childcomponent}), which accesses it via
\lstonelinejs{this.props}. 
Because \texttt{App} is re-rendered when the count is incremented,
the \texttt{Count} child component will also be re-rendered
with updated props.

\paragraph{Virtual DOM (VDOM) and Reconciliation}
The \texttt{render()} methods give the developer a declarative API
to specify what should be rendered. React uses a 
\textit{virtual DOM} abstraction, where the tree of React elements
are rendered on the virtual DOM, and React internally
runs a \textit{reconciliation} algorithm to update the browser DOM
accordingly using minimal operations.

For example, the \texttt{<button>} will not be re-rendered
on the browser DOM on every counter increment as it does not
depend on the updated state.
