\section{Approach}
\begin{itemize}
\item leverage existing front-end frameworks that support reactive programming -- identify the parallel ideas that the UI element rendered on the screen needs to be reactive with respect to -- not only user interaction, but also channel actions.
\item for example, UI should react to a receive state in the EFSM, and once the message is received on the channel, transition to a different UI with the supported actions
\item main goal -- statically ensure that whatever IO actions are `present' on the window at any given time must respect the protocol
\end{itemize}

\subsection{The React Framework}
\begin{itemize}
\item use a class-based Counter example to explain the terms: props, state, uni-directional data flow, virtual DOM, reconciliation
\end{itemize}

\paragraph{Note: from PLACES}
Our browser-side session type encodings for browser-side targets build upon the
\emph{React.js} framework, developed by Facebook \cite{React} for the
\textit{Model-View-Controller} (MVC) architecture.
React is widely used in industry to create scalable single-page TypeScript
applications, and we intend for our proposed workflow to be beneficial in an
industrial context.
We introduce the key features of the framework.

\paragraph{Components}
A component is a reusable UI element which
contains its own markup and logic.
Components implement a \texttt{render()} function which returns a React
element, the smallest building blocks of a React application, analogous to the
view function in the MVU architecture.
Components can keep \textit{state}s and the \texttt{render()} function is
invoked upon a change of state.

For example, a simple counter can be implemented as a component,
with its \texttt{count} stored as state.
When rendered, it displays a button which increments \texttt{count}
when clicked and a \texttt{div} that renders the current
\texttt{count}.
If the button is clicked, the \texttt{count} is incremented, which triggers a
re-rendering (since the state has changed), and the updated \texttt{count} is
displayed.

Components can also render other components, which gives rise
to a parent/child relationship between components.
Parents can pass data to children as \textit{props} (short for properties).
Going back to the aforementioned example, the counter component could
render a child component \texttt{<StyledDiv count=\{this.state.count\} />} in
its \texttt{render()} function, propagating the \texttt{count} from its state
to the child.
This enables reusability, and for our use case, gives control to the parent
on what data to pass to its children (e.g. pass the payload of a received
message to a child to render).

\subsection{Model Types Revisited}
\paragraph{Note: from PLACES paper}
For browser-side targets, we extend the approach presented in \cite{MVU2019} on
\textit{multiple model types} motivated by the \textit{Model-View-Update} (MVU)
architecture.
% FEEDBACK: introduce MVU and model type
An MVU application features a \textit{model} encapsulating application
state, a \textit{view function} rendering the state on the Document Object Model (DOM), and an
\textit{update function} handling \textit{messages} produced by the
rendered model to produce a new model.
The concept of model types express type dependencies between these
components: a \emph{model type} uniquely defines a \textit{view function},
set of \textit{messages} and \textit{update function} -- rather than
producing a new model, the update function defines valid transitions to
other model types.
We leverage the correspondence between model types and states in the EFSM:
each state in the EFSM is a model type, the set of messages represent
the possible (IO) actions available at that state,
and the update function defines which successor state to transition to,
given the supported IO actions at this state.