\section{Error Handling}
\label{section:error}

A robust error handling framework is critical
in a distributed setting.
It is naive to assume that endpoint implementations
are entirely free from exceptions.
As Fowler pointed out in \cite{Exceptional},
session type implementations that do not account for failure
are of limited use in distributed programming.

We motivate the need for handling a variety exceptions in
modern web programming in \cref{subsection:errormotivation} and
demonstrate how we implement a structured error handling
framework through extending the generated APIs 
(in \cref{subsection:errorapi}) and 
runtime (in \cref{subsection:errorruntime}).
We highlight the caveats of our error handling framework
when dealing with asynchronous operations 
(in \cref{subsection:errorasync}),
and conclude by explaining the limitations of our framework
in \cref{subsection:errorlimit}.

\subsection{Motivation}
\label{subsection:errormotivation}

Using the \tprotocol{2FA} example from
\cref{lst:2FA}, the database lookup function
might rely on a cloud service, and will throw an exception
if the cloud service cannot be reached.
This may cause \trole{Svr} to crash.
From the perspective of session types, we need to
ensure that \trole{Client} isn't blindly
waiting for the session to continue if \trole{Svr}
threw an exception.

We classify this as an example where \trole{Svr}
terminates the session because of a \textit{logical error}
in its implementation.
Existing work \cite{Exceptional,AffineSessions} 
on error handling for session types address this
through extending the process calculus with an
\textit{explicit cancellation} operator to ensure that
channel resources are closed during exception handling.

However, the context of modern web programming
introduces new possibilities for session cancellation
that are not addressed by existing work.
Browser-side endpoints can disconnect from the session due
to network connectivity issues or simply by closing
the browser. Similarly, the server may also face connectivity
issues. In order to deliver a robust error handling
framework tailored for integrating session types in 
modern web programming, we must also take these variants
of session cancellation into account.

We also need a novel approach for defining cancellation handlers.
The work by Fowler et al. \cite{Exceptional} integrates 
exception handling for session types in a 
\textit{functional language}, where their design of
exception handlers build upon existing work on effect handlers.
Our work targets a non-functional language in TypeScript,
which means any piece of code can be ``effectful'',
so the exception handler API we generate for developers
need to respect this property.
Exception handlers should also be well parameterised
to empower the developer to handle cancellations more effectively
-- for instance, if the developer knows which endpoint
emitted the cancellation, appropriate clean-up operations
(such as reverting database changes or application-specific
logging) can be performed.

\subsection{API Extension}
\label{subsection:errorapi}

We extend the generated APIs to
allow developers to provide a \textbf{session cancellation
handler}. The runtime will invoke this
to handle \textit{local} exceptions (e.g. exception
thrown by application logic) and \textit{global} session
cancellations (e.g. other endpoint disconnecting).

The session cancellation handler is uniformly
parameterised by the \texttt{role} which emitted the cancellation,
and some arbitrary \texttt{reason}. For exceptions thrown
by local application logic, the \texttt{reason} argument
is populated with the thrown exception. Because the 
\lstonelinejs{try} / \lstonelinejs{catch} block is not typed,
we give the \texttt{reason} parameter the dynamic type of
\lstonelinejs{any} to give developers the autonomy for 
handling the exceptions they throw locally.

Consider a binary session protocol specifying interactions
between an \trole{ATMSvr} and \trole{ATMClient}.
We demonstrate how the developer may choose to define
the session cancellation handlers in the context of this protocol.

\subparagraph{Server (\cref{lst:cancelsvr}):}
The \trole{ATMSvr}
may need to handle internal server errors 
(\cref{line:cancelsvrinternal}) as a result of
exceptions thrown by its implementation. 
If the \trole{ATMClient} disconnects prematurely, 
\trole{ATMSvr} can also notify the database 
to revert the transaction (\cref{line:cancelsvrexternal}). 
Note that session cancellation handlers can be 
\textit{asynchronous}, as we appreciate that any required
``clean-up'' operations may need to use asynchronous APIs
provided by databases or other persistent storage solutions.
The cancellation handler is passed into the runtime via
the constructor of the public API.

\begin{figure}[!h]
\begin{lstlisting}[language=javascript]
new ATMSvr(wss, logic, async (role: Roles.All, reason: any) => {
	if (role === Roles.Self) {
		// Handle internal server error	(*@\label{line:cancelsvrinternal}@*)
	} else {
		// Handle client error:
		// revert incomplete transactions from database
		await db.revertTransactions(role); (*@\label{line:cancelsvrexternal}@*)
	}
});
\end{lstlisting}
\captionof{lstlisting}{Example Cancellation Handler for Server Endpoint}
\label{lst:cancelsvr}
\end{figure}

\subparagraph{Client (\cref{lst:cancelclient})}
Session cancellation handlers for \fancyname{ReactMPST}
are also parameterised by the \texttt{role} and \texttt{reason} for
cancellation, but they must return a \texttt{ReactNode}
for the runtime to render upon session cancellation, which
guarantees that no channel actions will be rendered when a session
is cancelled. Here, the developer renders a header depending
on which role emitted the cancellation, and directly shows
the reason in a paragraph. 
The cancellation handler is passed into the runtime via a 
prop defined on the public API.

\begin{figure}[!h]
\begin{lstlisting}[language=javascript]
<ATMClient
	...
	cancellation={(role: Roles.All, reason: any) => (
		<div>
			// Customise heading to show which role cancelled
			{role === Roles.Self
				? <h1>Internal Error</h1>
				: <h1>Server Error</h1>}
			<p>{reason}</p>		// Render reason in paragraph
		</div>
	)}	
/>
\end{lstlisting}
\captionof{lstlisting}{Example Cancellation Handler for Browser Endpoint}
\label{lst:cancelclient}
\end{figure}

\subsection{Runtime Extension}
\label{subsection:errorruntime}

The generated session runtime is extended
with the capability of \textit{emitting} session
cancellation and \textit{handling} session cancellation.
We define session cancellation over WebSocket transport
by invoking the \texttt{close()} method to emit cancellations,
and overriding the \texttt{onclose} event listener to handle
cancellations. 

The WebSocket \texttt{close()} method accepts a 
\textit{close code}, which is analogue to exit codes for
executables. We extend this mechanism to distinguish
between the different variants of session cancellation,
as motivated in \cref{subsection:errormotivation}.
As shown in \cref{fig:cancelenum},
we define unique close codes for different types
of session cancellation under \filename{Cancellation.ts}
for both server-side (\cref{subfig:cancelenumsvr})
and browser-side endpoints (\cref{subfig:cancelenumclient}) 
respectively. 
To respect the convention specified in the WebSocket protocol
RFC \cite{WebSocketRFC} that the 1xxx close codes 
are reserved for internal use,
we define our custom close codes from 4000 onwards.
The enum values are named in a self-explanatory way.

\begin{figure}[!h]
\begin{subfigure}{0.48\textwidth}
\begin{lstlisting}[language=javascript,tabsize=2]
export enum Receive {
	NORMAL = 1000,
	CLIENT_BROWSER_CLOSED = 1001,
	LOGICAL_ERROR = 4001,
};


export enum Emit {
	CLIENT_BROWSER_CLOSED = 4000,
	LOGICAL_ERROR = 4001,
	ROLE_OCCUPIED = 4002,
};
\end{lstlisting}
\caption{Close Codes for \fancyname{NodeMPST}}
\label{subfig:cancelenumsvr}
\end{subfigure}
\hfill
\begin{subfigure}{0.48\textwidth}
\begin{lstlisting}[language=javascript,tabsize=2]
export enum Receive {
	NORMAL = 1000,
	SERVER_DISCONNECT = 1006,
	CLIENT_DISCONNECT = 4000,
	LOGICAL_ERROR = 4001,
	ROLE_OCCUPIED = 4002,
};

export enum Emit {
	NORMAL = 1000,
	LOGICAL_ERROR = 4001,
};
\end{lstlisting}
\caption{Close Codes for \fancyname{ReactMPST}}
\label{subfig:cancelenumclient}
\end{subfigure}
\caption{WebSocket Close Codes for Session Cancellation}
\label{fig:cancelenum}
\end{figure}

The session runtime features two phases:
\textbf{(1)} the \textit{connection} phase, when the server
accepts incoming connections from session participants;
and \textbf{(2)} the \textit{execution phase}, when all
participants execute their EFSM.
Cancellation can occur in both phases, and we outline
how the runtime handles cancellation in each phase.

\subsubsection{Cancellations during \textit{Connection} Phase}
As the global protocol has not began at this phase,
any cancellations that occur here do \textit{not} terminate
the connection phase; the endpoints that have already
joined just simply continue waiting for the remaining participants
to connect.

The server endpoint can emit a \texttt{ROLE_OCCUPIED} close event
to the browser endpoint if the role is already taken.
We know if a role is occupied if the payload of the
incoming connection request message is not one of the roles
that the server endpoint is waiting on.

It is possible for a browser endpoint to have already joined
the session but is still in the connection phase, because the
server endpoint is waiting on other participants. If this browser
endpoint disconnects prematurely (e.g. the user closes
the browser), the server endpoint detects this and adds the role
back to the set of roles it is waiting on. To do this, we keep
a mapping of socket to role identifier; this is populated
when the endpoint sends the connection request message,
so if the endpoint disconnects prematurely, the server knows
which role to wait for again. We show the changes made to 
the connection management logic generated by \fancyname{NodeMPST}
in \cref{lst:cancelconnectsvr}.

\begin{figure}[!h]
\begin{lstlisting}[language=javascript]
// Inside constructor of session runtime for server-side endpoint

const socketToRole = new Map<WebSocket, Roles.Peers>();

// Invoked when socket is closed
const onClose = ({ target: socket }) => {
	// Wait for the role again
	waiting.add(socketToRole.get(socket));
}

// Invoked when server receives connection request.
const onSubscribe = (event) => {
	...
	// Update role-WebSocket mapping.
	roleToSocket[role] = socket;
	socketToRole.set(socket, role);
	waiting.delete(role);
	...
}
\end{lstlisting}
\captionof{lstlisting}{Modified Connection Management in
Server Endpoint with Cancellation}
\label{lst:cancelconnectsvr}
\end{figure}


\subsubsection{Cancellations during \textit{Execution} Phase}
Any cancellation event that occurs during the execution of the
EFSM will terminate the protocol for all roles.
We see the concept of recovery to be an implementation detail
unique to each application, so we do not build recovery
mechanisms in our session cancellation framework.
There are four possible ways for a cancellation event
to be emitted during this phase.

\subparagraph{Server endpoint experiences logical error:}
This is when the developer's implementation for the server
endpoint throws an uncaught exception. 
We wrap the runtime EFSM execution methods in
\lstonelinejs{try}/\lstonelinejs{catch}
blocks, and handle the error with a \texttt{cancel()}
method we generate for the runtime in \fancyname{NodeMPST}.
The \texttt{cancel()} method invokes
the \texttt{close()} method on all browser endpoints,
then executes the user-defined cancellation handler.

Browser endpoints detect this through the \texttt{onclose}
WebSocket event listener. We extend the runtime
generated by \fancyname{ReactMPST} to handle this event
in the corresponding case of a switch statement
defined on the close code, 
as shown in \cref{lst:canceloncloseclient}.

\begin{figure}[!h]
\begin{lstlisting}[language=javascript,tabsize=2]
private onClose({ code, reason }: CloseEvent) {
	switch (code) {
		case Cancellation.Receive.NORMAL: { ... }
		case Cancellation.Receive.SERVER_DISCONNECT: {
			// Server role disconnected
			this.processCancellation(Roles.Server, 'server disconnected')
			return;
		}
		case Cancellation.Receive.CLIENT_DISCONNECT: { ... }
		case Cancellation.Receive.LOGICAL_ERROR: { ... }
		default: { ... }
	}
}
\end{lstlisting}
\captionof{lstlisting}{WebSocket Close Event Listener in
Browser Endpoints}
\label{lst:canceloncloseclient}
\end{figure}

Browser endpoints process the session cancellation
by simply rendering the UI element generated
by the user-supplied session cancellation handler.

\begin{lstlisting}[language=javascript,tabsize=2]
private processCancellation(role: Roles.All, reason?: any) {
	this.setState({
		// Calls the user-specified cancellation handler,
		// which returns a React node to render on DOM
		elem: this.props.cancellation(role, reason);
	});
}
\end{lstlisting}

\subparagraph{Server endpoint disconnects:}
This follows the same logic as the above,
except for the fact that the server does not
\textit{emit} the cancellation through code, since
it simply disconnects.

\subparagraph{Browser endpoint experiences logical error:}
This is when the developer's implementation for the browser
endpoint throws an uncaught exception. 
The runtime for the browser endpoint will catch
the exception and emit the cancellation to the server
using the corresponding close code.

The server endpoint will detect the cancellation event
using an \texttt{onclose} WebSocket event listener
similar to that defined in \cref{lst:canceloncloseclient},
Then, the server propagates the cancellation
to the other endpoints.
We present the propagation implementation
in \cref{lst:cancelpropagate}.
The ``payload'' of the propagated cancellation is also structured
as an object with the role and reason as properties,
so the browser endpoint can parse and handle using
the same framework presented in \cref{lst:canceloncloseclient}.
After propagating the cancellation to other endpoints,
runtime invokes the user-defined session cancellation
handler, then \textit{restarts} the runtime to accept
new connections once again. 
Any exceptions thrown by the developer's session cancellation
handler will be ignored.

\begin{figure}[!h]
\begin{lstlisting}[language=javascript,tabsize=2]
private propagateCancellation(cancelled: Roles.Peers, reason?: any)
{
	// Construct cancellation message as object
	// with role and reason as properties
	const message = Cancellation.toChannel(cancelled, reason);
	
	// Emit cancellation to other roles
	Object.entries(this.roleToSocket)
		.filter(([role, _]) => role !== cancelled)
		.forEach(([_, socket]) => {
			socket.removeAllListeners();
			socket.close(Cancellation.Emit.LOGICAL_ERROR,
				JSON.stringify(message));
		});
	
	try {
		// Execute user-defined cancellation handler
		const doCancel = this.cancellation(cancelled, reason);
		if (doCancel instanceof Promise) {
			doCancel.then(this.restart).catch(this.restart);		
		} else {
			this.restart();
		}
	} catch {
		this.restart();
	}
}
\end{lstlisting}
\captionof{lstlisting}{Propagating Session Cancellation Events in
Server Endpoint}
\label{lst:cancelpropagate}
\end{figure}

\subparagraph{Browser endpoint disconnects:}
% same as above, except the `emit' part is implicit
This follows the same logic as the above,
except for the fact that the browser does not
\textit{emit} the cancellation through code, since
it simply disconnects.

\subsection{Caveats with Asynchronous Operations}
\label{subsection:errorasync}

Implementing session cancellation whilst
preserving support for asynchronous operations introduces
subtle caveats. 

Consider the case where the developer implements
some asynchronous logic for a send state on the server endpoint. 
Whilst the \texttt{Promise} is being resolved, a cancellation
event occurs concurrently. The session will be cancelled, but
the asynchronous logic will \textit{not} be pre-empted; it
will run to completion, and proceed to try perform the send
action on a WebSocket that has already been closed.

Traditionally, this will throw an error. However,
the semantics of cancellation should dominate any asynchronous
operations from the developer, so we need to work around
the lack of pre-emption. We do this by defining a custom
error handler for the WebSocket send action: in short,
if a send action fails due to the session
being cancelled prior, then there is nothing to worry about.

\subsection{Limitations}
\label{subsection:errorlimit}

The expressiveness of our error handling framework is
limited by the expressiveness of the 
\lstonelinejs{try} / \lstonelinejs{catch} mechanism
in TypeScript.
Unlike languages like Java, we cannot enforce the type
of exceptions being thrown, so we have to resort to
typing the \texttt{reason} of the cancellation as 
\lstonelinejs{any} to be compatible with 
exception handling constructs defined by the developer.

We could change the session cancellation handler
to an object with specific handlers for each of the
custom WebSocket close codes we defined, but it
clutters the API which increases the learning
curve for the developer, and also makes it
more difficult to share cancellation handling logic
across different WebSocket close codes.