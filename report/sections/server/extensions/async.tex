\section{Supporting Asynchronous Implementations}
\label{section:async}

Asynchronous APIs are commonplace in modern web programming,
ranging from database operations to third-party API calls.
We motivate the need for supporting asynchronous operations
through an authentication protocol in
\cref{subsection:asyncmotivation},
demonstrate how we support this through extending
the generated APIs and runtime in
\cref{subsection:asyncapi,subsection:asyncruntime}
respectively, 
and highlight the limitations of our approach in
\cref{subsection:asynclimit}.

\subsection{Motivation}
\label{subsection:asyncmotivation}

Consider a basic example of two-factor authentication 
(\tprotocol{2FA}) in \cref{lst:2FA} motivated from \cite{Exceptional}:
\trole{Client} tries to log in to \trole{Svr}, which either
authorises or denies the login attempt. If \trole{Client} is logging 
in from a new device, the \trole{Svr} sends a challenge key
and waits for a response code before proceeding.

\begin{figure}[!h]
\begin{lstlisting}[language=scribble]
global protocol 2FA(role Svr, role Client) {
	Login(string, string) from Client to Svr;
	choice at Svr { (*@\label{line:2fachoice1}@*)
		Authorised() from Svr to Client;
		do Main(Svr, Client);		// Details are not important
	} or {
		Challenge(string)	from Svr to Client;
		Response(number) 	from Client to Svr;
		choice at Svr { (*@\label{line:2fachoice2}@*)
			Authorised() from Svr to Client;
			do Main(Svr, Client);
		} or {
			AccessDenied() from Svr to Client;
		}
	} or {
		AccessDenied() from Svr to Client;	
	}
}
\end{lstlisting}
\captionof{lstlisting}{The \tprotocol{Two Factor Authentication} Protocol}
\label{lst:2FA}
\end{figure}

The developer's implementation on 
\cref{line:2fachoice1,line:2fachoice2} is likely to query
a database for credential verification.
Database operations are implemented using \textit{asynchronous},
\textit{non-blocking} APIs; it is
recommended that file operations do not block the main execution,
so a JavaScript database API may require a \textit{callback}
to propagate the return value when it is available.
A server endpoint implementation of \cref{lst:2FA}
may resemble the following:

\begin{lstlisting}[language=javascript]
new Implementation.S40({
	Login: (username: string, password: string) => {
		db.lookup(username, password, (result) => {
			// returns Implementation.S42 here
		});
	}
});
\end{lstlisting}

Callbacks are commonplace in dealing with concurrency in JavaScript
\cite{CallbackHell}; we will go over other widely-used
concurrency primitives shortly.
Whatever the case may be, asynchronous developer implementations
using our generated APIs will not type-check as the function
returns immediately (hence, has return type \lstonelinejs{void}),
even though the callback function has the expected return type.
The developer will encounter the following:

\begin{lstlisting}[tabsize=2,numbers=none]
Type '(username: string, password: string) => void' is not 
assignable to type '(payload_0: string, payload_1: string) => S42'.
\end{lstlisting}

This is also possible for browser-side implementations:
when the user invokes a UI event, the browser-side logic
may first consult a third-party API for additional information
(which is not part of the communication protocol)
before performing a send action; the \textit{Fetch} API \cite{Fetch}
for making API calls is asynchronous.

In order to make \fancyname{SessionTS} more relevant and compatible
with modern web programming practices, it is important to
support asynchronous developer implementations as part of our
generated APIs and runtime. 

This means we should support
the concurrency primitives built into TypeScript:

\subparagraph{Callbacks}
Callback-based APIs are higher-order functions -- they take functions
as parameters and invoke them when the asynchronous operation is complete.
The \texttt{setTimeout(fn, delay)} function available on the browser
waits for \texttt{delay} milliseconds before invoking \texttt{fn()} --
or more precisely, before queuing the execution of \texttt{fn()}.
As a result, the code listing below prints \lstonelinejs{'Second!'}
before \lstonelinejs{'First!'}, even though \texttt{setTimeout}
was called with 0ms delay.

\begin{lstlisting}[language=javascript,numbers=none]
setTimeout(() => console.log('First!', 0);
console.log('Second!');
\end{lstlisting}

\subparagraph{Promises}
A Promise formalises the ``completion'' callback
and its error handling construct. The developer passes a ``success''
handler and ``error'' handler, conventionally named
``then'' and ``catch'' respectively, to construct a Promise.
This is analogue to the \texttt{Future} interface in Java \cite{JavaFuture}.
The Promise will invoke the correct handler on completion.
The benefits of Promises in our context is that TypeScript allows
us to specify the type of payload that the Promise resolves to.

\begin{lstlisting}[language=javascript,numbers=none]
declare function promise(n: number): Promise<number>
const promise = (n: number) => 
	new Promise((resolve, reject) => {
		if (n < 10)  resolve(n));
		else reject('Number too big!');
	});

declare function handleNumber(n: number) { ... } 
declare function handleError(err: string) { ... }

promise(10).then(handleNumber).catch(handleError)
\end{lstlisting}

\subparagraph{Async/Await}
The \lstonelinejs{async}/\lstonelinejs{await} construct
is syntactic sugar for Promises. An \lstonelinejs{async}
function always returns a Promise, and marking \lstonelinejs{await}
for the identifier of the return value for a function returning a Promise
``unwraps'' the resolved payload. A downside is that error handling
reverts to the \lstonelinejs{try ... catch} construct.

\begin{lstlisting}[language=javascript,numbers=none]
const asyncSum = async (x: number, y: number) => {
	return (await promise(x)) + (await promise(y));
}

asyncSum(1, 2);  // returns Promise<number> that resolves to 3
\end{lstlisting}

\subsection{API Extension}
\label{subsection:asyncapi}

This is rather straightforward.
We construct a generic union type, \texttt{MaybePromise<T>},
to specify that the type parameter might be wrapped in a Promise,
and leverage type inference in conditional types to
define a generic type, \texttt{FromPromise<T>}, to extract
the wrapped type.

\begin{lstlisting}[language=javascript,numbers=none]
type MaybePromise<T> = T | Promise<T>;
type FromPromise<T> = T extends MaybePromise<infer R> ? R : never;
\end{lstlisting}

Now we change the type signatures of our generated APIs
to use \texttt{MaybePromise<T>}.
For example, if the handler of a receive state generated by
\fancyname{NodeMPST} must return \texttt{Implementation.S42},
it is now permitted to return a \texttt{Promise} that resolves
to a value of type texttt{Implementation.S42}.
The same logic applies to other types of EFSM states
in both \fancyname{NodeMPST} and \fancyname{ReactMPST}.

Hence, the asynchronous \trole{Svr} implementation
for the 2FA protocol shown in \cref{lst:new2fa}
now type-checks. Note that, because 
\lstonelinejs{async} functions return \texttt{Promise}s,
we also support this newer piece of syntactic sugar.

\begin{figure}[!h]
\begin{lstlisting}[language=javascript,tabsize=2]
new Implementation.S40({
	Login: (*@\hl{async}@*) (username: string, password: string) => {
		const result = (*@\hl{await}@*) db.lookup(username, password);
		if (result.valid)
			return new Implementation.S42('Authorised', ...);
		else if (result.unauthorised)
			return new Implementation.S42('AccessDenied', ...);
		else
			return new Implementation.S42('Challenge', ..., challenger);
	}
});

const challenger = (key: number) => {
	return new (*@\hl{Promise}@*)((resolve, reject) => {
		keychain.check(key, ok => {
			if (ok)
				resolve(new Implementation.S44('Authorised', ...);
			else 
				resolve(new Implementation.S44('AccessDenied', ...);	
		});
	});
}
\end{lstlisting}
\captionof{lstlisting}{\tprotocol{2FA} \trole{Svr} Implementation using 
Callbacks, \texttt{Promise}s and
\lstonelinejs{async}/\lstonelinejs{await}}
\label{lst:new2fa}
\end{figure}

\subsection{Runtime Extension}
\label{subsection:asyncruntime}

Generally, minimal changes are required for the runtime.
The key observation is that the concrete types for the any
handlers are the same -- they just might be wrapped in a Promise
construct.
To handle potential Promises, we wrap existing logic
as a delayed computation, colloquially known as a \textit{thunk}
\cite{Thunk}.
We show an example in \cref{fig:thunk} of how we
adapt the \texttt{performSend()} function generated by
\fancyname{NodeMPST} to support optional \texttt{Promise}s:

\begin{enumerate}
\item
We wrap the original logic in \cref{subfig:originalthunk} into a
function (\cref{subfig:newthunk} \cref{line:thunk})
to delay the computation.

\item
We distinguish asynchronous implementations
from synchronous ones using the \lstonelinejs{instanceof} operator.

\begin{itemize}
\item
If the implementation is a \texttt{Promise}, we pass
the thunk as the resolve function invoked on completion.

\item
Otherwise, we directly invoke the computation.
\end{itemize}

\end{enumerate}

\begin{figure}[!h]
\begin{subfigure}{\textwidth}
\begin{lstlisting}[language=javascript,tabsize=2]
const [label, payload, succ] = this.handler;
send(label, payload);
next(succ);
\end{lstlisting}
\caption{Original \texttt{performSend()} Implementation}
\label{subfig:originalthunk}
\end{subfigure}
\hfill
\begin{subfigure}{\textwidth}
\begin{lstlisting}[language=javascript,tabsize=2]
const thunk = (label, payload, succ) => { (*@\label{line:thunk}@*)
	send(label, payload);
	next(succ);	
};

if (this.handler instanceof Promise) {
	this.handler.then(thunk);
} else {
	thunk(this.handler);
}
\end{lstlisting}
\caption{New \texttt{performSend()} for Optional \texttt{Promise}s}
\label{subfig:newthunk}
\end{subfigure}
\caption{Adapting Send Operations in \fancyname{NodeMPST}
using a Thunk}
\label{fig:thunk}
\end{figure}

Similar changes apply to other types of states in both
\fancyname{NodeMPST} and \fancyname{ReactMPST}.
For \fancyname{ReactMPST}, channel-related functions in the runtime
return the successor state identifier. Hence, aside from
refactoring the functions to delay computation using thunks,
the functions themselves need to
return \texttt{MaybePromise<State>} instead
to propagate the asynchrony
to the execution of the EFSM -- the transition occurs
\textit{after} the developer's handler finishes execution.

\subsection{Limitations}
\label{subsection:asynclimit}

Supporting asynchronous logic on top of our
existing implementation in \fancyname{ReactMPST} comes at a cost
of losing affine linearity guarantees.
If the send action handler is asynchronous, then the UI event handler
returns before the runtime transitions, so the UI event that can trigger
the send action is still active, which means the user can trigger
a channel linearity violation. This means we need a boolean flag
to keep track of channel usage, as done in \cite{Hybrid2016}.
We illustrate this in \cref{lst:asynclinearcheck},
with the linearity checks highlighted.

\begin{figure}[!h]
\begin{lstlisting}[language=javascript]
(*@\hl{let used = false;}@*)
...
return class extends React.Component {
	render() {
		const props = {
			[eventLabel as string]: (event) => {
				(*@\hl{if (used) return;}@*)
				(*@\hl{used = true;}@*)
				
				const result = handler(event);
				if (result instanceof Promise) {
					result.then(send);
				} else {
					send(result);
				}
			}		
		}
	}
}
\end{lstlisting}
\captionof{lstlisting}{Generating Runtime Linearity Checks for
React Send State Components}
\label{lst:asynclinearcheck}
\end{figure}

Referring back to the \texttt{sendMessage()} method
defined in \cref{subsection:reactruntimesend},
if we decouple the \texttt{send()} 
from the \texttt{advance()} function, we could advance to the successor state
immediately \textit{before} handling the asynchronous return value.
This restores affine channel linearity guarantees, but 
results in the UI being inconsistent with the channel
actions.