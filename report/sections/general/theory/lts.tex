\section{Labelled Transition System (LTS) Semantics}
\label{section:lts}

We define labelled transition system (LTS) semantics
over global types (\cref{subsection:newltsglobal}) 
and local types (\cref{subsection:newltslocal})
for \newtheory,
building upon the work of Deni\'elou and Yoshida in
\cite{characterisation}.
We show the soundness and completeness of projection
with respect to the LTSs through proving the
\textit{trace equivalence} of
a global type and the collection of local types projected
from the global type (\cref{subsection:newtraceeq}).
We then use this result to
conclude that \newtheory provide the same
communication safety guarantees from canonical
MPST theory for well-formed global types, 
namely deadlock freedom (\cref{subsection:newdeadlockfreedom}).

First, we extend the label in the LTS, as shown in 
\cref{fig:newlts}, to distinguish
the \textit{direct} sending (and reception) of a message
from the sending (and reception) of a message
\textit{via} an intermediate routing endpoint.
Labels range over $l, l', \dots$
We highlight and explain the new labels.

\begin{figure}[!h]
\doublespacing
\[
\begin{array}{rlr}
l ::= & & \text{Labels} \\
	\mid & \aout{p}{q}{j} & 
		\text{Direct Send} \\
	\mid & \ain{p}{q}{j} & 
		\text{Direct Receive} \\
	\mid & \via{s}{\aout{p}{q}{j}} & 
		\text{\hl{Routed Send}} \\
	\mid & \via{s}{\ain{p}{q}{j}} & 
		\text{\hl{Routed Receive}} \\
\end{array}
\]
\singlespacing
\captionof{figure}{LTS Labels in \newtheory}
\label{fig:newlts}
\end{figure}

\begin{itemize}
\item \textbf{Routed send:}

The label $\via{s}{\aout{p}{q}{j}}$ represents the
\textit{sending} (performed by $\role{p}$)
of a message labelled $j$ to $\role{q}$ through
the intermediate router $\role{s}$.

\item \textbf{Routed receive:}

The label $\via{s}{\ain{p}{q}{j}}$ represents the
\textit{reception} (initiated by $\role{q}$) 
of a message labelled $j$
send from $\role{p}$ through
the intermediate router $\role{s}$.
\end{itemize}

Labels represent communication actions, so we refer
to $l$ as labels and actions interchangeably,
as is the case in the literature.

Building upon \cite{characterisation},
the \textit{subject} of a label is the role
that initiates the action. 
Intuitively, the actions for routed send and routed
receive are still initiated by the original sender and
recipient respectively;
we extend the definition
of subjects in \cref{def:newsubj} to reflect this.

\begin{definition}[Subject]
The subject of a LTS label, or $\subj{l}$, is defined as:

\doublespacing
\[
\begin{array}{c}
\subj{\via{s}{\aout{p}{q}{j}}} = 
	\subj{\aout{p}{q}{j}} = \mrole{p} \\
\subj{\via{s}{\ain{p}{q}{j}}} = 
	\subj{\ain{p}{q}{j}} = \mrole{q} \\
\end{array}
\]
\singlespacing
\label{def:newsubj}
\end{definition}

\subsection{LTS Semantics over Global Types}
\label{subsection:newltsglobal}

The LTS semantics presented in \cite{characterisation}
models \textit{asynchronous communication},
which is consistent with our proposal.
In order to define LTS over global types for
asynchronous communication, we need to
represent intermediate states (i.e. messages in transit)
within the grammar of global types.

\cite{characterisation} added the construct
{$\gtrans{p}{q}{j}{l_i: G_i}{i \in I}$}
to represent that the message $l_j$ has been
sent by $\role{p}$ but not yet received by $\role{q}$.

We add a similar construct
{$\gtransroute{p}{q}{s}{j}{l_i: G_i}{i \in I}$}
to represent that the message $l_j$ has
been sent from $\role{p}$ to the router $\role{s}$
but not yet routed to $\role{q}$.

Because routed communication is treated differently
from normal send and receive actions, the notion
of asynchrony differs between the two types of communication
too. This definition allows us to extend
the LTS semantics defined in \cite{characterisation}
more naturally.

We define the LTS semantics 
over global types,
defined by $\treduce{G}{G'}{l}$,
in \cref{fig:newglobal}.
We highlight and explain the new rules.

\begin{figure}[!h]

\begin{prooftree}
\AxiomC{}
\RightLabel{\rulename{Gr1}}
\UnaryInfC{$
\treducelong
	{\gcomm{p}{q}{l_i: G_i}{i \in I}}
	{\gtrans{p}{q}{j}{l_i: G_i}{i \in I}}
	{\aout{p}{q}{j}}
$}
\end{prooftree}

\begin{prooftree}
\AxiomC{}
\RightLabel{\rulename{Gr2}}
\UnaryInfC{$
\treducelong
	{\gtrans{p}{q}{j}{l_i: G_i}{i \in I}}
	{G_j}
	{\ain{p}{q}{j}}
$}
\end{prooftree}

\begin{prooftree}
\AxiomC{$
\treduce
	{G[\trec{G} / \trecvar]}
	{G'}
	{l}
$}
\RightLabel{\rulename{Gr3}}
\UnaryInfC{$
\treduce
	{\trec{G}}
	{G'}
	{l}
$}
\end{prooftree}

\begin{prooftree}
\AxiomC{$\forall i \in I. ~ \treduce{G_i}{G'_i}{l}$}
\AxiomC{$\subj{l} \notin \{\mrole{p}, \mrole{q}\}$}
\RightLabel{\rulename{Gr4}}
\BinaryInfC{$
\treducelong
	{\gcomm{p}{q}{l_i: G_i}{i \in I}}
	{\gcomm{p}{q}{l_i: G'_i}{i \in I}}
	{l}
$}
\end{prooftree}

\begin{prooftree}
\AxiomC{$\treduce{G_j}{G'_j}{l}$}
\AxiomC{$\subj{l} \neq \mrole{q}$}
\AxiomC{$\forall i \in I \setminus \{ j \}. ~ G'_i = G_i$}
\RightLabel{\rulename{Gr5}}
\TrinaryInfC{$
\treducelong
	{\gtrans{p}{q}{j}{l_i: G_i}{i \in I}}
	{\gtrans{p}{q}{j}{l_i: G'_i}{i \in I}}
	{l}
$}
\end{prooftree}

\begin{prooftree}
\AxiomC{}
\RightLabel{\hlrulename{Gr6}}
\UnaryInfC{$
\treducelong
	{\groute{p}{q}{s}{l_i: G_i}{i \in I}}
	{\gtransroute{p}{q}{s}{j}{l_i: G_i}{i \in I}}
	{\via{s}{\aout{p}{q}{j}}}
$}
\end{prooftree}

\begin{prooftree}
\AxiomC{}
\RightLabel{\hlrulename{Gr7}}
\UnaryInfC{$
\treducelong
	{\gtransroute{p}{q}{s}{j}{l_i: G_i}{i \in I}}
	{G_j}
	{\via{s}{\ain{p}{q}{j}}}
$}
\end{prooftree}

\begin{prooftree}
\AxiomC{$\forall i \in I. ~ \treduce{G_i}{G'_i}{l}$}
\AxiomC{$\subj{l} \notin \{\mrole{p}, \mrole{q}\}$}
\RightLabel{\hlrulename{Gr8}}
\BinaryInfC{$
\treducelong
	{\groute{p}{q}{s}{l_i: G_i}{i \in I}}
	{\groute{p}{q}{s}{l_i: G'_i}{i \in I}}
	{l}
$}
\end{prooftree}

\begin{prooftree}
\AxiomC{$\treduce{G_j}{G'_j}{l}$}
\AxiomC{$\subj{l} \neq \mrole{q}$}
\AxiomC{$\forall i \in I \setminus \{ j \}. ~ G'_i = G_i$}
\RightLabel{\hlrulename{Gr9}}
\TrinaryInfC{$
\treducelong
	{\gtransroute{p}{q}{s}{j}{l_i: G_i}{i \in I}}
	{\gtransroute{p}{q}{s}{j}{l_i: G'_i}{i \in I}}
	{l}
$}
\end{prooftree}

\captionof{figure}{LTS Semantics over Global Types in \newtheory}
\label{fig:newglobal}
\end{figure}

\begin{itemize}

\item \rulename{Gr6}: 
\hl{TODO}
%The emission of a routed message $j$
%results in the message

\item \rulename{Gr7}:
\hl{TODO}

\item \rulename{Gr8}:
\hl{TODO}

\item \rulename{Gr9}:
\hl{TODO}

\end{itemize}

\subsection{LTS Semantics over Local Types}
\label{subsection:newltslocal}

We define the LTS semantics 
over local types,
defined by $\treduce{T}{T'}{l}$,
in \cref{fig:newlocal}.
We highlight and explain the new rules.

\begin{figure}[!h]

\begin{prooftree}
\AxiomC{}
\RightLabel{\rulename{Lr1}}
\UnaryInfC{$
\treducelong
	{\tsel{q}{l_i: T_i}{i \in I}}
	{T_j}
	{\aout{p}{q}{j}}
$}
\end{prooftree}

\begin{prooftree}
\AxiomC{}
\RightLabel{\rulename{Lr2}}
\UnaryInfC{$
\treducelong
	{\tbra{q}{l_i: T_i}{i \in I}}
	{T_j}
	{\ain{q}{p}{j}}
$}
\end{prooftree}

\begin{prooftree}
\AxiomC{$
\treduce
	{T[\trec{T} / \trecvar]}
	{T'}
	{l}
$}
\RightLabel{\rulename{Lr3}}
\UnaryInfC{$
\treduce
	{\trec{T}}
	{T'}
	{l}
$}
\end{prooftree}

\begin{prooftree}
\AxiomC{}
\RightLabel{\hlrulename{Lr4}}
\UnaryInfC{$
\treducelong
	{\tselproxy{q}{s}{l_i: T_i}{i \in I}}
	{T_j}
	{\via{s}{\aout{p}{q}{j}}}
$}
\end{prooftree}

\begin{prooftree}
\AxiomC{}
\RightLabel{\hlrulename{Lr5}}
\UnaryInfC{$
\treducelong
	{\tbraproxy{q}{s}{l_i: T_i}{i \in I}}
	{T_j}
	{\via{s}{\ain{q}{p}{j}}}
$}
\end{prooftree}

\begin{prooftree}
\AxiomC{}
\RightLabel{\hlrulename{Lr6}}
\UnaryInfC{$
\treducelong
	{\router{p}{q}{l_i: T_i}{i \in I}}
	{\routertrans{p}{q}{j}{l_i: T_i}{i \in I}}
	{\via{s}{\aout{p}{q}{j}}}
$}
\end{prooftree}

\begin{prooftree}
\AxiomC{}
\RightLabel{\hlrulename{Lr7}}
\UnaryInfC{$
\treducelong
	{\routertrans{p}{q}{j}{l_i: T_i}{i \in I}}
	{T_j}
	{\via{s}{\ain{p}{q}{j}}}
$}
\end{prooftree}

\begin{prooftree}
\AxiomC{$\forall i \in I. ~ \treduce{T_i}{T'_i}{l}$}
\AxiomC{$\subj{l} \notin \{\mrole{p}, \mrole{q}\}$}
\RightLabel{\hlrulename{Lr8}}
\BinaryInfC{$
\treducelong
	{\router{p}{q}{l_i: T_i}{i \in I}}
	{\router{p}{q}{l_i: T'_i}{i \in I}}
	{l}
$}
\end{prooftree}

\begin{prooftree}
\AxiomC{$\treduce{T_j}{T'_j}{l}$}
\AxiomC{$\subj{l} \neq \mrole{q}$}
\AxiomC{$\forall i \in I \setminus \{ j \}. ~ T'_i = T_i$}
\RightLabel{\hlrulename{Lr9}}
\TrinaryInfC{$
\treducelong
	{\routertrans{p}{q}{j}{l_i: T_i}{i \in I}}
	{\routertrans{p}{q}{j}{l_i: T'_i}{i \in I}}
	{l}
$}
\end{prooftree}

\begin{prooftree}
\AxiomC{$l = \via{s}{\cdot}$}
\AxiomC{$\subj{l} \neq \mrole{q}$}
\AxiomC{$\forall i \in I. ~ \treduce{T_i}{T'_i}{l}$}
\RightLabel{\hlrulename{Lr10}}
\TrinaryInfC{$
\treducelong
	{\tsel{q}{l_i: T_i}{i \in I}}
	{\tsel{q}{l_i: T'_i}{i \in I}}
	{l}
$}
\end{prooftree}

\begin{prooftree}
\AxiomC{$l = \via{s}{\cdot}$}
\AxiomC{$\subj{l} \neq \mrole{q}$}
\AxiomC{$\forall i \in I. ~ \treduce{T_i}{T'_i}{l}$}
\RightLabel{\hlrulename{Lr11}}
\TrinaryInfC{$
\treducelong
	{\tbra{q}{l_i: T_i}{i \in I}}
	{\tbra{q}{l_i: T'_i}{i \in I}}
	{l}
$}
\end{prooftree}

\captionof{figure}{LTS over Local Types in \newtheory}
\label{fig:newlocal}
\end{figure}

We explain rules \rulename{Lr4} and \rulename{Lr5}
from the perspective of role $\role{p}$.

\begin{itemize}

\item \rulename{Lr4}: 
\hl{TODO}

\item \rulename{Lr5}:
\hl{TODO}

\end{itemize}

We explain rules \rulename{Lr6}, \rulename{Lr7},
\rulename{Lr10} and \rulename{Lr11}
from the perspective of role $\role{s}$.

\begin{itemize}

\item \rulename{Lr6}:
\hl{TODO}

\item \rulename{Lr7}:
\hl{TODO}

\item \rulename{Lr10}:
\hl{TODO}

\item \rulename{Lr11}:
\hl{TODO}

\end{itemize}

We explain the remaining inductively defined rules.

\begin{itemize}

\item \rulename{Lr8}:
\hl{TODO}

\item \rulename{Lr9}:
\hl{TODO}

\end{itemize}
