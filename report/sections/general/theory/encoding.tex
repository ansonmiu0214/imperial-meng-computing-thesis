\section{From Canonical MPST to \newtheory}
\label{section:encoding}

We present an encoding from canonical MPST theory
to \newtheory.
This encoding is \textit{parameterised} by the router role 
(conventionally denoted as $\mrole{s}$);
the intuition is that we encode all communication interactions
to involve $\mrole{s}$.
For example, as motivated in \cref{chap:p2p},
we can encode the \tprotocol{Two Buyer} protocol
into routed multiparty session types with respect to the
\trole{S}eller role as the router.
When we extend \codegen to implement \newtheory in \cref{chap:impl},
the developer can implement the \tprotocol{Two Buyer} protocol
as an interactive web application over WebSocket transport.

We define the encoding in \cref{subsection:encoding}.
More importantly, we prove that the encoded routed communications
preserve the communication safety
properties (\cref{subsection:encodepreservewf}) 
and communication structure (\cref{subsection:encodepreservecomm}) 
of the original 
communication. 
This directly addresses the challenges discussed in
\cref{section:routerchallenges}


\subsection{Router-Parameterised Encoding}
\label{subsection:encoding}

We define the router-parameterised encoding
on global types, local types and LTS labels in
the canonical MPST theory.

We start with global types, 
as presented in \cref{def:encglobal}.
The main rule is \rulename{Enc-G-Comm}:
if the communication did not go through $\role{s}$,
then the encoded communication involves $\role{s}$ as the router.
The remaining rules are self-explanatory.

\begin{definition}[Encoding on Global Types]

\doublespacing
\[
\begin{array}{rclr}
\enc{\tend}{s} &= & \tend & \rulename{Enc-G-End} \\
\enc{\trecvar}{s} &= & 
	\trecvar & \rulename{Enc-G-RecVar} \\
\enc{\trec{G}}{s} &= & 
	\trec{\enc{G}{s}} & \rulename{Enc-G-Rec} \\
\enc{\gcomm{p}{q}{l_i: G_i}{i \in I}}{s} &= & 
	\begin{cases}
	\gcomm{p}{q}{l_i: \enc{G_i}{s}}{i \in I} 
		& \text{if } \mrole{s} \in \{ \mrole{p}, \mrole{q} \} \\
	\groute{p}{q}{s}{l_i: \enc{G_i}{s}}{i \in I} 
		& \text{otherwise} \\	
	\end{cases} & \rulename{Enc-G-Comm} \\
\end{array}
\]
\singlespacing

\label{def:encglobal}
\end{definition}

Readers may observe striking similarities between
the encoding on global types and the definition of a
centroid in routed communication (\cref{def:centroid}).
In fact, the two are the same -- the encoding on global
types simply expresses the centroid predicate as a function.
We arrive at \cref{lem:enccenter} -- the proof is straightforward
by induction on global types.

\begin{lemma}[Encoding Defines Centroid]
Given an encoding of global type $G$ with respect to the
router role $\mrole{s}$,
the router role is the centroid of the encoded communication.

\[
\centroid{\enc{G}{s}}{s}
\]

\label{lem:enccenter}
\end{lemma}

We now define the encoding on local types in \cref{def:enclocal}.
Local types express communication from the perspective of a 
particular role, say $\role{q}$. 
We need to capture this information in the encoding
to accurately encode into the routed communication,
so the encoding on local types is parameterised by
\textit{two} roles.

\begin{definition}[Encoding on Local Types]
The encoding of interpreting local type $T$ (from the
perspective of role $\mrole{q}$) with respect to
the router role $\mrole{s}$, or $\enclocal{T}{q}{s}$, is defined as:

\doublespacing
\[
\begin{array}{rclr}
\enclocal{\tend}{q}{s} &= & \tend & \rulename{Enc-L-End} \\
\enclocal{\trecvar}{q}{s} &= & \trecvar & \rulename{Enc-L-RecVar} \\
\enclocal{\trec{T}}{q}{s} &= & \trec{\enclocal{T}{q}{s}} 
	& \rulename{Enc-L-Rec} \\
\enclocal{\tsel{p}{l_i: T_i}{i\in I}}{q}{s} &= 
	& \begin{cases}
	\tsel{p}{l_i: \enclocal{T_i}{q}{s}}{i\in I}
		& \text{if } \mrole{s} \in \{ \mrole{p}, \mrole{q} \} \\
	\tselproxy{p}{s}{l_i: \enclocal{T_i}{q}{s}}{i\in I}
		& \text{otherwise} \\	
	\end{cases}
	& \rulename{Enc-L-Sel} \\
\enclocal{\tbra{p}{l_i: T_i}{i\in I}}{q}{s} &= 
	& \begin{cases}
	\tbra{p}{l_i: \enclocal{T_i}{q}{s}}{i\in I}
		& \text{if } \mrole{s} \in \{ \mrole{p}, \mrole{q} \} \\
	\tbraproxy{p}{s}{l_i: \enclocal{T_i}{q}{s}}{i\in I}
		& \text{otherwise} \\	
	\end{cases}
	& \rulename{Enc-L-Bra} \\
\end{array}
\]
\singlespacing

\label{def:enclocal}
\end{definition}

Similarly, if the original communication does not involve
the router from a local perspective, the local endpoint
will carry out the communication through the router
$\trole{s}$ as a proxy in the encoded local type.
This is expressed in both \rulename{Enc-L-Sel}
and \rulename{Enc-L-Bra}.

We establish a correspondence between the encodings for
global types and local types.

\[
\forall \mrole{r}, \mrole{s}, G. ~ \left(
\mrole{r} \neq \mrole{s} 
	\Longrightarrow
\proj{\enc{G}{s}}{r} = \enclocal{\proj{G}{r}}{r}{s}
\right)
\]

We formalise this as \cref{lem:enclink}.
The constraint $\mrole{r} \neq \mrole{s}$ is necessary
because we would otherwise lose information on the right-hand
side of the equality:
the projection of $\mrole{s}$ in the original
communication will not contain the routed interactions,
so applying the local type encoding cannot recover this 
information.
We prove \cref{lem:enclink} by induction on the structure of $G$;
the proof is mechanical and is left in \cref{section:proofs}
for the interested reader.

This correspondence lets us prove a more useful 
lemma about how our encoding preserves the equality of projections,
as presented in \cref{lem:encprojeq}.

\begin{lemma}
\[
\forall G_1, G_2, \mrole{r}, \mrole{s}. ~ \left(
(\proj{G_1}{r}) = (\proj{G_2}{r}) \wedge \mrole{r} \neq \mrole{s}
	\Longrightarrow
\proj{\enc{G_1}{s}}{r} = \proj{\enc{G_2}{s}}{r}
\right)
\]

\label{lem:encprojeq}
\end{lemma}

\begin{proof} By consequence from \cref{lem:enclink}.

Take $G_1, G_2, \mrole{r}, \mrole{s}$ arbitrarily. 
Assume $(\proj{G_1}{r}) = (\proj{G_2}{r})$ and $\mrole{r} \neq \mrole{s}$.

We need to show $\proj{\enc{G_1}{s}}{r} = \proj{\enc{G_2}{s}}{r}$,
but by \cref{lem:enclink}, it is sufficient to show
\[
\enclocal{\proj{G_1}{r}}{r}{s} = \enclocal{\proj{G_2}{r}}{r}{s}
\]

Define an ``inner'' function $f(z) = \enclocal{z}{r}{s}$.

By definition, $\forall x, y. (x = y \Longrightarrow f(x) = f(y))$.

We have $(\proj{G_1}{r}) = (\proj{G_2}{r})$ by assumption,
so we can conclude.

\begin{align*}
(\proj{G_1}{r}) = (\proj{G_2}{r}) 
&\Longrightarrow f(\proj{G_1}{r}) = f(\proj{G_2}{r})  \\
&\Longrightarrow
\enclocal{\proj{G_1}{r}}{r}{s} = \enclocal{\proj{G_2}{r}}{r}{s}
\end{align*}

%\[
%(\proj{G_1}{r}) = (\proj{G_2}{r}) 
%\Longrightarrow
%f(\proj{G_1}{r}) = f(\proj{G_2}{r}) 
%\Longrightarrow
%\enclocal{\proj{G_1}{r}}{r}{s} = \enclocal{\proj{G_2}{r}}{r}{s}
%\]

\end{proof}

We also encode LTS actions using the same idea. 
This is presented in \cref{def:enclts}, and is required for
proofs in \cref{subsection:encodepreservecomm}.

\begin{definition}[Encoding on LTS Actions]

\doublespacing
\[
\begin{array}{rclr}
\enc{\aout{p}{q}{j}}{s} &= & \begin{cases}
\aout{p}{q}{j}
	& \text{if } \mrole{s} \in \{ \mrole{p}, \mrole{q} \} \\
\via{s}{\aout{p}{q}{j}}
	& \text{otherwise} \\
\end{cases} & \rulename{Enc-L-Out} \\
\enc{\ain{p}{q}{j}}{s} &= & \begin{cases}
\ain{p}{q}{j}
	& \text{if } \mrole{s} \in \{ \mrole{p}, \mrole{q} \} \\
\via{s}{\ain{p}{q}{j}}
	& \text{otherwise} \\
\end{cases} & \rulename{Enc-L-In} \\
\end{array}
\]
\singlespacing

\label{def:enclts}
\end{definition}

\subsection{Preserving Well-formedness}
\label{subsection:encodepreservewf}

We show that our encoding preserves the well-formedness
of global types in \cref{th:encwf}.
Well-formedness in \newtheory defines a constraint
on the centroid of the routed communication.
As this directly follows from \cref{lem:enccenter},
it is sufficient to prove that our encoding preserves
projection (\cref{lem:encproj}), then arrive at \cref{th:encwf}
by consequence.

\begin{lemma}[Encoding Preserves Projection]
Let $G$ be a global type.
Take arbitrary roles $\mrole{r}, \mrole{s}$.

\[
\proj{G}{r} \text{ exists } 
	\Longrightarrow 
\proj{ \enc{G}{s} }{r} \text{ exists }
\]

\label{lem:encproj}
\end{lemma}

\begin{proof}
By induction on the structure of $G$.

\begin{enumerate}

\item $G = \tend$, $G = \trecvar$

As $\enc{G}{s} = G$, if $\proj{G}{r}$ exists, 
so does $\proj{ \enc{G}{s} }{r}$.

\item $G = \trec{G'}$

By assumption, $\proj{\trec{G'}}{r}$ exists.
Note that $\proj{G'}{r}$ exists regardless of $\mrole{r}$'s
participation in $G'$.

By induction, $\proj{\enc{G'}{s}}{r}$ exists.

To show $\proj{\enc{\trec{G'}}{s}}{r}$ exists,
consider $\mrole{r}$ by case:

\begin{itemize}
\item $\mrole{r} \in \pt{\enc{G'}{s}}$:

\[
\proj{\enc{\trec{G'}}{s}}{r}
= \proj{\trec{\enc{G'}{s}}}{r}
= \trec{\proj{\enc{G'}{s}}{r}}
\]

As $\proj{\enc{G'}{s}}{r}$ exists,
so does $\proj{\enc{\trec{G'}}{s}}{r}$.

\item $\mrole{r} \notin \pt{\enc{G'}{s}}$:

\[
\proj{\enc{\trec{G'}}{s}}{r}
= \proj{\trec{\enc{G'}{s}}}{r}
= \tend
\]

\end{itemize}

\item $G = \gcomm{p}{q}{l_i: G_i}{i \in I}$

To determine $\enc{G}{s}$, consider $\mrole{s}$ by case:

\begin{itemize}

\item $\mrole{s} \in \{ \mrole{p}, \mrole{q} \}$:

Then $\enc{G}{s} = \gcomm{p}{q}{l_i: \enc{G_i}{s}}{i \in I}$.

To show $\proj{\enc{G}{s}}{r}$ exists,
consider $\mrole{r}$ by case:

\begin{itemize}
\item $\mrole{r} = \mrole{p}$:
Then $\proj{G}{p} = \tsel{q}{l_i: \proj{G_i}{p}}{i \in I}$.

By induction, $\proj{\enc{G_i}{s}}{p}$ exists for $i \in I$.

$\proj{\enc{G}{s}}{p} = \tsel{q}{l_i: \proj{\enc{G_i}{s}}{p}}{i \in I}$.

As projections of the encoded continuations exist, 
so does $\proj{\enc{G}{s}}{p}$.

\item $\mrole{r} = \mrole{q}$: Follows similarly from above.

\item $\mrole{r} \notin \{\mrole{p},\mrole{q}\}$:
Then $\proj{G}{r} = \underset{i \in I}{\MERGEOP}\proj{G_i}{r}$,
so the merge exists.

By induction, $\proj{\enc{G_i}{s}}{r}$ exists for $i \in I$.

$\proj{\enc{G}{s}}{r} = 
	\underset{i \in I}{\MERGEOP}\proj{\enc{G_i}{s}}{r}$,
and this merge exists by \cref{lem:encglobalpreservemerge}.

\end{itemize}

\item $\mrole{s} \notin \{ \mrole{p}, \mrole{q} \}$:

Then $\enc{G}{s} = \groute{p}{q}{s}{l_i: \enc{G_i}{s}}{i \in I}$.

To show $\proj{\enc{G}{s}}{r}$ exists,
consider $\mrole{r}$ by case:

\begin{itemize}
\item $\mrole{r} = \mrole{p}$:
Then $\proj{G}{p} = \tsel{q}{l_i: \proj{G_i}{p}}{i \in I}$.

By induction, $\proj{\enc{G_i}{s}}{p}$ exists for $i \in I$.

$\proj{\enc{G}{s}}{p} = 
	\tselproxy{q}{s}{l_i: \proj{\enc{G_i}{s}}{p}}{i \in I}$.

As projections of the encoded continuations exist, 
so does $\proj{\enc{G}{s}}{p}$.

\item $\mrole{r} = \mrole{q}$: Follows similarly from above.
\item $\mrole{r} = \mrole{s}$:
Then $\proj{G}{s} = \underset{i \in I}{\MERGEOP}\proj{G_i}{s}$.

By induction, $\proj{\enc{G_i}{s}}{p}$ exists for $i \in I$.

$\proj{\enc{G}{s}}{s} = 
	\router{p}{q}{l_i: \proj{\enc{G_i}{s}}{s}}{i \in I}$.

As projections of the encoded continuations exist, 
so does $\proj{\enc{G}{s}}{s}$.

\item $\mrole{r} \notin \{\mrole{p},\mrole{q},\mrole{s}\}$:
Then $\proj{G}{r} = \underset{i \in I}{\MERGEOP}\proj{G_i}{r}$,
so the merge exists.

By induction, $\proj{\enc{G_i}{s}}{r}$ exists for $i \in I$.

$\proj{\enc{G}{s}}{r} = 
	\underset{i \in I}{\MERGEOP}\proj{\enc{G_i}{s}}{r}$,
and this merge exists by \cref{lem:encglobalpreservemerge}.

\end{itemize} % participation of r

\end{itemize} % participation of s

\end{enumerate} % structure of G

\end{proof}

\begin{theorem}[Encoding Preserves Well-Formedness]
Let $G$ be a global type, and $\mrole{s}$ be a role.

\[
\wf{G} \Longrightarrow \wfnew{\enc{G}{s}}{s}
\]

\label{th:encwf}
\end{theorem}

\begin{proof}
Direct consequence of \cref{lem:encproj,lem:enccenter,def:newwf}.
\end{proof}

\subsection{Preserving Communication}
\label{subsection:encodepreservecomm}

We present a crucial result that directly addresses 
the pitfalls discussed in \cref{section:routerchallenges} --
namely, that our encoding does not over-serialise the
original communication.
We show this through proving that our encoding preserves
the LTS semantics over global types -- or more precisely,
we can use the encodings over global types and LTS actions
to encode all possible transitions in the LTS for
global types in the canonical MPST theory.

\begin{theorem}[Encoding Preserves Semantics]
Let $G, G'$ be global types
such that $\treduce{G}{G'}{l}$.
Also take arbitrary role $\mrole{s}$.

\[
\treduce{G}{G'}{l}
	\Longrightarrow
\treducelong{\enc{G}{s}}{\enc{G'}{s}}{\enc{l}{s}}
\]

\end{theorem}

\begin{proof}
By induction on the structure of $\treduce{G}{G'}{l}$.

\begin{itemize}

\item \rulename{Gr1},
where $G = \gcomm{p}{q}{l_i: G_i}{i \in I},
G' = \gtrans{p}{q}{j}{l_i: G_i}{i \in I},
l = \aout{p}{q}{j}$.

To show $\treducelong
{\enc{G}{s}}{\enc{G'}{s}}{\enc{l}{s}}$,
consider $\mrole{s}$ by case:

\begin{itemize}

\item $\mrole{s} \in \{ \mrole{p}, \mrole{q} \}$:
Then we have

\begin{align*}
\enc{G}{s} &= \gcomm{p}{q}{l_i: \enc{G_i}{s}}{i \in I} \\
\enc{G'}{s} &= \gtrans{p}{q}{j}{l_i: \enc{G_i}{s}}{i \in I} \\
\enc{l}{s} &= \aout{p}{q}{j} \\
\end{align*}

The encoded transition is possible using \rulename{Gr1}.

\item $\mrole{s} \notin \{ \mrole{p}, \mrole{q} \}$:
Then we have

\begin{align*}
\enc{G}{s} &= \groute{p}{q}{s}{l_i: \enc{G_i}{s}}{i \in I} \\
\enc{G'}{s} &= \gtransroute{p}{q}{s}{j}{l_i: \enc{G_i}{s}}{i \in I} \\
\enc{l}{s} &= \via{s}{\aout{p}{q}{j}} \\
\end{align*}

The encoded transition is possible using \rulename{Gr6}.

\end{itemize}

\item \rulename{Gr2},
where $G = \gtrans{p}{q}{j}{l_i: G_i}{i \in I},
G' = G_j,
l = \ain{p}{q}{j}$.

We know $\enc{G'}{s} = \enc{G_j}{s}$

To show $\treducelong
{\enc{G}{s}}{\enc{G'}{s}}{\enc{l}{s}}$,
consider $\mrole{s}$ by case:

\begin{itemize}

\item $\mrole{s} \in \{ \mrole{p}, \mrole{q} \}$:
Then we have

\begin{align*}
\enc{G}{s} &= \gtrans{p}{q}{j}{l_i: \enc{G_i}{s}}{i \in I} \\
\enc{l}{s} &= \ain{p}{q}{j} \\
\end{align*}

The encoded transition is possible using \rulename{Gr2}.

\item $\mrole{s} \notin \{ \mrole{p}, \mrole{q} \}$:
Then we have

\begin{align*}
\enc{G}{s} &= \gtransroute{p}{q}{s}{j}{l_i: \enc{G_i}{s}}{i \in I} \\
\enc{l}{s} &= \via{s}{\ain{p}{q}{j}} \\
\end{align*}

The encoded transition is possible using \rulename{Gr7}.

\end{itemize}

\item \rulename{Gr3},
where $G = \trec{G''}$.

By hypothesis, $\treduce{G''[\trec{G''} / \trecvar]}{G'}{l}$.

By induction, $
\treducelong
	{\enc{G''[\trec{G''} / \trecvar]}{s}}
	{\enc{G'}{s}}
	{\enc{l}{s}}
$.

By \cref{lem:encsub}, 
$\enc{G''[\trec{G''} / \trecvar]}{s} 
= \enc{G''}{s}\left[ \trec{\enc{G''}{s}} / \trecvar \right]$.

We know $\enc{G}{s} = \enc{\trec{G''}}{s} = \trec{\enc{G''}{s}}$.

The encoded transition is possible using \rulename{Gr3} as shown:

\begin{prooftree}
\AxiomC{$
\treducelong
	{\enc{G''}{s}\left[ \trec{\enc{G''}{s}} / \trecvar \right]}
	{\enc{G'}{s}}
	{\enc{l}{s}}
$}
\RightLabel{\rulename{Gr3}}
\UnaryInfC{$
\treducelong{\trec{\enc{G''}{s}}}{\enc{G'}{s}}{\enc{l}{s}}
$}
\end{prooftree}

\item \rulename{Gr4},
where $G = \gcomm{p}{q}{l_i: G_i}{i \in I},
G' = \gcomm{p}{q}{l_i: G'_i}{i \in I}$.

By hypothesis,
$\forall i \in I. ~ \treduce{G_i}{G'_i}{l}$
and $\subj{l} \notin \{ \mrole{p}, \mrole{q} \}$.

By induction, 
$\forall i \in I. ~ \left(
\treducelong{\enc{G_i}{s}}{\enc{G'_i}{s}}{\enc{l}{s}}
\right)$.

By \cref{def:newsubj}, 
$\subj{\enc{l}{s}} \notin \{ \mrole{p}, \mrole{q} \}$.

To show $\treducelong
{\enc{G}{s}}{\enc{G'}{s}}{\enc{l}{s}}$,
consider $\mrole{s}$ by case:

\begin{itemize}

\item $\mrole{s} \in \{ \mrole{p}, \mrole{q} \}$:
Then we have

\begin{align*}
\enc{G}{s} &= \gcomm{p}{q}{l_i: \enc{G_i}{s}}{i \in I} \\
\enc{G'}{s} &= \gcomm{p}{q}{l_i: \enc{G'_i}{s}}{i \in I} \\
\end{align*}

The encoded transition is possible using \rulename{Gr4}.

\item $\mrole{s} \notin \{ \mrole{p}, \mrole{q} \}$:
Then we have

\begin{align*}
\enc{G}{s} &= \groute{p}{q}{s}{l_i: \enc{G_i}{s}}{i \in I} \\
\enc{G'}{s} &= \groute{p}{q}{s}{l_i: \enc{G'_i}{s}}{i \in I} \\
\end{align*}

The encoded transition is possible using \rulename{Gr8}.

\end{itemize}

\item \rulename{Gr5},
where $G = \gtrans{p}{q}{j}{l_i: G_i}{i \in I},
G' = \gtrans{p}{q}{j}{l_i: G'_i}{i \in I}$.

By hypothesis,
$\treduce{G_j}{G'_j}{l}$,
$\mrole{p'} \neq \subj{l}$,
and $\forall i \in I \setminus \{j\}. ~ G'_i = G_i$.

By induction, 
$\treducelong{\enc{G_j}{s}}{\enc{G'_j}{s}}{\enc{l}{s}}$.

By \cref{def:newsubj}, 
$\subj{\enc{l}{s}} \neq \mrole{q}$.

To show $\treducelong
{\enc{G}{s}}{\enc{G'}{s}}{\enc{l}{s}}$,
consider $\mrole{s}$ by case:

\begin{itemize}

\item $\mrole{s} \in \{ \mrole{p}, \mrole{q} \}$:
Then we have

\begin{align*}
\enc{G}{s} &= \gtrans{p}{q}{j}{l_i: \enc{G_i}{s}}{i \in I} \\
\enc{G'}{s} &= \gtrans{p}{q}{j}{l_i: \enc{G'_i}{s}}{i \in I} \\
\end{align*}

The encoded transition is possible using \rulename{Gr5}.

\item $\mrole{s} \notin \{ \mrole{p}, \mrole{q} \}$:
Then we have

\begin{align*}
\enc{G}{s} &= \gtransroute{p}{q}{s}{j}{l_i: \enc{G_i}{s}}{i \in I} \\
\enc{G'}{s} &= \gtransroute{p}{q}{s}{j}{l_i: \enc{G'_i}{s}}{i \in I} \\
\end{align*}

The encoded transition is possible using \rulename{Gr9}.

\end{itemize}

\end{itemize}

\end{proof}