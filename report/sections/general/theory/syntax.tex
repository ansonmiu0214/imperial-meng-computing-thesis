\section{Syntax for Global and Local Types}
\label{section:syntax}

We extend the syntax for global types
in \cref{subsection:newglobal} and
local types in \cref{subsection:newlocal}
to express routed communication.
The same conventions introduced in \cref{subsection:bgmpst}
apply here: 
we omit message payload types,
and deterministic send and receive actions
are represented by single selection and
single branching constructs respectively.
With these syntax extensions, we appropriately extend
the definition of \textit{projection} and 
\textit{well-formedness} in 
\cref{subsection:newprojection,subsection:newwf}
respectively.

\subsection{Global Types}
\label{subsection:newglobal}

Global types range over $G, G', G_i, \dots$
As motivated by \cref{section:routerchallenges},
we need to handle \textit{routed communication} separately
from direct send and receive actions.
We define the syntax of global types for \newtheory 
in \cref{fig:newsyntaxglobal}, and explain the 
new construct for expressing routed communication.

\begin{figure}[!h]
\doublespacing
\[
\begin{array}{rlr}
G ::= & & \text{Global Types} \\
\mid & \tend & \text{Termination} \\
\mid & \trecvar & \text{Type Variable} \\
\mid & \trec{G} & \text{Recursive Type} \\
\mid & \gcomm{p}{q}{l_i: G_i}{i \in I} & \text{Direct Communication} \\
\mid & \groute{p}{q}{s}{l_i: G_i}{i \in I}
	& \text{\hl{Routed Communication}} \\
\end{array}
\]
\singlespacing
\captionof{figure}{Global Types in \newtheory}
\label{fig:newsyntaxglobal}
\end{figure}

The new construct $\groute{p}{q}{s}{l_i: G_i}{i \in I}$ reads, 
\textit{
``the communication interaction 
from $\role{p}$ to $\role{q}$
is routed by $\role{s}$''
}.
To be more verbose,
$\role{p}$ offers $\role{q}$ a set of choices
$\left\{l_i\right\}_{i \in I}$,
$\role{q}$ sends their choice to
$\role{s}$, so $\role{p}$ receives the
selection $l_i$ made by $\role{q}$ via $\role{s}$,
and the communication system proceeds with continuation $G_i$.
We refer to $\role{s}$ hereafter as the \textit{router}.
$\role{s}$ ranges over the set of roles 
$\role{p},\role{q},\role{r},\dots$,
but we use $\role{s}$ by convention as the server endpoint
is usually the router.
Just as it is assumed that $\role{p} \neq \role{q}$ for
direct communication,
we also assume that
$\role{p} \neq \role{q} \neq \role{s}$ for routed communication.

We inherit the definitions of the participants function, $\pt{G}$,
introduced in \cref{table:pt}; we extend it in \cref{def:newpt}
to formalise that the router 
participates in the routed communication.

\begin{definition}[Participants]
The set of roles involved in the communication
interactions specified by $G$, written $\pt{G}$, is:

\doublespacing
\[
\begin{array}{>{\displaystyle}rc>{\displaystyle}l}

\pt{\groute{p}{q}{s}{l_i: G_i}{i \in I}} & = & 
\left\{ \mrole{p}, \mrole{q}, \mrole{s} \right\} 
\cup \bigcup_{i \in I}\pt{G_i} \\

\end{array}
\]
\singlespacing
\label{def:newpt}
\end{definition}

\subsection{Local Types}
\label{subsection:newlocal}

Local types range over $T, T', T_i, \dots$
Local types are obtained from global types via
\textit{projection}, so we also need new
constructs to express routed communication
from the \textit{local} perspective of individual
endpoints.
We explain the extensions to
projection in \cref{subsection:newprojection}.

We define the syntax of local types for \newtheory 
in \cref{fig:newsyntaxlocal}, and walk through
the new constructs below from the perspective of
some arbitrary role $\role{r}$.

\begin{figure}[!h]
\doublespacing
\[
\begin{array}{rlr}
T ::= &  & \text{Local Types} \\
\mid & \tend & \text{Termination} \\
\mid & \trecvar & \text{Type Variable} \\
\mid & \trec{T} & \text{Recursive Type} \\
\mid & \tsel{p}{l_i: T_i}{i \in I} & \text{Selection} \\
\mid & \tbra{p}{l_i: T_i}{i \in I} & \text{Branching} \\
\mid & \router{p}{q}{l_i: T_i}{i \in I}
& \text{\hl{Routing Communication}} \\
\mid & \tselproxy{p}{q}{l_i: T_i}{i \in I}
& \text{\hl{Routed Selection}} \\
\mid & \tbraproxy{p}{q}{l_i: T_i}{i \in I}
& \text{\hl{Routed Branching}} \\
\end{array}
\]
\singlespacing
\captionof{figure}{Global Types in \newtheory}
\label{fig:newsyntaxlocal}
\end{figure}

\begin{itemize}

\item \textbf{Routed Communication:}

If $\role{r}$ has local type $\router{p}{q}{l_i: T_i}{i \in I}$,
$\role{r}$ is routing the communication between 
$\role{p}$ and $\role{q}$.
$\role{r}$ routes the message $l_i$ sent by $\role{p}$
to the intended recipient $\role{q}$ and proceeds 
with continuation $T_i$.

\item \textbf{Routed Selection:}

If $\role{r}$ has local type $\tselproxy{p}{q}{l_i: T_i}{i \in I}$,
$\role{r}$ is making a selection from the
\textit{intended recipient} $\role{p}$,
but the selection is sent to the intermediate router $\role{q}$.,
instead of to $\role{p}$ directly.
$\role{r}$ sends their internal choice $l_i$ to $\role{q}$
and proceeds with continuation $T_i$.

\item \textbf{Routed Branching:}

If $\role{r}$ has local type $\tbraproxy{p}{q}{l_i: T_i}{i \in I}$,
$\role{r}$ is offering a choice to the
\textit{intended sender} $\role{p}$,
but the message is received from the intermediate router $\role{q}$,
instead of from $\role{p}$ directly.
$\role{r}$ receives the external choice $l_i$ from $\role{q}$
and proceeds with continuation $T_i$.

\end{itemize}

As a general comment, the new local types of routed selection
and routed branching should ``behave'' as normal selection and
receive types with respect to their intended recipient and sender
respectively. We keep track of the router role in the syntax
to allow us to distinguish between routing communications
from normal send and receive interactions.

\subsection{Projection}
\label{subsection:newprojection}

In \cref{def:newprojection},
we extend the projection operator
to be defined
on routed communication.

\begin{definition}[Projection]
The projection of $G$ onto $\role{r}$,
written $\proj{G}{r}$, is extended as:

\doublespacing
\[
\begin{array}{>{\displaystyle}rc>{\displaystyle}l}

\proj{
\left(\groute{p}{q}{s}{l_i: G_i}{i \in I}\right)
}{r} & = & 
\begin{cases}
\tselproxy{q}{s}{l_i: (\proj{G_i}{r})}{i \in I}
	& \text{if} ~ \mrole{r} = \mrole{p} \\
\tbraproxy{p}{s}{l_i: (\proj{G_i}{r})}{i \in I}
	& \text{if} ~ \mrole{r} = \mrole{q} \\
\router{p}{q}{l_i: (\proj{G_i}{r})}{i \in I}
	& \text{if} ~ \mrole{r} = \mrole{s} \\
\underset{i \in I}{\MERGEOP}\proj{G_i}{r}
	& \text{otherwise} \\
\end{cases}
\\

\end{array}
\]
\singlespacing

\label{def:newprojection}
\end{definition}

As shown in the last (4th) case, 
the same concept of \textit{merging}
applies for routed communication:
when projecting a routed communication
onto a non-participant, the projections of
all continuations must be ``compatible'',
namely they can be merged using the
the \textit{merging operator}, $\mergeop$.
As the merging operator is defined on local
types, we extend the merging operator 
(introduced in \cref{fig:merge}) to be defined
on the extended syntax in \cref{def:newmerge}.

\begin{definition}[Merging Operator]
The merging operator $\mergeop$ on local types
is extended as:

\doublespacing
\[
\begin{array}{rl}
\tmerge
{\tselproxy{p}{q}{l_i: T_i}{i \in I}}
{\tselproxy{p}{q}{l_i: T_i}{i \in I}}
	& = \tselproxy{p}{q}{l_i: T_i}{i \in I} \\
	
\tmerge
{\tbraproxy{p}{q}{l_i: T_i}{i \in I}}
{\tbraproxy{p}{q}{l_j: T'_j}{j \in J}}
	& = \tbraproxy{p}{q}{l_k: T''_k}{k \in I \cup J} \\
\text{where} & T''_k = \begin{cases}
T_k & \text{if} ~ k \in I \setminus J \\
T'_k & \text{if} ~ k \in J \setminus I \\
\tmerge{T_k}{T'_k} & \text{if } ~ k \in I \cap J \\
\end{cases} \\
\text{otherwise} & \text{undefined}
\end{array}
\]
\singlespacing

\label{def:newmerge}
\end{definition}

Recall that routed selection and routed
branching behave in the same way as their ``non-routed''
counterparts -- the merging operator reflects this
similarity.

\subsection{Well-formedness}
\label{subsection:newwf}

Recall that well-formedness is a predicate defined 
\textit{solely} on the global type
in canonical MPST theory: a global type $G$ is well-formed
if a projection is defined for all its participants.

\[
\wf{G} \iff 
\forall \mrole{p} \in \pt{G}. ~ \proj{G}{p} \text{ exists}
\]

In \newtheory, we need to express that a global type is
well-formed \textit{with respect to the role $\role{s}$
acting as the router}.

We need to define the characteristics that $\role{s}$
must display in $G$ to prove that it is a
router. 
We formalise this as an \textit{inductively} defined relation,
$\centroid{G}{s}$, which reads \textit{``$\role{s}$ is a
centroid in $G$''}.
The intuition is that $\role{s}$ is at the centre
of all communication interactions.
We define what it means to be a \textit{centroid} in
\cref{def:centroid}.

% characteristics of router as centroid
\begin{definition}[Centroid]
Let $\centroid{G}{s}$ denote that $\mrole{s}$ is the 
centroid of $G$.

\begin{prooftree}
\AxiomC{}
\RightLabel{\rulename{$\centroidop$-End}}
\UnaryInfC{$\centroid{\tend}{s}$}
\end{prooftree} 

\begin{prooftree}
\AxiomC{}
\RightLabel{\rulename{$\centroidop$-RecVar}}
\UnaryInfC{$\centroid{\trecvar}{s}$}
\end{prooftree}

\begin{prooftree}
\AxiomC{$\centroid{G}{s}$}
\RightLabel{\rulename{$\centroidop$-Rec}}
\UnaryInfC{$\centroid{\trec{G}}{s}$}
\end{prooftree}

\begin{prooftree}
\AxiomC{$\mrole{s} \in \left\{ \mrole{p}, \mrole{q} \right\}$}
\AxiomC{$\forall i \in I. ~ \centroid{G_i}{s}$}
\RightLabel{\rulename{$\centroidop$-Comm}}
\BinaryInfC{$\centroid{\gcomm{p}{q}{l_i: G_i}{i \in I}}{s}$}
\end{prooftree}

\begin{prooftree}
\AxiomC{$\mrole{r} = \mrole{s}$}
\AxiomC{$\forall i \in I. ~ \centroid{G_i}{s}$}
\RightLabel{\rulename{$\centroidop$-RouteComm}}
\BinaryInfC{$\centroid{\groute{p}{q}{r}{l_i: G_i}{i \in I}}{s}$}
\end{prooftree}

\label{def:centroid}
\end{definition}

\begin{itemize}
\item 
\rulename{$\centroidop$-Comm}:
For direct communication, $\role{s}$ must
be a participant and a centroid of all continuations.

\item 
\rulename{$\centroidop$-RoutedComm}:
For routed communication, $\role{s}$ must
be the router and be a centroid of all continuations.

\end{itemize}

Now we are in a position to formalise the definition
of well-formedness in \newtheory. We present this in
\cref{def:newwf}.

\begin{definition}[Well-formedness]
Let $\wfnew{G}{s}$ 
denote that the global type $G$
is well-formed with respect to the router $\role{s}$.

\[
\wfnew{G}{s} \iff 
(\forall \role{p} \in \pt{G}. ~ \proj{G}{p} \text{ exists})
\wedge
\centroid{G}{s}
\]

We also assume that the syntax of $G$ is \textit{contractive}, 
i.e. that type variables are guarded.

\label{def:newwf}
\end{definition}