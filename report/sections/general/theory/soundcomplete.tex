\section{LTS Soundness and Completeness with respect to Projection}
\label{section:newtraceeq}

We work towards proving the soundness and completeness
of our LTS semantics with respect to projection.
Our approach is motivated from \cite{characterisation}:

\begin{enumerate}

\item We first extend the LTS semantics to
a collection of local types (hereafter referred to
as a \textit{configuration} to be consistent with
the literature)
in \cref{subsection:newltsconfig};

\item Then, we extend the definition of projection to
obtain the configuration of a global type 
(hereafter referred to as the \textit{projected configuration})
in \cref{subsection:newltsprojection};

\item Finally, we prove the trace equivalence between 
the global type and its projected configuration 
in \cref{subsection:newtraceeq}.

\end{enumerate}

We use the trace equivalence result to prove deadlock freedom
in \cref{subsection:newdeadlockfreedom}.

\subsection{LTS Semantics over Configurations}
\label{subsection:newltsconfig}

Let $\mathcal{P}$ denote the set of participants in
the communication automaton.
Also let $\lty{p}$ denote the local type of a participant
$\pinP$.

A \textit{configuration} describes the
state of the communication automaton with respect to
each participant $\pinP$.
By definition of our LTS semantics, this includes
\textit{intermediate} states, so a configuration
would also need to express the state of messages
in transit.

We inherit the definition from \cite{characterisation},
restated in \cref{def:newconfig}.

\begin{definition}[Configuration]
A configuration $s = (\vec T; ~ \vec w)$ of a system of
local types $\{ \lty{p} \}_{\pinP}$
is defined as a pair of:

\begin{itemize}

\item $\vec T = (\lty{p})_{\pinP}$
is the collection of local types.
$\lty{p}$ describes the communication structure
from the local perspective of participant $\pinP$.

\item $\vec w = (w_{\mroles{p}{q}})_{\mrole{p} \neq \mrole{q} \in \mathcal{P}}$
is the collection of \textit{unbounded buffers}.
The unbounded buffer $w_{\mrole{p}\mrole{q}}$ represents a (FIFO)
queue of messages sent by $\role{p}$ but not yet
received by $\role{q}$.

\end{itemize}

\label{def:newconfig}
\end{definition}

\begin{remark}

The \textit{subtyping} relation defined on
local types (see \cref{subsection:bgbst})
can be extended to configurations:

\begin{prooftree}
\AxiomC{$\vec w = \vec w'$}
\AxiomC{$
\forall \mrole{p} \in \mathcal{P}. ~ 
T_\mrole{p} \subtype T'_\mrole{p}
$}
\BinaryInfC{$
(\vec T; \vec w) \subtype (\vec T'; \vec w')
$}
\end{prooftree}

\end{remark}

We proceed to define the LTS over configurations in 
\cref{def:newltsconfig}, highlighting the extensions
required for \newtheory.

\begin{definition}[LTS Semantics over Configurations]
The LTS semantics over configurations is defined by
the relation $\treduce{s_T}{s'_T}{l}$.

Let $s_T = (\vec T; ~ \vec w)$ and $s'_T = (\vec T'; ~ \vec w')$.
We define the specific transitions on $\vec T$ and $\vec w$
by case analysis on the label $l$.

\begin{itemize}

\item $l = \aout{p}{q}{j}$

Then $\treduce{T_\mrole{p}}{T'_\mrole{p}}{l}$ 
because $\role{p}$
initiates the action, so
$T'_\mrole{p'} = T_{\mrole{p'}}$ 
for all $\mrole{p'} \neq \mrole{p}$.

The message $j$ is in transit from $\role{p}$ to $\role{q}$, 
so $w'_{\mroles{p}{q}} = w_{\mroles{p}{q}} \cdot j$
($j$ is appended to the queue of in-transit messages
sent from $\role{p}$ to $\role{q}$),
and unrelated buffers $w'_{\mroles{p'}{q'}} = w_{\mroles{p}{q}}$ 
are untouched for all $\mroles{p'}{q'} \neq \mroles{p}{q}$.

\item $l = \ain{p}{q}{j}$

Then $\treduce{T_\mrole{q}}{T'_\mrole{q}}{l}$ 
because $\role{q}$
initiates the action, so
$T'_\mrole{p'} = T_{\mrole{p'}}$ 
for all $\mrole{p'} \neq \mrole{q}$.

The message $j$ is no longer in transit
from $\role{p}$ to $\role{q}$ as it is received by $\role{q}$,
so $w_{\mroles{p}{q}} = j \cdot w'_{\mroles{p}{q}}$ 
($j$ is removed from the front of the queue of in-transit
messages sent from $\role{p}$ to $\role{q}$),
and unrelated buffers $w'_{\mroles{p'}{q'}} = w_{\mroles{p}{q}}$ 
are untouched for all $\mroles{p'}{q'} \neq \mroles{p}{q}$.

\item \hl{$l = \via{s}{\aout{p}{q}{j}}$}

Then $\treduce{T_\mrole{p}}{T'_\mrole{p}}{l}$ 
because $\role{p}$
initiates the action.
Because the send action is routed, we also need 
$\treduce{T_\mrole{s}}{T'_\mrole{s}}{l}$.
This means
$T'_\mrole{p'} = T_{\mrole{p'}}$ 
for all $\mrole{p'} \notin \{\mrole{p} , \mrole{s} \}$.

The message $j$ is in transit from $\role{p}$ to $\role{q}$, 
so $w'_{\mroles{p}{q}} = w_{\mroles{p}{q}} \cdot j$
and unrelated buffers $w'_{\mroles{p'}{q'}} = w_{\mroles{p}{q}}$ 
are untouched for all $\mroles{p'}{q'} \neq \mroles{p}{q}$.

\item \hl{$l = \via{s}{\ain{p}{q}{j}}$}

Then $\treduce{T_\mrole{q}}{T'_\mrole{q}}{l}$ 
because $\role{q}$
initiates the action.
Because the receive action is routed, we also need
$\treduce{T_\mrole{s}}{T'_\mrole{s}}{l}$.
This means
$T'_\mrole{p'} = T_{\mrole{p'}}$ 
for all $\mrole{p'} \notin \{\mrole{q} , \mrole{s} \}$.

The message $j$ is no longer in transit
from $\role{p}$ to $\role{q}$ as it is received by $\role{q}$,
so $w_{\mroles{p}{q}} = j \cdot w'_{\mroles{p}{q}}$,
and unrelated buffers $w'_{\mroles{p'}{q'}} = w_{\mroles{p}{q}}$ 
are untouched for all $\mroles{p'}{q'} \neq \mroles{p}{q}$.

\end{itemize}

\label{def:newltsconfig}
\end{definition}

Routed actions are carried out by the router,
so it makes sense for the local type of the router 
to also makes a step.
The semantics of the message buffers for routed actions
are the same as their non-routed counterparts; the
only difference is that these message buffers are ``managed''
by the router, but this is a change of interpretation
which isn't reflected in the theory.

\subsection{Extending Projection for Configurations}
\label{subsection:newltsprojection}

When considering the grammar of global types $G$
extended to include intermediate states,
we can obtain the \textit{projected configuration}
from a global type $G$ with participants $\mathcal{P}$:

\[
\projconf{G} = 
\left(
	\{ \proj{G}{p} \}_{\pinP} ~ ; ~
	\projconf{G}_{\{ \epsilon \}_{\qqinP}}
\right)
\]

The collection of local types is obtained by
projecting $G$ onto each participant $\pinP$.
The contents of the buffers is defined as
$\projconf{G}_{\{ w_{\mroles{q}{q'}} \}_{\qqinP}}$.
We inherit the definitions presented in \cite{characterisation},
and introduce additional rules in
\cref{fig:buffer}.

\begin{figure}[!h]
\doublespacing
\[
\begin{array}{>{\displaystyle}rc>{\displaystyle}l}

\projconf{
\gtransroute{p}{p'}{s}{j}{l_i: G_i}{i \in I}
}_{\{ w_{\mroles{q}{q'}} \}_{\qqinP}} 
	& = & \projconf{G_j}_{
	\{ w_{\mroles{q}{q'}} \}_{\qqinP}
	[w_{\mroles{p}{p'}} \mapsto w_{\mroles{p}{p'}} \cdot j]
	} \\
	
\projconf{
\groute{p}{p'}{s}{l_i: G_i}{i \in I}
}_{\{ w_{\mroles{q}{q'}} \}_{\qqinP}} 
	& = & \projconf{G_1}_{
	\{ w_{\mroles{q}{q'}} \}_{\qqinP}
	} \\
	
\end{array}
\]
\singlespacing

\captionof{figure}{Projection of Buffer Contents from Global Type in
\newtheory}
\label{fig:buffer}
\end{figure}

As explained in \cref{subsection:newltsconfig},
the semantics of the message buffers
for routed actions are the same as their
non-routed counterparts,
so the projected contents of the buffers
for routed communication are
the same as those under non-routed communication.

\subsection{Trace Equivalence}
\label{subsection:newtraceeq}

A sequence of transitions is known as a \textit{trace}.
We want to prove that the set of traces
that can be obtained from reducing a global type
$G$ is the same as those that can be obtained
from reducing its projected configuration $\projconf{G}$.

Our approach is based on \cite{characterisation} --
namely, proving that this is the case for a single
transition (i.e. \textit{step equivalence}) is sufficient,
as we can obtain trace equivalence as a direct consequence.

\begin{lemma}[Step Equivalence]
For all global types $G$ and configurations $s$,
if $\projconf{G} \subtype s$,
then $\treduce{G}{G'}{l} \Longleftrightarrow \treduce{s}{s'}{l}$ 
such that $\projconf{G'} \subtype s'$.

\label{lem:stepeq}
\end{lemma}

\begin{proof}
By induction on the possible transitions in the LTSs
over global types (to prove $\Longrightarrow$,
i.e. \textit{soundness}) 
and configurations (to prove $\Longleftarrow$,
i.e. \textit{completeness}). 

\paragraph{Notation conventions} 
We use the following notation for decomposing configurations
and projected configurations.
\[
\def\arraystretch{1.6}
\begin{array}{rcl}
s &= & \{ T_\mrole{q} \}_{\mrole{q} \in \mathcal{P}},~
	\{ w_{\mroles{q}{q'}} \}_{\qqinP} \\
s' &= & \{ T'_\mrole{q} \}_{\mrole{q} \in \mathcal{P}},~
	\{ w'_{\mroles{q}{q'}} \}_{\qqinP} \\
\projconf{G} &= & \{ \hat{T_\mrole{q}} \}_{\mrole{q} \in \mathcal{P}},~
	\{ \hat{w}_{\mroles{q}{q'}} \}_{\qqinP} \\
\projconf{G'} &= & \{ \hat{T'_\mrole{q}} \}_{\mrole{q} \in \mathcal{P}},~
	\{ \hat{w'}_{\mroles{q}{q'}} \}_{\qqinP} \\
\end{array}
\]

\item \textbf{Soundness}

By induction on the structure of LTS semantics
over global types.

For each transition $\treduce{G}{G'}{l}$, we
take the configuration $s = \projconf{G}$,
derive $\treduce{G}{G'}{l}$ and $\treduce{s}{s'}{l}$
under the respective LTSs,
and show that $s' \subtype \projconf{G'}$.

The proofs for rules \rulename{Gr1-5} are
the same as in \cite{characterisation}.
We focus on the new rules introduced for routing.

\begin{itemize}

\item \rulename{Gr6}, 
where $G = \groute{p}{p'}{s}{l_i: G_i}{i \in I}, 
G' = \gtransroute{p}{p'}{s}{j}{l_i: G_i}{i \in I},
l = \via{s}{\aout{p}{p'}{j}}$

Then $s = \projconf{G}$ where
	
\[
\def\arraystretch{1.6}
\begin{array}{>{\displaystyle}r>{\displaystyle}l}
T_\mrole{p} = & \tselproxy{p'}{s}{l_i: \proj{G_i}{p}}{i \in I} \\
T_\mrole{p'} = & \tbraproxy{p}{s}{l_i: \proj{G_i}{p'}}{i \in I} \\
T_\mrole{s} = & \router{p}{p'}{l_i: \proj{G_i}{s}}{i \in I}\\
T_\mrole{r} = & \underset{i \in I}{\MERGEOP}\proj{G_i}{r} 
	\text{ for } \mrole{r} \notin \{ \mrole{p},\mrole{p'},\mrole{s} \}\\
\{ w_{\mroles{q}{q'}} \}_{\qqinP} = & \projconf{G_1}_{\{\vec\epsilon\}} \\
\end{array}
\]

\paragraph{Global transition:} 
We have 

\[
\def\arraystretch{1.6}
\begin{array}{>{\displaystyle}r>{\displaystyle}l}
\hat{T'_\mrole{p'}} = & \tbraproxy{p}{s}{l_i: \proj{G_i}{p'}}{i \in I} \\
\hat{T'_\mrole{s}} = & \routertrans{p}{p'}{j}{l_i: \proj{G_i}{s}}{i \in I}
\\
\hat{T'_\mrole{r}} = & \proj{G_j}{r} 
	\text{ for } \mrole{r} \notin \{ \mrole{p'},\mrole{s} \}\\
\{ \hat{w'}_{\mroles{q}{q'}} \}_{\qqinP} = & \projconf{G_1}_
	{\{ \vec\epsilon \}[w_{\mroles{p}{p'}} \mapsto w_{\mroles{p}{p'}} \cdot j]} \\
\end{array}
\]

So, $\hat{w'}_{\mroles{q}{q'}} = w_{\mroles{q}{q'}}$ for
$\mroles{q}{q'} \neq \mroles{p}{p'}$ and
$\hat{w'}_{\mroles{p}{p'}} = w_{\mroles{p}{p'}} \cdot j$.

\paragraph{Configuration transition:} 
Take $T'_\mrole{r} = T_\mrole{r}$ 
for $\mrole{r} \notin \{\mrole{p}, \mrole{s}\}$.

By \rulename{Lr4}, $\treduce{T_\mrole{p}}{T'_\mrole{p}}{l}$
where $T'_\mrole{p} = \proj{G_j}{p}$.

By \rulename{Lr6}, $\treduce{T_\mrole{s}}{T'_\mrole{s}}{l}$
where $T'_\mrole{s} = \routertrans{p}{p'}{j}{l_i: \proj{G_i}{s}}{i \in I}$.

Also, ${w'}_{\mroles{q}{q'}} = w_{\mroles{q}{q'}}$ for
$\mroles{q}{q'} \neq \mroles{p}{p'}$ and
${w'}_{\mroles{p}{p'}} = w_{\mroles{p}{p'}} \cdot j$.

\paragraph{Correspondence:}
We have 
${w'}_{\mroles{q}{q'}} = \hat{w}_{\mroles{q}{q'}}$
for $\qqinP$ and
$T'_\mrole{q} = \hat{T_\mrole{q}}$
for $\mrole{q} \in \{ \mrole{p}, \mrole{p'}, \mrole{s} \}$.

For $\mrole{q} \notin \{ \mrole{p}, \mrole{p'}, \mrole{s} \}$,
we have

\[
T'_\mrole{q} = \underset{i \in I}{\MERGEOP}\proj{G_i}{q}
	\subtype \proj{G_j}{q}
	= \hat{T_\mrole{q}}
\]

So, $s' \subtype \projconf{G'}$.

\item \rulename{Gr7}
where $G = \gtransroute{p}{p'}{s}{j}{l_i: G_i}{i \in I}, 
G' = G_j,
l = \via{s}{\ain{p}{p'}{j}}$

Then $s = \projconf{G}$ where
	
\[
\def\arraystretch{1.6}
\begin{array}{>{\displaystyle}r>{\displaystyle}l}
T_\mrole{p'} = & \tbraproxy{p}{s}{l_i: \proj{G_i}{p'}}{i \in I} \\
T_\mrole{s} = & \routertrans{p}{p'}{j}{l_i: \proj{G_i}{s}}{i \in I}\\
T_\mrole{r} = & \proj{G_j}{r} 
	\text{ for } \mrole{r} \notin \{ \mrole{p'},\mrole{s} \}\\
\{ w_{\mroles{q}{q'}} \}_{\qqinP} = & \projconf{G_j}_
	{\{ \vec\epsilon \}[w_{\mroles{p}{p'}} \mapsto w_{\mroles{p}{p'}} \cdot j]} \\
\end{array}
\]

\paragraph{Global transition:} 
We have 

\[
\def\arraystretch{1.6}
\begin{array}{>{\displaystyle}r>{\displaystyle}l}
\hat{T'_\mrole{r}} = & \proj{G_j}{r} 
	\text{ for } \mrole{r} \in \mathcal{P} \\
\{ \hat{w'}_{\mroles{q}{q'}} \}_{\qqinP} = & \projconf{G_j}_
	{\{ \vec\epsilon \}} \\
\end{array}
\]

So, $\hat{w'}_{\mroles{q}{q'}} = w_{\mroles{q}{q'}}$ for
$\mroles{q}{q'} \neq \mroles{p}{p'}$ and
$w_{\mroles{p}{p'}} = j \cdot \hat{w'}_{\mroles{p}{p'}}$.

\paragraph{Configuration transition:} 
Take $T'_\mrole{r} = T_\mrole{r}$ 
for $\mrole{r} \notin \{\mrole{p'}, \mrole{s}\}$.

By \rulename{Lr5}, $\treduce{T_\mrole{p}}{T'_\mrole{p}}{l}$
where $T'_\mrole{p} = \proj{G_j}{p}$.

By \rulename{Lr7}, $\treduce{T_\mrole{s}}{T'_\mrole{s}}{l}$
where $T'_\mrole{s} = \proj{G_j}{s}$.

Also, ${w'}_{\mroles{q}{q'}} = w_{\mroles{q}{q'}}$ for
$\mroles{q}{q'} \neq \mroles{p}{p'}$ and
$w_{\mroles{p}{p'}} = j \cdot {w'}_{\mroles{p}{p'}}$.

\paragraph{Correspondence:}
We have 
${w'}_{\mroles{q}{q'}} = \hat{w}_{\mroles{q}{q'}}$
for $\qqinP$ and
$T'_\mrole{q} = \hat{T_\mrole{q}}$
for $\mrole{q} \in \mathcal{P}$.

So, $s' = \projconf{G'}$.

\item \rulename{Gr8}
where $G = \groute{p}{p'}{s}{l_i: G_i}{i \in I}, 
G' = \groute{p}{p'}{s}{l_i: G'_i}{i \in I}$

By hypothesis,
$\forall i \in I. ~ \treduce{G_i}{G'_i}{l}$
and $\subj{l} \notin \{ \mrole{p}, \mrole{p'} \}$.

By induction,
$\forall i \in I. ~ \treduce{\projconf{G_i}}{\projconf{G'_i}}{l}$.

To show that $\treduce{\projconf{G}}{\projconf{G'}}{l}$,
it is sufficient to show that
$\treduce{\proj{G}{q}}{\proj{G'}{q}}{l}$
for $\mrole{q} = \subj{l}$,
since the projections for $\mrole{q'} \neq \subj{l}$ remain the same.

We know $\proj{G}{q} = \underset{i \in I}{\proj{G_i}{q}}$
and $\proj{G'}{q} = \underset{i \in I}{\proj{G'_i}{q}}$.

By induction,
$\treduce
	{\underset{i \in I}{\proj{G_i}{q}}}
	{\underset{i \in I}{\proj{G'_i}{q}}}
	{l}$,
so $\treduce{\proj{G}{q}}{\proj{G'}{q}}{l}$.

\item \rulename{Gr9}
where $G = \gtransroute{p}{p'}{s}{j}{l_i: G_i}{i \in I}, 
G' = \gtransroute{p}{p'}{s}{j}{l_i: G'_i}{i \in I}$

By hypothesis,
$\treduce{G_j}{G'_j}{l}$,
$\mrole{p'} \neq \subj{l}$,
and $\forall i \in I \setminus \{j\}. ~ G'_i = G_i$.

By induction,
$\treduce{\projconf{G_j}}{\projconf{G'_j}}{l}$.

To show that $\treduce{\projconf{G}}{\projconf{G'}}{l}$,
it is sufficient to show that
$\treduce{\proj{G}{q}}{\proj{G'}{q}}{l}$
for $\mrole{q} = \subj{l}$,
since the projections for $\mrole{q'} \neq \subj{l}$ remain the same.

We know $\proj{G}{q} = \proj{G_j}{q}$
and $\proj{G'}{q} = \proj{G'_j}{q}$.

By induction,
$\treduce
	{\proj{G_j}{q}}
	{\proj{G'_j}{q}}
	{l}$,
so $\treduce{\proj{G}{q}}{\proj{G'}{q}}{l}$.

\end{itemize}

\item \textbf{Completeness}

By considering the possible transitions in the LTS
over configurations, which is defined by
case analysis on the possible labels $l$.

For each transition $\treduce{s}{s'}{l}$,
we take the global type $G$ such that $s = \projconf{G}$,
derive $\treduce{s}{s'}{l}$ and $\treduce{G}{G'}{l}$
under the respective LTSs, 
and show that $s' \subtype \projconf{G'}$.

The proofs for $l = \aout{p}{q}{j}$ and $l = \ain{p}{q}{j}$
are the same as in \cite{characterisation}.
We focus on the new labels introduced for routing.

\begin{itemize}

\item $l = \via{s}{\aout{p}{q}{j}}$:

Then $T_\mrole{p} = \tselproxy{q}{s}{l_i: \proj{G_i}{p}}{i \in I}$.

Also, $T_\mrole{s}$ \textit{contains} 
$\router{p}{q}{l_i: \proj{G_i}{s}}{i \in I}$ as subterm. 
We denote this subterm $\tilde{T_\mrole{s}}$.

By definition of projection, $G$ has
$\groute{p}{q}{s}{l_i: G_i}{i \in I}$ as subterm. 
We denote this subterm $\tilde{G}$.

Also by definition of projection, no action in $G$
will involve $\mrole{p}$ before $\tilde{G}$.

\paragraph{Configuration transition:}

By \rulename{Lr4}, 
$\treduce{T_\mrole{p}}{T'_\mrole{p}}{l}$,
where $T'_\mrole{p} = \proj{G_j}{p}$.

By \rulename{Lr6}, 
$\treduce{\tilde{T}_\mrole{s}}{\tilde{T'}_\mrole{s}}{l}$,
where $\tilde{T}'_\mrole{s} = 
	\routertrans{p}{q}{j}{l_i: \proj{G_i}{s}}{i \in I}$.

We get $\treduce{T_\mrole{s}}{T'_\mrole{s}}{l}$
by inversion lemma, as illustrated below.

\begin{prooftree}
\AxiomC{}
\RightLabel{\rulename{Lr6}}
\UnaryInfC{$\treduce{\tilde{T}_\mrole{s}}{\tilde{T'}_\mrole{s}}{l}$}
\UnaryInfC{$\vdots$}
\RightLabel{\rulename{Lr8,9,10,11} as needed}
\UnaryInfC{$\qquad\treducelong{T_\mrole{s}}{T'_\mrole{s}}{l}\qquad$}
\end{prooftree}

\paragraph{Global transition:}

By \rulename{Gr6}, 
$\treduce{\tilde{G}}{\tilde{G}'}{l}$,
where $\tilde{G}' = \gtransroute{p}{q}{s}{j}{l_i: G_i}{i \in I}$.

We get $\treduce{G}{G'}{l}$ by inversion lemma,
as illustrated below.

\begin{prooftree}
\AxiomC{}
\RightLabel{\rulename{Gr6}}
\UnaryInfC{$\treduce{\tilde{G}}{\tilde{G}'}{l}$}
\UnaryInfC{$\vdots$}
\RightLabel{\rulename{Gr4,5,8,9} as needed}
\UnaryInfC{$\qquad\treducelong{G}{G'}{l}\qquad$}
\end{prooftree}

\paragraph{Correspondence:}
Since the projections for 
$\mrole{p'} \notin \{ \mrole{p}, \mrole{s} \}$
are unchanged,
it is sufficient to show that 
$T'_\mrole{p} \subtype (\proj{\tilde{G'}}{p})$ and
$\tilde{T'}_\mrole{s} \subtype (\proj{\tilde{G'}}{s})$.

\begin{align*}
\proj{\tilde{G'}}{p} 
	&= \proj{G_j}{p} 
	= T'_\mrole{p} \\
\proj{\tilde{G'}}{s} 
	&= \routertrans{p}{q}{j}{l_i: \proj{G_i}{s}}{i \in I}
	= \tilde{T'}_\mrole{s}
\end{align*}

\item $l = \via{s}{\ain{p}{q}{j}}$:

Then $T_\mrole{q} = \tbraproxy{p}{s}{l_i: \proj{G_i}{q}}{i \in I}$.

Also, $T_\mrole{s}$ \textit{contains} 
$\routertrans{p}{q}{j}{l_i: \proj{G_i}{s}}{i \in I}$ as subterm. 
We denote this subterm $\tilde{T_\mrole{s}}$.

By definition of projection, $G$ has
$\gtransroute{p}{q}{s}{j}{l_i: G_i}{i \in I}$ as subterm. 
We denote this subterm $\tilde{G}$.

Also by definition of projection, no action in $G$
will involve $\mrole{q}$ before $\tilde{G}$.

\paragraph{Configuration transition:}

By \rulename{Lr5}, 
$\treduce{T_\mrole{q}}{T'_\mrole{q}}{l}$,
where $T'_\mrole{q} = \proj{G_j}{q}$.

By \rulename{Lr7}, 
$\treduce{\tilde{T}_\mrole{s}}{\tilde{T'}_\mrole{s}}{l}$,
where $\tilde{T}'_\mrole{s} = \proj{G_j}{s}$.

We get $\treduce{T_\mrole{s}}{T'_\mrole{s}}{l}$
by inversion lemma, as illustrated below.

\begin{prooftree}
\AxiomC{}
\RightLabel{\rulename{Lr7}}
\UnaryInfC{$\treduce{\tilde{T}_\mrole{s}}{\tilde{T'}_\mrole{s}}{l}$}
\UnaryInfC{$\vdots$}
\RightLabel{\rulename{Lr8,9,10,11} as needed}
\UnaryInfC{$\qquad\treducelong{T_\mrole{s}}{T'_\mrole{s}}{l}\qquad$}
\end{prooftree}

\paragraph{Global transition:}

By \rulename{Gr7}, 
$\treduce{\tilde{G}}{\tilde{G}'}{l}$,
where $\tilde{G}' = G_j$.

We get $\treduce{G}{G'}{l}$ by inversion lemma,
as illustrated below.

\begin{prooftree}
\AxiomC{}
\RightLabel{\rulename{Gr7}}
\UnaryInfC{$\treduce{\tilde{G}}{\tilde{G}'}{l}$}
\UnaryInfC{$\vdots$}
\RightLabel{\rulename{Gr4,5,8,9} as needed}
\UnaryInfC{$\qquad\treducelong{G}{G'}{l}\qquad$}
\end{prooftree}

\paragraph{Correspondence:}
Since the projections for 
$\mrole{p'} \notin \{ \mrole{q}, \mrole{s} \}$
are unchanged,
it is sufficient to show that 
$T'_\mrole{q} \subtype (\proj{\tilde{G'}}{q})$ and
$\tilde{T'}_\mrole{s} \subtype (\proj{\tilde{G'}}{s})$.

\begin{align*}
\proj{\tilde{G'}}{q} 
	&= \proj{G_j}{q} 
	= T'_\mrole{q} \\
\proj{\tilde{G'}}{s} 
	&= \proj{G_j}{s}
	= \tilde{T'}_\mrole{s}
\end{align*}

\end{itemize}

\end{proof}

\begin{theorem}[Trace Equivalence]
Let $G$ be a global type with participants 
$\mathcal{P} = \pt{G}$.
Then $G \approx (\vec T, \vec \epsilon)$.

\label{th:traceeq}
\end{theorem}

\begin{proof}
Direct consequence of \cref{lem:stepeq}.
\end{proof}

\subsection{Deadlock Freedom}
\label{subsection:newdeadlockfreedom}

\begin{lemma}[Preservation of Well-formedness]
Let $G$ be a global type.
Suppose $G$ is well-formed with respect to some router $\mrole{s}$.

\[
\forall G', l. ~
(\treduce{G}{G'}{l}
	\Longrightarrow
\wfnew{G'}{s})
\]

\label{lem:preservewf}
\end{lemma}

\begin{proof}
By induction on the structure of $\treduce{G}{G'}{l}$.

\begin{itemize}
\item \rulename{Gr1}, where \dots

\hl{TODO}

\item \rulename{Gr2}, where \dots

\hl{TODO}

\item \rulename{Gr3}, where \dots

\hl{TODO}

\item \rulename{Gr4}, where \dots

\hl{TODO}

\item \rulename{Gr5}, where \dots

\hl{TODO}

\item \rulename{Gr6}, where \dots

\hl{TODO}

\item \rulename{Gr7}, where \dots

\hl{TODO}

\item \rulename{Gr8}, where \dots

\hl{TODO}

\item \rulename{Gr9}, where \dots

\hl{TODO}
\end{itemize}

\end{proof}

\begin{lemma}[Progress for Well-formed Global Types]
Let $G$ be a global type.
Suppose $G$ is well-formed with respect to some router $\mrole{s}$.

\[
(G = \tend) \vee \exists G', l. ~ (\treduce{G}{G'}{l})
\]

The following is logically equivalent.

\[
(G \neq \tend)
	\Longrightarrow 
\exists G', l. ~ (\treduce{G}{G'}{l})
\]

\label{lem:progresswf}
\end{lemma}

\begin{proof}
By induction on the structure of $G$.

We do not consider $G = \tend$ by assumption.

We also do not consider $G = \trecvar$ as the type variable is not guarded.

\begin{enumerate}

\item $G = \trec{G''}$

\hl{TODO}

\item $G = \gcomm{p}{q}{l_i: G_i}{i \in I}$

\hl{TODO}

\item $G = \groute{p}{q}{r}{l_i: G_i}{i \in I}$

$\wfnew{G}{s}$ holds by assumption, so $\mrole{r} = \mrole{s}$.

\hl{TODO}

\end{enumerate}

\end{proof}

\begin{theorem}[Deadlock Freedom]
Let $G$ be a global type.
Suppose $G$ is well-formed with respect to some router $\mrole{s}$.

\[
G \to^* G'
	\Longrightarrow
(G' = \tend) \vee \exists G'', l. ~ 
	(\treduce{G'}{G''}{l})
\]

\end{theorem}

\begin{proof}
Direct consequence of 
\cref{lem:preservewf,lem:progresswf}.
\end{proof}