\section{Proposal: Server as a Router}
\label{section:routerapproach}

We acknowledge that peer-to-peer browser interactions
could be achieved via different transport abstractions, 
such as the WebRTC framework \cite{WebRTC}. 
However, WebRTC has its own proprietary ``protocol'' 
for establishing connections and exchanging network information,
which isn't formalised in session type theory. This setup phase
generally requires a \textit{signalling server} \cite{WebRTCSignaling},
which brings our problem full circle, as it can be implemented
over WebSocket transport. Extending API generation to also include
the boilerplate for initialising WebRTC connections could be an option,
but would introduce complications in the session runtime and
error handling (e.g. how to handle cancellations during the connection
phase?).
We believe that adopting WebRTC deviates from the focus of project in 
implementing type-safe webservices.

Hence, for the purpose of our work in this part,
we establish that WebSocket transport is an invariant, 
and proceed to build up our approach. 
We start by breaking down the concept of 
``peer-to-peer communication''.
Focusing on \cref{lst:twobuyer} \cref{line:twobuyersplit},
we can say that the interaction satisfies three properties:

\begin{enumerate}
\item 
As far as \trole{A} is concerned, \trole{A} is sending
a message that will be received by \trole{B}.

\item
As far as \trole{B} is concerned, \trole{B} is receiving
a message that was sent by \trole{A}.

\item
The \trole{S}eller is not involved in this interaction.
\end{enumerate}

We proceed to relax point 3: for the \trole{S}eller to be
not involved in the interaction, it suffices to guarantee that
the \trole{S}eller cannot intervene or hijack the communication.
This allows \trole{S}eller to ``oversee'' the communication.

On this basis, we motivate our approach to empower the 
server endpoint to act as a \textit{router}. 
That is to say, if the server endpoint receives a message
of which it isn't the intended recipient for, 
it routes the message to the intended recipient. 
Based on the WebSocket transport invariant,
all browser endpoints must join the session
through the server. This means that the server has information about
all browser endpoints, 
and is capable to perform the routing.

Returning to \cref{lst:twobuyer} \cref{line:twobuyersplit},
\trole{A} sends the \tmsg{split} message through its existing
WebSocket connection (with \trole{S}eller on the other side),
noting that \trole{B} is the intended recipient;
\trole{S}eller receives this through the WebSocket bound to
\trole{A}, and routes the message to \trole{B}, who receives it through
its existing connection with \trole{S}eller.
It is trivial that the three properties\footnote{Here, we use the relaxed
version of point 3 mentioned previously.} listed previously are still
satisfied: 

\begin{enumerate}
\item
\textit{As far as \trole{A} is concerned, \trole{A} is sending
a message that will be received by \trole{B}} -- 
the intended recipient is specified in the message.

\item
\textit{As far as \trole{B} is concerned, \trole{B} is receiving
a message that was sent by \trole{A}} --
the original sender is specified in the message.

\item
\textit{\trole{S}eller cannot intervene or hijack the communication} --
this remains true, as \trole{S}eller simply receives the message
from the WebSocket bound to \trole{A} and sends it through the
WebSocket bound to \trole{B}.

\end{enumerate}