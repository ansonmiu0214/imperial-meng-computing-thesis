\section{Challenges}
\label{section:routerchallenges}

We need to formalise our concept of routing by extending session
type theory, such that, given a communication protocol implementing 
an arbitrary topology, we can encode it into our extended theory and
prove that safety properties (e.g. well-formedness, deadlock freedom)
are preserved, and that communication traces are preserved.
This must be done \textit{very carefully}, as
naive definitions risk over-serialising the original communication,
thereby not preserving all communication traces.

\begin{example}[Naive definition of routing in global types]
Consider the global type
\[
A \to B: \tmsg{M1}. S \to B: \tmsg{M2}. \texttt{end}
\]
By existing LTS semantics over
global types (\cref{subsubsection:lts}), then $SB!\tmsg{M2}$
is a prefix of a valid execution trace.
If we define our routing construct simply by replacing every single
$A \to B: \tmsg{M}$ interaction with $A \to S: \tmsg{M}.
S \to B: \tmsg{M}$, we transform the global type into
\[
A \to S: \tmsg{M1}. S \to B: \tmsg{M1}. S \to B: \tmsg{M2}. \texttt{end}
\]
our naive definition of routing has over-serialised the protocol
since $SB!\tmsg{M2}$ can no longer occur first because \trole{S}
is forced to send \tmsg{M1} first, which it cannot do so until receiving
from \trole{A}.
This implies that we need to handle routing interactions differently
from ``normal'' send/receive actions.
\end{example}

There is existing work \cite{LinearDecomp, BinaryDecomp, BinaryDuality}
that resembles our routing proposal through
\textit{decomposing} multiparty sessions into a collection
of binary sessions,
but these approaches
specify an endpoint to exclusively carry out
routing responsibilities and act as a centralised orchestrator 
for the multiparty session.
Our proposal recognises that, in the context of web applications,
there is a need for server-side endpoints to implement business logic
so enforcing the server to be an exclusive orchestrator makes our work
incompatible with webservice protocols; 
we also argue that the server can still participate in the 
multiparty session and concurrently perform its routing duties
in a way that is transparent to non-server endpoints, and crucially,
does not overserialise the permitted interactions. 
We introduce our extension as \textit{Routed Session Types},
and proceed to prove these claims in \cref{chap:theory}.

Likewise, the implementation of routed session types 
in \fancyname{SessionTS} needs to accurately reflect the theory
and be as transparent to the developer as possible to minimise
the learning curve overhead demanded 
by complicating the generated APIs -- we document our extensions
to \fancyname{SessionTS} in \cref{chap:impl}.