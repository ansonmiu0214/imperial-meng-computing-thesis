\chapter{Motivation: Supporting Peer-to-Peer Interactions}

\section{\fancyname{Two Buyer} Protocol}
\begin{itemize}
\item TwoBuyer example -- highlight the sections that show the `arbitrary topology'
\item Not possible with WebSocket topology
\end{itemize}

\section{Server as a Router}
\begin{itemize}
\item How to make TwoBuyer example work just with WebSockets
\item Quickly highlight WebRtC as an option, but it has its own proprietary connection `protocol', and will still need to rely on a centralised server role to set up the peer to peer connection; including the setup in the code generation will introduce unnecessary complications in error handling which deviates from the focus of implementing typesafe web services
\item Essentially `decomposing' multiparty session into binary interactions with respect to the server
\end{itemize}

\section{Challenges}
\begin{itemize}
\item Routed communication is different from `normal' communication
- server cannot overserialise interactions
\item Existing work in decomposing multiparty sessions (see LinearDecompScala paper and the binary one it references) assume the `server' takes on the pure role as a router -- our work argues that the server can still participate in the multiparty session and concurrently route messages between clients in a way that does not overserialise the interactions and allow the routing to be transparent to other roles.
\item theory side -- need to prove that it is indeed possible for the server to carry out its own interactions and perform routing in a transparent way, as if the messages were directly peer to peer
\item implementation side -- changes need to reflect the theory and be as transparent to the developer as possible
\end{itemize}