\chapter{\newtheory: A Theory of
Routed Multiparty Session Types}
\label{chap:theory}

In this chapter, we present
\newtheory, an extension of the
canonical multiparty session type (MPST)
theory with a message routing mechanism.
We build upon the variant of MPST theory introduced
in \cite{LessIsMore}
(which we have also discussed in \cref{subsection:bgmpst}), 
and extend it
with constructs for communication actions
that are routed through a participant.

\newtheory provide the theoretical basis
for allowing developers to implement
protocols with browser-to-browser interactions
(such as the \tprotocol{Two Buyer} protocol
introduced in \cref{section:twobuyer})
over a server-centric network topology
which uses WebSocket transport.
We introduce the extended syntax for types in 
\cref{section:syntax}, and give semantics of our
syntax extensions in \cref{section:lts}.
We define a router-parameterised encoding
for MPST into \newtheory in \cref{section:encoding}
and prove the preservation of well-formedness
and communication.
We proceed to extend \codegen to implement \newtheory
in \cref{chap:impl}.

\section{Syntax}
\label{section:syntax}

We extend the syntax for global types
in \cref{subsection:newglobal} and
local types in \cref{subsection:newlocal}.
The same conventions introduced in \cref{subsection:bgmpst}
apply here: 
we omit message payload types,
and deterministic send and receive actions
are represented by single selection and
single branching constructs respectively.
With these syntax extensions, we appropriately extend
the definition of \textit{projection} and 
\textit{well-formedness} in 
\cref{subsection:newprojection,subsection:newwf}
respectively.

\subsection{Global Types}
\label{subsection:newglobal}

As motivated by \cref{section:routerchallenges},
we need to handle \textit{routed communication} separately
from normal send and receive actions.
We define the syntax of global types for \newtheory 
in \cref{fig:newsyntaxglobal}, and explain the 
new construct for expressing routed communication.
Global types range over $G, G', G_i, \dots$

\begin{figure}[!h]
\doublespacing
\[
\begin{array}{rlr}
G ::= & \dots & \text{Global Types} \\
\mid & \groute{p}{q}{s}{l_i: G_i}{i \in I}
& \text{Routed Communication through $\role{s}$} \\
\end{array}
\]
\singlespacing
\captionof{figure}{Global Types in \newtheory}
\label{fig:newsyntaxglobal}
\end{figure}

The new construct $\groute{p}{q}{s}{l_i: G_i}{i \in I}$ reads, 
\textit{
``the communication $\gcomm{p}{q}{l_i: G_i}{i \in I}$
is routed by $\role{s}$''
}.
To be more verbose,
$\role{p}$ offers $\role{q}$ a set of choices
$\left\{l_i\right\}_{i \in I}$,
$\role{q}$ sends their choice to
$\role{s}$, so $\role{p}$ receives the
selection $l_i$ made by $\role{q}$ via $\role{s}$,
and the communication system proceeds with continuation $G_i$.
We refer to $\role{s}$ hereafter as the \textit{router}.
$\role{s}$ ranges over the set of roles 
$\role{p},\role{q},\role{r},\role{s},\dots$,
but we use $\role{s}$ by convention as the server endpoint
is usually the router.
Just as it is assumed that $\role{p} \neq \role{q}$ for
direct communication ($\gcomm{p}{q}{l_i: G_i}{i \in I}$),
we also assume that
$\role{p} \neq \role{q} \neq \role{s}$ for routed communication.

We also extend the definition of the participants function,
$\pt{G}$, in \cref{def:newpt}, to formalise that the router 
participates in the routed communication.

\begin{definition}[Participants]
The set of roles involved in the communication
interactions specified by $G$, written $\pt{G}$, is:

\doublespacing
\[
\begin{array}{>{\displaystyle}rc>{\displaystyle}l}

\pt{\dots} & = & \dots \\

\pt{\groute{p}{q}{s}{l_i: G_i}{i \in I}} & = & 
\left\{ \mrole{p}, \mrole{q}, \mrole{s} \right\} 
\cup \bigcup_{i \in I}\pt{G_i} \\

\end{array}
\]
\singlespacing
\label{def:newpt}
\end{definition}

\subsection{Local Types}
\label{subsection:newlocal}

Local types are obtained from global types via
\textit{projection}, so we also need new
constructs to express routed communication
from the \textit{local} perspective of individual
endpoints.
We explain the extensions to
projection in \cref{subsection:newprojection}.

We define the syntax of local types for \newtheory 
in \cref{fig:newsyntaxlocal}, and walk through
the new constructs below from the perspective of
some arbitrary role $\role{r}$.
Local types range over $T, T', T_i, \dots$

\begin{figure}[!h]
\doublespacing
\[
\begin{array}{rlr}
T ::= & \dots & \text{Local Types} \\
\mid & \router{p}{q}{l_i: T_i}{i \in I}
& \text{Routing communication from $\role{p}$ to $\role{q}$} \\
\mid & \tselproxy{p}{q}{l_i: T_i}{i \in I}
& \text{Selecting from $\role{p}$ via proxy $\role{q}$} \\
\mid & \tbraproxy{p}{q}{l_i: T_i}{i \in I}
& \text{Receiving from $\role{p}$ via proxy $\role{q}$} \\
\end{array}
\]
\singlespacing
\captionof{figure}{Global Types in \newtheory}
\label{fig:newsyntaxlocal}
\end{figure}

\begin{itemize}

\item \textbf{Routed communication:}

If $\role{r}$ has local type $\router{p}{q}{l_i: T_i}{i \in I}$,
$\role{r}$ is routing the communication between 
$\role{p}$ and $\role{q}$.
$\role{r}$ routes the message $l_i$ sent by $\role{p}$
to the intended recipient $\role{q}$ and proceeds 
with continuation $T_i$.

\item \textbf{Proxy selection:}

If $\role{r}$ has local type $\tselproxy{p}{q}{l_i: T_i}{i \in I}$,
$\role{r}$ is making a selection from the
\textit{intended recipient} $\role{p}$,
but the selection is sent to the intermediate proxy $\role{q}$. 
$\role{r}$ sends their internal choice $l_i$ to $\role{q}$
and proceeds with continuation $T_i$.

\item \textbf{Proxy branching:}

If $\role{r}$ has local type $\tbraproxy{p}{q}{l_i: T_i}{i \in I}$,
$\role{r}$ is offering a choice to the
\textit{intended sender} $\role{p}$,
but the message is received from the intermediate proxy $\role{q}$.
$\role{r}$ receives the external choice $l_i$ from $\role{q}$
and proceeds with continuation $T_i$.

\end{itemize}

As a general comment, the new local types of proxy selection
and proxy branching should ``behave'' as normal selection and
receive types with respect to their intended recipient and sender
respectively. We keep track of the proxy role in the syntax
to allow us to distinguish between routing communications
from normal send and receive interactions.

\subsection{Projection}
\label{subsection:newprojection}

We extend the projection operator to be defined
on routed communication. Given the explanations of the
new routing constructs for both global and local types,
\cref{def:newprojection} should be self-explanatory.

\begin{definition}[Projection]
The projection of $G$ onto $\role{r}$,
written $\proj{G}{r}$, is defined as:

\doublespacing
\[
\begin{array}{>{\displaystyle}rc>{\displaystyle}l}

\proj{
\left(\groute{p}{q}{s}{l_i: G_i}{i \in I}\right)
}{r} & = & 
\begin{cases}
\tselproxy{q}{s}{l_i: (\proj{G_i}{r})}{i \in I}
	& \text{if} ~ \mrole{r} = \mrole{p} \\
\tbraproxy{p}{s}{l_i: (\proj{G_i}{r})}{i \in I}
	& \text{if} ~ \mrole{r} = \mrole{q} \\
\router{p}{q}{l_i: (\proj{G_i}{r})}{i \in I}
	& \text{if} ~ \mrole{r} = \mrole{s} \\
\underset{i \in I}{\MERGEOP}\proj{G_i}{r}
	& \text{otherwise} \\
\end{cases}
\\

\end{array}
\]
\singlespacing

\label{def:newprojection}
\end{definition}

As shown in the last (4th) case, 
the same concept of \textit{merging}
applies for routed communication:
when projecting a routed communication
onto a non-participant, the projections of
all continuations must be ``compatible'',
namely they can be merged using the
the \textit{merging operator}, $\mergeop$.

As the merging operator is defined on local
types, we extend the merging operator to be defined
on the extended syntax in \cref{def:newmerge}.
Recall that proxy selection and proxy
branching behave in the same way as their ``non-proxified''
counterparts -- the merging operator reflects this
similarity.

\begin{definition}[Merging Operator]
The merging operator $\mergeop$ on local types
is defined as:

\doublespacing
\[
\begin{array}{rcl}
\tmerge
{(\tselproxy{p}{q}{l_i: T_i}{i \in I})}
{(\tselproxy{p}{q}{l_i: T_i}{i \in I})}
	& = & \tselproxy{p}{q}{l_i: T_i}{i \in I} \\
	
\tmerge
{(\tbraproxy{p}{q}{l_i: T_i}{i \in I})}
{(\tbraproxy{p}{q}{l_j: T'_j}{j \in J})}
	& = & \tbraproxy{p}{q}{l_k: T''_k}{k \in I \cup J} \\
\text{where} & & T''_k = \begin{cases}
T_k & \text{if} ~ k \in I \setminus J \\
T'_k & \text{if} ~ k \in J \setminus I \\
\tmerge{T_k}{T'_k} & \text{if} ~ k \in I \cap J \\
\end{cases} \\
\end{array}
\]
\singlespacing

\label{def:newmerge}
\end{definition}

\subsection{Well-formedness}
\label{subsection:newwf}

Recall that well-formedness is a predicate defined 
\textit{solely} on the global type
in canonical MPST theory: a global type $G$ is well-formed
if a projection exists for all its participants.

\[
\wf{G} \iff 
\forall \mrole{p} \in \pt{G}. ~ \proj{G}{p} \text{ exists}
\]

In \newtheory, we require that there is \textit{exactly
one} router, say $\role{s}$, in the global type, so
we need to express that a global type to be
well-formed \textit{with respect to the role $\role{s}$
acting as the router}.

We need to define the characteristics that $\role{s}$
must display in $G$ to prove that it is the \textit{only}
router. 
We formalise this as an \textit{inductively} defined relation,
$\centroid{G}{s}$, which reads \textit{``$\role{s}$ is the
centroid in $G$''}.
The intuition is that $\role{s}$ is at the centre
of all communication interactions.
We define what it means to be a \textit{centroid} in
\cref{def:centroid}.

% characteristics of router as centroid
\begin{definition}[Centroid]
Let $\centroid{G}{s}$ denote \dots

\begin{prooftree}
\AxiomC{}
\RightLabel{\rulename{$\centroidop$-End}}
\UnaryInfC{$\centroid{\tend}{s}$}
\end{prooftree} 

\begin{prooftree}
\AxiomC{}
\RightLabel{\rulename{$\centroidop$-RecVar}}
\UnaryInfC{$\centroid{\trecvar}{s}$}
\end{prooftree}

\begin{prooftree}
\AxiomC{$\centroid{G}{s}$}
\RightLabel{\rulename{$\centroidop$-Rec}}
\UnaryInfC{$\centroid{\trec{G}}{s}$}
\end{prooftree}

\begin{prooftree}
\AxiomC{$\mrole{s} \in \left\{ \mrole{p}, \mrole{q} \right\}$}
\AxiomC{$\forall i \in I. ~ \centroid{G_i}{s}$}
\RightLabel{\rulename{$\centroidop$-Comm}}
\BinaryInfC{$\centroid{\gcomm{p}{q}{l_i: G_i}{i \in I}}{s}$}
\end{prooftree}

\begin{prooftree}
\AxiomC{$\mrole{r} = \mrole{s}$}
\AxiomC{$\forall i \in I. ~ \centroid{G_i}{s}$}
\RightLabel{\rulename{$\centroidop$-RouteComm}}
\BinaryInfC{$\centroid{\groute{p}{q}{r}{l_i: G_i}{i \in I}}{s}$}
\end{prooftree}

\label{def:centroid}
\end{definition}

\begin{itemize}
\item 
\rulename{$\centroidop$-End}, 
\rulename{$\centroidop$-RecVar}, 
\rulename{$\centroidop$-Rec}:
Trivial.

\item 
\rulename{$\centroidop$-Comm}:
For normal communication, $\role{s}$ must
be a participant and the centroid of all continuations.

\item 
\rulename{$\centroidop$-RoutedComm}:
For routed communication, $\role{s}$ must
be the router and be the centroid of all continuations.

\end{itemize}

Now we are in a position to formalise the definition
of well-formedness in \newtheory. We present this in
\cref{def:newwf}.

\begin{definition}[Well-formedness]
Let $\wfnew{G}{s}$ 
denote that the global type $G$
is well-formed with respect to the router $\role{s}$.

\[
\wfnew{G}{s} \iff 
(\forall \role{p} \in \pt{G}. ~ \proj{G}{p} \text{ exists})
\wedge
\centroid{G}{s}
\]

\label{def:newwf}
\end{definition}

\section{Labelled Transition System (LTS) Semantics}
\label{section:lts}

We define labelled transition system (LTS) semantics
over global types (\cref{subsection:newltsglobal}) 
and local types (\cref{subsection:newltslocal})
for \newtheory,
building upon the work of Deni\'elou and Yoshida in
\cite{characterisation}.
We show the soundness and completeness of projection
with respect to the LTSs through proving the
\textit{trace equivalence} of
a global type and the collection of local types projected
from the global type (\cref{subsection:newtraceeq}).
We then use this result to
conclude that \newtheory provide the same
communication safety guarantees from canonical
MPST theory for well-formed global types, 
namely deadlock freedom (\cref{subsection:newdeadlockfreedom}).

First, we extend the label in the LTS, as shown in 
\cref{fig:newlts}, to distinguish
the \textit{direct} sending (and reception) of a message
from the sending (and reception) of a message
\textit{via} an intermediate routing endpoint.
Labels range over $l, l', \dots$
We highlight and explain the new labels.

\begin{figure}[!h]
\doublespacing
\[
\begin{array}{rlr}
l ::= & & \text{Labels} \\
	\mid & \aout{p}{q}{j} & 
		\text{Direct Send} \\
	\mid & \ain{p}{q}{j} & 
		\text{Direct Receive} \\
	\mid & \via{s}{\aout{p}{q}{j}} & 
		\text{\hl{Routed Send}} \\
	\mid & \via{s}{\ain{p}{q}{j}} & 
		\text{\hl{Routed Receive}} \\
\end{array}
\]
\singlespacing
\captionof{figure}{LTS Labels in \newtheory}
\label{fig:newlts}
\end{figure}

\begin{itemize}
\item \textbf{Routed send:}

The label $\via{s}{\aout{p}{q}{j}}$ represents the
\textit{sending} (performed by $\role{p}$)
of a message labelled $j$ to $\role{q}$ through
the intermediate router $\role{s}$.

\item \textbf{Routed receive:}

The label $\via{s}{\ain{p}{q}{j}}$ represents the
\textit{reception} (initiated by $\role{q}$) 
of a message labelled $j$
send from $\role{p}$ through
the intermediate router $\role{s}$.
\end{itemize}

Labels represent communication actions, so we refer
to $l$ as labels and actions interchangeably,
as is the case in the literature.

Building upon \cite{characterisation},
the \textit{subject} of a label is the role
that initiates the action. 
Intuitively, the actions for routed send and routed
receive are still initiated by the original sender and
recipient respectively;
we extend the definition
of subjects in \cref{def:newsubj} to reflect this.

\begin{definition}[Subject]
The subject of a LTS label, or $\subj{l}$, is defined as:

\doublespacing
\[
\begin{array}{c}
\subj{\via{s}{\aout{p}{q}{j}}} = 
	\subj{\aout{p}{q}{j}} = \mrole{p} \\
\subj{\via{s}{\ain{p}{q}{j}}} = 
	\subj{\ain{p}{q}{j}} = \mrole{q} \\
\end{array}
\]
\singlespacing
\label{def:newsubj}
\end{definition}

\subsection{LTS Semantics over Global Types}
\label{subsection:newltsglobal}

The LTS semantics presented in \cite{characterisation}
models \textit{asynchronous communication},
which is consistent with our proposal.
In order to define LTS over global types for
asynchronous communication, we need to
represent intermediate states (i.e. messages in transit)
within the grammar of global types.

\cite{characterisation} added the construct
{$\gtrans{p}{q}{j}{l_i: G_i}{i \in I}$}
to represent that the message $l_j$ has been
sent by $\role{p}$ but not yet received by $\role{q}$.

We add a similar construct
{$\gtransroute{p}{q}{s}{j}{l_i: G_i}{i \in I}$}
to represent that the message $l_j$ has
been sent from $\role{p}$ to the router $\role{s}$
but not yet routed to $\role{q}$.

Because routed communication is treated differently
from normal send and receive actions, the notion
of asynchrony differs between the two types of communication
too. This definition allows us to extend
the LTS semantics defined in \cite{characterisation}
more naturally.

We define the LTS semantics 
over global types,
defined by $\treduce{G}{G'}{l}$,
in \cref{fig:newglobal}.
We highlight and explain the new rules.

\begin{figure}[!h]

\begin{prooftree}
\AxiomC{}
\RightLabel{\rulename{Gr1}}
\UnaryInfC{$
\treducelong
	{\gcomm{p}{q}{l_i: G_i}{i \in I}}
	{\gtrans{p}{q}{j}{l_i: G_i}{i \in I}}
	{\aout{p}{q}{j}}
$}
\end{prooftree}

\begin{prooftree}
\AxiomC{}
\RightLabel{\rulename{Gr2}}
\UnaryInfC{$
\treducelong
	{\gtrans{p}{q}{j}{l_i: G_i}{i \in I}}
	{G_j}
	{\ain{p}{q}{j}}
$}
\end{prooftree}

\begin{prooftree}
\AxiomC{$
\treduce
	{G[\trec{G} / \trecvar]}
	{G'}
	{l}
$}
\RightLabel{\rulename{Gr3}}
\UnaryInfC{$
\treduce
	{\trec{G}}
	{G'}
	{l}
$}
\end{prooftree}

\begin{prooftree}
\AxiomC{$\forall i \in I. ~ \treduce{G_i}{G'_i}{l}$}
\AxiomC{$\subj{l} \notin \{\mrole{p}, \mrole{q}\}$}
\RightLabel{\rulename{Gr4}}
\BinaryInfC{$
\treducelong
	{\gcomm{p}{q}{l_i: G_i}{i \in I}}
	{\gcomm{p}{q}{l_i: G'_i}{i \in I}}
	{l}
$}
\end{prooftree}

\begin{prooftree}
\AxiomC{$\treduce{G_j}{G'_j}{l}$}
\AxiomC{$\subj{l} \neq \mrole{q}$}
\AxiomC{$\forall i \in I \setminus \{ j \}. ~ G'_i = G_i$}
\RightLabel{\rulename{Gr5}}
\TrinaryInfC{$
\treducelong
	{\gtrans{p}{q}{j}{l_i: G_i}{i \in I}}
	{\gtrans{p}{q}{j}{l_i: G'_i}{i \in I}}
	{l}
$}
\end{prooftree}

\begin{prooftree}
\AxiomC{}
\RightLabel{\hlrulename{Gr6}}
\UnaryInfC{$
\treducelong
	{\groute{p}{q}{s}{l_i: G_i}{i \in I}}
	{\gtransroute{p}{q}{s}{j}{l_i: G_i}{i \in I}}
	{\via{s}{\aout{p}{q}{j}}}
$}
\end{prooftree}

\begin{prooftree}
\AxiomC{}
\RightLabel{\hlrulename{Gr7}}
\UnaryInfC{$
\treducelong
	{\gtransroute{p}{q}{s}{j}{l_i: G_i}{i \in I}}
	{G_j}
	{\via{s}{\ain{p}{q}{j}}}
$}
\end{prooftree}

\begin{prooftree}
\AxiomC{$\forall i \in I. ~ \treduce{G_i}{G'_i}{l}$}
\AxiomC{$\subj{l} \notin \{\mrole{p}, \mrole{q}\}$}
\RightLabel{\hlrulename{Gr8}}
\BinaryInfC{$
\treducelong
	{\groute{p}{q}{s}{l_i: G_i}{i \in I}}
	{\groute{p}{q}{s}{l_i: G'_i}{i \in I}}
	{l}
$}
\end{prooftree}

\begin{prooftree}
\AxiomC{$\treduce{G_j}{G'_j}{l}$}
\AxiomC{$\subj{l} \neq \mrole{q}$}
\AxiomC{$\forall i \in I \setminus \{ j \}. ~ G'_i = G_i$}
\RightLabel{\hlrulename{Gr9}}
\TrinaryInfC{$
\treducelong
	{\gtransroute{p}{q}{s}{j}{l_i: G_i}{i \in I}}
	{\gtransroute{p}{q}{s}{j}{l_i: G'_i}{i \in I}}
	{l}
$}
\end{prooftree}

\captionof{figure}{LTS Semantics over Global Types in \newtheory}
\label{fig:newglobal}
\end{figure}

\begin{itemize}

\item \rulename{Gr6}: 
\hl{TODO}
%The emission of a routed message $j$
%results in the message

\item \rulename{Gr7}:
\hl{TODO}

\item \rulename{Gr8}:
\hl{TODO}

\item \rulename{Gr9}:
\hl{TODO}

\end{itemize}

\subsection{LTS Semantics over Local Types}
\label{subsection:newltslocal}

We define the LTS semantics 
over local types,
defined by $\treduce{T}{T'}{l}$,
in \cref{fig:newlocal}.
We highlight and explain the new rules.

\begin{figure}[!h]

\begin{prooftree}
\AxiomC{}
\RightLabel{\rulename{Lr1}}
\UnaryInfC{$
\treducelong
	{\tsel{q}{l_i: T_i}{i \in I}}
	{T_j}
	{\aout{p}{q}{j}}
$}
\end{prooftree}

\begin{prooftree}
\AxiomC{}
\RightLabel{\rulename{Lr2}}
\UnaryInfC{$
\treducelong
	{\tbra{q}{l_i: T_i}{i \in I}}
	{T_j}
	{\ain{q}{p}{j}}
$}
\end{prooftree}

\begin{prooftree}
\AxiomC{$
\treduce
	{T[\trec{T} / \trecvar]}
	{T'}
	{l}
$}
\RightLabel{\rulename{Lr3}}
\UnaryInfC{$
\treduce
	{\trec{T}}
	{T'}
	{l}
$}
\end{prooftree}

\begin{prooftree}
\AxiomC{}
\RightLabel{\hlrulename{Lr4}}
\UnaryInfC{$
\treducelong
	{\tselproxy{q}{s}{l_i: T_i}{i \in I}}
	{T_j}
	{\via{s}{\aout{p}{q}{j}}}
$}
\end{prooftree}

\begin{prooftree}
\AxiomC{}
\RightLabel{\hlrulename{Lr5}}
\UnaryInfC{$
\treducelong
	{\tbraproxy{q}{s}{l_i: T_i}{i \in I}}
	{T_j}
	{\via{s}{\ain{q}{p}{j}}}
$}
\end{prooftree}

\begin{prooftree}
\AxiomC{}
\RightLabel{\hlrulename{Lr6}}
\UnaryInfC{$
\treducelong
	{\router{p}{q}{l_i: T_i}{i \in I}}
	{\routertrans{p}{q}{j}{l_i: T_i}{i \in I}}
	{\via{s}{\aout{p}{q}{j}}}
$}
\end{prooftree}

\begin{prooftree}
\AxiomC{}
\RightLabel{\hlrulename{Lr7}}
\UnaryInfC{$
\treducelong
	{\routertrans{p}{q}{j}{l_i: T_i}{i \in I}}
	{T_j}
	{\via{s}{\ain{p}{q}{j}}}
$}
\end{prooftree}

\begin{prooftree}
\AxiomC{$\forall i \in I. ~ \treduce{T_i}{T'_i}{l}$}
\AxiomC{$\subj{l} \notin \{\mrole{p}, \mrole{q}\}$}
\RightLabel{\hlrulename{Lr8}}
\BinaryInfC{$
\treducelong
	{\router{p}{q}{l_i: T_i}{i \in I}}
	{\router{p}{q}{l_i: T'_i}{i \in I}}
	{l}
$}
\end{prooftree}

\begin{prooftree}
\AxiomC{$\treduce{T_j}{T'_j}{l}$}
\AxiomC{$\subj{l} \neq \mrole{q}$}
\AxiomC{$\forall i \in I \setminus \{ j \}. ~ T'_i = T_i$}
\RightLabel{\hlrulename{Lr9}}
\TrinaryInfC{$
\treducelong
	{\routertrans{p}{q}{j}{l_i: T_i}{i \in I}}
	{\routertrans{p}{q}{j}{l_i: T'_i}{i \in I}}
	{l}
$}
\end{prooftree}

\begin{prooftree}
\AxiomC{$l = \via{s}{\cdot}$}
\AxiomC{$\subj{l} \neq \mrole{q}$}
\AxiomC{$\forall i \in I. ~ \treduce{T_i}{T'_i}{l}$}
\RightLabel{\hlrulename{Lr10}}
\TrinaryInfC{$
\treducelong
	{\tsel{q}{l_i: T_i}{i \in I}}
	{\tsel{q}{l_i: T'_i}{i \in I}}
	{l}
$}
\end{prooftree}

\begin{prooftree}
\AxiomC{$l = \via{s}{\cdot}$}
\AxiomC{$\subj{l} \neq \mrole{q}$}
\AxiomC{$\forall i \in I. ~ \treduce{T_i}{T'_i}{l}$}
\RightLabel{\hlrulename{Lr11}}
\TrinaryInfC{$
\treducelong
	{\tbra{q}{l_i: T_i}{i \in I}}
	{\tbra{q}{l_i: T'_i}{i \in I}}
	{l}
$}
\end{prooftree}

\captionof{figure}{LTS over Local Types in \newtheory}
\label{fig:newlocal}
\end{figure}

We explain rules \rulename{Lr4} and \rulename{Lr5}
from the perspective of role $\role{p}$.

\begin{itemize}

\item \rulename{Lr4}: 
\hl{TODO}

\item \rulename{Lr5}:
\hl{TODO}

\end{itemize}

We explain rules \rulename{Lr6}, \rulename{Lr7},
\rulename{Lr10} and \rulename{Lr11}
from the perspective of role $\role{s}$.

\begin{itemize}

\item \rulename{Lr6}:
\hl{TODO}

\item \rulename{Lr7}:
\hl{TODO}

\item \rulename{Lr10}:
\hl{TODO}

\item \rulename{Lr11}:
\hl{TODO}

\end{itemize}

We explain the remaining inductively defined rules.

\begin{itemize}

\item \rulename{Lr8}:
\hl{TODO}

\item \rulename{Lr9}:
\hl{TODO}

\end{itemize}


\section{LTS Soundness and Completeness with respect to Projection}
\label{section:newtraceeq}

We work towards proving the soundness and completeness
of our LTS semantics with respect to projection.
Our approach is motivated from \cite{characterisation}:

\begin{enumerate}

\item We first extend the LTS semantics to
a collection of local types (hereafter referred to
as a \textit{configuration} to be consistent with
the literature)
in \cref{subsection:newltsconfig};

\item Then, we extend the definition of projection to
obtain the configuration of a global type 
(hereafter referred to as the \textit{projected configuration})
in \cref{subsection:newltsprojection};

\item Finally, we prove the trace equivalence between 
the global type and its projected configuration 
in \cref{subsection:newtraceeq}.

\end{enumerate}

We use the trace equivalence result to prove deadlock freedom
in \cref{subsection:newdeadlockfreedom}.

\subsection{LTS Semantics over Configurations}
\label{subsection:newltsconfig}

Let $\mathcal{P}$ denote the set of participants in
the communication automaton.
Also let $\lty{p}$ denote the local type of a participant
$\pinP$.

A \textit{configuration} describes the
state of the communication automaton with respect to
each participant $\pinP$.
By definition of our LTS semantics, this includes
\textit{intermediate} states, so a configuration
would also need to express the state of messages
in transit.

We inherit the definition from \cite{characterisation},
restated in \cref{def:newconfig}.

\begin{definition}[Configuration]
A configuration $s = (\vec T; ~ \vec w)$ of a system of
local types $\{ \lty{p} \}_{\pinP}$
is defined as a pair of:

\begin{itemize}

\item $\vec T = (\lty{p})_{\pinP}$
is the collection of local types.
$\lty{p}$ describes the communication structure
from the local perspective of participant $\pinP$.

\item $\vec w = (w_{\mroles{p}{q}})_{\mrole{p} \neq \mrole{q} \in \mathcal{P}}$
is the collection of \textit{unbounded buffers}.
The unbounded buffer $w_{\mrole{p}\mrole{q}}$ represents a (FIFO)
queue of messages sent by $\role{p}$ but not yet
received by $\role{q}$.

\end{itemize}

\label{def:newconfig}
\end{definition}

\begin{remark}

The \textit{subtyping} relation defined on
local types (see \cref{subsection:bgbst})
can be extended to configurations:

\begin{prooftree}
\AxiomC{$\vec w = \vec w'$}
\AxiomC{$
\forall \mrole{p} \in \mathcal{P}. ~ 
T_\mrole{p} \subtype T'_\mrole{p}
$}
\BinaryInfC{$
(\vec T; \vec w) \subtype (\vec T'; \vec w')
$}
\end{prooftree}

\end{remark}

We proceed to define the LTS over configurations in 
\cref{def:newltsconfig}, highlighting the extensions
required for \newtheory.

\begin{definition}[LTS Semantics over Configurations]
The LTS semantics over configurations is defined by
the relation $\treduce{s_T}{s'_T}{l}$.

Let $s_T = (\vec T; ~ \vec w)$ and $s'_T = (\vec T'; ~ \vec w')$.
We define the specific transitions on $\vec T$ and $\vec w$
by case analysis on the label $l$.

\begin{itemize}

\item $l = \aout{p}{q}{j}$

Then $\treduce{T_\mrole{p}}{T'_\mrole{p}}{l}~$ 
because $\role{p}$
initiates the action, so
$T'_\mrole{p'} = T_{\mrole{p'}}$ 
for all $\mrole{p'} \neq \mrole{p}$.

The message $j$ is in transit from $\role{p}$ to $\role{q}$, 
so $w'_{\mroles{p}{q}} = w_{\mroles{p}{q}} \cdot j$
($j$ is appended to the queue of in-transit messages
sent from $\role{p}$ to $\role{q}$),
and unrelated buffers $w'_{\mroles{p'}{q'}} = w_{\mroles{p}{q}}$ 
are untouched for all $\mroles{p'}{q'} \neq \mroles{p}{q}$.

\item $l = \ain{p}{q}{j}$

Then $\treduce{T_\mrole{q}}{T'_\mrole{q}}{l}~$ 
because $\role{q}$
initiates the action, so
$T'_\mrole{p'} = T_{\mrole{p'}}$ 
for all $\mrole{p'} \neq \mrole{q}$.

The message $j$ is no longer in transit
from $\role{p}$ to $\role{q}$ as it is received by $\role{q}$,
so $w_{\mroles{p}{q}} = j \cdot w'_{\mroles{p}{q}}$ 
($j$ is removed from the front of the queue of in-transit
messages sent from $\role{p}$ to $\role{q}$),
and unrelated buffers $w'_{\mroles{p'}{q'}} = w_{\mroles{p}{q}}$ 
are untouched for all $\mroles{p'}{q'} \neq \mroles{p}{q}$.

\item \hl{$l = \via{s}{\aout{p}{q}{j}}$}

Then $\treduce{T_\mrole{p}}{T'_\mrole{p}}{l}~$ 
because $\role{p}$
initiates the action.
Because the send action is routed, we also need 
$\treduce{T_\mrole{s}}{T'_\mrole{s}}{l}$.
This means
$T'_\mrole{p'} = T_{\mrole{p'}}$ 
for all $\mrole{p'} \notin \{\mrole{p} , \mrole{s} \}$.

The message $j$ is in transit from $\role{p}$ to $\role{q}$, 
so $w'_{\mroles{p}{q}} = w_{\mroles{p}{q}} \cdot j$
and unrelated buffers $w'_{\mroles{p'}{q'}} = w_{\mroles{p}{q}}$ 
are untouched for all $\mroles{p'}{q'} \neq \mroles{p}{q}$.

\item \hl{$l = \via{s}{\ain{p}{q}{j}}$}

Then $\treduce{T_\mrole{q}}{T'_\mrole{q}}{l}~$ 
because $\role{q}$
initiates the action.
Because the receive action is routed, we also need
$\treduce{T_\mrole{s}}{T'_\mrole{s}}{l}$.
This means
$T'_\mrole{p'} = T_{\mrole{p'}}$ 
for all $\mrole{p'} \notin \{\mrole{q} , \mrole{s} \}$.

The message $j$ is no longer in transit
from $\role{p}$ to $\role{q}$ as it is received by $\role{q}$,
so $w_{\mroles{p}{q}} = j \cdot w'_{\mroles{p}{q}}$,
and unrelated buffers $w'_{\mroles{p'}{q'}} = w_{\mroles{p}{q}}$ 
are untouched for all $\mroles{p'}{q'} \neq \mroles{p}{q}$.

\end{itemize}

\label{def:newltsconfig}
\end{definition}

Routed actions are carried out by the router,
so it makes sense for the local type of the router 
to also makes a step.
The semantics of the message buffers for routed actions
are the same as their non-routed counterparts; the
only difference is that these message buffers are ``managed''
by the router, but this is a change of interpretation
which isn't reflected in the theory.

\subsection{Extending Projection for Configurations}
\label{subsection:newltsprojection}

When considering the grammar of global types $G$
extended to include intermediate states,
we can obtain the \textit{projected configuration}
from a global type $G$ with participants $\mathcal{P}$:

\[
\projconf{G} = 
\left(
	\{ \proj{G}{p} \}_{\pinP} ~ ; ~
	\projconf{G}_{\{ \epsilon \}_{\qqinP}}
\right)
\]

The collection of local types is obtained by
projecting $G$ onto each participant $\pinP$.
The contents of the buffers is defined as
$\projconf{G}_{\{ w_{\mroles{q}{q'}} \}_{\qqinP}}$.
We inherit the definitions presented in \cite{characterisation},
and introduce additional rules in
\cref{fig:buffer}.

\begin{figure}[!h]
\doublespacing
\[
\begin{array}{>{\displaystyle}rc>{\displaystyle}l}

\projconf{
\gtransroute{p}{p'}{s}{j}{l_i: G_i}{i \in I}
}_{\{ w_{\mroles{q}{q'}} \}_{\qqinP}} 
	& = & \projconf{G_j}_{
	\{ w_{\mroles{q}{q'}} \}_{\qqinP}
	[w_{\mroles{p}{p'}} \mapsto w_{\mroles{p}{p'}} \cdot j]
	} \\
	
\projconf{
\groute{p}{p'}{s}{l_i: G_i}{i \in I}
}_{\{ w_{\mroles{q}{q'}} \}_{\qqinP}} 
	& = & \projconf{G_i}_{
	\{ w_{\mroles{q}{q'}} \}_{\qqinP}
	} \text{ for any } i \in I\\

\text{since}& & \forall i, j \in I. ~ 
\projconf{G_i}_{
	\{ w_{\mroles{q}{q'}} \}_{\qqinP}
} = \projconf{G_j}_{
	\{ w_{\mroles{q}{q'}} \}_{\qqinP}
} \\
\end{array}
\]
\singlespacing

\captionof{figure}{Projection of Buffer Contents from Global Type in
\newtheory}
\label{fig:buffer}
\end{figure}

As explained in \cref{subsection:newltsconfig},
the semantics of the message buffers
for routed actions are the same as their
non-routed counterparts,
so the projected contents of the buffers
for routed communication are
the same as those under non-routed communication.

\subsection{Trace Equivalence}
\label{subsection:newtraceeq}

A sequence of transitions is known as a \textit{trace}.
We want to prove that the set of traces
that can be obtained from reducing a global type
$G$ is the same as those that can be obtained
from reducing its projected configuration $\projconf{G}$.

Our approach is based on \cite{characterisation} --
namely, proving that this is the case for a single
transition (i.e. \textit{step equivalence}) is sufficient,
as we can obtain trace equivalence as a direct consequence.

\begin{lemma}[Step Equivalence]
For all global types $G$ and configurations $s$,
if $\projconf{G} \subtype s$,
then $\treduce{G}{G'}{l} \Longleftrightarrow \treduce{s}{s'}{l}$ 
such that $\projconf{G'} \subtype s'$.

\label{lem:stepeq}
\end{lemma}

\begin{proof}
By induction on the possible transitions in the LTSs
over global types (to prove $\Longrightarrow$,
i.e. \textit{soundness}) 
and configurations (to prove $\Longleftarrow$,
i.e. \textit{completeness}). 

\paragraph{Notation conventions} 
We use the following notation for decomposing configurations
and projected configurations.
\[
\def\arraystretch{1.6}
\begin{array}{rcl}
s &= & \{ T_\mrole{q} \}_{\mrole{q} \in \mathcal{P}},~
	\{ w_{\mroles{q}{q'}} \}_{\qqinP} \\
s' &= & \{ T'_\mrole{q} \}_{\mrole{q} \in \mathcal{P}},~
	\{ w'_{\mroles{q}{q'}} \}_{\qqinP} \\
\projconf{G} &= & \{ \hat{T_\mrole{q}} \}_{\mrole{q} \in \mathcal{P}},~
	\{ \hat{w}_{\mroles{q}{q'}} \}_{\qqinP} \\
\projconf{G'} &= & \{ \hat{T'_\mrole{q}} \}_{\mrole{q} \in \mathcal{P}},~
	\{ \hat{w'}_{\mroles{q}{q'}} \}_{\qqinP} \\
\end{array}
\]

\item \textbf{Soundness}

By induction on the structure of LTS semantics
over global types.

For each transition $\treduce{G}{G'}{l}$, we
take the configuration $s = \projconf{G}$,
derive $\treduce{G}{G'}{l}$ and $\treduce{s}{s'}{l}$
under the respective LTSs,
and show that $s' \subtype \projconf{G'}$.

The proofs for rules \rulename{Gr1-5} are
the same as in \cite{characterisation}.
We focus on the new rules introduced for routing.

\begin{itemize}

\item \rulename{Gr6}, 
where $G = \groute{p}{p'}{s}{l_i: G_i}{i \in I}, 
G' = \gtransroute{p}{p'}{s}{j}{l_i: G_i}{i \in I},
l = \via{s}{\aout{p}{p'}{j}}$

Then $s = \projconf{G}$ where
	
\[
\def\arraystretch{1.6}
\begin{array}{>{\displaystyle}r>{\displaystyle}l}
T_\mrole{p} = & \tselproxy{p'}{s}{l_i: \proj{G_i}{p}}{i \in I} \\
T_\mrole{p'} = & \tbraproxy{p}{s}{l_i: \proj{G_i}{p'}}{i \in I} \\
T_\mrole{s} = & \router{p}{p'}{l_i: \proj{G_i}{s}}{i \in I}\\
T_\mrole{r} = & \underset{i \in I}{\MERGEOP}~\proj{G_i}{r} 
	\text{ for } \mrole{r} \notin \{ \mrole{p},\mrole{p'},\mrole{s} \}\\
\{ w_{\mroles{q}{q'}} \}_{\qqinP} = & \projconf{G_i}_{\{\vec\epsilon\}} \text{ for some } i \in I \\
\end{array}
\]

\paragraph{Global transition:} 
We have 

\[
\def\arraystretch{1.6}
\begin{array}{>{\displaystyle}r>{\displaystyle}l}
\hat{T'_\mrole{p'}} = & \tbraproxy{p}{s}{l_i: \proj{G_i}{p'}}{i \in I} \\
\hat{T'_\mrole{s}} = & \routertrans{p}{p'}{j}{l_i: \proj{G_i}{s}}{i \in I}
\\
\hat{T'_\mrole{r}} = & \proj{G_j}{r} 
	\text{ for } \mrole{r} \notin \{ \mrole{p'},\mrole{s} \}\\
\{ \hat{w'}_{\mroles{q}{q'}} \}_{\qqinP} = & \projconf{G_i}_
	{\{ \vec\epsilon \}[w_{\mroles{p}{p'}} \mapsto w_{\mroles{p}{p'}} \cdot j]} \text{ for some } i \in I \\
\end{array}
\]

So, $\hat{w'}_{\mroles{q}{q'}} = w_{\mroles{q}{q'}}$ for
$\mroles{q}{q'} \neq \mroles{p}{p'}$ and
$\hat{w'}_{\mroles{p}{p'}} = w_{\mroles{p}{p'}} \cdot j$.

\paragraph{Configuration transition:} 
Take $T'_\mrole{r} = T_\mrole{r}$ 
for $\mrole{r} \notin \{\mrole{p}, \mrole{s}\}$.

By \rulename{Lr4}, $\treduce{T_\mrole{p}}{T'_\mrole{p}}{l}$
where $T'_\mrole{p} = \proj{G_j}{p}$.

By \rulename{Lr6}, $\treduce{T_\mrole{s}}{T'_\mrole{s}}{l}$
where $T'_\mrole{s} = \routertrans{p}{p'}{j}{l_i: \proj{G_i}{s}}{i \in I}$.

Also, ${w'}_{\mroles{q}{q'}} = w_{\mroles{q}{q'}}$ for
$\mroles{q}{q'} \neq \mroles{p}{p'}$ and
${w'}_{\mroles{p}{p'}} = w_{\mroles{p}{p'}} \cdot j$.

\paragraph{Correspondence:}
We have 
${w'}_{\mroles{q}{q'}} = \hat{w}_{\mroles{q}{q'}}$
for $\qqinP$ and
$T'_\mrole{q} = \hat{T_\mrole{q}}$
for $\mrole{q} \in \{ \mrole{p}, \mrole{p'}, \mrole{s} \}$.

For $\mrole{q} \notin \{ \mrole{p}, \mrole{p'}, \mrole{s} \}$,
we have

\[
T'_\mrole{q} = \underset{i \in I}{\MERGEOP}~\proj{G_i}{q}
	\subtype \proj{G_j}{q}
	= \hat{T_\mrole{q}}
\]

So, $s' \subtype \projconf{G'}$.

\item \rulename{Gr7}
where $G = \gtransroute{p}{p'}{s}{j}{l_i: G_i}{i \in I}, 
G' = G_j,
l = \via{s}{\ain{p}{p'}{j}}$

Then $s = \projconf{G}$ where
	
\[
\def\arraystretch{1.6}
\begin{array}{>{\displaystyle}r>{\displaystyle}l}
T_\mrole{p'} = & \tbraproxy{p}{s}{l_i: \proj{G_i}{p'}}{i \in I} \\
T_\mrole{s} = & \routertrans{p}{p'}{j}{l_i: \proj{G_i}{s}}{i \in I}\\
T_\mrole{r} = & \proj{G_j}{r} 
	\text{ for } \mrole{r} \notin \{ \mrole{p'},\mrole{s} \}\\
\{ w_{\mroles{q}{q'}} \}_{\qqinP} = & \projconf{G_j}_
	{\{ \vec\epsilon \}[w_{\mroles{p}{p'}} \mapsto w_{\mroles{p}{p'}} \cdot j]} \\
\end{array}
\]

\paragraph{Global transition:} 
We have 

\[
\def\arraystretch{1.6}
\begin{array}{>{\displaystyle}r>{\displaystyle}l}
\hat{T'_\mrole{r}} = & \proj{G_j}{r} 
	\text{ for } \mrole{r} \in \mathcal{P} \\
\{ \hat{w'}_{\mroles{q}{q'}} \}_{\qqinP} = & \projconf{G_j}_
	{\{ \vec\epsilon \}} \\
\end{array}
\]

So, $\hat{w'}_{\mroles{q}{q'}} = w_{\mroles{q}{q'}}$ for
$\mroles{q}{q'} \neq \mroles{p}{p'}$ and
$w_{\mroles{p}{p'}} = j \cdot \hat{w'}_{\mroles{p}{p'}}$.

\paragraph{Configuration transition:} 
Take $T'_\mrole{r} = T_\mrole{r}$ 
for $\mrole{r} \notin \{\mrole{p'}, \mrole{s}\}$.

By \rulename{Lr5}, $\treduce{T_\mrole{p}}{T'_\mrole{p}}{l}$
where $T'_\mrole{p} = \proj{G_j}{p}$.

By \rulename{Lr7}, $\treduce{T_\mrole{s}}{T'_\mrole{s}}{l}$
where $T'_\mrole{s} = \proj{G_j}{s}$.

Also, ${w'}_{\mroles{q}{q'}} = w_{\mroles{q}{q'}}$ for
$\mroles{q}{q'} \neq \mroles{p}{p'}$ and
$w_{\mroles{p}{p'}} = j \cdot {w'}_{\mroles{p}{p'}}$.

\paragraph{Correspondence:}
We have 
${w'}_{\mroles{q}{q'}} = \hat{w}_{\mroles{q}{q'}}$
for $\qqinP$ and
$T'_\mrole{q} = \hat{T_\mrole{q}}$
for $\mrole{q} \in \mathcal{P}$.

So, $s' = \projconf{G'}$.

\item \rulename{Gr8}
where $G = \groute{p}{p'}{s}{l_i: G_i}{i \in I}, 
G' = \groute{p}{p'}{s}{l_i: G'_i}{i \in I}$

By hypothesis,
$\forall i \in I. ~ \treduce{G_i}{G'_i}{l}$
and $\subj{l} \notin \{ \mrole{p}, \mrole{p'} \}$.

By induction,
$\forall i \in I. ~ \treduce{\projconf{G_i}}{\projconf{G'_i}}{l}$.

To show that $\treduce{\projconf{G}}{\projconf{G'}}{l}$,
it is sufficient to show that
$\treduce{\proj{G}{q}}{\proj{G'}{q}}{l}$
for $\mrole{q} = \subj{l}$,
since the projections for $\mrole{q'} \neq \subj{l}$ remain the same.

We know $\proj{G}{q} = \underset{i \in I}{\MERGEOP}\proj{G_i}{q}$
and $\proj{G'}{q} = \underset{i \in I}{\MERGEOP}\proj{G'_i}{q}$.

By induction,
$\treduce
	{\underset{i \in I}{\MERGEOP}\proj{G_i}{q}}
	{\underset{i \in I}{\MERGEOP}\proj{G'_i}{q}}
	{l}$,
so $\treduce{\proj{G}{q}}{\proj{G'}{q}}{l}$.

\item \rulename{Gr9}
where $G = \gtransroute{p}{p'}{s}{j}{l_i: G_i}{i \in I}, 
G' = \gtransroute{p}{p'}{s}{j}{l_i: G'_i}{i \in I}$

By hypothesis,
$\treduce{G_j}{G'_j}{l}$,
$\mrole{p'} \neq \subj{l}$,
and $\forall i \in I \setminus \{j\}. ~ G'_i = G_i$.

By induction,
$\treduce{\projconf{G_j}}{\projconf{G'_j}}{l}$.

To show that $\treduce{\projconf{G}}{\projconf{G'}}{l}$,
it is sufficient to show that
$\treduce{\proj{G}{q}}{\proj{G'}{q}}{l}$
for $\mrole{q} = \subj{l}$,
since the projections for $\mrole{q'} \neq \subj{l}$ remain the same.

We know $\proj{G}{q} = \proj{G_j}{q}$
and $\proj{G'}{q} = \proj{G'_j}{q}$.

By induction,
$\treduce
	{\proj{G_j}{q}}
	{\proj{G'_j}{q}}
	{l}$,
so $\treduce{\proj{G}{q}}{\proj{G'}{q}}{l}$.

\end{itemize}

\item \textbf{Completeness}

By considering the possible transitions in the LTS
over configurations, which is defined by
case analysis on the possible labels $l$.

For each transition $\treduce{s}{s'}{l}$,
we take the configuration $s$ from the reduction rule,
infer the structure of the global type $G$ such that $s = \projconf{G}$,
derive $\treduce{s}{s'}{l}$ and $\treduce{G}{G'}{l}$
under the respective LTSs, 
and show that $s' \subtype \projconf{G'}$.

The proofs for $l = \aout{p}{q}{j}$ and $l = \ain{p}{q}{j}$
are the same as in Appendix A.1 of \cite{characterisation}.
We focus on the new labels introduced for routing.

\begin{itemize}

\item $l = \via{s}{\aout{p}{q}{j}}$:

Then $T_\mrole{p} = \tselproxy{q}{s}{l_i: \proj{G_i}{p}}{i \in I}$.

Also, $T_\mrole{s}$ \textit{contains} 
$\router{p}{q}{l_i: \proj{G_i}{s}}{i \in I}$ as subterm. 
We denote this subterm $\tilde{T_\mrole{s}}$.

By definition of projection, $G$ has
$\groute{p}{q}{s}{l_i: G_i}{i \in I}$ as subterm. 
We denote this subterm $\tilde{G}$.

Also by definition of projection, no action in $G$
will involve $\mrole{p}$ before $\tilde{G}$.

\paragraph{Configuration transition:}

By \rulename{Lr4}, 
$\treduce{T_\mrole{p}}{T'_\mrole{p}}{l}$,
where $T'_\mrole{p} = \proj{G_j}{p}$.

By \rulename{Lr6}, 
$\treduce{\tilde{T}_\mrole{s}}{\tilde{T'}_\mrole{s}}{l}$,
where $\tilde{T}'_\mrole{s} = 
	\routertrans{p}{q}{j}{l_i: \proj{G_i}{s}}{i \in I}$.

We get $\treduce{T_\mrole{s}}{T'_\mrole{s}}{l}$
by inversion lemma, as illustrated below.

\begin{prooftree}
\AxiomC{}
\RightLabel{\rulename{Lr6}}
\UnaryInfC{$\treduce{\tilde{T}_\mrole{s}}{\tilde{T'}_\mrole{s}}{l}$}
\UnaryInfC{$\vdots$}
\RightLabel{\rulename{Lr8,9,10,11} as needed}
\UnaryInfC{$\qquad\treducelong{T_\mrole{s}}{T'_\mrole{s}}{l}\qquad$}
\end{prooftree}

\paragraph{Global transition:}

By \rulename{Gr6}, 
$\treduce{\tilde{G}}{\tilde{G}'}{l}$,
where $\tilde{G}' = \gtransroute{p}{q}{s}{j}{l_i: G_i}{i \in I}$.

We get $\treduce{G}{G'}{l}$ by inversion lemma,
as illustrated below.

\begin{prooftree}
\AxiomC{}
\RightLabel{\rulename{Gr6}}
\UnaryInfC{$\treduce{\tilde{G}}{\tilde{G}'}{l}$}
\UnaryInfC{$\vdots$}
\RightLabel{\rulename{Gr4,5,8,9} as needed}
\UnaryInfC{$\qquad\treducelong{G}{G'}{l}\qquad$}
\end{prooftree}

\paragraph{Correspondence:}
Since the projections for 
$\mrole{p'} \notin \{ \mrole{p}, \mrole{s} \}$
are unchanged,
it is sufficient to show that 
$T'_\mrole{p} \subtype (\proj{\tilde{G'}}{p})$ and
$\tilde{T'}_\mrole{s} \subtype (\proj{\tilde{G'}}{s})$.

\begin{align*}
\proj{\tilde{G'}}{p} 
	&= \proj{G_j}{p} 
	= T'_\mrole{p} \\
\proj{\tilde{G'}}{s} 
	&= \routertrans{p}{q}{j}{l_i: \proj{G_i}{s}}{i \in I}
	= \tilde{T'}_\mrole{s}
\end{align*}

\item $l = \via{s}{\ain{p}{q}{j}}$:

Then $T_\mrole{q} = \tbraproxy{p}{s}{l_i: \proj{G_i}{q}}{i \in I}$.

Also, $T_\mrole{s}$ \textit{contains} 
$\routertrans{p}{q}{j}{l_i: \proj{G_i}{s}}{i \in I}$ as subterm. 
We denote this subterm $\tilde{T_\mrole{s}}$.

By definition of projection, $G$ has
$\gtransroute{p}{q}{s}{j}{l_i: G_i}{i \in I}$ as subterm. 
We denote this subterm $\tilde{G}$.

Also by definition of projection, no action in $G$
will involve $\mrole{q}$ before $\tilde{G}$.

\paragraph{Configuration transition:}

By \rulename{Lr5}, 
$\treduce{T_\mrole{q}}{T'_\mrole{q}}{l}$,
where $T'_\mrole{q} = \proj{G_j}{q}$.

By \rulename{Lr7}, 
$\treduce{\tilde{T}_\mrole{s}}{\tilde{T'}_\mrole{s}}{l}$,
where $\tilde{T}'_\mrole{s} = \proj{G_j}{s}$.

We get $\treduce{T_\mrole{s}}{T'_\mrole{s}}{l}$
by inversion lemma, as illustrated below.

\begin{prooftree}
\AxiomC{}
\RightLabel{\rulename{Lr7}}
\UnaryInfC{$\treduce{\tilde{T}_\mrole{s}}{\tilde{T'}_\mrole{s}}{l}$}
\UnaryInfC{$\vdots$}
\RightLabel{\rulename{Lr8,9,10,11} as needed}
\UnaryInfC{$\qquad\treducelong{T_\mrole{s}}{T'_\mrole{s}}{l}\qquad$}
\end{prooftree}

\paragraph{Global transition:}

By \rulename{Gr7}, 
$\treduce{\tilde{G}}{\tilde{G}'}{l}$,
where $\tilde{G}' = G_j$.

We get $\treduce{G}{G'}{l}$ by inversion lemma,
as illustrated below.

\begin{prooftree}
\AxiomC{}
\RightLabel{\rulename{Gr7}}
\UnaryInfC{$\treduce{\tilde{G}}{\tilde{G}'}{l}$}
\UnaryInfC{$\vdots$}
\RightLabel{\rulename{Gr4,5,8,9} as needed}
\UnaryInfC{$\qquad\treducelong{G}{G'}{l}\qquad$}
\end{prooftree}

\paragraph{Correspondence:}
Since the projections for 
$\mrole{p'} \notin \{ \mrole{q}, \mrole{s} \}$
are unchanged,
it is sufficient to show that 
$T'_\mrole{q} \subtype (\proj{\tilde{G'}}{q})$ and
$\tilde{T'}_\mrole{s} \subtype (\proj{\tilde{G'}}{s})$.

\begin{align*}
\proj{\tilde{G'}}{q} 
	&= \proj{G_j}{q} 
	= T'_\mrole{q} \\
\proj{\tilde{G'}}{s} 
	&= \proj{G_j}{s}
	= \tilde{T'}_\mrole{s}
\end{align*}

\end{itemize}

\end{proof}

\begin{theorem}[Trace Equivalence]
Let $G$ be a global type with participants 
$\mathcal{P} = \pt{G}$, 
and let $\vec T = \{ \proj{G}{p} \}_{\pinP}$ be the local
types projected from $G$.
Then $G \approx (\vec T, \vec \epsilon)$.

\label{th:traceeq}
\end{theorem}

\begin{proof}
Direct consequence of \cref{lem:stepeq}.
\end{proof}

\subsection{Deadlock Freedom}
\label{subsection:newdeadlockfreedom}

\begin{lemma}[Preservation of Well-formedness]
Let $G$ be a global type.
Suppose $G$ is well-formed with respect to some router $\mrole{s}$,
i.e. $\wfnew{G}{s}$.

\[
\forall G', l. ~
\left(\treduce{G}{G'}{l}
	\Longrightarrow
\wfnew{G'}{s}\right)
\]

\label{lem:preservewf}
\end{lemma}

\begin{proof}
By induction on the structure of $\treduce{G}{G'}{l}$.

For each transition, we show the two conjuncts for $\wfnew{G'}{s}$: 
\textbf{(1)} $\proj{G'}{r}$ exists for $\mrole{r}$ 
	such that $\proj{G}{r}$ exists;
and, \textbf{(2)} $\centroid{G'}{s}$.

\begin{itemize}
\item \rulename{Gr1}, 
where 
$G = \gcomm{p}{q}{l_i: G_i}{i \in I}$,
$G' = \gtrans{p}{q}{j}{l_i: G_i}{i \in I}$,
$l = \aout{p}{q}{j}$.

\paragraph{(1)}
We know $\proj{G}{r}$ by assumption.
To show $\proj{G'}{r}$, consider $\mrole{r}$ by case:

\begin{itemize}
\item $\mrole{r} = \mrole{p}$:
Then $\proj{G}{p} = \tsel{q}{l_i: \proj{G_i}{p}}{i \in I}$,
so $\forall i \in I. ~ \proj{G_i}{p}$ exists.

$\proj{G'}{p} = \proj{G_j}{p}$, which exists as $j \in I$. 

\item $\mrole{r} = \mrole{q}$:
Then $\proj{G'}{q} = \tbra{p}{l_i: \proj{G_i}{q}}{i \in I} \proj{G}{q}$, 
which exists.

\item $\mrole{r} \notin \{ \mrole{p}, \mrole{q} \}$:
Then $\proj{G}{r} = \underset{i \in I}{\MERGEOP}\proj{G_i}{r}$,
so $\forall i \in I. ~ \proj{G_i}{r}$ exists.

$\proj{G'}{r} = \proj{G_j}{r}$, which exists as $j \in I$. 
\end{itemize}

\paragraph{(2)}
We know $\centroid{G}{s}$ by assumption.
We deduce $\centroid{G'}{s}$ by consequence.

\[
\centroid{G}{s} 
\Longrightarrow 
	\mrole{s} \in \{ \mrole{p}, \mrole{q} \} 
		\wedge 
	\underset{i \in I}{\bigwedge}\centroid{G_i}{s}
\Longrightarrow	
	\mrole{s} \in \{ \mrole{p}, \mrole{q} \} 
		\wedge 
	\centroid{G_j}{s}
\Longrightarrow \centroid{G'}{s}
\]

\item \rulename{Gr2}, 
where
$G = \gtrans{p}{q}{j}{l_i: G_i}{i \in I}$,
$G' = G_j$,
$l = \ain{p}{q}{j}$.

\paragraph{(1)}
We know $\proj{G}{r}$ by assumption.
To show $\proj{G'}{r}$, consider $\mrole{r}$ by case:

\begin{itemize}
\item $\mrole{r} = \mrole{p}$:
Then $\proj{G'}{p} = \proj{G_j}{p} = \proj{G}{p}$, which exists.

\item $\mrole{r} = \mrole{q}$:
Then $\proj{G}{q} = \tbra{p}{l_i: \proj{G_i}{q}}{i \in I}$,
so $\forall i \in I. ~ \proj{G_i}{q}$ exists.

$\proj{G'}{q} = \proj{G_j}{q}$, which exists as $j \in I$. 

\item $\mrole{r} \notin \{ \mrole{p}, \mrole{q} \}$:
Then $\proj{G'}{r} = \proj{G_j}{r} = \proj{G}{r}$,
which exists.

\end{itemize}

\paragraph{(2)}
We know $\centroid{G}{s}$ by assumption.
We deduce $\centroid{G'}{s}$ by consequence.

\[
\centroid{G}{s} 
	\Longrightarrow 
		\mrole{s} \in \{ \mrole{p}, \mrole{q} \} 
			\wedge 
		\centroid{G_j}{s}
	\Longrightarrow
		\centroid{G'}{s}
\]

\item \rulename{Gr3}, 
where $\treduce{\trec{G}}{G'}{l}$.
By hypothesis, $\treduce{G[\trec{G} / \trecvar]}{G'}{l}$.

We first show that $\wfnew{G[\trec{G} / \trecvar]}{s}$.

\subparagraph{(1)}
$\proj{\trec{G}}{r}$ exists for some $\mrole{r}$.

Note that $\proj{G}{r}$ exists regardless of $\mrole{r}$'s 
participation in $G$.

\begin{itemize}
\item 
If $\mrole{r} \in \pt{G}$, 
then $\proj{\trec{G}}{r} = \trec{\proj{G}{r}}$,
so $\proj{G}{r}$ exists.

\item 
Otherwise, $\proj{G}{r} = \tend$, which exists.
\end{itemize}

Projection is homomorphic under recursion, so 
$\proj{G[\trec{G}/\trecvar]}{r}$ exists.

\subparagraph{(2)} 
By assumption, $\centroid{(\trec{G})}{s}$,
so $\centroid{G}{s}$.

The $\centroid{}{}$ relation is also homomorphic under recursion,
so we get $\centroid{G[\trec{G}/\trecvar]}{s}$.

We conclude by induction to obtain $\wfnew{G'}{s}$.

\item \rulename{Gr4},
where
$G = \gcomm{p}{q}{l_i: G_i}{i \in I}$,
$G' = \gcomm{p}{q}{l_i: G'_i}{i \in I}$.

By hypothesis,
$\forall i \in I. ~ (\treduce{G_i}{G'_i}{l})$
and $\mrole{p} \neq \mrole{q} \neq \subj{l}$.

If $\proj{G}{r}$ exists, so does $\proj{G_i}{r}$ for $i \in I$.

By assumption, $\centroid{G}{s}$, 
so $\mrole{s} \in \{ \mrole{p}, \mrole{q} \} \wedge 
\underset{i \in I}{\bigwedge}\centroid{G_i}{s}$.

By induction, $\forall i \in I. ~ 
(\proj{G'_i}{r} \text{ exists } \wedge
\centroid{G'_i}{s})$.

\paragraph{(1)}
To show $\proj{G'}{r}$, consider $\mrole{r}$ by case:

\begin{itemize}
\item $\mrole{r} = \mrole{p}$:
Then $\proj{G'}{p} = \tsel{q}{l_i: \proj{G'}{p}}{i \in I}$.

\item $\mrole{r} = \mrole{q}$:
Then $\proj{G'}{q} = \tbra{p}{l_i: \proj{G'_i}{q}}{i \in I}$.

\item $\mrole{r} \notin \{ \mrole{p}, \mrole{q} \}$:
Then $\proj{G'}{r} = \underset{i \in I}{\MERGEOP} \proj{G'_i}{r}$.
We know that $\proj{G}{r} = \underset{i \in I}{\MERGEOP} \proj{G_i}{r}$
exists. 
By \cref{lem:localltspreservemerge}, 
$\underset{i \in I}{\MERGEOP} \proj{G'_i}{r}$ exists too.

\end{itemize}

\paragraph{(2)}
We have $\mrole{s} \in \{ \mrole{p}, \mrole{q} \}$
from assumption and
$\underset{i \in I}{\bigwedge}\centroid{G'_i}{s}$ from induction, 
so $\centroid{G'}{s}$.

\item \rulename{Gr5},
where
$G = \gtrans{p}{q}{j}{l_i: G_i}{i \in I}$,
$G' = \gtrans{p}{q}{j}{l_i: G'_i}{i \in I}$.

By hypothesis,
$\treduce{G_j}{G'_j}{l}$,
$\forall i \in I \setminus \{ j \}. ~ G'_i = G_i$,
and $\mrole{q} \neq \subj{l}$.

If $\proj{G}{r}$ exists, so does $\proj{G_i}{r}$ for $i \in I$.

By assumption, $\centroid{G}{s}$, 
so $\mrole{s} \in \{ \mrole{p}, \mrole{q} \} \wedge 
\centroid{G_j}{s}$.

By induction on $\treduce{G_j}{G'_j}{l}$
and hypothesis $\forall i \in I \setminus \{ j \}. ~ G'_i = G_i$,
we get
$\forall i \in I. ~ 
(\proj{G'_i}{r} \text{ exists } \wedge
\centroid{G'_i}{s})$.

\paragraph{(1)}
To show $\proj{G'}{r}$, consider $\mrole{r}$ by case:

\begin{itemize}
\item $\mrole{r} = \mrole{p}$:
Then $\proj{G'}{p} = \proj{G'_j}{p}$.

\item $\mrole{r} = \mrole{q}$:
Then $\proj{G'}{q} = \tbra{p}{l_i: \proj{G'_i}{q}}{i \in I}$.

\item $\mrole{r} \notin \{ \mrole{p}, \mrole{q} \}$:
Then $\proj{G'}{r} = \proj{G'_j}{r}$.

\end{itemize}

\paragraph{(2)}
We have $\mrole{s} \in \{ \mrole{p}, \mrole{q} \}$
from assumption and
$\centroid{G'_j}{s}$ from induction, so $\centroid{G'}{s}$.

\item \rulename{Gr6},
where
$G = \groute{p}{q}{t}{l_i: G_i}{i \in I}$,
$G' = \gtransroute{p}{q}{t}{j}{l_i: G_i}{i \in I}$,
$l = \via{s}{\aout{p}{q}{j}}$.

By assumption, $\wfnew{G}{s}$, so $\mrole{t} = \mrole{s}$.

\paragraph{(1)}
We know $\proj{G}{r}$ by assumption.
To show $\proj{G'}{r}$, consider $\mrole{r}$ by case:

\begin{itemize}
\item $\mrole{r} = \mrole{p}$:
Then $\proj{G}{p} = \tselproxy{q}{s}{l_i: \proj{G_i}{p}}{i \in I}$,
so $\forall i \in I. ~ \proj{G_i}{p}$ exists.

$\proj{G'}{p} = \proj{G_j}{p}$, which exists as $j \in I$. 

\item $\mrole{r} = \mrole{q}$:
Then $\proj{G'}{q} = \tbraproxy{p}{s}{l_i: \proj{G_i}{q}}{i \in I} 
= \proj{G}{q}$, 
which exists.

\item $\mrole{r} = \mrole{s}$:
Then $\proj{G}{s} = \router{p}{q}{l_i: \proj{G_i}{s}}{i \in I}$,
so $\forall i \in I. ~ \proj{G_i}{s}$ exists.

$\proj{G'}{s} = \routertrans{p}{q}{j}{l_i: \proj{G_i}{s}}{i \in I}$, 
which exists.

\item $\mrole{r} \notin \{ \mrole{p}, \mrole{q}, \mrole{s} \}$:
Then $\proj{G}{r} = \underset{i \in I}{\MERGEOP}\proj{G_i}{r}$,
so $\forall i \in I. ~ \proj{G_i}{r}$ exists.

$\proj{G'}{r} = \proj{G_j}{r}$, which exists as $j \in I$. 
\end{itemize}

\paragraph{(2)}
We know $\centroid{G}{s}$ by assumption.
We deduce $\centroid{G'}{s}$ by consequence.

\[
\centroid{G}{s} 
\Longrightarrow 
	\mrole{t} = \mrole{s} 
		\wedge 
	\underset{i \in I}{\bigwedge}\centroid{G_i}{s}
\Longrightarrow	
	\mrole{t} = \mrole{s}
		\wedge 
	\centroid{G_j}{s}
\Longrightarrow \centroid{G'}{s}
\]

\item \rulename{Gr7},
where
$G = \gtransroute{p}{q}{t}{j}{l_i: G_i}{i \in I}$,
$G' = G_j$,
$l = \via{s}{\ain{p}{q}{j}}$.

\paragraph{(1)}
We know $\proj{G}{r}$ by assumption.
To show $\proj{G'}{r}$, consider $\mrole{r}$ by case:

By assumption, $\wfnew{G}{s}$, so $\mrole{t} = \mrole{s}$.

\begin{itemize}
\item $\mrole{r} = \mrole{p}$:
Then $\proj{G'}{p} = \proj{G_j}{p} = \proj{G}{p}$, which exists.

\item $\mrole{r} = \mrole{q}$:
Then $\proj{G}{q} = \tbraproxy{p}{s}{l_i: \proj{G_i}{q}}{i \in I}$,
so $\forall i \in I. ~ \proj{G_i}{q}$ exists.

$\proj{G'}{q} = \proj{G_j}{q}$, which exists as $j \in I$. 

\item $\mrole{r} = \mrole{s}$:
Then $\proj{G}{s} = \routertrans{p}{q}{j}{l_i: \proj{G_i}{s}}{i \in I}$,
so $\forall i \in I. ~ \proj{G_i}{s}$ exists.

$\proj{G'}{s} = \proj{G_j}{s}$, 
which exists as $j \in I$.

\item $\mrole{r} \notin \{ \mrole{p}, \mrole{q}, \mrole{s} \}$:
Then $\proj{G}{r} = \underset{i \in I}{\MERGEOP}\proj{G_i}{r}$,
so $\forall i \in I. ~ \proj{G_i}{r}$ exists.

$\proj{G'}{r} = \proj{G_j}{r}$,
which exists as $j \in I$.

\end{itemize}

\paragraph{(2)}
We know $\centroid{G}{s}$ by assumption.
We deduce $\centroid{G'}{s}$ by consequence.

\[
\centroid{G}{s} 
	\Longrightarrow 
		\mrole{s} \in \{ \mrole{p}, \mrole{q} \} 
			\wedge 
		\centroid{G_j}{s}
	\Longrightarrow
		\centroid{G'}{s}
\]

\item \rulename{Gr8},
where
$G = \groute{p}{q}{t}{l_i: G_i}{i \in I}$,
$G' = \groute{p}{q}{t}{l_i: G'_i}{i \in I}$.

By hypothesis,
$\forall i \in I. ~ (\treduce{G_i}{G'_i}{l})$
and $\mrole{p} \neq \mrole{q} \neq \subj{l}$.

If $\proj{G}{r}$ exists, so does $\proj{G_i}{r}$ for $i \in I$.

By assumption, $\centroid{G}{s}$, 
so $\mrole{t} = \mrole{s} \wedge 
\underset{i \in I}{\bigwedge}\centroid{G_i}{s}$.

By induction, $\forall i \in I. ~ 
(\proj{G'_i}{r} \text{ exists } \wedge
\centroid{G'_i}{s})$.

\paragraph{(1)}
To show $\proj{G'}{r}$, consider $\mrole{r}$ by case:

\begin{itemize}
\item $\mrole{r} = \mrole{p}$:
Then $\proj{G'}{p} = \tselproxy{q}{s}{l_i: \proj{G'}{p}}{i \in I}$.

\item $\mrole{r} = \mrole{q}$:
Then $\proj{G'}{q} = \tbraproxy{p}{s}{l_i: \proj{G'_i}{q}}{i \in I}$.

\item $\mrole{r} = \mrole{s}$:
Then $\proj{G'}{s} = \router{p}{q}{l_i: \proj{G'_i}{s}}{i \in I}$.

\item $\mrole{r} \notin \{ \mrole{p}, \mrole{q}, \mrole{s} \}$:
Then $\proj{G'}{r} = \underset{i \in I}{\MERGEOP} \proj{G'_i}{r}$.
We know that $\proj{G}{r} = \underset{i \in I}{\MERGEOP} \proj{G_i}{r}$
exists. 
By \cref{lem:localltspreservemerge}, 
$\underset{i \in I}{\MERGEOP} \proj{G'_i}{r}$ exists too.

\end{itemize}

\paragraph{(2)}
We have $\mrole{t} = \mrole{s}$
from assumption and
$\underset{i \in I}{\bigwedge}\centroid{G'_i}{s}$ from induction, 
so $\centroid{G'}{s}$.

\item \rulename{Gr9},
where
$G = \gtransroute{p}{q}{t}{j}{l_i: G_i}{i \in I}$,
$G' = \gtransroute{p}{q}{t}{j}{l_i: G'_i}{i \in I}$.

By hypothesis,
$\treduce{G_j}{G'_j}{l}$,
$\forall i \in I \setminus \{ j \}. ~ G'_i = G_i$,
and $\mrole{q} \neq \subj{l}$.

If $\proj{G}{r}$ exists, so does $\proj{G_i}{r}$ for $i \in I$.

By assumption, $\centroid{G}{s}$, 
so $\mrole{t} = \mrole{s} \wedge 
\centroid{G_j}{s}$.

By induction on $\treduce{G_j}{G'_j}{l}$
and hypothesis $\forall i \in I \setminus \{ j \}. ~ G'_i = G_i$,
we get
$\forall i \in I. ~ 
(\proj{G'_i}{r} \text{ exists } \wedge
\centroid{G'_i}{s})$.

\paragraph{(1)}
To show $\proj{G'}{r}$, consider $\mrole{r}$ by case:

\begin{itemize}
\item $\mrole{r} = \mrole{p}$:
Then $\proj{G'}{p} = \proj{G'_j}{p}$.

\item $\mrole{r} = \mrole{q}$:
Then $\proj{G'}{q} = \tbraproxy{p}{s}{l_i: \proj{G'_i}{q}}{i \in I}$.

\item $\mrole{r} = \mrole{s}$:
Then $\proj{G'}{s} = \routertrans{p}{q}{j}{l_i: \proj{G'_i}{s}}{i \in I}$.

\item $\mrole{r} \notin \{ \mrole{p}, \mrole{q}, \mrole{s} \}$:
Then $\proj{G'}{r} = \proj{G'_j}{r}$.

\end{itemize}

\paragraph{(2)}
We have $\mrole{t} = \mrole{s}$
from assumption and
$\centroid{G'_j}{s}$ from induction, so $\centroid{G'}{s}$.

\end{itemize}

\end{proof}

\begin{lemma}[Progress for Well-formed Global Types]
Let $G$ be a global type.
Suppose $G$ is well-formed with respect to some router $\mrole{s}$,
i.e. $\wfnew{G}{s}$.

\[
(G = \tend) \vee \exists G', l. ~ (\treduce{G}{G'}{l})
\]

\label{lem:progresswf}
\end{lemma}

\begin{proof}
The following is logically equivalent.

\[
(G \neq \tend)
	\Longrightarrow 
\exists G', l. ~ (\treduce{G}{G'}{l})
\]

We prove this by induction on the structure of $G$.

We do not consider $G = \tend$ by assumption.

We also do not consider $G = \trecvar$ as the type variable is not guarded.

\begin{enumerate}

\item $G = \trec{G''}$

$\trecvar$ must occur in $G$, so $G[\trec{G} / \trecvar] \neq \tend$.

By induction, 
$\exists G', l. ~ (\treduce{G[\trec{G} / \trecvar]}{G'}{l})$.

Apply \rulename{Gr3} to get 
$\exists G', l. ~ (\treduce{\trec{G}}{G'}{l})$.

\item $G = \gcomm{p}{q}{l_i: G_i}{i \in I}$

Apply \rulename{Gr1} to get 
$\treducelong
	{G}
	{\gtrans{p}{q}{j}{l_i: G_i}{i \in I}}
	{\aout{p}{q}{j}}$.

\item $G = \groute{p}{q}{r}{l_i: G_i}{i \in I}$

By assumption, $\wfnew{G}{s}$, so $\mrole{r} = \mrole{s}$.

Apply \rulename{Gr6} to get
$\treducelong
	{G}
	{\gtransroute{p}{q}{s}{j}{l_i: G_i}{i \in I}}
	{\via{s}{\aout{p}{q}{j}}}$.

\item $G = \gtrans{p}{q}{j}{l_i: G_i}{i \in I}$

Apply \rulename{Gr2} to get 
$\treducelong
	{G}
	{G_j}
	{\ain{p}{q}{j}}$.

\item $G = \gtransroute{p}{q}{r}{j}{l_i: G_i}{i \in I}$

By assumption, $\wfnew{G}{s}$, so $\mrole{r} = \mrole{s}$.

Apply \rulename{Gr7} to get
$\treducelong
	{G}
	{G_j}
	{\via{s}{\ain{p}{q}{j}}}$.

\end{enumerate}

\end{proof}

\begin{theorem}[Deadlock Freedom]
Let $G$ be a global type.
Suppose $G$ is well-formed with respect to some router $\mrole{s}$,
i.e. $\wfnew{G}{s}$.

\[
\forall G'. ~ \left(
G \to^* G'
	\Longrightarrow
(G' = \tend) \vee \exists G'', l. ~ 
	(\treduce{G'}{G''}{l})
\right)\]

\end{theorem}

\begin{proof}
Direct consequence of 
\cref{lem:preservewf,lem:progresswf}.
\end{proof}

\section{Encoding Session Types in TypeScript}
% Design choices

Developers can implement their application using APIs generated from the EFSM
to guarantee correctness by construction.
Our approach integrates the EFSM into the development workflow by encoding
session types as TypeScript types.
Communication over the WebSocket protocol introduces additional constraints:
communication is always initiated in the front-end and driven by user interactions,
whilst back-end roles can only accept connections.
This motivates our design of encoding the session types differently for server
(\cref{section:server}) and client (\cref{section:browser}) targets.

\subsection{Server-Side API Generation}
\label{section:server}

% Francisco: mention that it corresponds to the messages for the Noughts and Crosses example too

\begin{wrapfigure}{r}{0.6\textwidth}
  \vspace{-5mm}
  \begin{center}
    \includegraphics[width=0.58\textwidth]{figures/efsm_svr.png}
  \end{center}

  \vspace{-5mm}
  \captionof{figure}{EFSM for \texttt{Svr}.}
  \label{fig:efsmsvr}
\vspace{-1cm}
\end{wrapfigure}

We refer to the \texttt{Svr} EFSM (\cref{fig:efsmsvr}) as a running example in
this section.
For server-side targets, we encode EFSM states into TypeScript types and
consider branching (receiving) and selection (sending) states separately.
We assign TypeScript encodings of states to their state identifiers in the
EFSM, providing syntactic sugar when referring to the successor state when
encoding the current state.
For any state $S$ in the EFSM, we refer to the TypeScript type alias of its
encoding as $\llbracket S \rrbracket$.
We outline the encoding below using examples from the
\textit{Noughts and Crosses} game (\cref{lst:svr}).

\paragraph{Branching State}
We consider a receiving state as a unary branching state for conciseness.
A branching state is encoded as an \textit{object literal}
\cite{TypeScriptSpec} (a record type), with each branch $i \in I$ ($I$ denoting set of all
branches), corresponding to a member field.
A branch expecting to receive a message labelled $\texttt{label}_i$ carrying
payload of type $\texttt{T}_i$ with successor state $S_i$ is encoded as an
\textit{member field} named $\texttt{label}_i$ of function type
$(payload:\texttt{T}_i) \to \llbracket S_i \rrbracket$.
The developer implements a branching operation by passing callbacks for each
branch, parameterised by the expected message payload type for that branch.

\paragraph{Selection State}
We consider a sending state as a unary selection state for conciseness.
A selection state is encoded as a \textit{union type}
\cite{TypeScriptSpec} of internal choice encodings: each internal choice $i \in
I$ ($I$ denoting set of all choices), sending a message labelled
$\texttt{label}_i$ carrying payload of type $\texttt{T}_i$ with successor state
$S_i$ is encoded as a \textit{tuple type} of \texttt{[Labels.label$_i$, T$_i$,
  $\llbracket S_i \rrbracket$]}.
The developer implements a selection operation by passing the selected label
and payload to send in the message.
We generate a \textit{string enum} (named \texttt{Labels}) wrapping the labels
in the protocol.

\begin{figure}[ht]
\begin{lstlisting}[language=JavaScript]
export type S13 = { Pos: (payload: Point) => S15 };
export type S15 = [ Labels.Lose, Point, S16 ]
                | [ Labels.Draw, Point, S17 ]
                | [ Labels.Update, Point, S18 ];
\end{lstlisting}
\captionof{lstlisting}{Example encodings from \textit{Noughts and Crosses} \texttt{Svr} EFSM.}
\label{lst:svr}
\end{figure}

% REVIEW:
% Also, when describing Listing 2, it would be helpful to say a bit more explicitly that the client is supposed to implement S13, and the code in S13 returns an object of type S15. Then you can point to Listing 4 if they are coordinated.
In the case of \cref{lst:svr}, the developer is expected to implement
\texttt{S13} which handles the \texttt{Pos} message sent by \texttt{P1},
and the code in \texttt{S13} returns a value of type \texttt{S15}, which
corresponds to a selection of messages to send to \texttt{P2}. \Cref{lst:svrprotocol}
illustrates how the developer may implement these types.

We make a key design decision \textit{not} to expose communication channels in
the TypeScript session type encodings to provide \textit{static} linearity
guarantees (\cref{section:serverlinear}).
Our encoding sufficiently exposes seams for the developer to inject their
program logic, whilst the generated session API
(\cref{section:serversessionapi}) handles the sending and receiving of
messages.

\subsubsection{Session Runtime}
\label{section:serversessionapi}

The generated code for our session runtime performs communication in a protocol-conformant manner, but
does not expose these IO actions to the developer by delegating the
aforementioned responsibilities to an inner class.
% REVIEW: it is not clear what is
% generated on the server-side and what actually is the session runtime
The runtime executes the EFSM by keeping track of
the current state (similar to the generated code in \cite{javatypestate})
and only permitting the specified IO actions at the current state.
The runtime listens to message (receiving) events on the communication channel,
invokes the corresponding callback to obtain the value to send next, and
performs the sending.
The developer instantiates a session by constructing an instance of the
session runtime class, providing the WebSocket endpoint URL (to open the
connection) and the initial state (to execute the EFSM).

\subsubsection{Linear Channel Usage}
\label{section:serverlinear}
Developers writing their implementation using the generated APIs 
enjoy channel linearity by construction.
Our library design prevents the two conditions detailed below:

\paragraph{Repeated Usage}
We do not expose channels to the developer, which makes \textit{reusing
  channels} impossible.
For example, to send a message, the generated API only requires the payload
that needs to be sent, and the session runtime performs the send internally,
guaranteeing this action is done \textit{exactly once} by construction.

\paragraph{Unused Channels}
The initial state must be supplied to the session runtime
constructor in order to instantiate a session;
this initial state is defined
in terms of the successor states, which in turn has references to its
successors and so forth.
The developer's implementation will cover the terminal state
(if it exists), and the
session runtime guarantees this terminal state will be reached
by construction.

\subsection{The React Framework}
Our browser-side session type encodings for browser-side targets build upon the
\emph{React.js} framework, developed by Facebook \cite{React} for the
\textit{Model-View-Controller} (MVC) architecture.
React is widely used in industry to create scalable single-page TypeScript
applications, and we intend for our proposed workflow to be beneficial in an
industrial context.
We introduce the key features of the framework.

\paragraph{Components}
A component is a reusable UI element which
contains its own markup and logic.
Components implement a \texttt{render()} function which returns a React
element, the smallest building blocks of a React application, analogous to the
view function in the MVU architecture.
Components can keep \textit{state}s and the \texttt{render()} function is
invoked upon a change of state.

For example, a simple counter can be implemented as a component,
with its \texttt{count} stored as state.
When rendered, it displays a button which increments \texttt{count}
when clicked and a \texttt{div} that renders the current
\texttt{count}.
If the button is clicked, the \texttt{count} is incremented, which triggers a
re-rendering (since the state has changed), and the updated \texttt{count} is
displayed.

Components can also render other components, which gives rise
to a parent/child relationship between components.
Parents can pass data to children as \textit{props} (short for properties).
Going back to the aforementioned example, the counter component could
render a child component \texttt{<StyledDiv count=\{this.state.count\} />} in
its \texttt{render()} function, propagating the \texttt{count} from its state
to the child.
This enables reusability, and for our use case, gives control to the parent
on what data to pass to its children (e.g. pass the payload of a received
message to a child to render).

\subsection{Browser-Side API Generation}

\label{section:browser}

\begin{wrapfigure}{R}{0.5\textwidth}
  \begin{center}
    \includegraphics[width=0.5\textwidth]{figures/efsm_p1.png}
  \end{center}

  \captionof{figure}{EFSM for \texttt{P1}.}
  \label{fig:efsmp1}
\end{wrapfigure}

We refer to the \texttt{P1} EFSM (\cref{fig:efsmp1}) as a running example in
this section.
Preserving behavioural typing and channel linearity is challenging
for browser-side applications due to EFSM transitions being triggered by user
events:
in the case of \textit{Noughts and Crosses}, once the user makes a move by
clicking on a cell on the game board, this click event must be deactivated
until the user's next turn, otherwise the user can click again and violate
channel linearity.
Our design goal is to enforce this statically through the generated APIs.

For browser-side targets, we extend the approach presented in \cite{MVU2019} on
\textit{multiple model types} motivated by the \textit{Model-View-Update} (MVU)
architecture.
% FEEDBACK: introduce MVU and model type
An MVU application features a \textit{model} encapsulating application
state, a \textit{view function} rendering the state on the Document Object Model (DOM), and an
\textit{update function} handling \textit{messages} produced by the
rendered model to produce a new model.
The concept of model types express type dependencies between these
components: a \emph{model type} uniquely defines a \textit{view function},
set of \textit{messages} and \textit{update function} -- rather than
producing a new model, the update function defines valid transitions to
other model types.
We leverage the correspondence between model types and states in the EFSM:
each state in the EFSM is a model type, the set of messages represent
the possible (IO) actions available at that state,
and the update function defines which successor state to transition to,
given the supported IO actions at this state.

\subsubsection{Model Types in React}

\paragraph{State}
An EFSM state is encoded as an \textit{abstract} React
component.
This is an abstract class to require the developer to provide their
own view function, which translates conveniently to the \texttt{render()}
function of React components.
Our session runtime (\cref{section:clientruntime}) ``executes'' the EFSM and
renders the current state.
Upon transitioning to a successor state, the successor's view function will be
invoked, as per the semantics expressed in \cite{MVU2019}.

\paragraph{Model transitions}
Transitions are encoded as React component props onto the encoded states by the
session runtime (\cref{section:clientruntime}).
We motivate the design choice of not exposing channel resources to provide
guarantees on channel usage.
React components in TypeScript are
\textit{generic} \cite{TypeScriptSpec}, parameterised by the permitted
types of prop and state.
The parameters allow us to leverage the TypeScript compiler to
verify that the props for model transitions stay local to the state they are
defined for.
The model transitions for EFSMs are message send and receive.

\subparagraph{Sending}
We make the assumption that message sending is triggered by
some user-driven UI event (e.g. clicking a button, pressing a key on the
keyboard) which interacts with some DOM element.
We could pass a
\texttt{send()} function as a prop to the sending state, but the developer
would be free to call the function multiple times which makes channel reuse
possible.
Instead, we pass a \textit{factory function} as a prop, which will,
given an HTML event and an event handler function, return a fresh React
component that binds the sending action on construction.
So once the bound event is triggered, our session runtime executes the event
handler function to obtain the payload to send, perform the send
\textit{exactly once} and transition to (which, in practice, means render) the
successor state.

\begin{figure}[!h]
\begin{lstlisting}[language=JavaScript, tabsize=4]
// Inside some render() function..
{board.map((row, x) => (
	row.map((col, y) => {
		const SelectPoint = this.props.Pos('click', (event: UIEvent) => {
			event.preventDefault();
			return { x: x, y: y };}
		return <SelectPoint><td>.</td></SelectPoint>;
});}
\end{lstlisting}
\captionof{lstlisting}{Model transition for message sending in
\textit{Noughts and Crosses} \texttt{P1} implementation.}
\label{lst:clientapp}
\end{figure}

We demonstrate the semantics using the \textit{Noughts and Crosses} example in
\cref{lst:clientapp}.
The session runtime passes the factory function \texttt{this.props.Pos} as a prop.
For each x-y coordinate on the game board, we
create a \texttt{SelectPoint} React component from the factory function (which
reads ``build a React component that sends the \texttt{Pos} message with x-y
coordinates as payload when the user clicks on it'') and we wrap a table cell
(the game board is rendered as an HTML table) inside the \texttt{SelectPoint}
component to bind the click event on the table cell.

\subparagraph{Receiving}
The React component for a receiving state is required to
define a handler for each supported branch. 
Upon a message receive event, the session runtime invokes the
handler of the corresponding branch with the message payload and 
renders the successor state upon completion.

\subsubsection{Session Runtime}
\label{section:clientruntime}

The session runtime can be interpreted as an abstraction on top of the React
VDOM that implements the EFSM by construction.
The session runtime itself is a React component too, named after the endpoint
role identifier:
it opens the WebSocket connection to the server, keeps track of the current
EFSM state as part of its React component state, and most importantly, renders
the React component encoding of the active EFSM state.
Channel communications are managed by the runtime, which allows it to render
the successor of a receive state upon receiving a message from the channel.
Similarly, the session runtime is responsible for passing the required props
for model transitions to EFSM state React components.
The session runtime component is rendered by the developer and requires, as
props, the \textit{endpoint URL} (so it can open the connection) and a list of
\textit{concrete state components}.

The developer writes their own implementation of each state (mainly to
customise how the state is rendered and inject business logic into state
transitions) by extending the abstract React class components.
The session runtime requires references to these concrete components in order to
render the user implementation accordingly.

\subsubsection{Affine Channel Usage}
A limitation of our browser-side session type encoding is only being able to
guarantee that channel resources are used \textit{at most once} as opposed to
\textit{exactly once}.

Communication channels are not exposed to the developer so multiple sends are
impossible.
This does not restrict the developer from binding the send action to exactly
one UI event: for \textit{Noughts and Crosses}, we bind the \texttt{Pos(Point)}
send action to each unoccupied cell on the game board, but the generated
runtime ensures that, once the cell is clicked, the send is only performed once
and the successor state is rendered on the DOM, so the channel resource used to
send becomes unavailable.

However, our approach \textit{does not} statically detect whether all
transitions in a certain state are bound to some UI event.
This means that it is possible for an implementation to \textit{not} handle
transitions to a terminal state but still type-check, so we cannot prevent
unused states. Equally, our approach does not prevent a client closing the browser, which would drop the connection.


\section{Summary}
We have presented \newtheory, a variant of
the canonical MPST theory to express \textit{routed communication}.
We introduced extensions to syntax and semantics, and proved
that our extended semantics preserve deadlock-freedom
for well-formed protocols. We defined an encoding from the
canonical theory onto \newtheory, and proved the preservation
of well-formedness and communication.