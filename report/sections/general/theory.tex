\chapter{\newtheory: A Theory of
Routed Multiparty Session Types}
\label{chap:theory}

In this chapter, we present
\newtheory, an extension of the
canonical multiparty session type (MPST)
theory with a message routing mechanism.
We build upon the variant of MPST theory introduced
in \cite{LessIsMore}
(which we have also discussed in\cref{subsection:bgmpst}), 
and extend it
with constructs for communication actions
that are routed through a participant.
We introduce the extended syntax for types in 
\cref{section:syntax}, and give semantics of our
syntax extensions in \cref{section:lts}.

\newtheory provides the theoretical basis
for allowing developers to implement
protocols with browser-to-browser interactions
(such as the \tprotocol{Two Buyer} protocol
introduced in \cref{section:twobuyer})
over a server-centric network topology
which uses WebSocket transport.
We define a router-parameterised encoding
for MPST into \newtheory in \cref{section:encoding}
and prove the preservation of well-formedness
and communication.
We proceed to extend \codegen to implement \newtheory
in \cref{chap:impl}.

\section{Syntax}
\label{section:syntax}

\subsection{Global Types}
As motivated by \ref{section:routerchallenges},
we need to handle \textit{routed communication} separately
from normal send and receive actions.
We define the syntax of global types for \newtheory 
in \cref{fig:newsyntaxglobal}, and explain the 
new construct for expressing routed communication.

\begin{figure}[!h]
\doublespacing
\[
\begin{array}{rlr}
G ::= & \dots & \text{Global Types} \\
\mid & \groute{p}{q}{s}{l_i: G_i}{i \in I}
& \text{Routed Communication through $\role{s}$} \\
\end{array}
\]
\singlespacing
\captionof{figure}{Global Types in \newtheory}
\label{fig:newsyntaxglobal}
\end{figure}

The new construct $\groute{p}{q}{s}{l_i: G_i}{i \in I}$ reads, 
\textit{
``the communication $\gcomm{p}{q}{l_i: G_i}{i \in I}$
between $\role{p}$ and $\role{q}$
is routed by $\role{s}$''
}.
We refer to $\role{s}$ hereafter as the \textit{router}.
$\role{s}$ ranges over the set of roles 
$\role{p},\role{q},\role{r},\role{s},\dots$,
but we use $\role{s}$ by convention as the server endpoint
is usually the router.
We also extend the definition of the participants function,
$\pt{G}$, in \cref{def:newpt}.

\begin{definition}[Participants]
\textit{The participants function 
represents the set of roles involved in
the communication specified by $G$.}

\doublespacing
\[
\begin{array}{>{\displaystyle}rc>{\displaystyle}l}

\pt{\dots} & = & \dots \\

\pt{\groute{p}{q}{s}{l_i: G_i}{i \in I}} & = & 
\left\{ \role{p}, \role{q}, \role{s} \right\} 
\cup \bigcup_{i \in I}\pt{G_i} \\

\end{array}
\]
\singlespacing
\label{def:newpt}
\end{definition}

\subsection{Local Types}
Local types are obtained from global types via
\textit{projection}, so we also need new
constructs to express routed communication
from the \textit{local} perspective of individual
endpoints.
We explain the extensions to
projection in \cref{subsection:newprojection}.

We define the syntax of local types for \newtheory 
in \cref{fig:newsyntaxlocal}, and walk through
the new constructs below.

\begin{figure}[!h]
\doublespacing
\[
\begin{array}{rlr}
T ::= & \dots & \text{Local Types} \\
\mid & \router{p}{q}{l_i: T_i}{i \in I}
& \text{Routing communication from $\role{p}$ to $\role{q}$} \\
\mid & \tselproxy{p}{q}{l_i: T_i}{i \in I}
& \text{Selecting from $\role{p}$ via proxy $\role{q}$} \\
\mid & \tbraproxy{p}{q}{l_i: T_i}{i \in I}
& \text{Receiving from $\role{p}$ via proxy $\role{q}$} \\
\end{array}
\]
\singlespacing
\captionof{figure}{Global Types in \newtheory}
\label{fig:newsyntaxlocal}
\end{figure}

\begin{itemize}

\item \textbf{Routed communication:}

\dots

\item \textbf{Proxy selection:}

\dots

\item \textbf{Proxy receive:}

\dots

\end{itemize}

% routing construct
% proxy send and receive


\subsection{Projection}
\label{subsection:newprojection}

Projection

Full Merge

% projection onto local types
% full merge

\subsection{Well-Formedness}
Recall that well-formedness is a predicate defined 
\textit{solely} on the global type
in canonical MPST theory: a global type is well-formed 
if a projection xists for all its participants.

\[
\dots
\]

In \newtheory, we require that there is \textit{exactly
one} router, say $\role{s}$, in the global type, so
we need to express that a global type to be
well-formed \textit{with respect to the role $\role{s}$
acting as the router}.



% act as server role


\begin{definition}[Centroid]
\textit{Let} $\centroid{G}{s}$ \textit{denote \dots}

\end{definition}

% well-formedness

\section{Labelled Transition System (LTS) Semanics}
\label{section:lts}

\begin{itemize}
\item extend action syntax
\end{itemize}

\subsection{LTS over Global Types}

\subsection{LTS over Local Types}

\subsection{LTS over Collections of Local Types}

\subsection{Trace Equivalence}

\subsection{Deadlock Freedom}

\section{Router-Parameterised Encoding}
\label{section:encoding}

\begin{itemize}
\item define encoding on global types
\item define encoding on local types
\item define encoding on LTS actions
\item motivate with twobuyer example
\end{itemize}

\subsection{Correspondence between Global and Local Type Encodings}
\begin{itemize}
\item prove proposition
\item state corollary for subsequent proof
\end{itemize}

\subsection{Preserving Well-formedness}

\subsection{Preserving Communication}