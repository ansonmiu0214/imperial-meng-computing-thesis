\chapter{\newtheory: A Theory of
Routed Multiparty Session Types}
\label{chap:theory}

In this chapter, we present
\newtheory, an extension of the
canonical multiparty session type (MPST)
theory with a message routing mechanism.
We build upon the variant of MPST theory introduced
in \cite{LessIsMore}
(which we have also discussed in \cref{subsection:bgmpst}), 
and extend it
with constructs for communication actions
that are routed through a participant.

\newtheory provide the theoretical basis
for allowing developers to implement
protocols with browser-to-browser interactions
(such as the \tprotocol{Two Buyer} protocol
introduced in \cref{section:twobuyer})
over a server-centric network topology
which uses WebSocket transport.
We introduce the extended syntax for types in 
\cref{section:syntax}, and give semantics of our
syntax extensions in \cref{section:lts}.
Regarding the extended semantics on routed session types,
we prove the soundness and completeness of
the extended semantics in \cref{section:newtraceeq}.
In \cref{section:encoding},
we define a router-parameterised encoding
for canonical MPST theory into \newtheory
and prove the preservation of well-formedness
and communication.
We proceed to extend \codegen to implement \newtheory
in \cref{chap:impl}.

\section{Syntax}
\label{section:syntax}

We extend the syntax for global types
in \cref{subsection:newglobal} and
local types in \cref{subsection:newlocal}.
The same conventions introduced in \cref{subsection:bgmpst}
apply here: 
we omit message payload types,
and deterministic send and receive actions
are represented by single selection and
single branching constructs respectively.
With these syntax extensions, we appropriately extend
the definition of \textit{projection} and 
\textit{well-formedness} in 
\cref{subsection:newprojection,subsection:newwf}
respectively.

\subsection{Global Types}
\label{subsection:newglobal}

As motivated by \cref{section:routerchallenges},
we need to handle \textit{routed communication} separately
from normal send and receive actions.
We define the syntax of global types for \newtheory 
in \cref{fig:newsyntaxglobal}, and explain the 
new construct for expressing routed communication.
Global types range over $G, G', G_i, \dots$

\begin{figure}[!h]
\doublespacing
\[
\begin{array}{rlr}
G ::= & \dots & \text{Global Types} \\
\mid & \groute{p}{q}{s}{l_i: G_i}{i \in I}
& \text{Routed Communication through $\role{s}$} \\
\end{array}
\]
\singlespacing
\captionof{figure}{Global Types in \newtheory}
\label{fig:newsyntaxglobal}
\end{figure}

The new construct $\groute{p}{q}{s}{l_i: G_i}{i \in I}$ reads, 
\textit{
``the communication $\gcomm{p}{q}{l_i: G_i}{i \in I}$
is routed by $\role{s}$''
}.
To be more verbose,
$\role{p}$ offers $\role{q}$ a set of choices
$\left\{l_i\right\}_{i \in I}$,
$\role{q}$ sends their choice to
$\role{s}$, so $\role{p}$ receives the
selection $l_i$ made by $\role{q}$ via $\role{s}$,
and the communication system proceeds with continuation $G_i$.
We refer to $\role{s}$ hereafter as the \textit{router}.
$\role{s}$ ranges over the set of roles 
$\role{p},\role{q},\role{r},\role{s},\dots$,
but we use $\role{s}$ by convention as the server endpoint
is usually the router.

We also extend the definition of the participants function,
$\pt{G}$, in \cref{def:newpt}, to formalise that the router 
participates in the routed communication.

\begin{definition}[Participants]
The set of roles involved in the communication
interactions specified by $G$, written $\pt{G}$, is:

\doublespacing
\[
\begin{array}{>{\displaystyle}rc>{\displaystyle}l}

\pt{\dots} & = & \dots \\

\pt{\groute{p}{q}{s}{l_i: G_i}{i \in I}} & = & 
\left\{ \mrole{p}, \mrole{q}, \mrole{s} \right\} 
\cup \bigcup_{i \in I}\pt{G_i} \\

\end{array}
\]
\singlespacing
\label{def:newpt}
\end{definition}

\subsection{Local Types}
\label{subsection:newlocal}

Local types are obtained from global types via
\textit{projection}, so we also need new
constructs to express routed communication
from the \textit{local} perspective of individual
endpoints.
We explain the extensions to
projection in \cref{subsection:newprojection}.

We define the syntax of local types for \newtheory 
in \cref{fig:newsyntaxlocal}, and walk through
the new constructs below from the perspective of
some arbitrary role $\role{r}$.
Local types range over $T, T', T_i, \dots$

\begin{figure}[!h]
\doublespacing
\[
\begin{array}{rlr}
T ::= & \dots & \text{Local Types} \\
\mid & \router{p}{q}{l_i: T_i}{i \in I}
& \text{Routing communication from $\role{p}$ to $\role{q}$} \\
\mid & \tselproxy{p}{q}{l_i: T_i}{i \in I}
& \text{Selecting from $\role{p}$ via proxy $\role{q}$} \\
\mid & \tbraproxy{p}{q}{l_i: T_i}{i \in I}
& \text{Receiving from $\role{p}$ via proxy $\role{q}$} \\
\end{array}
\]
\singlespacing
\captionof{figure}{Global Types in \newtheory}
\label{fig:newsyntaxlocal}
\end{figure}

\begin{itemize}

\item \textbf{Routed communication:}

If $\role{r}$ has local type $\router{p}{q}{l_i: T_i}{i \in I}$,
$\role{r}$ is routing the communication between 
$\role{p}$ and $\role{q}$.
$\role{r}$ routes the message $l_i$ sent by $\role{p}$
to the intended recipient $\role{q}$ and proceeds 
with continuation $T_i$.

\item \textbf{Proxy selection:}

If $\role{r}$ has local type $\tselproxy{p}{q}{l_i: T_i}{i \in I}$,
$\role{r}$ is making a selection from the
\textit{intended recipient} $\role{p}$,
but the selection is sent to the intermediate proxy $\role{q}$. 
$\role{r}$ sends their internal choice $l_i$ to $\role{q}$
and proceeds with continuation $T_i$.

\item \textbf{Proxy branching:}

If $\role{r}$ has local type $\tbraproxy{p}{q}{l_i: T_i}{i \in I}$,
$\role{r}$ is offering a choice to the
\textit{intended sender} $\role{p}$,
but the message is received from the intermediate proxy $\role{q}$.
$\role{r}$ receives the external choice $l_i$ from $\role{q}$
and proceeds with continuation $T_i$.

\end{itemize}

As a general comment, the new local types of proxy selection
and proxy branching should ``behave'' as normal selection and
receive types with respect to their intended recipient and sender
respectively. We keep track of the proxy role in the syntax
to allow us to distinguish between routing communications
from normal send and receive interactions.

\subsection{Projection}
\label{subsection:newprojection}

We extend the projection operator to be defined
on routed communication. Given the explanations of the
new routing constructs for both global and local types,
\cref{def:newprojection} should be self-explanatory.

\begin{definition}[Projection]
The projection of $G$ onto $\role{r}$,
written $\proj{G}{r}$, is defined as:

\doublespacing
\[
\begin{array}{>{\displaystyle}rc>{\displaystyle}l}

\proj{
\left(\groute{p}{q}{s}{l_i: G_i}{i \in I}\right)
}{r} & = & 
\begin{cases}
\tselproxy{q}{s}{l_i: (\proj{G_i}{r})}{i \in I}
	& \text{if} ~ \mrole{r} = \mrole{p} \\
\tbraproxy{p}{s}{l_i: (\proj{G_i}{r})}{i \in I}
	& \text{if} ~ \mrole{r} = \mrole{q} \\
\router{p}{q}{l_i: (\proj{G_i}{r})}{i \in I}
	& \text{if} ~ \mrole{r} = \mrole{s} \\
\underset{i \in I}{\MERGEOP}\proj{G_i}{r}
	& \text{otherwise} \\
\end{cases}
\\

\end{array}
\]
\singlespacing

\label{def:newprojection}
\end{definition}

As shown in the last (4th) case, 
the same concept of \textit{merging}
applies for routed communication:
when projecting a routed communication
onto a non-participant, the projections of
all continuations must be ``compatible'',
namely they can be merged using the
the \textit{merging operator}, $\mergeop$.

As the merging operator is defined on local
types, we extend the merging operator to be defined
on the extended syntax in \cref{def:newmerge}.
Recall that proxy selection and proxy
branching behave in the same way as their ``non-proxified''
counterparts -- the merging operator reflects this
similarity.

\begin{definition}[Merging Operator]
The merging operator $\mergeop$ on local types
is defined as:

\doublespacing
\[
\begin{array}{rcl}
\tmerge
{(\tselproxy{p}{q}{l_i: T_i}{i \in I})}
{(\tselproxy{p}{q}{l_i: T_i}{i \in I})}
	& = & \tselproxy{p}{q}{l_i: T_i}{i \in I} \\
	
\tmerge
{(\tbraproxy{p}{q}{l_i: T_i}{i \in I})}
{(\tbraproxy{p}{q}{l_j: T'_j}{j \in J})}
	& = & \tbraproxy{p}{q}{l_k: T''_k}{k \in I \cup J} \\
\text{where} & & T''_k = \begin{cases}
T_k & \text{if} ~ k \in I \setminus J \\
T'_k & \text{if} ~ k \in J \setminus I \\
\tmerge{T_k}{T'_k} & \text{if} ~ k \in I \cap J \\
\end{cases} \\
\end{array}
\]
\singlespacing

\label{def:newmerge}
\end{definition}

\subsection{Well-formedness}
\label{subsection:newwf}

Recall that well-formedness is a predicate defined 
\textit{solely} on the global type
in canonical MPST theory: a global type $G$ is well-formed
if a projection exists for all its participants.

\[
\wf{G} \iff 
\forall \mrole{p} \in \pt{G}. ~ \proj{G}{p} \text{ exists}
\]

In \newtheory, we require that there is \textit{exactly
one} router, say $\role{s}$, in the global type, so
we need to express that a global type to be
well-formed \textit{with respect to the role $\role{s}$
acting as the router}.

We need to define the characteristics that $\role{s}$
must display in $G$ to prove that it is the \textit{only}
router. 
We formalise this as an \textit{inductively} defined relation,
$\centroid{G}{s}$, which reads \textit{``$\role{s}$ is the
centroid in $G$''}.
The intuition is that $\role{s}$ is at the centre
of all communication interactions.
We define what it means to be a \textit{centroid} in
\cref{def:centroid}.

% characteristics of router as centroid
\begin{definition}[Centroid]
Let $\centroid{G}{s}$ denote \dots

\begin{prooftree}
\AxiomC{}
\RightLabel{\rulename{$\centroidop$-End}}
\UnaryInfC{$\centroid{\tend}{s}$}
\end{prooftree} 

\begin{prooftree}
\AxiomC{}
\RightLabel{\rulename{$\centroidop$-RecVar}}
\UnaryInfC{$\centroid{\trecvar}{s}$}
\end{prooftree}

\begin{prooftree}
\AxiomC{$\centroid{G}{s}$}
\RightLabel{\rulename{$\centroidop$-Rec}}
\UnaryInfC{$\centroid{\trec{G}}{s}$}
\end{prooftree}

\begin{prooftree}
\AxiomC{$\mrole{s} \in \left\{ \mrole{p}, \mrole{q} \right\}$}
\AxiomC{$\forall i \in I. ~ \centroid{G_i}{s}$}
\RightLabel{\rulename{$\centroidop$-Comm}}
\BinaryInfC{$\centroid{\gcomm{p}{q}{l_i: G_i}{i \in I}}{s}$}
\end{prooftree}

\begin{prooftree}
\AxiomC{$\mrole{r} = \mrole{s}$}
\AxiomC{$\forall i \in I. ~ \centroid{G_i}{s}$}
\RightLabel{\rulename{$\centroidop$-RouteComm}}
\BinaryInfC{$\centroid{\groute{p}{q}{r}{l_i: G_i}{i \in I}}{s}$}
\end{prooftree}

\label{def:centroid}
\end{definition}

\begin{itemize}
\item 
\rulename{$\centroidop$-End}, 
\rulename{$\centroidop$-RecVar}, 
\rulename{$\centroidop$-Rec}:
Trivial.

\item 
\rulename{$\centroidop$-Comm}:
For normal communication, $\role{s}$ must
be a participant and the centroid of all continuations.

\item 
\rulename{$\centroidop$-RoutedComm}:
For routed communication, $\role{s}$ must
be the router and be the centroid of all continuations.

\end{itemize}

Now we are in a position to formalise the definition
of well-formedness in \newtheory. We present this in
\cref{def:newwf}.

\begin{definition}[Well-formedness]
Let $\wfnew{G}{s}$ 
denote that the global type $G$
is well-formed with respect to the router $\role{s}$.

\[
\wfnew{G}{s} \iff 
(\forall \role{p} \in \pt{G}. ~ \proj{G}{p} \text{ exists})
\wedge
\centroid{G}{s}
\]

\label{def:newwf}
\end{definition}

\section{Labelled Transition System (LTS) Semantics}
\label{section:lts}

We define the labelled transition system (LTS) semantics
over global types (\cref{subsection:newltsglobal}) 
and local types (\cref{subsection:newltslocal})
for \newtheory,
building upon the work of Deni\'elou and Yoshida
\cite{characterisation}.
We show the soundness and completeness of projection
with respect to the LTSs through proving the
\textit{trace equivalence} of
a global type and the collection of local types projected
from the global type (\cref{subsection:newtraceeq}).
We then use this result to
conclude that \newtheory provide the same
communication safety guarantees from canonical
MPST theory for well-formed global types, 
namely deadlock freedom (\cref{subsection:newdeadlockfreedom}).

First, we extend the label in the LTS, as shown in 
\cref{fig:newlts}, to distinguish
the \textit{direct} sending (and reception) of a message
from the sending (and reception) of a message
\textit{via} an intermediate routing endpoint.
Labels range over $l, l', \dots$
We highlight and explain the new labels.

\begin{figure}[!h]
\doublespacing
\[
\begin{array}{rlr}
l ::= & & \text{Labels} \\
	\mid & \aout{p}{q}{j} & 
		\text{Direct Send} \\
	\mid & \ain{p}{q}{j} & 
		\text{Direct Receive} \\
	\mid & \via{s}{\aout{p}{q}{j}} & 
		\text{\hl{Routed Send}} \\
	\mid & \via{s}{\ain{p}{q}{j}} & 
		\text{\hl{Routed Receive}} \\
\end{array}
\]
\singlespacing
\captionof{figure}{LTS Labels in \newtheory}
\label{fig:newlts}
\end{figure}

\begin{itemize}
\item \textbf{Routed Send:}

The label $\via{s}{\aout{p}{q}{j}}$ represents the
\textit{sending} (performed by $\role{p}$)
of a message labelled $j$ to $\role{q}$ through
the intermediate router $\role{s}$.

\item \textbf{Routed Receive:}

The label $\via{s}{\ain{p}{q}{j}}$ represents the
\textit{reception} (initiated by $\role{q}$) 
of a message labelled $j$
send from $\role{p}$ through
the intermediate router $\role{s}$.
\end{itemize}

Labels represent communication actions, so we refer
to $l$ as labels and actions interchangeably,
as is the case in the literature.

Building upon \cite{characterisation},
the \textit{subject} of a label is the role
that initiates the action. 
Intuitively, the actions for routed send and routed
receive are still initiated by the original sender and
recipient respectively;
we extend the definition
of subjects in \cref{def:newsubj} to reflect this.

\begin{definition}[Subject]
The subject of a LTS label, or $\subj{l}$, is defined as:

\doublespacing
\[
\begin{array}{c}
\subj{\via{s}{\aout{p}{q}{j}}} = 
	\subj{\aout{p}{q}{j}} = \mrole{p} \\
\subj{\via{s}{\ain{p}{q}{j}}} = 
	\subj{\ain{p}{q}{j}} = \mrole{q} \\
\end{array}
\]
\singlespacing
\label{def:newsubj}
\end{definition}

\subsection{LTS Semantics over Global Types}
\label{subsection:newltsglobal}

The LTS semantics presented in \cite{characterisation}
models \textit{asynchronous communication},
which is consistent with our proposal.
In order to define LTS over global types for
asynchronous communication, we need to
represent intermediate states (i.e. messages in transit)
within the grammar of global types.
Deni\'elou and Yoshida \cite{characterisation}
added the construct
{$\gtrans{p}{q}{j}{l_i: G_i}{i \in I}$}
to represent that the message $l_j$ has been
sent by $\role{p}$ but not yet received by $\role{q}$.

We add a similar construct
{$\gtransroute{p}{q}{s}{j}{l_i: G_i}{i \in I}$}
to represent that the message $l_j$ has
been sent from $\role{p}$ to the router $\role{s}$
\textit{but not yet routed to $\role{q}$}.
We extend the projection operator to 
support this new construct.

\[
\proj{\gtransroute{p}{q}{s}{j}{l_i: G_i}{i \in I}}{r} = \begin{cases}
\tbraproxy{p}{s}{l_i: \proj{G_i}{r}}{i \in I} 
	& \text{if } \mrole{r} = \mrole{q} \\
\routertrans{p}{q}{j}{l_i: \proj{G_i}{r}}{i \in I}
	& \text{if } \mrole{r} = \mrole{s} \\
\proj{G_j}{r}
	& \text{otherwise} \\
\end{cases}
\]

Because the router $\role{s}$ holds
Because the router holds a ``global perspective''
on the routed communication, we also need to
represent the intermediate state (at which the message is 
being routed by the router) within the grammar
of \textit{local types}.
We extend the grammar of local types with the construct
$\routertrans{p}{q}{j}{l_i: T_i}{i \in I}$
to represent that, from the local perspective of
the router, the message $l_j$
has been received from $\role{p}$ \textit{but not yet routed
to $\role{q}$}.

Because routed communication is treated differently
from direct send and receive actions, the notion
of asynchrony differs between the two types of communication
too. This definition allows us to extend
the LTS semantics from \cite{characterisation}
more naturally.

We define the LTS semantics 
over global types,
denoted by $\treduce{G}{G'}{l}$,
in \cref{fig:newglobal}.
We highlight and explain the new rules.\\

\begin{figure}[!h]

\begin{prooftree}
\AxiomC{}
\RightLabel{\rulename{Gr1}}
\UnaryInfC{$
\treducelong
	{\gcomm{p}{q}{l_i: G_i}{i \in I}}
	{\gtrans{p}{q}{j}{l_i: G_i}{i \in I}}
	{\aout{p}{q}{j}}
$}
\end{prooftree}

\begin{prooftree}
\AxiomC{}
\RightLabel{\rulename{Gr2}}
\UnaryInfC{$
\treducelong
	{\gtrans{p}{q}{j}{l_i: G_i}{i \in I}}
	{G_j}
	{\ain{p}{q}{j}}
$}
\end{prooftree}

\begin{prooftree}
\AxiomC{$
\treduce
	{G[\trec{G} / \trecvar]}
	{G'}
	{l}
$}
\RightLabel{\rulename{Gr3}}
\UnaryInfC{$
\treduce
	{\trec{G}}
	{G'}
	{l}
$}
\end{prooftree}

\begin{prooftree}
\AxiomC{$\forall i \in I. ~ \treduce{G_i}{G'_i}{l}$}
\AxiomC{$\subj{l} \notin \{\mrole{p}, \mrole{q}\}$}
\RightLabel{\rulename{Gr4}}
\BinaryInfC{$
\treducelong
	{\gcomm{p}{q}{l_i: G_i}{i \in I}}
	{\gcomm{p}{q}{l_i: G'_i}{i \in I}}
	{l}
$}
\end{prooftree}

\begin{prooftree}
\AxiomC{$\treduce{G_j}{G'_j}{l}$}
\AxiomC{$\subj{l} \neq \mrole{q}$}
\AxiomC{$\forall i \in I \setminus \{ j \}. ~ G'_i = G_i$}
\RightLabel{\rulename{Gr5}}
\TrinaryInfC{$
\treducelong
	{\gtrans{p}{q}{j}{l_i: G_i}{i \in I}}
	{\gtrans{p}{q}{j}{l_i: G'_i}{i \in I}}
	{l}
$}
\end{prooftree}

\begin{prooftree}
\AxiomC{}
\RightLabel{\hlrulename{Gr6}}
\UnaryInfC{$
\treducelong
	{\groute{p}{q}{s}{l_i: G_i}{i \in I}}
	{\gtransroute{p}{q}{s}{j}{l_i: G_i}{i \in I}}
	{\via{s}{\aout{p}{q}{j}}}
$}
\end{prooftree}

\begin{prooftree}
\AxiomC{}
\RightLabel{\hlrulename{Gr7}}
\UnaryInfC{$
\treducelong
	{\gtransroute{p}{q}{s}{j}{l_i: G_i}{i \in I}}
	{G_j}
	{\via{s}{\ain{p}{q}{j}}}
$}
\end{prooftree}

\begin{prooftree}
\AxiomC{$\forall i \in I. ~ \treduce{G_i}{G'_i}{l}$}
\AxiomC{$\subj{l} \notin \{\mrole{p}, \mrole{q}\}$}
\RightLabel{\hlrulename{Gr8}}
\BinaryInfC{$
\treducelong
	{\groute{p}{q}{s}{l_i: G_i}{i \in I}}
	{\groute{p}{q}{s}{l_i: G'_i}{i \in I}}
	{l}
$}
\end{prooftree}

\begin{prooftree}
\AxiomC{$\treduce{G_j}{G'_j}{l}$}
\AxiomC{$\subj{l} \neq \mrole{q}$}
\AxiomC{$\forall i \in I \setminus \{ j \}. ~ G'_i = G_i$}
\RightLabel{\hlrulename{Gr9}}
\TrinaryInfC{$
\treducelong
	{\gtransroute{p}{q}{s}{j}{l_i: G_i}{i \in I}}
	{\gtransroute{p}{q}{s}{j}{l_i: G'_i}{i \in I}}
	{l}
$}
\end{prooftree}

\captionof{figure}{LTS Semantics over Global Types in \newtheory}
\label{fig:newglobal}
\end{figure}

\begin{itemize}

\item \rulename{Gr6} and \rulename{Gr7}
are analogue to \rulename{Gr1} and \rulename{Gr2}
for describing the emission and reception of
messages in routed communication, but uses
the ``routed in-transit'' construct instead.

\item \rulename{Gr8} and \rulename{Gr9}
are analogue to \rulename{Gr4} and \rulename{Gr5}
in the sense that we only enforce the
syntactic order of messages for the participants
involved in the action $l$.

\end{itemize}

An important observation from
\rulename{Gr8} and \rulename{Gr9} is that,
for the router,
the syntactic order of routed communication
can be freely interleaved between
the syntactic order of direct communication.
This is a crucial result for proving
that the router does not over-serialise
communication -- 
we show this in \cref{subsection:encodepreservecomm}.

\subsection{LTS Semantics over Local Types}
\label{subsection:newltslocal}

We define the LTS semantics 
over local types,
denoted by $\treduce{T}{T'}{l}$,
in \cref{fig:newlocal}.
We highlight and explain the new rules.

\begin{figure}[!h]

\begin{prooftree}
\AxiomC{}
\RightLabel{\rulename{Lr1}}
\UnaryInfC{$
\treducelong
	{\tsel{q}{l_i: T_i}{i \in I}}
	{T_j}
	{\aout{p}{q}{j}}
$}
\end{prooftree}

\begin{prooftree}
\AxiomC{}
\RightLabel{\rulename{Lr2}}
\UnaryInfC{$
\treducelong
	{\tbra{q}{l_i: T_i}{i \in I}}
	{T_j}
	{\ain{q}{p}{j}}
$}
\end{prooftree}

\begin{prooftree}
\AxiomC{$
\treduce
	{T[\trec{T} / \trecvar]}
	{T'}
	{l}
$}
\RightLabel{\rulename{Lr3}}
\UnaryInfC{$
\treduce
	{\trec{T}}
	{T'}
	{l}
$}
\end{prooftree}

\begin{prooftree}
\AxiomC{}
\RightLabel{\hlrulename{Lr4}}
\UnaryInfC{$
\treducelong
	{\tselproxy{q}{s}{l_i: T_i}{i \in I}}
	{T_j}
	{\via{s}{\aout{p}{q}{j}}}
$}
\end{prooftree}

\begin{prooftree}
\AxiomC{}
\RightLabel{\hlrulename{Lr5}}
\UnaryInfC{$
\treducelong
	{\tbraproxy{q}{s}{l_i: T_i}{i \in I}}
	{T_j}
	{\via{s}{\ain{q}{p}{j}}}
$}
\end{prooftree}

\begin{prooftree}
\AxiomC{}
\RightLabel{\hlrulename{Lr6}}
\UnaryInfC{$
\treducelong
	{\router{p}{q}{l_i: T_i}{i \in I}}
	{\routertrans{p}{q}{j}{l_i: T_i}{i \in I}}
	{\via{s}{\aout{p}{q}{j}}}
$}
\end{prooftree}

\begin{prooftree}
\AxiomC{}
\RightLabel{\hlrulename{Lr7}}
\UnaryInfC{$
\treducelong
	{\routertrans{p}{q}{j}{l_i: T_i}{i \in I}}
	{T_j}
	{\via{s}{\ain{p}{q}{j}}}
$}
\end{prooftree}

\begin{prooftree}
\AxiomC{$\forall i \in I. ~ \treduce{T_i}{T'_i}{l}$}
\AxiomC{$\subj{l} \notin \{\mrole{p}, \mrole{q}\}$}
\RightLabel{\hlrulename{Lr8}}
\BinaryInfC{$
\treducelong
	{\router{p}{q}{l_i: T_i}{i \in I}}
	{\router{p}{q}{l_i: T'_i}{i \in I}}
	{l}
$}
\end{prooftree}

\begin{prooftree}
\AxiomC{$\treduce{T_j}{T'_j}{l}$}
\AxiomC{$\subj{l} \neq \mrole{q}$}
\AxiomC{$\forall i \in I \setminus \{ j \}. ~ T'_i = T_i$}
\RightLabel{\hlrulename{Lr9}}
\TrinaryInfC{$
\treducelong
	{\routertrans{p}{q}{j}{l_i: T_i}{i \in I}}
	{\routertrans{p}{q}{j}{l_i: T'_i}{i \in I}}
	{l}
$}
\end{prooftree}

\begin{prooftree}
\AxiomC{$l = \via{s}{\cdot}$}
\AxiomC{$\subj{l} \neq \mrole{q}$}
\AxiomC{$\forall i \in I. ~ \treduce{T_i}{T'_i}{l}$}
\RightLabel{\hlrulename{Lr10}}
\TrinaryInfC{$
\treducelong
	{\tsel{q}{l_i: T_i}{i \in I}}
	{\tsel{q}{l_i: T'_i}{i \in I}}
	{l}
$}
\end{prooftree}

\begin{prooftree}
\AxiomC{$l = \via{s}{\cdot}$}
\AxiomC{$\subj{l} \neq \mrole{q}$}
\AxiomC{$\forall i \in I. ~ \treduce{T_i}{T'_i}{l}$}
\RightLabel{\hlrulename{Lr11}}
\TrinaryInfC{$
\treducelong
	{\tbra{q}{l_i: T_i}{i \in I}}
	{\tbra{q}{l_i: T'_i}{i \in I}}
	{l}
$}
\end{prooftree}

\captionof{figure}{LTS over Local Types in \newtheory}
\label{fig:newlocal}
\end{figure}

We walk through rules \rulename{Lr4} and \rulename{Lr5}
from the perspective of role $\role{p}$.

\begin{itemize}

\item \rulename{Lr4} and \rulename{Lr5} are
analogue to \rulename{Lr1} and \rulename{Lr2}
for sending and receiving messages respectively.
The exception is that the new rules pattern-match
on the router role $\role{s}$ on the local type
and the routed label.

\end{itemize}

We walk through rules \rulename{Lr6}, \rulename{Lr7},
\rulename{Lr10} and \rulename{Lr11}
from the perspective of role $\role{s}$.

\begin{itemize}

\item \rulename{Lr6} and \rulename{Lr7} are
analogue to \rulename{Gr1} and \rulename{Gr2}.
Intuitively, the router $\role{s}$ holds a
``global'' perspective on the interaction
between $\role{p}$ and $\role{q}$, which explains
the correspondence with the LTS semantics over global
types.

\item \rulename{Lr10} and \rulename{Lr11} allow
the router to perform routing actions before handling
their own direct communication. The syntax $l = \via{s}{\cdot}$
means that the label $l$ is ``of the form'' of a routing
action, i.e. there exists some $\role{p}, \role{q}, j$
such that $l = \via{s}{\aout{p}{q}{j}}$ or 
$l = \via{s}{\ain{p}{q}{j}}$.
The constraint of $\subj{l} \neq \mrole{q}$
prevents the violation of the syntactic order of messages
sent and received by $\mrole{q}$.

Curious readers can consider the examples 
$\proj{G_1}{s}$ and $\proj{G_2}{s}$
to see why this constraint is needed.

\begin{align*}
G_1 &= \gcommone{s}{r}{\tmsg{M1}}.~ \grouteone{r}{q}{s}{\tmsg{M2}}.~ \tend \\
\proj{G_1}{s} &= \tselone{r}{\tmsg{M1}}. ~ 
	\routerone{r}{q}{\tmsg{M2}}{\tend} \\~\\
G_2 &= \gcommone{s}{r}{\tmsg{M1}}.~ \grouteone{p}{r}{s}{\tmsg{M2}}.~ \tend \\
\proj{G_2}{s} &= \tselone{r}{\tmsg{M1}}. ~ 
	\routerone{p}{r}{\tmsg{M2}}{\tend} \\
\end{align*}

\end{itemize}

As for the remaining rules,
\rulename{Lr8} and \rulename{Lr9} are analogue to
\rulename{Gr4} and \rulename{Gr5} because the router
holds a ``global'' perspective on the communication,
so transitions that do not violate the syntactic order
of messages between roles $\mrole{p}$ and $\mrole{q}$
are allowed.


\section{LTS Soundness and Completeness with respect to Projection}
\label{section:newtraceeq}

We work towards proving the soundness and completeness
of our LTS semantics with respect to projection.
Our approach is motivated from \cite{characterisation}:

\begin{enumerate}

\item We first extend the LTS semantics to
a collection of local types (hereafter referred to
as a \textit{configuration} to be consistent with
the literature)
in \cref{subsection:newltsconfig};

\item Then, we extend the definition of projection to
obtain the configuration of a global type 
(hereafter referred to as the \textit{projected configuration})
in \cref{subsection:newltsprojection};

\item Finally, we prove the trace equivalence between 
the global type and its projected configuration 
in \cref{subsection:newtraceeq}.

\end{enumerate}

We use the trace equivalence result to prove deadlock freedom
in \cref{subsection:newdeadlockfreedom}.

\subsection{LTS Semantics over Configurations}
\label{subsection:newltsconfig}

Let $\mathcal{P}$ denote the set of participants in
the communication automaton.
Also let $\lty{p}$ denote the local type of a participant
$\pinP$.

A \textit{configuration} describes the
state of the communication automaton with respect to
each participant $\pinP$.
By definition of our LTS semantics, this includes
\textit{intermediate} states, so a configuration
would also need to express the state of messages
in transit.

We inherit the definition from \cite{characterisation},
restated in \cref{def:newconfig}.

\begin{definition}[Configuration]
A configuration $s = (\vec T; ~ \vec w)$ of a system of
local types $\{ \lty{p} \}_{\pinP}$
is defined as a pair of:

\begin{itemize}

\item $\vec T = (\lty{p})_{\pinP}$
is the collection of local types.
$\lty{p}$ describes the communication structure
from the local perspective of participant $\pinP$.

\item $\vec w = (w_{\mroles{p}{q}})_{\mrole{p} \neq \mrole{q} \in \mathcal{P}}$
is the collection of \textit{unbounded buffers}.
The unbounded buffer $w_{\mrole{p}\mrole{q}}$ represents a (FIFO)
queue of messages sent by $\role{p}$ but not yet
received by $\role{q}$.

\end{itemize}

\label{def:newconfig}
\end{definition}

\begin{remark}

The \textit{subtyping} relation defined on
local types (see \cref{subsection:bgbst})
can be extended to configurations:

\begin{prooftree}
\AxiomC{$\vec w = \vec w'$}
\AxiomC{$
\forall \mrole{p} \in \mathcal{P}. ~ 
T_\mrole{p} \subtype T'_\mrole{p}
$}
\BinaryInfC{$
(\vec T; \vec w) \subtype (\vec T'; \vec w')
$}
\end{prooftree}

\end{remark}

We proceed to define the LTS over configurations in 
\cref{def:newltsconfig}, highlighting the extensions
required for \newtheory.

\begin{definition}[LTS Semantics over Configurations]
The LTS semantics over configurations is defined by
the relation $\treduce{s_T}{s'_T}{l}$.

Let $s_T = (\vec T; ~ \vec w)$ and $s'_T = (\vec T'; ~ \vec w')$.
We define the specific transitions on $\vec T$ and $\vec w$
by case analysis on the label $l$.

\begin{itemize}

\item $l = \aout{p}{q}{j}$

Then $\treduce{T_\mrole{p}}{T'_\mrole{p}}{l}$ 
because $\role{p}$
initiates the action, so
$T'_\mrole{p'} = T_{\mrole{p'}}$ 
for all $\mrole{p'} \neq \mrole{p}$.

The message $j$ is in transit from $\role{p}$ to $\role{q}$, 
so $w'_{\mroles{p}{q}} = w_{\mroles{p}{q}} \cdot j$
($j$ is appended to the queue of in-transit messages
sent from $\role{p}$ to $\role{q}$),
and unrelated buffers $w'_{\mroles{p'}{q'}} = w_{\mroles{p}{q}}$ 
are untouched for all $\mroles{p'}{q'} \neq \mroles{p}{q}$.

\item $l = \ain{p}{q}{j}$

Then $\treduce{T_\mrole{q}}{T'_\mrole{q}}{l}$ 
because $\role{q}$
initiates the action, so
$T'_\mrole{p'} = T_{\mrole{p'}}$ 
for all $\mrole{p'} \neq \mrole{q}$.

The message $j$ is no longer in transit
from $\role{p}$ to $\role{q}$ as it is received by $\role{q}$,
so $w_{\mroles{p}{q}} = j \cdot w'_{\mroles{p}{q}}$ 
($j$ is removed from the front of the queue of in-transit
messages sent from $\role{p}$ to $\role{q}$),
and unrelated buffers $w'_{\mroles{p'}{q'}} = w_{\mroles{p}{q}}$ 
are untouched for all $\mroles{p'}{q'} \neq \mroles{p}{q}$.

\item \hl{$l = \via{s}{\aout{p}{q}{j}}$}

Then $\treduce{T_\mrole{p}}{T'_\mrole{p}}{l}$ 
because $\role{p}$
initiates the action.
Because the send action is routed, we also need 
$\treduce{T_\mrole{s}}{T'_\mrole{s}}{l}$.
This means
$T'_\mrole{p'} = T_{\mrole{p'}}$ 
for all $\mrole{p'} \notin \{\mrole{p} , \mrole{s} \}$.

The message $j$ is in transit from $\role{p}$ to $\role{q}$, 
so $w'_{\mroles{p}{q}} = w_{\mroles{p}{q}} \cdot j$
and unrelated buffers $w'_{\mroles{p'}{q'}} = w_{\mroles{p}{q}}$ 
are untouched for all $\mroles{p'}{q'} \neq \mroles{p}{q}$.

\item \hl{$l = \via{s}{\ain{p}{q}{j}}$}

Then $\treduce{T_\mrole{q}}{T'_\mrole{q}}{l}$ 
because $\role{q}$
initiates the action.
Because the receive action is routed, we also need
$\treduce{T_\mrole{s}}{T'_\mrole{s}}{l}$.
This means
$T'_\mrole{p'} = T_{\mrole{p'}}$ 
for all $\mrole{p'} \notin \{\mrole{q} , \mrole{s} \}$.

The message $j$ is no longer in transit
from $\role{p}$ to $\role{q}$ as it is received by $\role{q}$,
so $w_{\mroles{p}{q}} = j \cdot w'_{\mroles{p}{q}}$,
and unrelated buffers $w'_{\mroles{p'}{q'}} = w_{\mroles{p}{q}}$ 
are untouched for all $\mroles{p'}{q'} \neq \mroles{p}{q}$.

\end{itemize}

\label{def:newltsconfig}
\end{definition}

Routed actions are carried out by the router,
so it makes sense for the local type of the router 
to also makes a step.
The semantics of the message buffers for routed actions
are the same as their non-routed counterparts; the
only difference is that these message buffers are ``managed''
by the router, but this is a change of interpretation
which isn't reflected in the theory.

\subsection{Extending Projection for Configurations}
\label{subsection:newltsprojection}

When considering the grammar of global types $G$
extended to include intermediate states,
we can obtain the \textit{projected configuration}
from a global type $G$ with participants $\mathcal{P}$:

\[
\projconf{G} = 
\left(
	\{ \proj{G}{p} \}_{\pinP} ~ ; ~
	\projconf{G}_{\{ \epsilon \}_{\qqinP}}
\right)
\]

The collection of local types is obtained by
projecting $G$ onto each participant $\pinP$.
The contents of the buffers is defined as
$\projconf{G}_{\{ w_{\mroles{q}{q'}} \}_{\qqinP}}$.
We inherit the definitions presented in \cite{characterisation},
and introduce additional rules in
\cref{fig:buffer}.

\begin{figure}[!h]
\doublespacing
\[
\begin{array}{>{\displaystyle}rc>{\displaystyle}l}

\projconf{
\gtransroute{p}{p'}{s}{j}{l_i: G_i}{i \in I}
}_{\{ w_{\mroles{q}{q'}} \}_{\qqinP}} 
	& = & \projconf{G_j}_{
	\{ w_{\mroles{q}{q'}} \}_{\qqinP}
	[w_{\mroles{p}{p'}} \mapsto w_{\mroles{p}{p'}} \cdot j]
	} \\
	
\projconf{
\groute{p}{p'}{s}{l_i: G_i}{i \in I}
}_{\{ w_{\mroles{q}{q'}} \}_{\qqinP}} 
	& = & \projconf{G_1}_{
	\{ w_{\mroles{q}{q'}} \}_{\qqinP}
	} \\
	
\end{array}
\]
\singlespacing

\captionof{figure}{Projection of Buffer Contents from Global Type in
\newtheory}
\label{fig:buffer}
\end{figure}

As explained in \cref{subsection:newltsconfig},
the semantics of the message buffers
for routed actions are the same as their
non-routed counterparts,
so the projected contents of the buffers
for routed communication are
the same as those under non-routed communication.

\subsection{Trace Equivalence}
\label{subsection:newtraceeq}

The sequence of labels $l_1, l_2, \dots, l_n$
that reduce the LTS is known as a trace.
We want to prove that the set of traces
that can be obtained from reducing a global type
$G$ is the same as those that can be obtained
from reducing its projected configuration $\projconf{G}$.

Our approach is based on \cite{characterisation} --
namely, proving that this is the case for a single
transition (i.e. \textit{step equivalence}) is sufficient,
as we can obtain trace equivalence as a direct consequence.

\begin{lemma}[Step Equivalence]
For all global types $G$ and configurations $s$,
if $\projconf{G} \subtype s$,
then $\treduce{G}{G'}{l} \Longleftrightarrow \treduce{s}{s'}{l}$ 
such that $\projconf{G'} \subtype s'$.

\label{lem:stepeq}
\end{lemma}

\begin{proof}
By induction on the possible transitions in the LTSs
over global types (to prove $\Longrightarrow$,
i.e. \textit{soundness}) 
and configurations (to prove $\Longleftarrow$,
i.e. \textit{completeness}). 

\item \textbf{Soundness}

By induction on the structure of LTS semantics
over global types.
The proofs for rules \rulename{Gr1-5} are
the same as in \cite{characterisation}.
We focus on the new rules introduced for routing.

\begin{itemize}

\item \rulename{Gr6}

\hl{TODO}

\item \rulename{Gr7}

\hl{TODO}

\item \rulename{Gr8}

\hl{TODO}

\item \rulename{Gr9}

\hl{TODO}

\end{itemize}

\item \textbf{Completeness}

By considering the possible transitions in the LTS
over configurations, which is defined by
case analysis on the possible labels $l$.
The proof for $l = \aout{p}{q}{j}$ and $l = \ain{p}{q}{j}$
is the same as in \cite{characterisation}.
We focus on the new labels introduced for routing.

\begin{itemize}

\item $l = \via{s}{\aout{p}{q}{j}}$:

\hl{TODO}

\item $l = \via{s}{\ain{p}{q}{j}}$:

\hl{TODO}

\end{itemize}

\end{proof}

\begin{theorem}[Trace Equivalence]
Let $G$ be a global type with participants 
$\mathcal{P} = \pt{G}$.
Then $G \approx (\vec T, \vec \epsilon)$.

\label{th:traceeq}
\end{theorem}

\begin{proof}
Direct consequence of \cref{lem:stepeq}.
\end{proof}

\subsection{Deadlock Freedom}
\label{subsection:newdeadlockfreedom}

\begin{theorem}[Preservation of Well-formedness]
Let $G$ be a global type.
Suppose $G$ is well-formed with respect to some router $\mrole{s}$.

\[
\forall G', l. ~
(\treduce{G}{G'}{l}
	\Longrightarrow
\wfnew{G'}{s})
\]

\label{th:preservewf}
\end{theorem}

\begin{proof}
By induction on the structure of $\treduce{G}{G'}{l}$.

\begin{itemize}
\item \rulename{Gr1}, where \dots

\hl{TODO}

\item \rulename{Gr2}, where \dots

\hl{TODO}

\item \rulename{Gr3}, where \dots

\hl{TODO}

\item \rulename{Gr4}, where \dots

\hl{TODO}

\item \rulename{Gr5}, where \dots

\hl{TODO}

\item \rulename{Gr6}, where \dots

\hl{TODO}

\item \rulename{Gr7}, where \dots

\hl{TODO}

\item \rulename{Gr8}, where \dots

\hl{TODO}

\item \rulename{Gr9}, where \dots

\hl{TODO}
\end{itemize}

\end{proof}

\begin{theorem}[Progress for Well-formed Global Types]
Let $G$ be a global type.
Suppose $G$ is well-formed with respect to some router $\mrole{s}$.

\[
(G = \tend) \vee \exists G', l. ~ (\treduce{G}{G'}{l})
\]

The following is logically equivalent.

\[
(G \neq \tend)
	\Longrightarrow 
\exists G', l. ~ (\treduce{G}{G'}{l})
\]

\label{th:progresswf}
\end{theorem}

\begin{proof}
By induction on the structure of $G$.

We do not consider $G = \tend$ by assumption.

We also do not consider $G = \trecvar$ as the type variable is not guarded.

\begin{enumerate}

\item $G = \trec{G''}$

\hl{TODO}

\item $G = \gcomm{p}{q}{l_i: G_i}{i \in I}$

\hl{TODO}

\item $G = \groute{p}{q}{r}{l_i: G_i}{i \in I}$

$\wfnew{G}{s}$ holds by assumption, so $\mrole{r} = \mrole{s}$.

\hl{TODO}

\end{enumerate}

\end{proof}

\begin{theorem}[Deadlock Freedom]
Let $G$ be a global type.
Suppose $G$ is well-formed with respect to some router $\mrole{s}$.

\[
G \to^* G'
	\Longrightarrow
(G' = \tend) \vee \exists G'', l. ~ 
	(\treduce{G'}{G''}{l})
\]

\end{theorem}

\begin{proof}
Direct consequence of 
\cref{th:preservewf,th:progresswf}.
\end{proof}

\section{Router-Parameterised Encoding}
\label{section:encoding}

\begin{itemize}
\item define encoding on global types
\item define encoding on local types
\item define encoding on LTS actions
\item motivate with twobuyer example
\end{itemize}

\subsection{Correspondence between Global and Local Type Encodings}
\label{subsection:encodelink}
\begin{itemize}
\item prove proposition
\item state corollary for subsequent proof
\end{itemize}

\subsection{Preserving Well-formedness}
\label{subsection:encodepreservewf}

\subsection{Preserving Communication}
\label{subsection:encodepreservecomm}

\section{Summary}
We have presented \newtheory, a variant of
the canonical MPST theory to express \textit{routed communication}.
We introduced extensions to syntax and semantics, and proved
that our extended semantics are sound and complete,
and preserve deadlock-freedom
for well-formed protocols. 
We defined an encoding from the
canonical theory onto \newtheory, and proved the preservation
of well-formedness and communication.