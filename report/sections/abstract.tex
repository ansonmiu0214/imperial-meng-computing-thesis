\begin{abstract}

\noindent
Modern web programming
involves coordinating interactions
between browser clients and a web server.
Typically, the interactions are
informally described, which makes
it difficult to verify communication
correctness in the distributed system.

Multiparty Session Types (MPST)
is a typing discipline for concurrent
processes that communicate via
message-passing. MPST theory can ensure
communication safety properties and
protocol conformance.
Existing work in session-typed
web development over WebSocket transport
is incompatible with modern web
programming practices and are limited
to supporting communication protocols
that implement the server-centric network
topology.

We address limitations in the current 
state-of-the-art by:
\textbf{(1)} implementing \codegen, a code generation
toolchain for session-typed web development
over WebSocket transport using TypeScript;
and \textbf{(2)} presenting \newtheory, a new
multiparty session type theory that supports
routed communications.

\codegen provides
developers with TypeScript APIs generated
from a communication protocol specification 
based on multiparty session type theory.
Our work is compatible with modern web programming
industrial practices. The generated APIs
build upon TypeScript concurrency practices,
complement the event-driven style of programming
in full-stack web development,
and are compatible
with the Node.js runtime and the React.js framework
for server-side and browser-side endpoints respectively.
We evaluate the expressiveness of \codegen for modern
web programming
using case studies of protocols found in web-services,
and analyse the performance overhead
through 
running benchmarks against a baseline implementation.

\newtheory can express interactions to be routed
via an intermediate participant.
Using \newtheory, 
we propose an approach for supporting 
peer-to-peer communication
between browser-side endpoints through 
routing communication via the server
in a way that avoids excessive serialisation
and preserves communication safety. 
We evaluate the correctness of \newtheory
by proving communication safety properties, such as
deadlock freedom.

\end{abstract}