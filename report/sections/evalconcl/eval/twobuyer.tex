\section{Routed Multiparty Sessions: \tprotocol{Two Buyers}}
\label{section:evaltwobuyer}

We implement the \tprotocol{Two Buyer} protocol
introduced in \cref{section:twobuyer}. 
This protocol cannot be implemented using
existing proposals \cite{PureScript2019,MVU2020} for integrating
session types into web development.
\codegen overcomes the limitations of existing work
through implementing novel theory of routed multiparty
session types formalised in \cref{chap:theory};
the implementation of \newtheory in \codegen is 
explained in \cref{chap:impl}.

We present the implementation of the \trole{S}eller
in \cref{lst:evaltwobuyerseller}, 
and the implementation
of the peer-to-peer interaction by buyer \trole{A}
in \cref{lst:evaltwobuyerA}
The main point to note is that the routing mechanism
is completely transparent to the developer,
which shows the elegance of our solution.

\begin{figure}[!h]
\begin{lstlisting}[language=javascript,tabsize=2]
new Implementation.Initial({
	[Labels.S32.title]: async (title) => {
		const quote = await db.getQuote(title);
		return new Implementation.S34([
			Labels.S34.quote, [quote], new Implementation.S35([
				Labels.S35.quote, [quote], new Implementation.S36({
					[Labels.S36.buy]: () => {
						return new Implementation.Terminal();
					},
					[Labels.S36.cancel]: () => {
						return new Implementation.Terminal();
					}
				})
			])
		])
	}
})
\end{lstlisting}
\captionof{lstlisting}{Two Buyer Seller Implementation}
\label{lst:evaltwobuyerseller}
\end{figure}

\begin{figure}[!h]
\begin{lstlisting}[language=javascript]
type State = { split: number };

class ProposeSplit extends S11<State> {
	state = { split: 0 };
	
	render() {
		const SendSplit = this.split('onClick', ev => {
			ev.preventDefault();
			return [this.state.split];		
		});
		
		return <div>
			<input
				type='number'
				value={this.state.split}
				onChange={ev => this.setState({
					split: Number(ev.target.value)
				})}
				placeholder='Enter split'
			/>
			<SendSplit><button>Propose</button></SendSplit>	
		</div>;	
	}
}
\end{lstlisting}
\captionof{lstlisting}{Developer Implementation of Peer-to-Peer Interaction
in Two Buyer}
\label{lst:evaltwobuyerA}
\end{figure}

The \newtheory implementation does not affect
the compatibility of our generated APIs with external libraries.
The \trole{S}eller endpoint is set up as an Express.js
application, and both buyers use the \textit{React Context API}\footnote{
React Contexts allow components
to pass data (such as application state)
through the component tree without
having to propagate via props at every level.
}
for application state management. 
%We show an example of
%the latter in the developer's implementation for
%buyer \trole{A} receiving and keeping track of the server's quote
%using a React Context in \cref{lst:evaltwobuyercontext}.
%
%\begin{figure}[!h]
%\begin{lstlisting}[language=javascript]
%// two buyer A context
%\end{lstlisting}
%\captionof{lstlisting}{\dots}
%\label{lst:evaltwobuyercontext}
%\end{figure}

%We include the implementations of the remaining EFSM states
%for buyer \trole{A} in \cref{section:evalcodetwobuyer}
%for completeness;
Interested readers can find the full implementations for the other
endpoints alongside the generated code on
GitHub\footnote{
\url{https://github.com/ansonmiu0214/SessionTS-Examples/TwoBuyer}
}.
