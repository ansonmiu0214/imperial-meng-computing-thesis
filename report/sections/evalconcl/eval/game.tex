\section{Multiparty Sessions: \tprotocol{Noughts and Crosses}}
\label{section:evalgame}

We implement the classic turn-based board game of 
\textit{Noughts and Crosses}
between two players,
as introduced in \cite{PLACES2020}.
Both players, identified by either \textit{noughts (O's)}
or \textit{crosses (X's)} respectively, take turns
to place a mark on an unoccupied cell of a 3-by-3 grid
until one player wins (when their markers form one straight
line on the board) or a stalemate is reached (when all cells
are occupied and no one wins).

We formalise the game interactions using a Scribble protocol
presented in \cref{lst:game}.
The \tprotocol{Game} protocol describes one turn:
\trole{P1} makes a move by sending the coordinates of a
vacant cell on the game board to \trole{Svr},
then \trole{Svr} reports the outcome of that move
to both players. If another round is required
to determine the game result, the \tprotocol{Game}
protocol is recursively invoked (\cref{line:gamerec}) with
roles \trole{P1} and \trole{P2} swapped.

\begin{figure}[!ht]
\begin{lstlisting}[language=Scribble]
module NoughtsAndCrosses;

// TypeScript definition:
// interface Point {x: number, y: number}
type <typescript> "Coordinate" from "./Types" as Point;

global protocol Game(role Svr, role P1, role P2) {
	Pos(Point) from P1 to Svr;
	choice at Svr {
		Lose(Point)   from Svr to P2;
		Win(Point)    from Svr to P1;
	} or {
		Draw(Point)   from Svr to P2;
		Draw(Point)   from Svr to P1;
	} or {
		Update(Point) from Svr to P2;
		Update(Point) from Svr to P1;
		do Game(Svr, P2, P1); (*@\label{line:gamerec}@*)
	}
}
\end{lstlisting}
\captionof{lstlisting}{The \tprotocol{Noughts and Crosses} Protocol}
\label{lst:game}
\end{figure}

We focus on the implementation details that
best illustrate the expressiveness of our work;
the interested reader can find the full implementation on
GitHub\footnote{
\url{https://github.com/ansonmiu0214/SessionTS-Examples/NoughtsAndCrosses}
}, and consult \texttt{README.md} to navigate between the generated code
and developer implementation.

\subsection{Game Server}
We set up the WebSocket server as an \textit{Express.js}\footnote{
Express is a commonly used library for writing lightweight
web servers in JavaScript.
} \cite{Express} application
on top of the Node.js runtime.
We define our own game logic in a \texttt{Board} class
to keep track of the game state and expose methods 
to query the result -- the implementation for \texttt{Board} is included
in \cref{section:evalcodegame}.
This custom logic is integrated into the 
\texttt{handleP1Move} and \texttt{handleP2Move}
handlers implemented by the developer, defined to
handle the moves made by \trole{P1} and \trole{P2} respectively.
We illustrate this in \cref{lst:gamesvr}.

\begin{figure}[!h]
\begin{lstlisting}[language=javascript,tabsize=2]
const handleP1Move = new Implementation.S13({
	Pos: (*@\hl{async}@*) (move: Point) => {
		// `board` manages game state;
		// `board.P1` registers the move and returns the game result
		const result = (*@\hl{await}@*) board.P1(move);
		switch (result) {
			case MoveResult.Win: {
				return new Implementation.S15([
					// Send losing result to P2
					[Labels.S15.Lose, [move], new Implementation.S16(
						// Send winning result to P1
						[Labels.S16.Win, [move], new Implementation.Terminal()]
					)]
				]);
			}
			case MoveResult.Draw: { ... }
			case MoveResult.Continue: {
				return new Implementation.S15(
					// Notify both players and proceed
					// with next round using `handleP2Move`
					[Labels.S15.Update, [move], new Implementation.S18(
						[Labels.S18.Update, [move], handleP2Move]
					)]
				);	
			}		
		}
	}
});

const handleP2Move = ...		// Defined similarly as handleP1Move

const cancellation = (role: Roles.All, reason: string) => {
	console.log(`${role} cancelled because of ${reason}`);
}

new Svr(
	wss,						// WebSocket server
	handleP1Move,		// Game logic
	cancellation		// Error handler
);
\end{lstlisting}
\captionof{lstlisting}{Implementing \tprotocol{Noughts and Crosses} Game Server using
\codegen}
\label{lst:gamesvr}
\end{figure}

When the server receives a move, it notifies
the game logic to update the game state and return the game
result caused by that move.
The game logic is likely to keep track of move history
using a database; we simulate this with a delay, so 
the game result returned by the game logic is a \texttt{Promise}.
The expressiveness of our generated APIs enable the developer
to define the handlers as \lstonelinejs{async} functions
to use the asynchronous game logic API intuitively -- this is something
prevalent in modern web programming, but not directly addressed in existing
session type implementations for web development \cite{PureScript2019,MVU2020}.

\subsection{Game Players}
For simplicity, our game uses the same implementation for both
\trole{P1} and \trole{P2}, although they can be different in theory --
the developer could implement \trole{P1} using a GUI and provide
\trole{P2} with a text-based game experience on the browser.

The main implementation detail for players
is to make moves. 
Intuitively, the developer implements a grid and binds a handler
to the \lstonelinejs{'onClick'} event of each vacant cell to send that cell's coordinate 
in a \tmsg{Pos(Point)} message to the game server.
A common source of bugs would be not preventing the user from selecting
a second cell when waiting for the game server's response,
which violates the game rules (and the global protocol).

Our approach of providing \textit{component factories} for send states
in \reactcodegen makes this very intuitive and guarantees communication safety.
First, it gives the developer the flexibility to trigger
the same send action (in this case, \tmsg{Pos(Point)} via multiple
UI elements -- the developer can generate a send action wrapper component 
for each vacant cell on the game board.
Moreover, each generated wrapper component sends a different payload
corresponding to the coordinates of the cell:
our generated APIs support this as the handler supplied
to the send component factory can access the cell's coordinates in the closure.
Finally, the send action is always followed by a transition to the receive
state component, so the user cannot violate channel linearity by selecting
two cells.

We demonstrate how this works in \cref{lst:gamesendfactory}.
The factory function for binding the \tmsg{Pos(Point)} send action
is defined under \lstonelinejs{this.props.Pos}.
For each x-y coordinate on the game board, if the cell is vacant,
we create a \texttt{<SelectPoint>} React component from the
component factory function (which reads ``build a react
component that sends the \tmsg{Pos} message with x-y coordinate
as payload when the user clicks on it''), and we wrap
a \texttt{<td>} table cell (since the game board is rendered as an
HTML table) inside the generated component to bind the click event
to the table cell.

\begin{figure}[!h]
\begin{lstlisting}[language=javascript,tabsize=2]
// Inside some render() function...
{board.map((row, x) =>
	<tr>
	{row.map((cell, y) => {
		if (cell === Cells.VACANT) {
			const sendPoint = (event: React.MouseEvent) => {
				// Don't refresh the screen
				event.preventDefault();
				return { x, y };
			});
			const SelectPoint = this.pros.Pos('onClick', sendPoint);
			return <SelectPoint><td>{cell}</td></SelectPoint>
		} else {
			// Render nought or cross,
			// but clicking on this cell will *not* send anything
			return <td>{cell}</td>
		}		
	})}
	</tr>
)}
\end{lstlisting}
\captionof{lstlisting}{Safely Binding Send Actions to 
\tprotocol{Noughts and Crosses} Game Board}
\label{lst:gamesendfactory}
\end{figure}

The session cancellation handler allows the developer
to render useful messages to the player, since
a different component can be rendered depending on whether
the server or the opposition has disconnected,
and make \textit{application-specific} interpretations of the cancellation.
For example, if the opposition has disconnected,
the developer can interpret this as a forfeit and
render a winning message to the user.

\subsection{Summary}
We demonstrated how the developer can use the generated APIs
from \codegen to implement a complex multiparty protocol
which features branching, selection and recursion.
We highlighted specific features in the generated APIs for both
server and browser endpoints that allow the developer
to intuitively implement their application logic.
In particular, we observed that the extensions introduced
in \cref{chap:ext} play crucial roles in
improving the usability of the generated APIs when
compared with existing work on session types for web development 
\cite{PureScript2019,MVU2020,LINKS}.

Code is available at 
\url{https://github.com/ansonmiu0214/SessionTS-Examples/NoughtsAndCrosses}.