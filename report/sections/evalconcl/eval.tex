\chapter{Evaluation}
\label{chap:eval}

In this chapter, we
evaluate the expressiveness
and performance of our generated APIs
with respect to the main objective of
our project in
providing
developers with a development workflow
that offers communication safety guarantees in modern
web programming through multiparty session types.

\subsubsection{Expressiveness}
To evaluate the expressiveness
of our work,
we use two case studies
of protocols found in webservices
to demonstrate strengths and weaknesses
of our work:

\begin{itemize}

\item
\textbf{\tprotocol{Noughts and Crosses} Game, as introduced
in \cite{PLACES2020}:}

In \cref{section:evalgame},
we implement a classic multiplayer game which involves a game server
and two players interacting with the game via the browser,
to show that our work is compatible with multiparty sessions,
which is not the case in \cite{MVU2020,Exceptional}.
We show how our generated APIs are compatible with the
state management solution used by the game, and give examples of 
how the generated APIs empower the developer to intuitively 
implement the game interface.

\item
\textbf{\tprotocol{Two Buyers} Protocol, as introduced in 
\cref{section:twobuyer} and \cite{JACM2016,MPAST}:}

In \cref{section:evaltwobuyer},
we walk through an implementation of the \tprotocol{Two Buyers}
protocol. This \textit{cannot} be implemented in
\cite{PureScript2019,MVU2020,Exceptional}, 
which illustrates the novelty
of our \newtheory theory on routed multiparty session types.
We emphasise how the routing mechanism is transparent to the
developer implementing the protocol, which demonstrates
the practicality of our extensions to \codegen to support
\newtheory.

\end{itemize}

\subsubsection{Performance}
To evaluate the performance of our work,
we run micro-benchmarks
on a variant of the \tprotocol{Ping Pong}
protocol parameterised by the number
of round trips
to analyse the overhead of our implementation,
compared with implementations written
without using our generated APIs,
as the number of round trips increase.
We describe the experiment methodology
and comment on our findings in \cref{section:benchmarks}.

\section{Multiparty Sessions: \tprotocol{Noughts and Crosses}}
\label{section:evalgame}

We implement the classic turn-based board game of 
\textit{Noughts and Crosses}
between two players,
as introduced in \cite{PLACES2020}.
Both players, identified by either \textit{noughts (O's)}
or \textit{crosses (X's)} respectively, take turns
to place a mark on an unoccupied cell of a 3-by-3 grid
until one player wins (when their markers form one straight
line on the board) or a stalemate is reached (when all cells
are occupied and no one wins).

We formalise the game interactions using a Scribble protocol
presented in \cref{lst:game}.
The \tprotocol{Game} protocol describes one turn:
\trole{P1} makes a move by sending the coordinates of a
vacant cell on the game board to \trole{Svr},
then \trole{Svr} reports the outcome of that move
to both players. If another round is required
to determine the game result, the \tprotocol{Game}
protocol is recursively invoked (\cref{line:gamerec}) with
roles \trole{P1} and \trole{P2} swapped.

\begin{figure}[!ht]
\begin{lstlisting}[language=Scribble]
module NoughtsAndCrosses;

// TypeScript definition:
// interface Point {x: number, y: number}
type <typescript> "Coordinate" from "./Types" as Point;

global protocol Game(role Svr, role P1, role P2) {
	Pos(Point) from P1 to Svr;
	choice at Svr {
		Lose(Point)   from Svr to P2;
		Win(Point)    from Svr to P1;
	} or {
		Draw(Point)   from Svr to P2;
		Draw(Point)   from Svr to P1;
	} or {
		Update(Point) from Svr to P2;
		Update(Point) from Svr to P1;
		do Game(Svr, P2, P1); (*@\label{line:gamerec}@*)
	}
}
\end{lstlisting}
\captionof{lstlisting}{The \tprotocol{Noughts and Crosses} Protocol}
\label{lst:game}
\end{figure}

We focus on the implementation details that
best illustrate the expressiveness of our work;
the interested reader can find the full implementation on
GitHub\footnote{
\url{https://github.com/ansonmiu0214/SessionTS-Examples/NoughtsAndCrosses}
}, and consult \texttt{README.md} to navigate between the generated code
and developer implementation.

\subsection{Game Server}
We set up the WebSocket server as an \textit{Express.js}\footnote{
Express is a commonly used library for writing lightweight
web servers in JavaScript.
} \cite{Express} application
on top of the Node.js runtime.
We define our own game logic in a \texttt{Board} class
to keep track of the game state and expose methods 
to query the result -- the implementation for \texttt{Board} is included
in \cref{section:evalcodegame}.
This custom logic is integrated into the 
\texttt{handleP1Move} and \texttt{handleP2Move}
handlers implemented by the developer, defined to
handle the moves made by \trole{P1} and \trole{P2} respectively.
We illustrate this in \cref{lst:gamesvr}.

\begin{figure}[!h]
\begin{lstlisting}[language=javascript,tabsize=2]
const handleP1Move = new Implementation.S13({
	Pos: (*@\hl{async}@*) (move: Point) => {
		// `board` manages game state;
		// `board.P1` registers the move and returns the game result
		const result = (*@\hl{await}@*) board.P1(move);
		switch (result) {
			case MoveResult.Win: {
				return new Implementation.S15([
					// Send losing result to P2
					[Labels.S15.Lose, [move], new Implementation.S16(
						// Send winning result to P1
						[Labels.S16.Win, [move], new Implementation.Terminal()]
					)]
				]);
			}
			case MoveResult.Draw: { ... }
			case MoveResult.Continue: {
				return new Implementation.S15(
					// Notify both players and proceed
					// with next round using `handleP2Move`
					[Labels.S15.Update, [move], new Implementation.S18(
						[Labels.S18.Update, [move], handleP2Move]
					)]
				);	
			}		
		}
	}
});

const handleP2Move = ...		// Defined similarly as handleP1Move

const cancellation = (role: Roles.All, reason: string) => {
	console.log(`${role} cancelled because of ${reason}`);
}

new Svr(
	wss,						// WebSocket server
	handleP1Move,		// Game logic
	cancellation		// Error handler
);
\end{lstlisting}
\captionof{lstlisting}{Implementing \tprotocol{Noughts and Crosses} Game Server}
\label{lst:gamesvr}
\end{figure}

When the server receives a move, it notifies
the game logic to update the game state and return the game
result caused by that move.
The game logic is likely to keep track of move history
using a database; we simulate this with a delay, so 
the game result returned by the game logic is a \texttt{Promise}.
The expressiveness of our generated APIs enable the developer
to define the handlers as \lstonelinejs{async} functions
to use the asynchronous game logic API intuitively -- this is something
prevalent in modern web programming, but not directly addressed in existing
session type implementations for web development \cite{PureScript2019,MVU2020}.

\subsection{Game Players}
For simplicity, our game uses the same implementation for both
\trole{P1} and \trole{P2}, although they can be different in theory --
the developer could implement \trole{P1} using a GUI and provide
\trole{P2} with a text-based game experience on the browser.

The main implementation detail for players
is to make moves. 
Intuitively, the developer implements a grid and binds a handler
to the \lstonelinejs{'onClick'} event of each vacant cell to send that cell's coordinate 
in a \tmsg{Pos(Point)} message to the game server.
A common source of bugs would be not preventing the user from selecting
a second cell when waiting for the game server's response,
which violates the game rules (and the global protocol).

Our approach of providing \textit{component factories} for send states
in \reactcodegen makes this very intuitive and guarantees communication safety.
First, it gives the developer the flexibility to trigger
the same send action (in this case, \tmsg{Pos(Point)} via multiple
UI elements -- the developer can generate a send action wrapper component 
for each vacant cell on the game board.
Moreover, each generated wrapper component sends a different payload
corresponding to the coordinates of the cell:
our generated APIs support this as the handler supplied
to the send component factory can access the cell's coordinates in the closure.
Finally, the send action is always followed by a transition to the receive
state component, so the user cannot violate channel linearity by selecting
two cells.

We demonstrate how this works in \cref{lst:gamesendfactory}.
The factory function for binding the \tmsg{Pos(Point)} send action
is defined under \lstonelinejs{this.props.Pos}.
For each x-y coordinate on the game board, if the cell is vacant,
we create a \texttt{<SelectPoint>} React component from the
component factory function (which reads ``build a react
component that sends the \tmsg{Pos} message with x-y coordinate
as payload when the user clicks on it''), and we wrap
a \texttt{<td>} table cell (since the game board is rendered as an
HTML table) inside the generated component to bind the click event
to the table cell.

\begin{figure}[!h]
\begin{lstlisting}[language=javascript,tabsize=2]
// Inside some render() function...
{board.map((row, x) =>
	<tr>
	{row.map((cell, y) => {
		if (cell === Cells.VACANT) {
			const sendPoint = (event: React.MouseEvent) => {
				return { x, y };
			});
			const SelectPoint = this.pros.Pos('onClick', sendPoint);
			return <SelectPoint><td>{cell}</td></SelectPoint>
		} else {
			// Render nought or cross,
			// but clicking on this cell will *not* send anything
			return <td>{cell}</td>
		}		
	})}
	</tr>
)}
\end{lstlisting}
\captionof{lstlisting}{Safely Binding Send Actions to 
\tprotocol{Noughts and Crosses} Game Board}
\label{lst:gamesendfactory}
\end{figure}

The session cancellation handler allows the developer
to render useful messages to the player, since
a different component can be rendered depending on whether
the server or the opposition has disconnected,
and make \textit{application-specific} interpretations of the cancellation.
For example, if the opposition has disconnected,
the developer can interpret this as a forfeit and
render a winning message to the user.

\subsection{Summary}
We demonstrated how the developer can use the generated APIs
from \codegen to implement a complex multiparty protocol
which features branching, selection and recursion.
We highlighted specific features in the generated APIs for both
server and browser endpoints that allow the developer
to intuitively implement their application logic.
In particular, we observed that the extensions introduced
in \cref{chap:ext} play crucial roles in
improving the usability of the generated APIs when
compared with existing work on session types for web development 
\cite{PureScript2019,MVU2020,LINKS}.

Code is available at 
\url{https://github.com/ansonmiu0214/SessionTS-Examples/NoughtsAndCrosses}.

\section{Routed Multiparty Sessions: \tprotocol{Two Buyers}}
\label{section:evaltwobuyer}

We implement the \tprotocol{Two Buyer} protocol
introduced in \cref{section:twobuyer}. 


Interested readers can find the full implementation on
GitHub\footnote{\url{https://github.com/ansonmiu0214/SessionTS-Examples/TwoBuyer}}.

The \newtheory implementation does not affect
the compatibility of our generated APIs with external libraries.
The \trole{S}eller endpoint is set up as an Express.js
application, and both buyers still use the React Context API
for application state management.

\begin{itemize}
\item show that the routing aspect is transparent to both sides
\item same flexibilities apply -- we use react context for state management
\end{itemize}


%\section{Session Delegation: \fancyname{ATM}}
\label{section:evalatm}

The developer may define a communication protocol
in a way that requires implicit \textit{session delegation}
upon connection: when the browser endpoint connects
to the server endpoint, the server delegates a private channel
to carry out the interactions specified by the protocol,
whilst concurrently accepting new connections.
We illustrate the limitations of our work with respect
to supporting session delegation, albeit this is also
a limiting factor in similar existing work 
\cite{PureScript2019,MVU2020}.

Consider an \tprotocol{ATM} protocol involving a 
\trole{Bank} server endpoint and 
a \trole{Client} browser endpoint,
as presented in \cref{lst:atm}.
This is a recursive protocol, where the \trole{Client}
can choose to \tmsg{WITHDRAW} or \tmsg{DEPOSIT} some money
into their account, or \tmsg{QUIT}.
For \tmsg{WITHDRAW} and \tmsg{DEPOSIT} operations,
the \trole{Bank} will respond with the updated \tmsg{BALANCE},
or a \tmsg{FAIL}ure message if an operation fails (specifically,
\tmsg{WITHDRAW}ing more money than available).

\begin{figure}[!ht]
\begin{lstlisting}[language=Scribble, tabsize=2]
module ATM;

global protocol ATM(role Bank, role Client) {
	choice at Client {
		// The string contains the client's bank account ID
		WITHDRAW(string, number) from Client to Bank;
		choice at Svr {
			BALANCE(number) from Bank to Client;		
		} or {
			FAIL(string) 		from Bank to Client;		
		}
		do ATM(Bank, Client);
	} or {
		DEPOSIT(string, number) from Client to Bank;
		BALANCE(number) 				from Bank to Client;
		do ATM(Bank, Client);
	} or {
		QUIT(string) from Client to Bank;
		BYE() 			 from Bank to Client;	
	}
}
\end{lstlisting}
\captionof{lstlisting}{The \tprotocol{ATM} Protocol}
\label{lst:atm}
\end{figure}

An important detail is that, for any operation,
the \trole{Client} provides their bank account ID
in the message. By specifying the communication protocol
in this way, the developer intends to allow the
\trole{Bank} to manage the operations of multiple
\trole{Client}s concurrently.
More precisely, the developer makes an assumption
that the interactions specified in the protocol
are delegated to a private channel upon connection.

However, the runtime generated by \nodecodegen
does not make this assumption, so we do not get
the behaviour intended by the developer. If a second
\trole{Client} tries to connect to the \trole{Bank},
the connection request will be unsuccessful because
the role is already occupied.
Interested readers can validate this
behaviour by running the full implementation available
on GitHub\footnote{
\url{https://github.com/ansonmiu0214/SessionTS-Examples/ATM}
}.

At the same time, the runtime generated by \nodecodegen
can be adapted to ``support'' \textit{binary} session delegation.
The generated runtime includes
an \trole{Svr} class used to listen for
and manage incoming connections, and a
private \texttt{Session} class to execute the EFSM.
When the instance of the \trole{Svr} class
is no longer waiting for participants,
it delegates to a fresh instance of the \texttt{Session} class
to execute the protocol.
To support delegation, we can simply modify this piece of 
logic to also create a fresh instance of the \trole{Svr}
class to concurrently manage new connections, whilst the
instance of the \texttt{Session} class is bound to the
participants that have already joined.
We show this workaround in \cref{lst:atmdelegation} -- 
specifically, on \cref{line:atmdelegation}.

\begin{figure}[!h]
\begin{lstlisting}[language=javascript]
// ATM delegation workaround (*@\label{line:atmdelegation}@*)
\end{lstlisting}
\captionof{lstlisting}{Workaround for Binary Session
Delegation in \tprotocol{ATM} protocol}
\label{lst:atmdelegation}
\end{figure}

This workaround does not generalise well for
\textit{multiparty} session delegation. 
Suppose we want to do the same for the 
\tprotocol{Noughts and Crosses} protocol discussed
in \cref{section:evalcodegame}.
Assume an user connects as \trole{P1} and the game server
is still waiting for \trole{P2}. If another user attempts
to connect as \trole{P1}, connection will be unsuccessful
as the first game hasn't began yet -- this workaround will
only begin accepting connections again \textit{once}
the protocol begins executing.
Formally supporting session delegation will also require
extensions to the Scribble language, since we should not
make the assumption that all protocols require
session delegation upon connection.

This would be an interesting area of future work and
enable \codegen to support a wider range of protocols
for webservices; we discuss further in \cref{section:future}. 

\section{Performance Benchmarks}

Whilst web applications that implement our generated APIs enjoy
communication safety guarantees, the presence of the session runtime acts
as an additional layer of abstraction between the application logic and the
WebSocket transport, which is likely to present a performance trade-off.

\begin{figure}[!ht]
\begin{lstlisting}[language=Scribble]
global protocol PingPong(role Client, role Svr) {
	PING(number) from Client to Svr;
	choice at Svr {
		PONG(number) from Svr to Client;
		do PingPong(Client, Svr);
	} or {
		BYE(number) from Svr to Client;	
	}
}
\end{lstlisting}
\captionof{lstlisting}{Ping Pong Protocol}
\label{lst:pingpong}
\end{figure}

To measure the overhead of our implementation, we compare the
execution time of an interactive web application implementing the
Ping Pong protocol (\cref{lst:pingpong}) using our generated APIs,
against implementations of the protocol \textit{without} session types.

We parameterise the Ping Pong protocol by $n > 0$, the number of
round-trip messages. This is standardised in the application logic
across experiments.
Upon establishing a connection, the experiment proceeds as follow:

\begin{enumerate}

\item \trole{Client} sends \tmsg{PING($m$:number)} to \trole{Svr}, 
with $m = 0$ initially.

\item \trole{Svr} receives \tmsg{PING($m$:number)}, and
conditionally responds based on $n$:

\begin{enumerate}
\item If $m + 1 < n$, then \trole{Svr} replies \tmsg{PONG($m + 1$)}.
\trole{Client} responds to \tmsg{PONG} by returning to
step 1 with $m$ set as the payload from \tmsg{PONG}.

\item Otherwise, $m + 1 = n$, then \trole{Svr} responds with 
\tmsg{BYE($m + 1$)}, as $n$ round trips have taken place. 
\trole{Client} responds to \tmsg{BYE} by 
closing the connection, thus ending the experiment.
\end{enumerate}

\end{enumerate}

We note that the Ping Pong protocol implements a \textit{binary} session. 
It would be interesting to observe the overhead in a \textit{multiparty}
context, but due to limited time constraints, we were unable to 
extend our benchmarking suite to support multiple browser targets.
Benchmarking multiparty protocols would also require writing multiple
distinct React applications using the generated APIs -- as this is currently
a manual process, doing this for multiple roles requires more time than
available.

\subsection{Setup}

In order to measure the overhead as accurately as possible,
we outline the logic that all implementations must follow:

\subparagraph{Ping Pong \trole{Client} on React:}
\begin{itemize}

\item All \trole{Client}s implement the same user interface 
(\cref{fig:pingpongclient}), rendering
a \texttt{<button>} which triggers the send, and
a \texttt{<div>} captioned with the number of \tmsg{PONG}s received.

\item \trole{Client}s will use the React Context API \cite{reactcontext}
for application state management, i.e. the number of \tmsg{PONG}s received. 
We wrap the session logic in a \texttt{<Benchmark>} component 
that acts as the \texttt{ContextProvider} using its component state.

\item To automate the benchmark, we use the React Refs API \cite{reactrefs}
to access the DOM \texttt{<button>} node programmatically, in order to
simulate the click event and send a \tmsg{PING} message upon establishing
the WebSocket connection, or upon receiving a \tmsg{PONG}.

\item We use the production build generated by 
\texttt{create-react-app} \cite{cra} for all experiments, which performs the
transpilation into JavaScript. We serve the production build using the
\texttt{serve} package \cite{npmserve} available on \texttt{npm}.

\end{itemize}

\begin{figure}[!ht]
\captionof{figure}{User interface of Ping Pong \trole{Client}}
\label{fig:pingpongclient}
\end{figure}

\subparagraph{Ping Pong \trole{Svr} on Node:}
\begin{itemize}

\item We use the built-in \texttt{console.time} function to record
the execution time of all experiments. 
The timer starts when a WebSocket connection has been established at
\trole{Server}, and stops when on a \texttt{CloseEvent}.

\item To observe the execution pattern, the \trole{Svr} will log the running 
elapsed time for every \tmsg{PING} message received.

\item All \trole{Svr}s run the benchmarks without a real web browser,
using headless browsing functionality from 
the Zombie.js \cite{zombie} package.

\item The entry point to the \trole{Svr} logic is parameterised by the
number of messages, $n$, configured through an environment variable
passed through the command line.

\item We use the transpiled JavaScript versions of all \trole{Svr}
for the experiments.

\end{itemize}

The interested reader may follow the instructions under
\texttt{benchmarks/README.md} in the project repository \cite{repo}
to run the benchmarks and visualise the logs using the interactive
notebook in the same directory.

We run the experiments under a network of latency 0.165ms
(64 bytes ping), and repeat each experiment 20 times.
Execution time measurements  are taken using a machine 
equipped with Intel i7-4850HQ CPU (2.3 GHz, 4 cores, 8 threads), 
16 GB RAM, macOS operating system version 10.15.4, 
Node.js runtime version 12.12.0, and
TypeScript compiler version 3.7.4.
We standardise all packages used in the front- and back-end
implementations across experiments. Details can be found in their
corresponding \texttt{package.json} manifests (\cref{appendix:eval}).

We run our experiments using three implementations of the Ping Pong protocol:

\subparagraph{bare:}
The \texttt{bare} implementation directly interfaces with 
WebSocket primitives for sending and receiving. 
The implementation executes the Ping Pong protocol, but does not 
guarantee communication safety by construction -- e.g. the user can click
the \tmsg{PING} button multiple times before a \tmsg{PONG} message 
is received, violating channel linearity. 
This represents the typical developer implementation without using the
MPST framework.

\subparagraph{bare_safe:}
The \texttt{bare_safe} implementation also directly interfaces with
WebSocket primitives for communication, but assumes the developer
implements minimal viable workarounds to address the lack of
communication safety. Here, the developer renders an inactive version
of the \tmsg{PING} button when the \tmsg{PING} message has been sent
but a response has yet to be received; a \texttt{visible} boolean flag
is used to explicitly manage which \texttt{<button>} to render.

\subparagraph{mpst:}
The \texttt{mpst} implementation follows the MPST framework and
uses the APIs generated from \fancyname{SessionTS}, so it enjoys the
communication safety guarantees from our methodology.

\subsection{Execution Pattern}
We compare the execution patterns of exchanging 10,000 Ping-Pongs
throughout 20 repeated experiments across the three implementations.
We visualise the elapsed time with respect to the number of \tmsg{PING}s
received in \cref{fig:execution}.

\begin{figure}[!h]
\centering
\begin{subfigure}[b]{0.3\textwidth}
\centering
\includegraphics[width=\textwidth]{execbare10000}
\caption{\texttt{bare}}
\label{fig:executionbare}
\end{subfigure}
\hfill
\begin{subfigure}[b]{0.3\textwidth}
\centering
\includegraphics[width=\textwidth]{execbaresafe10000}
\caption{\texttt{bare_safe}}
\label{fig:executionbaresafe}
\end{subfigure}
\hfill
\begin{subfigure}[b]{0.3\textwidth}
\centering
\includegraphics[width=\textwidth]{execmpst10000}
\caption{\texttt{mpst}}
\label{fig:executionmpst}
\end{subfigure}
\captionof{figure}{Comparison of Execution Pattern for 10,000 Ping-Pongs}
\label{fig:execution}
\end{figure}



\subsection{Overhead}
We compare the total execution time (\textit{Exe. Time}) 
and execution time per round trip (\textit{Exe. Time / Ping-Pong}) -- 
averaged over 20 repeated experiments -- across the three implementations,
for $n \in \{10^2, 10^3, 10^4\}$.
We summarise the results in \cref{table:overhead}.

\renewcommand{\arraystretch}{1.5}
\begin{table}[!h]
\centering
\begin{tabular}{||c||c|c|c||c|c|c||}
\hline
\multirow{2}{*}{$n$} & 
\multicolumn{3}{c||}{Exe. Time} & 
\multicolumn{3}{c||}{Exe. Time / Ping-Pong} \\
\cline{2-7}
 & \texttt{bare} & \texttt{bare_safe} & \texttt{mpst} 
 & \texttt{bare} & \texttt{bare_safe} & \texttt{mpst} \\
\hline\hline
$10^2$ & 89.64ms & 107.09ms & 186.23ms & 0.90ms & 1.07ms & 1.86ms \\
$10^3$ & 642.92ms & 663.91ms & 1155.48ms & 0.64ms & 0.66ms & 1.16ms \\
$10^4$ & 3542.16ms & 3837.97ms & 7015.25ms & 0.35ms & 0.38ms & 0.70ms \\
\hline
\end{tabular}
\captionof{table}{Comparison of Execution Time for 
100, 1,000 and 10,000 Ping-Pongs}
\label{table:overhead}
\end{table}
\renewcommand{\arraystretch}{1}

We note that the addition of a session runtime for all roles in the 
\texttt{mpst} implementation \textit{does} incur a performance overhead. 
This is made apparent when looking closely at 
\textit{Exe. Time / Ping-Pong};
we visualise this in \cref{fig:timepermsg}.

\begin{figure}[!h]
\centering
\includegraphics[width=\textwidth]{timepermsg}
\captionof{figure}{Comparing Average Time per Ping-Pong 
Across Implementations}
\label{fig:timepermsg}
\end{figure}

The \texttt{mpst} implementation records greater round trip times compared
to both \texttt{bare} and \texttt{bare_safe} variants. This is expected,
as \dots

\section{Summary}
We have demonstrated
the expressiveness of our generated APIs
through implementing two web services describing
multiparty sessions. We highlighted how our
approach towards session-typed GUI programming
prevents channel reuse and 
makes it intuitive to implement the game board
of \tprotocol{Noughts and Crosses}.
We also showed
that the extensions added to \nodecodegen
and \reactcodegen
to support \newtheory
are transparent to the developer.

Through performance benchmarks against a
baseline implementation of the \tprotocol{Ping Pong}
protocol, we analysed the overhead of our implementation
and reasoned that this overhead will minimise when compared
with baseline implementations of more complex, multiparty
protocols involving routed communication.