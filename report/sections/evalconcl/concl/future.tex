\section{Future Work}
\label{section:future}

We highlight several areas of future work
to support a wider range of communication protocols
for web applications and increase the 
industrial relevance of our work.

\paragraph{Session Delegation}
Delegation allows channels to be sent between endpoints.
This directly addresses the limitations presented in
\cref{section:evalatm}: in the \tprotocol{ATM} protocol,
when the \trole{Client}
connects to the \trole{Svr},
the \trole{Svr} should be able to send a private channel
for \trole{Client} to perform bank operations,
whilst concurrently accepting new connections.

The API generation solution 
presented by Scalas et al. \cite{LinearDecomp} supports
\textit{distributed multiparty delegation} in Scala.
We have also shown in \cref{section:evalatm} that
the developer can implement delegation with small changes
to the session runtime.
Future work can apply the findings from \cite{LinearDecomp},
extend the Scribble protocol language for sending
channels through messages, and formalise the workaround
in \cref{section:evalatm} with respect to the extended 
Scribble syntax for delegation.

\paragraph{``Serverless'' Server-Side Endpoints}
\dots

\paragraph{Explicit Connection Actions}
Hu and Yoshida \cite{ExplicitConnections} extends 
multiparty session type theory
with \textit{explicit connection actions} to support protocols 
with optional participants.
Supporting optional participants is a relevant concept
for web applications, particularly for browser-side endpoints.

Our work defines a way to handle connection requests
and detect disconnection events. The implementation assumes that
all participants join the session when the protocol begins.
We also focus on disconnection events as a result of session
cancellation rather than those stated in the protocol.

\cite{ExplicitConnections} uses a variant of the Scribble protocol
language that supports explicit connection actions. Future work
can use that as a basis for code generation, and generalise
our existing mechanisms for session connection and cancellation
to support explicit connection actions.

\paragraph{Transport Abstraction}
The MPST-based web development framework presented by 
King et al. \cite{PureScript2019} defines the session runtime
in a way that parameterises the transportation mechanism.
The developer can run the session over different transportation,
provided that the custom transport implements the behaviour
of the \texttt{Transport} type class.

Future work can generalise the session runtime for 
\fancyname{NodeMPST} and \fancyname{ReactMPST} to provide
similar abstractions. A successful implementation could
possibly allow direct interactions between two Node.js-based
endpoints, but this will also require the routed MPST
theory to be adapted accordingly to reason about
such interactions in a web-based context.

\paragraph{EFSM Encoding as Typestates}
The concept of \textit{typestates} define the interface of
an object to be dependent on its private state, meaning
it can change at runtime.
This is compatible with the EFSM abstraction from the
MPST framework: the permitted methods (i.e. channel actions)
of the EFSM depends on its current state.
In fact, Kouzapas et al. \cite{StMungo} presents a way to
generate typestate specifications for each endpoint of a
Scribble protocol; Gay et al. \cite{ModularST} proposes
an encoding of session types in object-oriented languages
and discuss their approach with respect to typestates.

Future work can explore implementing typestates
in TypeScript. A linear type system would be useful towards 
implementing typestates, but this is lacking in TypeScript.
One interesting possibility for approximating this would
be to combine \textit{decorators} with \textit{transformers}:
the former lets the developer define typestate-related metadata,
whilst the latter can parse the metadata and
transform the source code into another TypeScript program that acts
as an ``intermediate representation'',
such that only implementations that respect the typestate specification
will type-check with its intermediate representation.