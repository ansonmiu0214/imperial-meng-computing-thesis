\documentclass{article}
\usepackage[utf8]{inputenc}
\usepackage{listings}

\title{Project Notes}
\author{Anson Miu}


\begin{document}

\maketitle

\section{Progress}

\paragraph{01/04/2020}
\begin{itemize}
\item 
Set up Node and TypeScript sandbox for testing.
The idea is to have a workflow where the
tool generates code for some protocol,
then copies into the sandbox and runs
the TypeScript compiler with the predefined
\texttt{tsconfig.json}. 

\item
This guarantees that ``correct'' TypeScript is
generated with respect to the TypeScript 
language, not about correctness with respect
to session.
\end{itemize}


\paragraph{02/04/2020}
\begin{itemize}
\item
Set up React TypeScript sandbox for testing.
No need \texttt{create-react-app} at
the moment (which creates full application
boilerplate) as we are just trying to check
for TypeScript syntax.

\item
Piped the \texttt{tsfmt} into the code 
generation workflow so no need to deal with
spacing issues on Jinja template -- timesaver.

\item
The full code generation workflow ``works'' for 
both targets. The validity of the generated 
code works only for Node targets.

\item
Now focusing on formalising the React generated 
code -- taking more time than planned because
the TicTacToe example made more assumptions
about the protocol than I thought.
\end{itemize}


\paragraph{03/04/2020}
\begin{itemize}
\item
Build system testing pipeline -- register which
Scribble protocols to test, specify which role
(e.g. Bob) corresponds with which target (e.g. Node),
then the pipeline will perform the code generation,
check for success exit code, then invoke the
TypeScript compiler to check the TypeScript code
and also check for success exit code.
\end{itemize}

\paragraph{05/04/2020}
\begin{itemize}
\item
Implemented templates for React code generation.
The main breakthrough was to better leverage
the type system and reduce duplication in the
generated code. The TicTacToe example had a lot
of duplication by hardcoding the state machine.
The new template pushes the logic to the actual
states and makes the runtime more lightweight.

\item
The generated code for BinaryNoPayload.scr is functional for both Node and browser targets.
\end{itemize}

\paragraph{06/04/2020}
\begin{itemize}
\item
Modified Node.js code generation templates to
take inspiration from the browser ones -- namely,
to make the runtime more lightweight by pushing
logic to the individual states; again, this means
no more huge switch statement with duplication.

\item
New structure for Node.js generated code generally
works. Explicitly used partial application to
focus the types (e.g. State $\to$ ReceiveState $\to$
ReceiveState.S12), and define each non-terminal
state as a separate module to internally use
its known message/label types to parse messages
without having one large switch statement in the
runtime. 

\item
Taking cues from Express middleware (with
the \texttt{next} function) to allow the
runtime to pass the send and receive functions
into the individual states, and give a callback
function for the individual states to pass any
data back to the runtime. 

\item
An example of this is for receive states -- 
the receive state needs to construct the message 
receive handler \textbf{tailored to the types/labels of 
that particular state}. This message receive handler
will do the parsing, invoke the correct receive
function, and advance the EFSM to the correct
successor state (achieved by having the runtime
pass the state-advancing function as param).

\item
Problem -- still need a type cast for the time being
with the handler functions; not sure why the types
aren't specific enough , trying to make sense of the 
error message at the moment.
\end{itemize}

\paragraph{07/04/2020}
\begin{itemize}
\item
Address ``technical debt'' in EFSM API -- it has
been inconsistent through being adapted alongside
the Node/React codegen template generation. 
Make use of dunder methods and properties
throughout to remove some of the internal logic
required in the templates.

\end{itemize}

\paragraph{08/04/2020}
\begin{itemize}
\item
Different MPST TypeSrcript encoding for Node targets --
we consider the previous encodings (labelled callbacks for
receive states, union types for send states) as ``handlers'',
and the developer needs to wrap them in an ``implementation''
class instance.

\item
The aim is for the runtime to be as lightweight as possible
-- it only needs to concern with whether the new state
is sending, receiving or terminal; and the state implementation
deals with performing the corresponding channel action
based on the particular label/payload types of that state.

\item
The class-based implementation/handler distinction allows
the runtime to use \texttt{instanceof} to safely narrow the
type of implementation. Each type of state essentially acts as
an interface (but we use classes in TypeScript to take advantage
of the \texttt{instanceof} function -- the equivalent notion
of a Java interface can be expressed as a TypeScript abstract
class with abstract methods and no properties defined). Because
all receive states will expose the same method signature,
everything type-checks.

\end{itemize}

\paragraph{09/04/2020}
\begin{itemize}
\item
Fix label regex problem in EFSM parsing -- currently just
allow alphabets, should allow more (i.e. alphanumeric
+ ``typical symbols'' like underscores and dashes).

\item
Could also look up source code of Scribble-Java to see what they use.
\end{itemize}

\section{MPST Encoding}

\subsection{Node}

TODO - on OneNote

\subsection{Browser}

TODO - on OneNote

\section{Code generation workflow}

TODO - on OneNote

\section{Testing}

\subsection{Unit tests}
TODO -- typical, use \texttt{unittest} module
to check for functionality of EFSM parsing
etc.

\begin{itemize}
\item EFSM parsing -- regexes, \dots
\item EFSM API
\end{itemize}

\subsection{Integration tests}
TODO -- integrate with external components, namely
scribble-java (which would be mocked in unit tests).

\subsection{Systems tests}
For a given protocol and specified role/target,
generate the TypeScript code, put into the
sandbox environment and run \texttt{npm test} 
(which is just a wrapper for 
\texttt{tsc --noEmit})
to expect success return code.

\texttt{npm test} is already set up -- just need to
set up the Python script to call for testing.

\section{To fix later (most important first)}

\subsection*{Receive transitions in React}
This is the current workflow -- need to verify whether the receive transition is safe?

\begin{enumerate}
\item
Assume you are on send state, user triggers
event to perform send.
\item
Runtime performs the send over the WebSocket
and renders the receive state \textbf{by
calling \texttt{setState}}. Calling setState 
schedules the state transition, just like
how a Promise resolve scheduels the call in
the high-priority callback queue.

\item When the receive state is rendered (as a 
result of \texttt{setState}), the lifecycle
method \texttt{componentDidMount} will call \texttt{register} to add the receive handlers 

\item \textbf{At the same time}, message is sent from
other role to this user -- this triggers the 
onMessage event listener on the WebSocket. 
\textbf{There is no guarantee that \texttt{setState} 
is called before the message event listener.}

\end{enumerate}

If the onMessage event listener
is called before the setState,
then the implementation breaks because there is no
handler to use.

\subsubsection*{Possible solution}

Use the same approach as the backend.
\textit{Ignoring the exact TypeScript types for now,}
keep a message queue and a handler queue.

On mounting the receive state (which is the actual
transition), check whether there is a message in the 
queue.
If yes -- pop message and call handler based on label
(and cast payload accordingly).
If no -- add handler to queue.

On message receive, check whether there are handlers
in the queue. 
If yes -- call the correct one based on
the label (and cast payload accordingly).
If no -- add message to queue.

\begin{itemize}
\item
Do we need a queue? Is it possible for more than
one message to be queued up, since you advance to
a receive state and only expect to receive exactly
one message? 

\item
Could use an optional-type variable
but need to consider the types...
\end{itemize}

\subsection*{Exception handling}

TODO read Exceptional GV paper \cite{ExceptionalGV}.

\subsection*{Send transition in React}
Sends can only be event-driven --
currently not possible to do this:

\begin{enumerate}
\item Server sends browser 
\texttt{Offer(int)}
\item Browser immediately sends 
\texttt{Reject()} if payload is less than 0;
otherwise attach a \texttt{Counter(int)} 
send action to a user event.
\end{enumerate}

\section{Extensions}

\subsection*{Explicit connection actions}

See \cite{FASE2017} -- also refer to Battleships 
example from Jonathan.

\begin{lstlisting}
explicit global protocol Game(role Svr, role P1, role P2) {
  Play() connect P1 to Svr;
  Play() connect P2 to Svr;
  do OneGame(Svr, P1, P2);
  disconnect P1 and Svr;
  disconnect P2 and Svr;
 }

global protocol OneGame(role Svr, role P1, role P2)
  Pos(Point) from P1 to Svr;
  choice at Svr {
    Lose(Point) from Svr to P2;
    Win(Point) from Svr to P1;
  } or {
    Draw(Point) from Svr to P2;
    Draw(Point) from Svr to P1;
  } or {
    Update(Point) from Svr to P2;
    Update(Point) from Svr to P1;
    do OneGame(Svr, P2, P1);
  }
}

\end{lstlisting}

\subsection*{Typestate in TypeScript}

Look into TyoeScript transfomers -- the idea is
to define typestate decorators, then augment the
TypeScript compiler to transform the code into some
intermediate layer to perform validation on
typestate: for example, consider the example of a 
stack that can either be full or empty.

\section{Report Structure}

\begin{enumerate}
\item 
Introduction -- as per PLACES (for now)

\item
Background -- process calculus, BST, MPST, code 
generation, TypeScript; think about where React goes?

\item
MPST encoding in TypeScript (section 2 in these notes).
Split into Node and browser targets, also talk about
the different ideas building up to this (pros and
cons of the experiments I did with Calculator in
2019).

Any of the new concepts (exception handling, etc.) will 
go in this section too.

TODO think about how to split into 
more logical sections.

\item
Code generation workflow (section 3 in these notes)
-- software architecture of
toolchain implemented in Python.

\end{enumerate}

\nocite{*}
\bibliographystyle{acm}
\bibliography{notes}

\end{document}
