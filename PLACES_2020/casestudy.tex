\section{Case Study}
\label{section:example}

\begin{figure}
\begin{lstlisting}[language=JavaScript, tabsize=4]
const handleP1Move: S13 = (move: Point) => {
	board.P1(move);			// User logic
	if (board.won()) { return [Labels.Lose, move, move]; }
	else if (board.draw()) { return [Labels.Draw, move, move]; }
	else { return [Labels.Update, move, handleP2Move]; }}

// Instantiate session - `handleP2Move` defined similarly as S18
new NoughtsAndCrosses.Svr(webSocketServer, handleP1Move);
\end{lstlisting}
\captionof{lstlisting}{Session runtime instantiation for \textit{Noughts and Crosses} \texttt{Svr}.}
\label{lst:svrprotocol}
\end{figure}

We apply our framework to implement a web-based implementation of the
\textit{Noughts and Crosses} running example in TypeScript;
the interested reader can find the full implementation in
\cite{NoughtsAndCrosses}.
In addition to MPST-safety, we show that our library design welcomes idiomatic
JavaScript practices in the user implementation and is interoperable with
common front- and back-end frameworks.

\paragraph{Game server}
We set up the WebSocket server as an Express.js \cite{ExpressJS}
application on top of a Node.js \cite{NodeJS} runtime.
We define our own game logic in a \texttt{Board} class to keep track of the
game state and expose methods to query the result.
This custom logic is integrated into our \texttt{handleP1Move} and
\texttt{handleP2Move} handlers (\cref{lst:svrprotocol}), so the session runtime
can handle \texttt{Pos(Point)} messages from players and transition according
to the injected game logic.

Note that, by TypeScript's structural typing \cite{TypeScriptSpec}, replacing
\texttt{handleP2Move} on line 5 with \texttt{handleP1Move} would be
type-correct -- this allows for better code reuse as opposed to defining
additional abstractions to work around the limitations of nominal typing in
\cite{Hybrid2016}.
There is also full type erasure when transpiling to JavaScript to run the
server code, so state space explosion is not a runtime consideration.

\paragraph{Game clients}
We implement the game client for \texttt{P1} and
\texttt{P2} by extending from the generated abstract React (EFSM state)
components and registering those to the session runtime component.

For the sake of code reuse for demonstration purposes, \cite{NoughtsAndCrosses}
uses \textit{higher-order components} (HOC) to build the correct state
implementations depending on which player the user chooses to be.
We leverage the \textit{Redux} \cite{Redux} state management library to keep track of game
state, thus showing the flexibility of our library design in being
interoperable with other libraries and idiomatic JavaScript practices.

