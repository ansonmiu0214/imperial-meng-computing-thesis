\section{Introduction}
% Situation: stateful communication over WebSocket and interactive web applications

Modern interactive web applications aim to provide a highly responsive user
experience by minimising the communication latency between clients and servers.
Whilst the HTTP request-response model is sufficient for retrieving static
assets, applying the same stateless communication approach for interactive use
cases (such as real-time multiplayer games) introduces undesirable performance
overhead.
Developers have since adopted other communication transport abstractions over
HTTP connections such as the WebSockets protocol \cite{WebSocketRFC} to enjoy
low-latency full-duplex client-server communication in their applications over
a single persistent connection.
Enabling more complex communication patterns caters for more interactive use
cases, but introduces additional correctness concerns to the developer.

% Example: noughts and crosses
Consider a classic turn-based board game of \textit{Noughts and Crosses}
between two players.
Both players, identified by either noughts or crosses respectively, take
turns to place a mark on an unoccupied cell of a 3-by-3 grid until one player
wins (when their markers form one straight line on the board) or a stalemate is
reached (when all cells are occupied).
A web-based implementation may involve players connected to a game server via
WebSocket connections.
The players interact with the game from their web browser, which shows a
\textit{single-page application} (SPA) of the game client written in a popular
framework like \textit{React.js} \cite{React}.
SPAs feature a single HTML page and dynamically render content via JavaScript
in the browser.
Players take turns to make a move on the game board, which sends a message to
the server, and the server implements the game logic to progress the game
forward until a result (either a win/loss or draw) can be declared, where
either the move of the other player or the game result is sent to players.

Whilst WebSockets make this web-based implementation possible, it introduces
the developer to a new family of communication errors, even for this simple
game.
In addition to the usual testing for game logic correctness, the
developer needs to test against \textit{deadlocks} (e.g. both players waiting
for each other to make a move at the same time) and \textit{communication
 mismatches} (e.g. player 1 sending a boolean to the game server instead of
the board coordinates).
The complexity of these errors correlate to the
complexity of the required tests and scale with the complexity of
communication patterns involved.

\textit{Multiparty Session Types} (MPST) \cite{MPST} provide a framework for
formally specifying a structured communication pattern between concurrent
processes and verifying implementations for correctness with respect to the
communications aspect.
By specifying the client-server interactions of our game as an MPST protocol
and verifying the implementations against the protocol for conformance,
MPST theory guarantees well-formed implementations to be free from
communication errors.

We see the application of the MPST methodology to generating interactive 
TypeScript web applications to be an interesting design space 
-- to what extent can the MPST methodology be applied to
deliver a workflow where developers use the generated TypeScript APIs in
their application to guarantee protocol conformance by construction?
Such a workflow would ultimately decrease the overhead for incorporating MPST
into mainstream web development, which reduces development time by programmatically
verifying communication correctness of the implementation.



\paragraph{Contributions}
This paper presents a workflow for developing type-safe interactive SPAs
motivated by the MPST framework:
\textbf{(1)} An endpoint API code generation
workflow targeting TypeScript-based web applications for multiparty sessions;
\textbf{(2)} An encoding of session types in server-side TypeScript that
enforces static linearity;
and \textbf{(3)} An encoding of session types in
browser-side TypeScript using the React framework that guarantees affine usage
of communication channels.

%Figure \ref{fig:workflow} illustrates our proposed development workflow: we extend the Scribble framework (\sectionref{section:scribble}) and generate TypeScript endpoints for server-side (\sectionref{section:server}) and browser-side (\sectionref{section:browser}) targets. We will use the \textit{Noughts and Crosses} game as our running example in the rest of the paper and show how our approach is a practical compromise that combines benefits from static session typing and being compatible with common libraries and idiomatic JavaScript practices (\sectionref{section:example}).

